%% Generated by Sphinx.
\def\sphinxdocclass{report}
\documentclass[a4paper,10pt,icelandic]{sphinxmanual}
\ifdefined\pdfpxdimen
   \let\sphinxpxdimen\pdfpxdimen\else\newdimen\sphinxpxdimen
\fi \sphinxpxdimen=.75bp\relax

\PassOptionsToPackage{warn}{textcomp}
\usepackage[utf8]{inputenc}
\ifdefined\DeclareUnicodeCharacter
% support both utf8 and utf8x syntaxes
\edef\sphinxdqmaybe{\ifdefined\DeclareUnicodeCharacterAsOptional\string"\fi}
  \DeclareUnicodeCharacter{\sphinxdqmaybe00A0}{\nobreakspace}
  \DeclareUnicodeCharacter{\sphinxdqmaybe2500}{\sphinxunichar{2500}}
  \DeclareUnicodeCharacter{\sphinxdqmaybe2502}{\sphinxunichar{2502}}
  \DeclareUnicodeCharacter{\sphinxdqmaybe2514}{\sphinxunichar{2514}}
  \DeclareUnicodeCharacter{\sphinxdqmaybe251C}{\sphinxunichar{251C}}
  \DeclareUnicodeCharacter{\sphinxdqmaybe2572}{\textbackslash}
\fi
\usepackage{cmap}
\usepackage[T1]{fontenc}
\usepackage{amsmath,amssymb,amstext}
\usepackage{babel}
\usepackage{times}
\usepackage[Sonny]{fncychap}
\usepackage{sphinx}

\fvset{fontsize=\small}
\usepackage{geometry}

% Include hyperref last.
\usepackage{hyperref}
% Fix anchor placement for figures with captions.
\usepackage{hypcap}% it must be loaded after hyperref.
% Set up styles of URL: it should be placed after hyperref.
\urlstyle{same}

\addto\captionsicelandic{\renewcommand{\figurename}{Mynd}}
\addto\captionsicelandic{\renewcommand{\tablename}{Tafla}}
\addto\captionsicelandic{\renewcommand{\literalblockname}{Listi}}

\addto\captionsicelandic{\renewcommand{\literalblockcontinuedname}{framhald frá fyrri síðu}}
\addto\captionsicelandic{\renewcommand{\literalblockcontinuesname}{continues on next page}}
\addto\captionsicelandic{\renewcommand{\sphinxnonalphabeticalgroupname}{Non-alphabetical}}
\addto\captionsicelandic{\renewcommand{\sphinxsymbolsname}{Tákn}}
\addto\captionsicelandic{\renewcommand{\sphinxnumbersname}{Numbers}}

\addto\extrasicelandic{\def\pageautorefname{page}}




\usepackage{amsmath}
\usepackage{amssymb}
\usepackage{hyperref}


\title{NAME Documentation}
\date{jan. 08, 2020}
\release{2019}
\author{AUTHOR}
\newcommand{\sphinxlogo}{\sphinxincludegraphics{hi_horiz_raunvisindadeild.png}\par}
\renewcommand{\releasename}{Útgáfa}
\makeindex
\begin{document}

\ifdefined\shorthandoff
  \ifnum\catcode`\=\string=\active\shorthandoff{=}\fi
  \ifnum\catcode`\"=\active\shorthandoff{"}\fi
\fi

\pagestyle{empty}
\maketitle
\pagestyle{plain}
\sphinxtableofcontents
\pagestyle{normal}
\phantomsection\label{\detokenize{index::doc}}



\chapter{Formáli}
\label{\detokenize{formali:formali}}\label{\detokenize{formali::doc}}
Þetta kennsluefni er haft til hliðsjónar í fyrirlestrum í áfanganum
Stærðfræðigreining IV við Háskóla Íslands. Það er aðgengilegt sem
vefsíða, \sphinxurl{http://notendur.hi.is/sigurdur/stae401}, og verður aðgengilegt sem \sphinxhref{https://notendur.hi.is/sigurdur/stae401/stae401.pdf}{pdf-skjal} sem hentar
til útprentunar. Efnið byggir að miklu leyti á bókinni \sphinxstyleemphasis{Tvinnfallagreining, afleiðujöfnur, Fourier-greining og hlutafleiðujöfnur} eftir Ragnar Sigurðsson. Bók Ragnars er einnig kennslubók námskeiðsins.

\sphinxstylestrong{Janúar 2019, Sigurður Örn Stefánsson og Valentina Giangreco M Puletti}


\chapter{Gagnlegar upplýsingar}
\label{\detokenize{umnamskeidid:gagnlegar-upplysingar}}\label{\detokenize{umnamskeidid::doc}}

\section{Námsefnið}
\label{\detokenize{umnamskeidid:namsefni}}
Aðallesefni námskeiðsins er bók Ragnars Sigurðssonar sem þið finnið í
námsefnismöppunni á UGLU.

\medskip


\section{Fyrirlestrar}
\label{\detokenize{umnamskeidid:fyrirlestrar}}
Fyrirlestrar verða á mánudögum 8:20\textendash{}9:50 og á miðvikudögum 10:00\textendash{}11:30.
Í fyrirlestraáætlun og á dæmablöðunum er sagt nánar frá efni fyrirlestra
hverrar viku og vísað á viðeigandi efnisgreinar í kennsluefni. Að mestu
verður fylgt þeirri efnisröð sem er í fyrirlestranótum Ragnars. Í
fyrirlestrum mun aðeins gefast tími til að fara yfir helstu atriði
námsefnisins og verðið þið sjálf að kynna ykkur mikinn hluta af
námsefninu upp á eigin spýtur. Yfirferð námsefnisins er hröð og farið er
yfir mikið efni.


\section{Dæmatímar og vinnustofur}
\label{\detokenize{umnamskeidid:daematimar-og-vinnustofur}}
Sameiginlegur dæmatími með dæmareikningi á töflu verður á mánudögum
15:50\textendash{}17:20 (nema í fyrstu vikunni). Í námskeiðinu er einnig boðið upp á
vinnustofur síðdegis á miðvikudögum, 15:00-16:30 og fimmtudögum, 15:50-17:20 (nema í fyrstu
vikunni) sem eru hugsaðar þannig að þið getið komið og fengið aðstoð við
dæmareikning eða fengið nánari útskýringar á atriðum sem hafa vafist
fyrir ykkur.


\section{Skiladæmi}
\label{\detokenize{umnamskeidid:skiladaemi}}
Á misserinu eru skiladæmi, alls 8 talsins sem sett eru fyrir samkvæmt kennsluáætlun í viðauka. Ætlast er til að lausnir séu sjálfstæðar og
afritaðar lausnir, hvorki frumrit né afrit eru teknar gildar. Þó eruð
þið að sjálsögðu hvött til að skiptast á skoðunum og hugmyndum við að leysa verkefnin.

\begin{sphinxadmonition}{attention}{Athugið:}
Til að fá próftökurétt þarf nemandi að skila dæmum í að minnsta kosti \sphinxstylestrong{5 af 8} skiptum.
\end{sphinxadmonition}

Til að fá skil metin þarf að hafa leyst í hvert skipti að
minnsta kosti helming dæmanna (við munum ekki gefa einkunn fyrir
skiladæmi en þið getið hugsað þetta þannig að þið þurfið að fá a.m.k. 5
fyrir skiladæmi til að þau gildi) og frágangur þarf að vera
sómasamlegur. Þið eigið að miða við að úrlausnin sé þannig að jafningjar
ykkar ættu að geta lesið og skilið hana frá upphafi til enda án
hjálpargagna. Takið vinsamlegast tillit til eftirfarandi atriða
\begin{enumerate}
\def\theenumi{\arabic{enumi}}
\def\labelenumi{\theenumi .}
\makeatletter\def\p@enumii{\p@enumi \theenumi .}\makeatother
\item {} 
Merkið lausnina ykkar efst hægra megin á forsíðu.

\item {} 
Skrifið upp dæmatextann og tilgreinið hvar lausnin byrjar.

\item {} 
Hafið útreikninga og röksemdafærslur í vel uppsettum texta og hafið í
huga að stærðfræðitexti lýtur sömu reglum og venjulegt skrifað mál,
t.d. byrja setningar á stórum staf og enda á punkti. Stærri formúlur
er gott að afmarka sérstaklega en stundum fer betur á því að hafa
smærri formúlur inni í texta. \sphinxstylestrong{Dæmi:}

\end{enumerate}

\begin{sphinxadmonition}{attention}{Athugið:}\begin{quote}

Fallið \(f(x) = \sin(x)/x^2\) hefur afleiðuna
\end{quote}
\begin{equation*}
\begin{split}f'(x) = \frac{\cos(x)x^2-2\sin(x)x}{x^4}.\end{split}
\end{equation*}\end{sphinxadmonition}

Takið eftir að málsgreinin hefst á stórum staf og endar á punkti, jafnvel þótt formúla sé í lok málsgreinar. Smærri formúlan er inni í texta en sú stærri er afmörkuð sérstaklega.


\section{Námsmat}
\label{\detokenize{umnamskeidid:namsmat}}
\sphinxstylestrong{Tvisvar á misserinu verða próf} sem gilda 15\% hvort en þó aðeins til hækkunar.
Stefnt er að því að hafa prófin á dagsetningum sem koma fram í kennsluáætlun í viðauka. Þá verður prófað
úr lesnu efni, skilgreiningum og setningum (u.þ.b. 25\%) og jafnframt úr skiladæmunum (u.þ.b. 75\%).

Undir lok námskeiðs er \sphinxstylestrong{eitt stórt skilaverkefni} þar sem fengist er við töluleg verkefni. Verkefnið gildir 20\% af lokaeinkunn en þó aðeins til hækkunar. Nauðsynlegt er að fá einkunnina 5 eða hærra í verkefninu til að ljúka námskeiðinu. Nemendur vinna 2-3 saman í hóp að verkefninu. Nánari fyrirmæli verða gefin síðar.

Í lok námskeiðsins er þriggja tíma skriflegt próf sem gildir 50\%.  Nauðsynlegt er að þið fáið
a.m.k. 5 í lokaprófinu til að standast námskeiðið. Engin hjálpargögn eru
heimil í lokaprófinu, nema hvað prófverkefni fylgir formúlublað sem má
finna á UGLU í námsefnis möppunni.


\section{Tölvunotkun}
\label{\detokenize{umnamskeidid:tolvunotkun}}
Í Stærðfræðigreiningu IV er hluti umfjöllunarefnisins tölulegar aðferðir við að leysa hlutafleiðujöfnur og mun því reyna meira á notkun tölvukerfa en í fyrri stærðfræðigreiningarnámskeiðum. Við munum notast við Octave/Matlab í stærri tölulegum verkefnum.

Í dæmareikningi getur verið gagnlegt að nota aðstoð reiknikerfa til að sannreyna niðurstöður og teikna myndir. Forritin Octave/Matlab geta hentað vel til þess og einnig getur verið gagnlegt að nota Geogebra, Sage eða Wolfram alpha. Athugið að hægt er að keyra létta reikninga í Octave í vafra \sphinxurl{http://octave-online.com}. Athugið samt að á prófi þurfið þið að reikna sjálf án aðstoðar vasareikna og tölvukerfa.


\section{Að taka námskeiðið í annað sinn}
\label{\detokenize{umnamskeidid:a-taka-namskeii-i-anna-sinn}}
Þau sem sátu námskeiðið í fyrra og unnu sér inn próftökurétt þá halda
próftökuréttinum en eldri próftökuréttur gildir ekki. Vinsamlegast
sendið tölvupóst á kennara ef þið viljið halda próftökuréttinum frá því í
fyrra. \sphinxstylestrong{Einkunnir úr miðmisserisprófum/ vetrareinkunn gilda ekki frá
því í fyrra.} Við hvetjum ykkur eindregið til að taka þátt í námskeiðinu af
fullum krafti og skila dæmum þó svo að þið hafið eldri próftökurétt.


\section{Viðtalstímar kennara og fyrirspurnir}
\label{\detokenize{umnamskeidid:vitalstimar-kennara-og-fyrirspurnir}}
Kennarar námskeiðsins eru Sigurður Örn Stefánsson (fyrri hluti) og Valentina Giangreco M Puletti (síðari hluti) en þau eru með skrifstofur 330 og 309 á 3. hæð í Tæknigarði. Viðtalstímar við kennara eru samkvæmt samkomulagi.

Við munum notast við Piazza vefinn þar sem þið getið spjallað um efni námskeiðsins, skipulag, heimaverkefni og fleira. Ég legg áherslu á að þetta er hugsað sem vettvangur fyrir \sphinxstylestrong{ykkur} til að ræða saman og þið getið ekki treyst því að öllum fyrirspurnum verði svarað þar samstundis.

\begin{sphinxadmonition}{important}{Mikilvægt:}
Þar sem mjög margir nemendur eru í námskeiðinu biðjum við  ykkur um að íhuga áður en þið sendið tölvupóst hvort svarið við spurningunni sé að finna í þessu skjali eða hvort þið gætuð borið spurninguna fram í fyrirlestri, dæmatíma, stoðtíma, á Piazza vefnum eða í viðtalstíma.
\end{sphinxadmonition}


\chapter{Hlutafleiðujöfnur}
\label{\detokenize{Kafli01:hlutafleiujofnur}}\label{\detokenize{Kafli01::doc}}

\section{Inngangur}
\label{\detokenize{Kafli01:inngangur}}
Mörg verkefni í t.a.m. verkfræði og eðlisfræði krefjast þess að ákvarða fall af mörgum breytistærðum sem lýsir einhverjum eiginleika kerfis. Slíkum föllum er oft lýst með afleiðujöfnum þar sem hlutafleiður með tilliti til mismunandi breytistærða koma við sögu. Slíkar afleiðujöfnur nefnast \sphinxstyleemphasis{hlutafleiðujöfnur}.

Ef \(u\) er fall af breytistærðunum \(x_1,x_2,\ldots,x_m\) skrifum við hlutafleiður þess með tilliti til \(x_j\) með
einhverjum eftirfarandi tákna
\begin{equation*}
\begin{split}\partial_j u, \quad \frac{\partial u}{\partial x_j},\quad  \partial_{x_j} u, \quad u'_{x_j} \quad \text{eða}\quad  u_{x_j}.\end{split}
\end{equation*}
Í sumum tilfellum hefur skapast venja að nota ákveðið táknmál fyrir breytistærðirnar. Til dæmis er \(t\) gjarnan notað fyrir tíma og \(x,y,z\) fyrir rúmvíddirnar þrjár.

Þegar breytistærðirnar eru margar getur verið þægilegt að nota eftirfarandi rithátt:


\subsection{Ritháttur - Fjölvísir}
\label{\detokenize{Kafli01:rithattur-fjolvisir}}
Ef \(\alpha = (\alpha_1,\ldots,\alpha_m)\) er vigur af ekki neikvæðum heiltölum skilgreinum við hlutafleiðuvirkjann \(\partial^\alpha\) með
\begin{equation*}
\begin{split}\partial^\alpha u = \partial_1^{\alpha_1}\cdots \partial_m^{\alpha_m} u.\end{split}
\end{equation*}
Vigurinn \(\alpha\) nefnist í þessu samhengi \sphinxstyleemphasis{fjölvísir}.


\subsection{Ritháttur - Laplace-virki}
\label{\detokenize{Kafli01:rithattur-laplace-virki}}
Virkinn
\begin{equation*}
\begin{split}\Delta = \partial_{x_1}^2+\cdots + \partial_{x_m}^2\end{split}
\end{equation*}
kallast Laplace-virkinn í \(m\) breytistærðum.


\subsection{Skilgreining - Stig hlutafleiðu og hlutafleiðujöfnu}
\label{\detokenize{Kafli01:skilgreining-stig-hlutafleiu-og-hlutafleiujofnu}}
Hlutafleiðan \(\partial^\alpha u\) hefur stig \(|\alpha| = \alpha_1 + \cdots + \alpha_m\).  Hæsta stig á afleiðu sem kemur fyrir í hlutafleiðujöfnu nefnist \sphinxstyleemphasis{stig} hlutafleiðujöfnunnar.

Við munum skoða \sphinxstylestrong{dæmi um hlutafleiðujöfnur} og læra \sphinxstylestrong{mismunandi aðferðir við að leysa þær}. Í sumum tilfellum má leysa jöfnurnar og skrifa niður beina lausn en oft þarf að nota \sphinxstylestrong{tölulegar aðferðir} til að leysa verkefnin. Töluleg verkefni verða viðfangsefni síðara hluta námskeiðsins.


\subsection{Línulegar hlutafleiðujöfnur}
\label{\detokenize{Kafli01:linulegar-hlutafleiujofnur}}
Við munum eingöngu fást við línulegar hlutafleiðujöfnur í þessu námskeiði. Hlutafleiðujafna er sögð vera \sphinxstylestrong{línulegt} ef hægt er að rita hana á forminu
\begin{equation*}
\begin{split}\sum_{|\alpha|\leq m} a_\alpha(x) \partial^\alpha u(x) = f(x), \quad x\in X \subseteq \mathbb{R}^n.\end{split}
\end{equation*}
Fallið \(u\) er óþekkta stærðin sem við viljum reikna, \(a_\alpha(x)\) eru stuðlar sem geta verið háðir \(x\) og fallið \(f\) er gefið. Ef \(f\) er núllfallið segjum við að hlutafleiðujafan sé \sphinxstylestrong{óhliðruð} en annars að hún sé \sphinxstylestrong{hliðruð}.

Við munum einnig nota ritháttin
\begin{equation*}
\begin{split}Lu = f\end{split}
\end{equation*}
þar sem við lítum á
\begin{equation*}
\begin{split}L = \sum_{|\alpha|\leq m} a_\alpha(x) \partial^\alpha\end{split}
\end{equation*}
sem línulegan virkja \(L: C^m(X) \to C(X), X\subseteq \mathbb{R}^n\) sem úthlutar falli línulegri samantekt af fallinu sjálfu og hlutafleiðum þess upp að stigi \(m\). Virkinn \(L\) er línulegur því
\begin{equation*}
\begin{split}L(au + bv) = aL(u) + bL(v)\end{split}
\end{equation*}
fyrir hvaða tölur \(a\) og \(b\) sem er. \sphinxstylestrong{Kjarni} eða \sphinxstylestrong{núllrúm} virkjans \(L\) er skilgreint sem mengi allra þeirra \(u\in C^m(X)\) sem eru lausnir á óhliðruðu jöfnunni \(Lu=0\). Ef \(u_p\) er lausn á \(Lu = f\) þá er sérhver önnur lausn á forminu \(u = v+u_p\) þar sem \(v\) er í núllrúminu.


\section{Dæmi um hlutafleiðujöfnur í eðlisfræði}
\label{\detokenize{Kafli01:daemi-um-hlutafleiujofnur-i-elisfraei}}

\subsection{Varmaleiðnijafnan}
\label{\detokenize{Kafli01:varmaleinijafnan}}
Ef \(T\) er fall af \(m+1\) breytistærðum \(x_1,\ldots,x_m,t\) kallast jafnan
\begin{equation*}
\begin{split}\frac{\partial T}{\partial t} - \kappa \Delta T = f(x_1,\ldots,x_m,t)\end{split}
\end{equation*}
\sphinxstyleemphasis{varmaleiðnijafnan} í \(m\) rúmvíddum. Talan \(\kappa\) ákvarðast af eiginleikum þess kerfis sem fengist er við og fallið \(f\) svarar til ytri áhrifa á kerfið.

Varmaleiðnijafnan lýsir því hvernig hitastig \(T\) í hlut breytist með tíma. Þá svarar \(f\) til áhrifa ytri varmagjafa. Jafnan getur einnig lýst dreifingu efnis sem leyst er upp í vökva og er þá gjarnan nefnd \sphinxstyleemphasis{sveimjafna}.


\subsection{Bylgjujafnan}
\label{\detokenize{Kafli01:bylgjujafnan}}
Ef \(u\) er fall af \(m+1\) breytistærðum \(t, x_1,\ldots,x_m\) kallast jafnan
\begin{equation*}
\begin{split}\frac{\partial^2 u}{\partial t^2} - c^2 \Delta u = f(x_1,\ldots,x_m,t)\end{split}
\end{equation*}
\sphinxstyleemphasis{bylgjujafnan} í \(m\) rúmvíddum. Talan \(c\) hefur einingu hraða og ákvarðast af eiginleikum þess kerfis sem fengist er við og fallið \(f\) svarar til svarar til ytri áhrifa á kerfið.

Bylgjujafnan kemur mjög víða við sögu í eðlisfræði. Hún getur til dæmis lýst sveiflu á einvíðum streng eða tvívíðri trommu en þá táknar \(u\) færslu strengsins eða trommuskinnsins frá jafnvægisstöðu og \(f\) svarar til ytri krafts, t.d. ef strengurinn er plokkaður eða tromman slegin. Annað dæmi er lýsing á útbreiðslu rafsegulbylgna en í því tilfelli má leiða bylgjujöfnuna út frá jöfnum Maxwells.


\subsection{Dæmi - Sveifla á einvíðum streng}
\label{\detokenize{Kafli01:daemi-sveifla-a-einvium-streng}}
Hér má sjá lausn á bylgjujöfnunni
\begin{equation*}
\begin{split}\frac{\partial^2 u(x,t)}{\partial t^2} - c^2  \frac{\partial^2 u(x,t)}{\partial x^2} = 0\end{split}
\end{equation*}
fyrir \(x\) á bilinu \([0,L]\) með jaðarskilyrðunum \(u(0,t) = u(L,t)=0\) (strengurinn er fastur í báða enda) og upphafsskilyrðunum \(u(x,0) = a(x),~\partial_t u(x,0) = b(x)\). Upphaflegu stillingarnar eru \(L=2\pi\), \(a(x) = \sin(2x)+\sin(3x)\) og \(b(x) = \sin(x)\).


\begin{center}
\includegraphics[width=4cm,keepaspectratio=true]{polarggb.png}
\end{center}


\begin{DUlineblock}{0em}
\item[] 
\item[] 
\end{DUlineblock}


\section{Hliðarskilyrði. Vel framsett verkefni}
\label{\detokenize{Kafli01:hliarskilyri-vel-framsett-verkefni}}
Skoðum verkefnið að ákvarða fall \(u\) sem uppfyllir hlutafleiðujöfnu \(Lu = f\) á mengi \(X \times I \in \mathbb{R}^{n+1}\), þar sem \(X\subseteq \mathbb{R}^n\) er opið mengi og \(I \subseteq \mathbb{R}\) er bil. Hugsum um breytuna \(x\in X\) sem rúmbreytu og breytuna \(t\in I\) sem tíma.

Til að ákvarða \(u\)  ótvírætt þarf oft hliðarskilyrði á fallið. Þau geta verið á eftirfarandi formi.


\subsection{Upphafsskilyrði}
\label{\detokenize{Kafli01:upphafsskilyri}}
Þá eru gildi á fallinu \(u\) og einhverjum tímaafleiðum þess \(\partial_t u,\partial_t^2 u,\ldots\) gefin á ákveðnum upphafstíma. Nefnast einnig \sphinxstyleemphasis{Cauchy-skilyrði}.


\subsection{Jaðarskilyrði}
\label{\detokenize{Kafli01:jaarskilyri}}
Skilgreinum stefnuafleiðu \(u\) út um jaðar \(X\) með
\begin{equation*}
\begin{split}\frac{\partial u}{\partial n} = \nabla u \cdot \vec n\end{split}
\end{equation*}
þar sem \(\nabla\) er stigull með tilliti til rúmbreytanna og \(\vec n\) er einingarþvervigur sem stefnir út úr \(X\) (þegar það hefur merkingu).

Mikilvæg jaðarskilyrði sem koma upp víða í eðlisfræði eru á eftirfarandi formi
\begin{enumerate}
\def\theenumi{\arabic{enumi}}
\def\labelenumi{\theenumi .}
\makeatletter\def\p@enumii{\p@enumi \theenumi .}\makeatother
\item {} 
Lausnin \(u\) er tilgreind á jaðri svæðisins. Nefnist \sphinxstyleemphasis{Dirichlet-skilyðri} eða \sphinxstyleemphasis{fallsjaðarskilyrði}.

\item {} 
Stefnuafleiðan \(\partial u/\partial n\) er tilgreind á jaðri svæðisins. Nefnist \sphinxstyleemphasis{Neumann-skilyrði} eða \sphinxstyleemphasis{flæðisskilyrði}.

\item {} 
Línuleg samantekt af \(u\) og \(\partial u/\partial n\) er tilgreind á jaðri svæðis. Nefnist \sphinxstyleemphasis{Robin-skilyrði} eða \sphinxstyleemphasis{blandað jaðarskilyrði}.

\end{enumerate}

Athugið að jaðarskilyrði fyrir venjulegar afleiðujöfnur eru yfirleitt í 1 eða 2 punktum en jaðar mengis \(X \subseteq \mathbb{R}\) getur verið mjög almennur.


\subsection{Vel framsett verkefni}
\label{\detokenize{Kafli01:vel-framsett-verkefni}}
Úrlausn á hlutafleiðujöfnu með hliðarskilyrðum nefnist \sphinxstyleemphasis{vel framsett verkefni}, ef eftirfarandi
þrjú skilyrði eru uppfyllt:
\begin{enumerate}
\def\theenumi{\arabic{enumi}}
\def\labelenumi{\theenumi .}
\makeatletter\def\p@enumii{\p@enumi \theenumi .}\makeatother
\item {} 
\sphinxstylestrong{Tilvist:} Til er lausn sem uppfyllir jöfnuna og öll hliðarskilyrðin.

\item {} 
\sphinxstylestrong{Ótvíræðni:} Aðeins ein lausn er til.

\item {} 
\sphinxstylestrong{Stöðugleiki:} Lausnin er stöðug í þeim skilningi að lítilsháttar frávik frá hliðarskilyrðum kemur fram í lítilsháttar fráviki frá lausninni. Í hverju verkefni um sig þarf að skigreina hvaða mælikvarði er lagður á frávik í hliðarskilyrðum og í lausn.

\end{enumerate}

Við munum leggja mesta áherslu á skilyrðið \sphinxstylestrong{1. Tilvist} í þessu námskeiði.


\section{Fyrsta stigs jöfnur}
\label{\detokenize{Kafli01:fyrsta-stigs-jofnur}}
Línuleg fyrsta stigs hlutafleiðujafna af tveimur breytistærðum \((x,y)\) er af gerðinni
\begin{equation*}
\begin{split}a(x,y)\frac{\partial u}{\partial x} + b(x,y) \frac{\partial u}{\partial y} + c(x,y)u = f(x,y).\end{split}
\end{equation*}
Skoðum aðferðir við að leysa slíkar jöfnur.


\subsection{Kennilínuaðferðin}
\label{\detokenize{Kafli01:kennilinuaferin}}

\subsection{Setning}
\label{\detokenize{Kafli01:setning}}
Fall \(u\in C^1(\mathbb{R}^2)\) er lausn á jöfnunni
\begin{equation*}
\begin{split}a\frac{\partial u}{\partial x}+ b\frac{\partial u}{\partial y} = 0\end{split}
\end{equation*}
þar sem \((a,b)\in\mathbb{R}^2\) og \((a,b)\neq (0,0)\) þá og því aðeins að \(u\) sé af gerðinni
\begin{equation*}
\begin{split}u(x,y) = f(bx-ay)\end{split}
\end{equation*}
með \(f\in C^1(\mathbb{R})\).


\subsection{Setning}
\label{\detokenize{Kafli01:id1}}
Upphafsgildisverkefnið
\begin{equation*}
\begin{split}\left\{\begin {array}{l}
a\frac{\partial u}{\partial x}+ b\frac{\partial u}{\partial y} = 0, \quad (x,y)\in \mathbb{R}^2, \\
u(x,0) = \phi(x),\quad x \in \mathbb{R}
\end{array}\right.\end{split}
\end{equation*}
þar sem \(\phi \in C^1(\mathbb{R})\) er gefið fall og \(b\neq 0\) hefur ótvírætt ákvarðaða lausn
\begin{equation*}
\begin{split}u(x,y) = \phi(x-ay/b).\end{split}
\end{equation*}

\subsection{Skilgreining}
\label{\detokenize{Kafli01:skilgreining}}
Lína sem hefur stefnuvigur samsíða \((a, b)\) nefnist kennilína afleiðuvirkjans \(a\partial_x + b\partial_y\).


\subsection{Skilgreining}
\label{\detokenize{Kafli01:id2}}
Sérhver lausn  á afleiðujöfnuhneppinu
\begin{equation*}
\begin{split}\xi' = a(\xi,\eta), \qquad \eta' = b(\xi,\eta),\end{split}
\end{equation*}
nefnist kenniferill eða kennilína afleiðuvirkjans
\begin{equation*}
\begin{split}a(x,y)\frac{\partial}{\partial x} + b(x,y) \frac{\partial}{\partial y}\end{split}
\end{equation*}

\subsection{Reikniaðferð}
\label{\detokenize{Kafli01:reikniafer}}
Finna skal lausn á upphafsgildisverkefninu
\begin{equation*}
\begin{split}\left\{\begin {array}{l}
a(x,y)\frac{\partial u}{\partial x}+ b(x,y)\frac{\partial u}{\partial y} = 0, \quad (x,y)\in \mathbb{R}^2, \\
u(x,0) = \phi(x), \quad x \in \mathbb{R}.
\end{array}\right.\end{split}
\end{equation*}\begin{enumerate}
\def\theenumi{\arabic{enumi}}
\def\labelenumi{\theenumi .}
\makeatletter\def\p@enumii{\p@enumi \theenumi .}\makeatother
\item {} 
Tökum punkt \((x,y)\) í \((\xi,\eta)\) plani. Leysum verkefnið

\end{enumerate}
\begin{equation*}
\begin{split}\xi' = a(\xi,\eta), \qquad \eta' = b(\xi,\eta), \qquad \xi(0) = x, \quad \eta(0) = y.\end{split}
\end{equation*}\begin{enumerate}
\def\theenumi{\arabic{enumi}}
\def\labelenumi{\theenumi .}
\makeatletter\def\p@enumii{\p@enumi \theenumi .}\makeatother
\setcounter{enumi}{1}
\item {} 
Ef til er ótvírætt ákvörðuð lausn \((\xi(t),\eta(t))\) á einhverju opnu bili fyrir sérhvert \((x,y)\) og ferillinn sker \(\xi\)-ásinn í nákvæmlega einum punkti \((s(x,y),0)\) þá er lausnin gefin með formúlunni

\end{enumerate}
\begin{equation*}
\begin{split}u(x,y) = \phi(s(x,y)).\end{split}
\end{equation*}

\subsection{Úrlausn með Laplace-ummyndun}
\label{\detokenize{Kafli01:urlausn-me-laplace-ummyndun}}
Laplace ummyndun er gagnleg þegar leysa skal upphafsgildisverkefni og virkar einnig þegar um hlutafleiðujöfnur er að ræða. Eftirfarandi reikniaðferð má beita á fyrsta stigs hlutafleiðujöfnu falls \(u(x,t)\) þegar stuðlarnir eru ekki háðir \(t\).
\begin{enumerate}
\def\theenumi{\arabic{enumi}}
\def\labelenumi{\theenumi .}
\makeatletter\def\p@enumii{\p@enumi \theenumi .}\makeatother
\item {} 
Tökum Laplace-mynd af báðum hliðum miðað við breytistærðina \(t\). Gert er ráð fyrir að víxla megi á afleiðum og heildum þar sem þarf.

\item {} 
Þá fæst fyrsta stigs venjuleg afleiðujafna í \(x\) fyrir fallið

\end{enumerate}
\begin{equation*}
\begin{split}U(x,s) = \mathcal{L}\{u(x,t)\}(s) = \int_{0}^\infty e^{-st}u(x,t) dt\end{split}
\end{equation*}
sem má leysa með almennri lausnarformúlu.
\begin{enumerate}
\def\theenumi{\arabic{enumi}}
\def\labelenumi{\theenumi .}
\makeatletter\def\p@enumii{\p@enumi \theenumi .}\makeatother
\setcounter{enumi}{2}
\item {} 
Lausn upphaflega verkefnisins fæst með því að taka andhverfu Laplace-myndina af \(U(x,s)\).

\end{enumerate}


\subsection{Dæmi}
\label{\detokenize{Kafli01:daemi}}
Upphafsgildisverkefnið
\begin{equation*}
\begin{split}\left\{\begin {array}{l}
\frac{\partial u}{\partial t}+ x\frac{\partial u}{\partial x} + u = f(x,t), x>0, t>0, \\
u(x,0) = u(0,t) = 0.
\end{array}\right.\end{split}
\end{equation*}
hefur lausnina
\begin{equation*}
\begin{split}u(x,t) = x^{-1}\int_{0}^x H(t-\ln(x/\xi)) f(\xi,t-\ln(x/\xi)) d\xi\end{split}
\end{equation*}
þar sem \(H\) táknar Heaviside-fallið.


\chapter{Fourier-raðir}
\label{\detokenize{Kafli02:fourier-rair}}\label{\detokenize{Kafli02::doc}}

\section{Inngangur}
\label{\detokenize{Kafli02:inngangur}}
Rifjum upp að ef við höfum grunn af vigrum \(x_1,x_2,\ldots,x_n\) í \(\mathbb{R}^n\) má rita sérhvern vigur \(y\) sem
\begin{equation*}
\begin{split}y = a_1 x_1 + \cdots + a_n x_n\end{split}
\end{equation*}
þar sem stuðlarnir \(a_1,\ldots,a_n\) eru ótvírætt ákvarðaðir. Við munum nú spyrja okkur spurningarinnar, er hægt að gera eitthvað sambærilegt þegar vigurrúmið er óendanlega vítt, t.d. þegar það samanstendur af föllum. Við þekkjum dæmi um slíkt, þegar rita má óendanlega oft diffranlegt fall \(f\) með Taylor-röð þess
\begin{equation*}
\begin{split}f(x) = a_0\cdot 1 + a_1 x + a_2 x^2 + \cdots, \qquad \text{þar sem} \quad a_n = \frac{f^{(n)}(0)}{n!}.\end{split}
\end{equation*}
Í þessum kafla munum við skilgreina svokallaðar Fourier-raðir sem líta svipað út en í stað fallanna \(1,x,x^2,\ldots\) munum við liða lotubundin föll \(f\) í grunn sem samanstendur af hornaföllum (eða jafngilt, veldisvísisföllum), finna formúlur fyrir stuðlunum í framsetningunni og loks skoða hvernig má nota raðirnar við lausn hlutafleiðujafna.


\subsection{Skilgreining}
\label{\detokenize{Kafli02:skilgreining}}
Fall \(f: \mathbb{R}\to \mathbb{R}\) er sagt vera \(T\)-lotubundið ef \(f(x+T) = f(x)\) fyrir öll \(x\in\mathbb{R}\).


\section{Fourier-raðir}
\label{\detokenize{Kafli02:id1}}
Við munum skoða tilfellið þegar föllin eru lotubundin með lotu \(2\pi\) og sjá hvaða skilyrði tryggja að hægt sé að liða slík föll í grunn sem samanstendur af hornaföllunum \(\sin(nx)\) og \(\cos(nx)\) annars vegar eða \(e^{nix}\) hins vegar, þar sem \(n\geq 0\).


\subsection{Skilgreining}
\label{\detokenize{Kafli02:id2}}
Látum \(I\subseteq \mathbb{R}\) vera bil.
\begin{enumerate}
\def\theenumi{\arabic{enumi}}
\def\labelenumi{\theenumi )}
\makeatletter\def\p@enumii{\p@enumi \theenumi )}\makeatother
\item {} 
Rúmið \(L^1(I)\) er mengi þeirra falla \(f: I \to \mathbb{C}\) þannig að

\end{enumerate}
\begin{equation*}
\begin{split}\int_I |f(x)| dx < \infty.\end{split}
\end{equation*}\begin{enumerate}
\def\theenumi{\arabic{enumi}}
\def\labelenumi{\theenumi )}
\makeatletter\def\p@enumii{\p@enumi \theenumi )}\makeatother
\setcounter{enumi}{1}
\item {} 
Rúmið \(L^2(I)\) er mengi þeirra falla \(f: I \to \mathbb{C}\) þannig að

\end{enumerate}
\begin{equation*}
\begin{split}\int_I |f(x)|^2 dx < \infty.\end{split}
\end{equation*}
Ef \(f\) og \(g\) eru föll í \(L^2(I)\) kallast
\begin{equation*}
\begin{split}\langle f, g \rangle = \frac{1}{|I|} \int_I f(x) \overline{g(x)} dx\end{split}
\end{equation*}
innfeldi þeirra (misjafnt er hvort deilt er með \(|I|\), lengdinni á \(I\), í skilgreiningunni). Ef \(\langle f, g \rangle = 0\) segjum við að \(f\) og \(g\) séu hornrétt.

\begin{sphinxadmonition}{attention}{Athugið:}
\(L^j(I)\), \(j=1,2\) eru vigurrúm, af því að
\begin{enumerate}
\def\theenumi{\arabic{enumi}}
\def\labelenumi{\theenumi .}
\makeatletter\def\p@enumii{\p@enumi \theenumi .}\makeatother
\item {} 
Ef \(f \in L^j(I)\) og \(g \in L^j(I)\) þá er fallið \(f+g \in L^j(I)\)

\item {} 
Ef \(f \in L^j(I)\) þá er \(\alpha f \in L^j(I)\), þar sem \(\alpha \in\mathbb R\)

\end{enumerate}
\end{sphinxadmonition}


\subsection{Skilgreining}
\label{\detokenize{Kafli02:id3}}
Ef \(f \in L^1([-\pi,\pi])\) er \(2\pi\)-lotubundið þá skilgreinum við Fourier-stuðla þess með
\begin{equation*}
\begin{split}c_n = c_n(f) = \frac{1}{2\pi} \int_{-\pi}^\pi e^{-inx} f(x) dx, \quad n = \ldots,-2,-1,0,1,2,\ldots,\end{split}
\end{equation*}
Fourier-kósínus-stuðla \(f\) með
\begin{equation*}
\begin{split}a_n = a_n(f) = \frac{1}{\pi} \int_{-\pi}^\pi f(x) \cos(nx) dx, \quad n = 0,1,2,\ldots,\end{split}
\end{equation*}
og Fourier-sínus-stuðla \(f\) með
\begin{equation*}
\begin{split}b_n = b_n(f) = \frac{1}{\pi} \int_{-\pi}^\pi f(x) \sin(nx) dx, \quad n = 1,2,\ldots.\end{split}
\end{equation*}
Raðirnar
\begin{equation*}
\begin{split}\frac{a_0}{2} + \sum_{n\geq 1} \left(a_n \cos(nx) + b_n \sin(nx)\right) \quad \text{og} \quad \sum_{n=-\infty}^\infty c_n e^{inx}\end{split}
\end{equation*}
kallast Fourier-raðir \(f\) og til aðgreiningar er sú fyrri oft nefnd hornafallaröð \(f\).

\begin{DUlineblock}{0em}
\item[] 
\item[] 
\end{DUlineblock}


\begin{center}
\includegraphics[width=4cm,keepaspectratio=true]{polarggb.png}
\end{center}


\(2\pi\)-lotubundna fallið er skilgreint með því að gefa formúlu fyrir því á bilinu \([0,2\pi]\).

\begin{DUlineblock}{0em}
\item[] 
\item[] 
\end{DUlineblock}

\begin{sphinxadmonition}{attention}{Athugið:}
Þegar \(T\)-lotubundið fall er heildað yfir eina lotu skiptir ekki máli hvar upphafspunktur heildisins er valinn, þ.e.
\begin{equation*}
\begin{split}\int_{-T/2}^{T/2} f(x) dx = \int_0^T f(x) dx = \int_\alpha^{\alpha + T}f(x)dx, \quad \text{fyrir öll $\alpha\in\mathbb{R}$.}\end{split}
\end{equation*}\end{sphinxadmonition}


\subsection{Setning - Reiknireglur}
\label{\detokenize{Kafli02:setning-reiknireglur}}
Látum \(f,g\in L^1([-\pi,\pi])\) vera \(2\pi\)-lotubundin föll.
\begin{enumerate}
\def\theenumi{\arabic{enumi}}
\def\labelenumi{\theenumi .}
\makeatletter\def\p@enumii{\p@enumi \theenumi .}\makeatother
\item {} 
Fourier-stuðlarnir eru línulegar varpanir á \(L^1([-\pi,\pi])\),

\end{enumerate}
\begin{equation*}
\begin{split}\begin {align*}
    a_n(\alpha f+\beta g) &= \alpha a_n(f) + \beta a_n(g) \\
    b_n(\alpha f+\beta g) &= \alpha b_n(f) + \beta b_n(g) \\
    c_n(\alpha f+\beta g) &= \alpha c_n(f) + \beta c_n(g)
\end{align*}\end{split}
\end{equation*}\begin{enumerate}
\def\theenumi{\arabic{enumi}}
\def\labelenumi{\theenumi .}
\makeatletter\def\p@enumii{\p@enumi \theenumi .}\makeatother
\setcounter{enumi}{1}
\item {} 
Eftirfarandi samband gildir

\end{enumerate}
\begin{equation*}
\begin{split}\begin {align*}
a_0 &= 2c_0, \qquad a_n = c_n + c_{-n}, \qquad b_n = i(c_n-c_{-n}),  \\
c_0 &= \frac{a_0}{2}, \qquad c_n = \frac{1}{2}(a_n-ib_n), \qquad c_{-n} = \frac{1}{2}(a_n+ib_n), \quad n\geq 1.
\end {align*}\end{split}
\end{equation*}\begin{enumerate}
\def\theenumi{\arabic{enumi}}
\def\labelenumi{\theenumi .}
\makeatletter\def\p@enumii{\p@enumi \theenumi .}\makeatother
\setcounter{enumi}{2}
\item {} 
Ef \(g(x) = f(x+\alpha)\) , þar sem \(\alpha \in \mathbb{R}\) þá er \(c_n(g) = e^{i n\alpha} c_n(f)\) fyrir öll \(n=0,\pm 1,\pm2,\ldots\).

\item {} 
Ef \(f\) er raungilt fall þá eru \(a_n(f)\) og \(b_n(f)\) rauntölur og \(c_{-n}(f) = \overline{c_n(f)}\).

\item {} 
Ef \(f\) er jafnstætt fall þá er \(b_n(f) = 0\) fyrir öll \(n=1,2,3,\ldots\) og

\end{enumerate}
\begin{equation*}
\begin{split}a_n(f) = \frac{2}{\pi} \int_0^\pi f(x) \cos(nx) dx.\end{split}
\end{equation*}
6 Ef \(f\) er oddstætt fall þá er \(a_n(f) = 0\) fyrir öll \(n=0,1,2,\ldots\) og
\begin{equation*}
\begin{split}b_n(f) = \frac{2}{\pi} \int_0^\pi f(x) \sin(nx) dx.\end{split}
\end{equation*}\begin{enumerate}
\def\theenumi{\arabic{enumi}}
\def\labelenumi{\theenumi .}
\makeatletter\def\p@enumii{\p@enumi \theenumi .}\makeatother
\setcounter{enumi}{6}
\item {} 
Ef \(f,f',\ldots,f^{(m)}\) eru í \(L_1([-\pi,\pi])\) þá er

\end{enumerate}
\begin{equation*}
\begin{split}c_n(f^{(k)}) = (in)^k c_n(f), \quad 0\leq k \leq m, \quad n \in \mathbb{Z}.\end{split}
\end{equation*}

\subsection{Skilgreining}
\label{\detokenize{Kafli02:id4}}
Ef \(f \in L^1([-T/2,T/2])\) er \(T\)-lotubundið þá setjum við \(\omega = 2\pi/T\) og skilgreinum Fourier-stuðla þess með
\begin{equation*}
\begin{split}c_n = c_n(f) = \frac{1}{T} \int_{-T/2}^{T/2} e^{-in \omega x} f(x) dx, \quad n = \ldots,-2,-1,0,1,2,\ldots,\end{split}
\end{equation*}
Fourier-kósínus-stuðla \(f\) með
\begin{equation*}
\begin{split}a_n = a_n(f) = \frac{2}{T} \int_{-T/2}^{T/2} f(x) \cos(n\omega x) dx, \quad n = 0,1,2,\ldots,\end{split}
\end{equation*}
og Fourier-sínus-stuðla \(f\) með
\begin{equation*}
\begin{split}b_n = b_n(f) = \frac{2}{T} \int_{-T/2}^{T/2} f(x) \sin(n\omega x) dx, \quad n = 1,2,\ldots.\end{split}
\end{equation*}
Raðirnar
\begin{equation*}
\begin{split}\frac{a_0}{2} + \sum_{n\geq 1} \left(a_n \cos(n\omega x) + b_n \sin(n\omega x)\right) \quad \text{og} \quad \sum_{n=-\infty}^\infty c_n e^{in\omega x}\end{split}
\end{equation*}
kallast Fourier-raðir \(f\) og til aðgreiningar er sú fyrri oft nefnd hornafallaröð \(f\).

\begin{sphinxadmonition}{attention}{Athugið:}
Sambærilegar reiknireglur fyrir \(T\)-lotubundin fást út frá reglunum fyrir \(2\pi\)-lotubundin föll, með því að „skipta \(2\pi\) út fyrir \(T\) “ á viðeigandi stöðum.
\end{sphinxadmonition}


\section{Samleitni Fourier-raða}
\label{\detokenize{Kafli02:samleitni-fourier-raa}}
Í þessari grein fjöllum við um skilyrði sem tryggja samleitni Fourier-raða falls og hvenær og í hvaða skilningi fallið er jafnt Fourier-röð sinni. Við munum notast talsvert við innfeldið sem skilgreint er á \(L^2([-\pi,\pi])\) og setjum því fram nokkrar reiknireglur um innfeldi


\subsection{Reiknireglur um innfeldi}
\label{\detokenize{Kafli02:reiknireglur-um-innfeldi}}
Ef \(u,v,w\in L^2([-\pi,\pi])\) og \(\alpha,\beta \in \mathbb{C}\) þá gilda eftirfarandi reiknireglur
\begin{equation*}
\begin{split}\begin{gathered}
 {{\langle \alpha u + \beta v,w\rangle}}= \alpha{{\langle u,w\rangle}} + \beta {{\langle v,w\rangle}},\\
 {{\langle u,\alpha v + \beta w\rangle}}= \overline\alpha {{\langle u,v\rangle}} + \overline
 \beta {{\langle u,w\rangle}},\\
 {{\langle u,v\rangle}} = \overline{{{\langle v,u\rangle}}},\\
 {{\langle u,u\rangle}}\geq 0.\end{gathered}\end{split}
\end{equation*}
Síðasta reglan leyfir okkur að skilgreina lengd
eða staðal fallsins \(u\) sem
\begin{equation*}
\begin{split}\| u\|= \sqrt{{{\langle u,u\rangle}}}.\end{split}
\end{equation*}
Ein mikilvægasta ójafna stærðfræðinnar er Cauchy-Schwarz ójafnan


\subsection{Cauchy-Schwarz ójafna}
\label{\detokenize{Kafli02:cauchy-schwarz-ojafna}}
Fyrir \(u,v\in L^2([-\pi,\pi])\) gildir
\begin{equation*}
\begin{split}|\langle u,v \rangle| \leq \frac{1}{2\pi} \int_{-\pi}^\pi |u(x)v(x)|dx \leq \| u\|\| v\|.\end{split}
\end{equation*}
Athugum nú að föllin \(e^{inx}\) og \(e^{imx}\) eru hornrétt ef \(n\neq m\) því þá gildir
\begin{equation*}
\begin{split}\langle e^{inx},e^{imx}\rangle = \frac{1}{2\pi}\int_{-\pi}^\pi e^{(n-m)ix} dx =  \left[\frac{e^{(n-m)ix}}{i(n-m)}\right]_{-\pi}^\pi = 0.\end{split}
\end{equation*}
Ef \(n=m\) gildir hins vegar að \(\langle e^{inx},e^{imx}\rangle = 1\).

Ef rita má \(2\pi\)-lotubundið fall \(f\) með röð á forminu
\begin{equation*}
\begin{split}f(x) = \sum_{n=-\infty}^\infty c_n e^{inx}\end{split}
\end{equation*}
og ef víxla má á heildi og óendanlegri summu í eftirfarandi reikningum þá fæst
\begin{equation*}
\begin{split}\frac{1}{2\pi}\int_{-\pi}^\pi f(x) e^{-imx} dx = \langle f,e^{imx} \rangle = \sum_{n=-\infty}^\infty c_n \langle e^{inx},e^{-imx} \rangle = c_m.\end{split}
\end{equation*}
Þar með eru stuðlarnir \(c_n\) ótvírætt ákvarðaðir og jafnir Fourier-stuðlum fallsins \(f\) og \(f\) er jafnt Fourier-röð sinni. Í framhaldinu munum við fjalla betur um þessa reikninga og undir hvaða skilyrðum þeir eru rættlætanlegir.


\subsection{Regla Pýþagórasar}
\label{\detokenize{Kafli02:regla-pyagorasar}}
Ef \(u, v\in L^2[-\pi,\pi]\) eru hornrétt, þá er
\begin{equation*}
\begin{split}\| u+v\|^2 = \|u\|^2 + \| v\|^2.\end{split}
\end{equation*}
Nokkuð einfalt er að sanna eftirfarandi ójöfnu.


\subsection{Bessel-ójafnan}
\label{\detokenize{Kafli02:bessel-ojafnan}}
Ef \(f\in L^2([-\pi,\pi])\) er \(2\pi\)\textendash{}lotubundið og hefur
Fourier-stuðla \(c_n=c_n(f)\), þá er
\begin{equation*}
\begin{split}\sum\limits_{n=-\infty}^{+\infty}|c_n|^2 \leq \dfrac
 1{2\pi}\int_{-\pi}^\pi |f(x)|^2\, dx.\end{split}
\end{equation*}
Losaralegir reikningar leyfa okkur að færa rök fyrir því að sterkari niðurstaða gildir, ójafnan er í raun jafna:

Ef rita má
\begin{equation*}
\begin{split}f(x) = \sum_{n=-\infty}^\infty c_n e^{inx}\end{split}
\end{equation*}
og að því gefnu að víxla megi á óendanlegum summum og heildum í eftirfarandi reikningum þá er
\begin{equation*}
\begin{split}\begin {align*}
 \dfrac
1{2\pi}\int_{-\pi}^\pi |f(x)|^2\, dx &= \langle f, f\rangle = \langle \sum_{n=-\infty}^\infty c_n e^{inx}, \sum_{m=-\infty}^\infty c_m e^{imx}\rangle \\
&= \sum_{n=-\infty}^\infty \sum_{m=-\infty}^\infty c_n \overline{c_m} \langle  e^{inx},  e^{imx}\rangle = \sum_{n=-\infty}^\infty \sum_{m=-\infty}^\infty c_n \overline{c_m} \delta_{nm} = \sum_{n=-\infty}^\infty |c_n|^2.
 \end{align*}\end{split}
\end{equation*}
Táknið \(\delta_{nm}\) sem kallast Kronecker-\(\delta\) og uppfyllir \(\delta_{mn} = 1\) ef \(m=n\) en \(\delta_{mn}=0\) annars. Það er talsvert flóknara að réttlæta þessa niðurstöðu með fullnægjandi hætti en það er hægt og við ræðum niðurstöðuna aftur þegar við fjöllum um Parseval-jöfnuna.


\subsection{Skilgreining}
\label{\detokenize{Kafli02:id5}}
Fall \(f\) á \(\mathbb{R}\) er sagt vera samfellt deildanlegt á köflum ef skipta má \(\mathbb{R}\) í endanlega mörg bil með skiptipunktum \(x_1,x_2,\ldots, x_k\) þannig að fallið er samfellt diffranlegt á opnu bilunum \(]x_j,x_{j+1}[\) og afleiðan hefur markgildi frá hægri í vinstri endapunkti bils og markgildi frá vinstri í hægri endapunkti bils. Mengi falla sem eru samfellt deildanleg á köflum er táknað með \(PC^1(\mathbb{R})\).

Við munum skoða föll sem eru \(2\pi\)-lotubundin og tilheyra menginu \(PC^1(\mathbb{R})\cap C(\mathbb{R})\), þ.e. eru samfellt diffranleg á köflum og samfelld. Dæmi um slíkt fall er \(2\pi\)-lotubundna fallið sem er skilgreint með formúlunni \(f(x) = x^2\) á \([-\pi,\pi]\).


\subsection{Setning}
\label{\detokenize{Kafli02:setning}}
Ef \(f\in PC^1({{\mathbb  R}})\cap C({{\mathbb  R}})\) er
\(2\pi\)\textendash{}lotubundið, þá er \(c_n(f{{^{\prime}}})=inc_n(f)\),
\begin{equation*}
\begin{split}\sum\limits_{n=-\infty}^{+\infty} |c_n(f)|< +\infty,\end{split}
\end{equation*}
og þar með er Fourier\textendash{}röðin
\(\sum_{-\infty}^{+\infty}c_n(f)e^{inx}\) samleitin í jöfnum mæli á
\({{\mathbb  R}}\).

Meginniðurstaða þessarar greinar er eftirfarandi setning sem sýnir undir hvaða skilyrðum og í hvaða skilningi fall er jafnt Fourier-röð sinni. Rifjum upp ritháttinn
\begin{equation*}
\begin{split}f(x+) = \lim_{y \to x^+} f(y) \quad \text{og} \quad f(x-) = \lim_{y \to x^-} f(y)\end{split}
\end{equation*}

\subsection{Setning - Andhverfuformúla Fouriers}
\label{\detokenize{Kafli02:setning-andhverfuformula-fouriers}}
Ef \(f\in PC^1({{\mathbb  R}})\) er \(2\pi\)\textendash{}lotubundið fall með
Fourier\textendash{}stuðla \(c_n=c_n(f)\), Fourier-kósínus\textendash{}stuðla
\(a_n=a_n(f)\) og Fourier\textendash{}sínus\textendash{}stuðla \(b_n=b_n(f)\), þá gildir
\begin{equation*}
\begin{split}\begin{aligned}
 \tfrac 12\big(f(x+)+f(x-)\big) &=
 \sum\limits_{n=-\infty}^{+\infty} c_ne^{inx} =
 \lim\limits_{N\to+\infty}\sum\limits_{n=-N}^{N} c_ne^{inx}\\
 &=\tfrac 12 a_0 + \sum\limits_{n=1}^\infty \big(a_n \cos nx + b_n\sin
 nx\big).\end{aligned}\end{split}
\end{equation*}
Í punktum \(x\) þar sem \(f\) er samfellt gildir
\(f(x)=\tfrac 12\big(f(x+)+f(x-)\big)\) og þar með er
\begin{equation*}
\begin{split}f(x)=
 \sum\limits_{n=-\infty}^{+\infty} c_ne^{inx}
 =\tfrac 12 a_0 + \sum\limits_{n=1}^\infty \big(a_n \cos nx + b_n\sin
 nx\big).\end{split}
\end{equation*}
Ef \(f\in PC^1({{\mathbb  R}})\cap C({{\mathbb  R}})\), þá eru
raðirnar samleitnar í jöfnum mæli á \({{\mathbb  R}}\).

Þegar \(2\pi\)-lotubundið fall \(f\in L^2([-\pi,\pi])\) er ósamfellt gildir almennt ekki að það sé jafnt Fourier-röð sinni í ósamfelldnipunktunum. Við getum samt spurt okkur hvort hægt sé að tala um að fallið sé jafnt Fourier-röð sinni í einhverjum öðrum skilningi. Eftirfarandi setning segir okkur að hlutsumman
\begin{equation*}
\begin{split}s_N = \sum_{n=-N}^{N}
c_n(f) e^{in x}\end{split}
\end{equation*}
stefnir á fallið \(f\) í staðlinum \(\|\cdot\|\) á \(L^2([-\pi,\pi])\).


\subsection{Setning - Parseval-jafnan}
\label{\detokenize{Kafli02:setning-parseval-jafnan}}
Ef \(f\in L^2[-\pi,\pi]\) er \(2\pi\)\textendash{}lotubundið, þá gildir
\begin{equation*}
\begin{split}\|f-s_N\|^2=\dfrac 1{2\pi}\int_{-\pi}^{\pi} |f(x)-\sum_{n=-N}^{N}
 c_n(f) e^{in x}|^2\, dx \to 0, \qquad N\to +\infty\end{split}
\end{equation*}
og af þessu leiðir jafna Parseval
\begin{equation*}
\begin{split}\sum_{n=-\infty}^{+\infty} |c_n(f)|^2 = \dfrac 1{2\pi}\int_{-\pi}^{\pi}
 |f(x)|^2 \, dx,\end{split}
\end{equation*}
\begin{sphinxadmonition}{attention}{Athugið:}
Mismunurinn \(\|f-\sum_{n=-N}^{N}   c^\ast_n e^{in x}\|^2\) nefnist ferskekkja nálgunar \(f\) með \(\sum_{n=-N}^{N} c^\ast_n e^{in x}\). Hægt er að sýna að lágmarks ferskekkja fæst með því að velja stuðlana \(c^\ast_n = c_n(f)\).
\end{sphinxadmonition}


\section{Úrlausn á hlutafleiðujöfnum}
\label{\detokenize{Kafli02:urlausn-a-hlutafleiujofnum}}
Í þessari grein munum við líta á dæmi þar sem hagnýta má Fourier-raðir við lausn jaðargildisverkefna. Byrjum á tveimur mikilvægum skilgreiningum.

Þegar fengist er við ákveðnar tegundir jaðargildisverkefna getur verið gagnlegt að skilgreina lotubundna framlengingu af falli á bili sem annað hvort er oddstæð eða jafnstæð. Með þeim hætti má skilgreina raðir sem uppfylla sjálfkrafa jaðarskilyrðin sem gefin eru.


\subsection{Jafnstæð framlenging og kósínus-röð}
\label{\detokenize{Kafli02:jafnstae-framlenging-og-kosinus-ro}}
Ef \(L>0\) og \(f: [0,L]\to \mathbb{C}\) er fall á endanlegu bili skilgreinum við jafnstæða \(2L\)-lotubundna framlengingu á \(f\) með því að setja
\begin{equation*}
\begin{split}f_J(x)=\begin{cases} f(x), & x\in [0,L],\\  f(-x), & x\in
  [-L,0],\end{cases}\end{split}
\end{equation*}
og framlengja \(f_J\) í \(2L\)-lotubundið fall.

\noindent{\hspace*{\fill}\sphinxincludegraphics[width=1.000\linewidth]{{jafnstaett}.png}\hspace*{\fill}}

\sphinxstyleemphasis{Jafnstæð framlenging falls} \(f:[0,L]\to \mathbb{C}\) \sphinxstyleemphasis{í} \(2L\) \sphinxstyleemphasis{-lotubundið fall} \(f_J\).

Fourier-stuðlar \(f_J\) eru gefnir með
\begin{equation*}
\begin{split}\begin{aligned}
 a_n(f_J)&=\dfrac 1L \int_{-L}^L f_J(x)\cos \dfrac {n\pi}L
 x \, dx\\
 &=\dfrac 2L \int_{0}^L f_J(x)\cos \dfrac {n\pi}L
 x \, dx\\
 &=\dfrac 2L \int_{0}^L f(x)\cos\dfrac {n\pi}L
 x \, dx, \qquad n=0,1,2,\dots,\\
 b_n(f_J)&=0 \qquad \qquad \qquad\qquad n=1,2,3,\dots.\end{aligned}\end{split}
\end{equation*}
Stuðlarnir \(a_n\) nefnast \sphinxstyleemphasis{Fourier\textendash{}kósínus\textendash{}stuðlar} fallsins \(f\) og röðin
\begin{equation*}
\begin{split}\tfrac 12 a_0 + \sum_{n=1}^\infty a_n \cos  \dfrac {n\pi} L x\end{split}
\end{equation*}
kallast \sphinxstyleemphasis{Fourier\textendash{}kósínus\textendash{}röð} fallsins \(f\).


\subsection{Oddstæð framlenging og sínus-röð}
\label{\detokenize{Kafli02:oddstae-framlenging-og-sinus-ro}}
Ef \(L>0\) og \(f: [0,L]\to \mathbb{C}\) er fall á endanlegu bili skilgreinum við oddstæða \(2L\)-lotubundna framlengingu á \(f\) með því að setja
\begin{equation*}
\begin{split}f_O(x)=\begin{cases} f(x), & x\in [0,L],\\  -f(-x), & x\in
  [-L,0],\end{cases}\end{split}
\end{equation*}
og framlengja \(f_O\) í \(2L\)-lotubundið fall.

\noindent{\hspace*{\fill}\sphinxincludegraphics[width=1.000\linewidth]{{oddstaett}.png}\hspace*{\fill}}

\sphinxstyleemphasis{Oddstæð framlenging falls} \(f:[0,L]\to \mathbb{C}\) \sphinxstyleemphasis{í} \(2L\) \sphinxstyleemphasis{-lotubundið fall} \(f_O\).

Fourier-stuðlar \(f_O\) eru gefnir með
\begin{equation*}
\begin{split}\begin{aligned}
 a_n(f_O)&=0 \qquad\qquad\qquad \qquad n=0,1,2,\dots,\\
 b_n(f_O)&=\dfrac 1L \int_{-L}^L f_O(x)\sin \dfrac {n\pi}L
 x  \, dx\\
 &=\dfrac 2L \int_{0}^L f_O(x)\sin \dfrac {n\pi}L
 x  \, dx\\
 &=\dfrac 2L \int_{0}^L f(x)\sin \dfrac {n\pi}L
 x  \, dx, \qquad n=1,2,\dots.\\\end{aligned}\end{split}
\end{equation*}
Stuðlarnir \(b_n\) nefnast \sphinxstyleemphasis{Fourier\textendash{}sínus\textendash{}stuðlar} fallsins \(f\) og röðin
\begin{equation*}
\begin{split}\sum_{n=1}^\infty b_n \sin  \dfrac {n\pi} L x\end{split}
\end{equation*}
kallast \sphinxstyleemphasis{Fourier\textendash{}sínus\textendash{}röð} fallsins \(f\).

\begin{sphinxadmonition}{attention}{Athugið:}
Hægt er að yfirfæra allar reiknireglur og fræðilegar niðurstöður líkt og t.d. andhverfusetninguna beint á Fourier-kósínus og Fourier-sínus raðir. Vísað er í kennslubók fyrir frekari smáatriði.
\end{sphinxadmonition}


\subsection{Setning}
\label{\detokenize{Kafli02:id6}}
Látum \(P\) vera margliðu af stigi \(m\) og lítum á jöfnuna
\begin{equation*}
\begin{split}P(D)u=(a_mD^m+a_{m-1}D^{m-1}+\cdots+a_1 D +a_0)u=f(x),\end{split}
\end{equation*}
þar sem \(f\in PC^1({{\mathbb  R}})\cap C({{\mathbb  R}})\) er
\(T\)\textendash{}lotubundið fall og setjum \(\omega=2\pi/T\). Ef
\(c_n(f)=0\) fyrir öll \(n\) þannig að \(P(in\omega)=0\), þá
fæst \(T\)\textendash{}lotubundin lausn af gerðinni
\begin{equation*}
\begin{split}u(x)=\sum_{\substack{n=-\infty\\ P(in\omega)\neq 0}}^{+\infty}
 \dfrac{c_n(f)} {P(in\omega)} e^{in\omega x}, \qquad x\in {{\mathbb  R}}.\end{split}
\end{equation*}
Eftirfarandi dæmi má finna í kennslubók og þar eru reikningar framkvæmdir í smáatriðum.


\subsection{Dæmi}
\label{\detokenize{Kafli02:daemi}}
Notum Fourier-raðir til að leysa jaðargildisverkefnið
\begin{equation*}
\begin{split}u{{^{\prime\prime}}}+{\omega}^2 u=f(x), \qquad u(0)=u(1)=0.\end{split}
\end{equation*}
Það hefur ótvírætt ákvarðaða lausn fyrir sérhvert \(f\) ef og aðeins ef \({\omega}\) er ekki
heiltölumargfeldi af \({\pi}\). Prófum að liða \(u\) í Fourier-sínus-röð en þá eru jaðarskilyrðin uppfyllt.

Lausnin er
\begin{equation*}
\begin{split}u(x)=\sum\limits_{n=1}^{\infty} \dfrac{f_n}{{\omega}^2-n^2{\pi}^2}
 \sin (n{\pi}x)\end{split}
\end{equation*}
þar sem \(f_n\) eru Fourier-sínus-stuðlar fallsins \(f\).


\subsection{Dæmi - Sveiflandi strengur}
\label{\detokenize{Kafli02:daemi-sveiflandi-strengur}}
Lítum á einvíðan streng af lengd \(L\) sem festur er í báða enda. Táknum frávik hans frá jafnvægi í punkti \(x\) á tíma \(t\) með \(u(x,t)\). Fallið \(u(x,t)\) uppfyllir þá bylgjujöfnuna í einni rúmbreytu ásamt jaðarskilyrðunum
\begin{equation*}
\begin{split}\dfrac{{\partial}^2u}{{\partial}t^2}-
 c^2\dfrac{{\partial}^2u}{{\partial}x^2}=0, \qquad u(0,t)=u(L,t)=0.\end{split}
\end{equation*}
Gerum einnig ráð fyrir því að upphafsstaðan og hraðinn séu þekkt
\begin{equation*}
\begin{split}u(x,0)=\varphi(x), \qquad {\partial}_tu(x,0)={\psi}(x), \qquad x\in
 ]0,L[.\end{split}
\end{equation*}
Þetta verkefni má leysa með því að liða \(u(x,t)\) í Fourier-sínus\textendash{}röð
með miðað við breytuna \(x\). Þannig eru jaðarskilyrðin sjálfkrafa uppfyllt.

Lausnin verður
\begin{equation*}
\begin{split}u(x,t)=\sum\limits_{n=1}^{\infty}
 \bigg(\varphi_n\cos\big(n{\pi}ct/L\big) +
 \dfrac{{\psi}_nL}{n{\pi}c} \sin\big(n{\pi}ct/L\big)\bigg)
 \sin(n{\pi}x/L)\end{split}
\end{equation*}
þar sem \(\phi_n\) og \(\psi_n\) eru Fourier-sínus-stuðlar fallanna \(\phi\) og \(\psi\).

Lausnina má einnig rita
\begin{equation*}
\begin{split}u(x,t)=\sum\limits_{n=1}^{\infty}
 C_n\cos\big(n{\pi}ct/L-{\alpha}_n\big)
 \sin(n{\pi}x/L)\end{split}
\end{equation*}
þar sem
\begin{equation*}
\begin{split}C_n=\sqrt{\varphi_n^2+({\psi}_nL/n{\pi}c)^2}\end{split}
\end{equation*}
kallast sveifluvídd og \({\alpha}_n\) kallast fasahliðrun og uppfyllir
\begin{equation*}
\begin{split}\cos{\alpha}_n= \varphi_n/C_n, \qquad
 \sin{\alpha}_n= ({\psi}_nL)/(n{\pi}cC_n).\end{split}
\end{equation*}

\subsection{Dæmi - Varmaleiðni}
\label{\detokenize{Kafli02:daemi-varmaleini}}
Reiknum út hitastig \(u(x,t)\), í punkti \(x\) á tíma \(t\),  í einvíðri stöng af lengd \(L\), sem er einangruð í báðum endapunktunum. Jaðarskilyrðin eru þá að ekkert varmaflæði er í endapunktum stangarinnar,  sem þýðir að afleiða hitastigsins er núll í jaðarpunktunum 0 og \(L\). Fallið \(u\) uppfyllir varmaleiðnijöfnuna og við höfum því eftirfarandi jaðargildisverkefni
\begin{equation*}
\begin{split}\begin{cases} \dfrac{{\partial} u}{{\partial}t}-{\kappa}
 \dfrac{{\partial}^2 u}{{\partial}x^2}=f(x,t), &0<x<L, \quad t>0,\\
 {\partial}_xu(0,t)={\partial}_xu(L,t)=0, &t>0
 \end{cases}\end{split}
\end{equation*}
með upphafsskilyrðinu
\begin{equation*}
\begin{split}u(x,0)=\varphi(x), \qquad x\in ]0,L[.\end{split}
\end{equation*}
Föllin \(f\) og \(\phi\) eru ótiltekin.

Fallið \(u\) er liðað í Fourier-kósínus-röð til þess að jaðarskilyrði séu uppfyllt. Þá má sýna að lausnin er
\begin{equation*}
\begin{split}u(x,t)=\sum_{n=0}^{\infty}
 \bigg(\varphi_ne^{-{\kappa}(n{\pi}/L)^2t}+
 \int_0^te^{-{\kappa}(n{\pi}/L)^2(t-{\tau})}f_n({\tau})\, d{\tau}\bigg)
 \cos(n{\pi}x/L)\end{split}
\end{equation*}
þar sem \(\phi_n\) og \(f_n\) eru Fourier-kósínus-stuðlar fallanna \(\phi\) og \(f\).


\chapter{Eigingildisverkefni}
\label{\detokenize{Kafli03:eigingildisverkefni}}\label{\detokenize{Kafli03::doc}}

\section{Eigingildisverkefni}
\label{\detokenize{Kafli03:id1}}
Verkefnið að leysa afleiðujöfnu
\begin{equation*}
\begin{split}Lu = \lambda u, \quad x\in I\end{split}
\end{equation*}
þar sem
\begin{equation*}
\begin{split}L = a_m(x) D^m + \cdots + a_1(x) D + a_0(x)\end{split}
\end{equation*}
\(\lambda\) er tvinntala og \(I\) er bil, ásamt jaðalskilyrðum á \(u\) á \(I\), kallast \sphinxstyleemphasis{eigingildisverkefni}. Verkefnið felst í að finna öll \(\lambda \in \mathbb{C}\) þannig að afleiðujafnan hafi lausn, segjum \(u_\lambda\), sem uppfyllir gefin jaðarskilyrði og er ekki núllfallið. Slík gildi á \(\lambda\) kallast eigingildi verkefnisins og tilsvarandi lausnir \(u_\lambda\) kallast eiginföll.

\begin{sphinxadmonition}{attention}{Athugið:}
Athugið að ef við lítum á föll sem vigra og línulega virkja sem línulegar varpanir þá er greinileg samsvörun milli eigingildisverkefna og að finna eigingildi og eiginvigra fylkis eins og við þekkjum úr línulegri algebru.
\end{sphinxadmonition}

Eigingildisverkefni koma til að mynda upp þegar hlutafleiðujöfnur eru leystar með aðskilnaði breytistærða eins og verður fjallað um síðar.

Hugmyndin að baki eigingildisverkefnum er sú að ef eiginföllinn mynda grunn í einhverju fallarúmi þá má rita sérhvert fall í því sem línulega (mögulega óendanlega) samantekt af eiginföllum. Lítum til dæmis á verkefnið
\begin{equation*}
\begin{split}Lu = f\end{split}
\end{equation*}
þar sem \(L\) hefur eiginföll \(v_j\) með tilssvarandi eigingildum \(\lambda_j\). Ef liða má \(f\) í grunn eiginvigranna
\begin{equation*}
\begin{split}f = \sum_{j} c_j v_j\end{split}
\end{equation*}
þá fæst með losaralegum reikningum að
\begin{equation*}
\begin{split}L \sum_{j} \frac{c_j}{\lambda_j} v_j =  \sum_{j} \frac{c_j}{\lambda_j} L v_j  = \sum_{j} \frac{c_j}{\lambda_j} \lambda_j v_j = f\end{split}
\end{equation*}
og þar með er \(u = \sum_{j} \frac{c_j}{\lambda_j} v_j\) lausn á verkefninu.


\subsection{Dæmi}
\label{\detokenize{Kafli03:daemi}}
Ef \(L = D\) þá er sérhvert fall \(u_\alpha(x) = e^{\alpha x}\) eiginfall með eigingildi \(\alpha\) því
\begin{equation*}
\begin{split}L u_\alpha(x) = D e^{\alpha x} = \alpha e^{\alpha x} = \alpha u_\alpha(x).\end{split}
\end{equation*}
Ef \(L = P(D)\) þar sem \(P\) er margliða með fasta stuðla þá er sérhvert fall \(u_\alpha(x)=e^{\alpha x}\) eiginfall með eigingildi \(P(\alpha)\) því
\begin{equation*}
\begin{split}L u_\alpha(x) = P(D) e^{\alpha x} = P(\alpha) e^{\alpha x} = P(\alpha) u_\alpha(x).\end{split}
\end{equation*}
Í síðara tilfellinu er formleg lausn á verkefninu
\begin{equation*}
\begin{split}Lu = f\end{split}
\end{equation*}
á forminu
\begin{equation*}
\begin{split}u(x) = \sum_{j} \frac{c_n(f)}{P(\alpha_j)}e^{\alpha_j x}\end{split}
\end{equation*}
ef \(P(\alpha_j) \neq 0\) fyrir öll \(j\), eins og við þekkjum úr umræðunni um Fourier-raðir.

\begin{DUlineblock}{0em}
\item[] 
\item[] 
\end{DUlineblock}

Eins og dæmið gefur til kynna má líta á Fourier-raðir sem sértilfelli af þeirri almennu hugmynd að liða föll í grunn eiginfalla afleiðuvirkja. Lítum nú nánar á það í næstu dæmum. Skoðum eigingildisverkefnið
\begin{equation*}
\begin{split}Tu = \lambda u, \quad x\in ]0,L]\end{split}
\end{equation*}
þar sem \(T = -D^2\) og \(L>0\), með ýmsum ólíkum jaðarskilyrðum. Tökum sérstaklega eftir því hvernig ólík jaðarskilyrði geta gefið ólík eigingildi og/eða ólík eiginföll.


\subsection{Fallsjaðarskilyrði í báðum endapunktum}
\label{\detokenize{Kafli03:fallsjaarskilyri-i-baum-endapunktum}}
Lítum á jaðarskilyrðin
\begin{equation*}
\begin{split}u(0) = u(L) = 0.\end{split}
\end{equation*}
Ef \(\lambda = 0\) er lausnin á forminu \(u(x) = A+ Bx\) en jaðarskilyrðin ákvarða \(A=B=0\) svo núllfallið er eina lausnin. Þar með er 0 ekki eigingildi.

Ef \(\lambda \neq 0\) þá er lausnin á forminu \(u(x) = A \sin(\beta x) + B\cos(\beta x)\) þar sem \(\beta\) er tvinntala sem uppfyllir \(\beta^2 = \lambda\) og má velja þannig að \(\operatorname{Re}(\beta)\geq 0\). Skilyrðið \(u(0) = 0\) gefur \(B=0\) en skilyrðið \(u(L) = 0\) gefur
\begin{equation*}
\begin{split}0 = \sin(\beta L)\end{split}
\end{equation*}
en þessi jafna hefur lausn þegar \(\beta L\) er heilt margfeldi af \(\pi\) svo eigingildin eru
\begin{equation*}
\begin{split}\lambda_n = \left(\frac{n\pi}{L}\right)^2, \quad n = 1,2,3,\ldots\end{split}
\end{equation*}
og tilsvarandi eiginföll
\begin{equation*}
\begin{split}u(x) = \sin(n\pi x/L).\end{split}
\end{equation*}
Línulegar samantektir eiginfallanna
\begin{equation*}
\begin{split}\sum_{n=1}^\infty C_n \sin(n\pi x/L)\end{split}
\end{equation*}
eru Fourier-sínus-raðir á bilinu \([0,L]\).


\subsection{Afleiðuskilyrði í báðum endapunktum}
\label{\detokenize{Kafli03:afleiuskilyri-i-baum-endapunktum}}
Lítum á jaðarskilyrðin
\begin{equation*}
\begin{split}u'(0) = u'(L) = 0.\end{split}
\end{equation*}
Með svipuðum hætti og áður fæst að eigingildin eru
\begin{equation*}
\begin{split}\lambda_n = \left(\frac{n\pi}{L}\right)^2, \quad n=0,1,2,\ldots\end{split}
\end{equation*}
(athugið að \(\lambda = 0\) er núna með) og tilsvarandi eiginföll
\begin{equation*}
\begin{split}u(x) = \cos(n\pi x/L).\end{split}
\end{equation*}
Línulegar samantektir eiginfallanna
\begin{equation*}
\begin{split}\sum_{n=0}^\infty C_n \cos(n\pi x/L)\end{split}
\end{equation*}
eru Fourier-kósínus-raðir á bilinu \([0,L]\).


\subsection{Fallsjaðarskilyrði í öðrum endapunkti og afleiðuskilyrði í hinum}
\label{\detokenize{Kafli03:fallsjaarskilyri-i-orum-endapunkti-og-afleiuskilyri-i-hinum}}
Lítum á jaðarskilyrðin
\begin{equation*}
\begin{split}u(0) = u'(L) = 0.\end{split}
\end{equation*}
Með svipuðum hætti og áður fæst að eigingildin eru
\begin{equation*}
\begin{split}\lambda_n = \left(\frac{(n-1/2)\pi}{L}\right)^2, \quad n=1,2,\ldots\end{split}
\end{equation*}
og tilsvarandi eiginföll
\begin{equation*}
\begin{split}u(x) = \sin((n-1/2)\pi x/L).\end{split}
\end{equation*}
Í kennslubók má lesa eitt viðamikið sýnidæmi til viðbótar þar sem blönduð jaðarskilyrði eru í báðum endapunktum bilsins.


\section{Aðskilnaður breytistærða}
\label{\detokenize{Kafli03:askilnaur-breytistaera}}
Við lausn línulegra óhliðraðra hlutafleiðujafna þar sem breyturnar \(x_1,x_2,\ldots,x_k\) koma við sögu getur verið gagnlegt að leita að lausnum sem eru á forminu \(X_1(x_1)X_2(x_2)\cdots X_k(x_k)\), þ.e.a.s. lausnir sem má þátta í föll sem hvert um sig er háð aðeins einni breytistærð. Línuleg samantekt lausna á slíku formi er einnig lausn en ekki endilega þáttanleg eins og gildir um sérhvern lið samantektarinnar. Í sumum tilfellum mynda lausnir á þessu formi grunn þannig að liða má föll upp í línulegar samantektir af grunnföllunum. Þannig má finna almenna lausn á hlutafleiðujöfnunni.

Þessi aðferð, að skoða lausnir sem þáttast, kallast \sphinxstyleemphasis{aðskilnaður breytistærða} og þegar henni er beitt fást venjulega eigingildisverkefni fyrir hvert fall í þáttuninni. Lítum á kunnuglegt dæmi.


\subsection{Sveiflandi strengur - aftur}
\label{\detokenize{Kafli03:sveiflandi-strengur-aftur}}
Lítum á einvíðan streng af lengd \(L\) sem festur er í báða enda. Táknum frávik hans frá jafnvægi í punkti \(x\) á tíma \(t\) með \(u(x,t)\). Fallið \(u(x,t)\) uppfyllir þá bylgjujöfnuna í einni rúmbreytu ásamt jaðarskilyrðunum
\begin{equation*}
\begin{split}\dfrac{{\partial}^2u}{{\partial}t^2}-
 c^2\dfrac{{\partial}^2u}{{\partial}x^2}=0, \qquad u(0,t)=u(L,t)=0.\end{split}
\end{equation*}
Leysum verkefnið með aðskilnaði breytistærða. Leitum að lausn á forminu \(u(x,t) = T(t)X(x)\). Stingum slíkri lausn inn í afleiðujöfnuna og fáum
\begin{equation*}
\begin{split}T''(t) X(x) - c^2 T(t)X''(x) = 0\end{split}
\end{equation*}
sem má umrita í
\begin{equation*}
\begin{split}\frac{T''(t)}{c^2 T(t)} = \frac{X''(x)}{X(x)}.\end{split}
\end{equation*}
Vinstri hliðin er aðeins háð \(t\) og sú hægri aðeins háð \(x\) og þar með hlýtur hvor um sig að vera jöfn fasta, köllum hann \(\lambda\). Fáum því afleiðujöfnu
\begin{equation*}
\begin{split}-T''(t) = c^2 \lambda T\end{split}
\end{equation*}
og eigingildisverkefni
\begin{equation*}
\begin{split}-X''(x) = \lambda X \quad X(0) = X(L) = 0.\end{split}
\end{equation*}
Eigingildisverkefnið hefur eigingildi \(\lambda_n = (n\pi/L)^2\) og tilsvarandi eiginföll \(\sin(n\pi x/L)\), \(n=1,2,3,\ldots\). Afleiðujafnan fyrir \(T\) hefur því lausn, fyrir \(\lambda = \lambda_n\), á forminu
\begin{equation*}
\begin{split}A_n \cos(n\pi ct/L) + B_n \sin(n\pi c t/L).\end{split}
\end{equation*}
Lausnin á hlutafleiðujöfnunni er því á forminu
\begin{equation*}
\begin{split}T(t)X(x) = (A_n \cos(n\pi ct/L) + B_n \sin(n\pi c t/L))\sin(n\pi x/L).\end{split}
\end{equation*}
Almenn lausn hlutafleiðujöfnunnar er línulega samantekt af svona liðum
\begin{equation*}
\begin{split}u(x,t) = \sum_{n\geq 1} (A_n\cos(n\pi ct/L) + B_n \sin(n\pi c t/L))\sin(n\pi x/L)\end{split}
\end{equation*}
og stuðlarnir \(A_n\) og \(B_n\) ákvarðast af upphafsskilyrðum
\begin{equation*}
\begin{split}u(x,0)=\varphi(x), \qquad {\partial}_tu(x,0)={\psi}(x), \qquad x\in
 ]0,L[.\end{split}
\end{equation*}

\subsection{Annað dæmi}
\label{\detokenize{Kafli03:anna-daemi}}
Notum aðskilnað breytistærða til að leysa
\begin{equation*}
\begin{split}a\partial_t^2u+b\partial_tu+cu-\Delta u=0, \quad x,y,z \in [0,1], t>0\end{split}
\end{equation*}
þar sem \(u\) er fall af tíma \(t\) og þremur
rúmbreytum \(x,y\) og \(z\) og \(\Delta=\partial_x^2+\partial_y^2+\partial_z^ 2\) er
Laplace\textendash{}virkinn miðað við rúmbreyturnar.  Gerum ráð fyrir jaðarskilyrðunum \(u(x,y,z,t) = 0\) ef eitthvert hnitanna \(x,y,x\) er jafnt 0 eða 1.

Leitum að lausn á forminu
\(u(x,y,z,t)=T(t)X(x)Y(y)Z(z)\). Stingum henni inn og umritum á formið
\begin{equation*}
\begin{split}\dfrac{aT{{^{\prime\prime}}}(t)+bT(t){{^{\prime}}}+cT(t)}{T(t)}-
 \dfrac{X{{^{\prime\prime}}}(x)}{X(x)}-\dfrac{Y{{^{\prime\prime}}}(y)}{Y(y)}=\dfrac{Z{{^{\prime\prime}}}(z)}{Z(z)}.\end{split}
\end{equation*}
Hægri hlið er háð \(z\) en sú vinstri ekki. Ályktum að hægri hlið sé fasti og með sömu rökum að sérhver liður í jöfnunni sé fasti. Vegna jaðarskilyrða fáum við því þrjú eigingildisverkefni
\begin{equation*}
\begin{split}\begin {align*}
-X''(x) &= \lambda X(x),\quad X(0) = X(1) = 0 \\
-Y''(y) &= \lambda Y(z),\quad Y(0) = Y(1) = 0 \\
-Z''(z) &= \lambda Z(z),\quad Z(0) = Z(1) = 0 \\
\end{align*}\end{split}
\end{equation*}
og afleiðujöfnu
\begin{equation*}
\begin{split}aT{{^{\prime\prime}}}(t)+bT{{^{\prime}}}(t)+(c+\lambda+\mu+\nu)T=0,\end{split}
\end{equation*}
þar sem \(\lambda, \mu\) og \(\nu\) eru fastar. Við þekkjum lausnir eigingildisverkefnana og þáttanlega lausnin er á forminu
\begin{equation*}
\begin{split}u_{\ell,m,n}(x) = T_{\ell, m, n}(t) \sin (\ell \pi x) \sin (m\pi y) \sin
(n\pi z), \qquad \ell, m, n=1,2,3,\dots,\end{split}
\end{equation*}
þar sem \(T_{\ell, m,n}\) uppfyllir afleiðujöfnuna
\begin{equation*}
\begin{split}aT{{^{\prime\prime}}}+ bT{{^{\prime}}}+\big(c+\pi^2(\ell^2+m^2+n^2)\big)T=0.\end{split}
\end{equation*}

\section{Virkjar af Sturm-Liouville-gerð}
\label{\detokenize{Kafli03:virkjar-af-sturm-liouville-ger}}
Í þessari grein munum við skoða eigingildisverkefni virkja af tiltekinni gerð. Við byrjum á því að ræða virkjann og fjöllum því næst um jaðarskilyrðin sem skilgreina eigingildisverkefnið.

Við lítum á annars stigs afleiðuvirkja af eftirfarandi gerð
\begin{equation*}
\begin{split}Lu=P(x,D)u= a_2(x) u{{^{\prime\prime}}}+a_1(x)u{{^{\prime}}}+ a_0(x)u,\end{split}
\end{equation*}
þar sem \(a_0,a_1,a_2\) eru samfelld raungild föll á bili \([a,b]\) og \(a_2(x)\neq 0\) fyrir öll
\(x\in [a,b]\). Í útreikningum hentar betur að setja virkjann fram á svokölluðu \sphinxstyleemphasis{Sturm-Liouville formi}
\begin{equation*}
\begin{split}Lu ={{\dfrac {1}{\varrho}
 \bigg(-\dfrac d{dx}\bigg(p\dfrac {du}{dx}\bigg)+qu\bigg)}}.\end{split}
\end{equation*}
\begin{sphinxadmonition}{attention}{Athugið:}
Sambandið milli framsetninganna tveggja er eftirfarandi. Veljum
\begin{equation*}
\begin{split}p(x)=\exp\bigg(C +\int_a^x\dfrac{a_1({\xi})}{a_2({\xi})}\, d{\xi}\bigg),
\quad
q(x)=\dfrac{-a_0(x)p(x)}{a_2(x)}, \quad
{\varrho}(x)=\dfrac{-p(x)}{a_2(x)},\end{split}
\end{equation*}
þar sem \(C\) er einhver ótiltekinn fasti.
\end{sphinxadmonition}

Þar sem \(a_2(x)\neq 0\) fyrir öll \(x\in [a,b]\), má gera ráð fyrir
að \(a_2(x)<0\). Þar með gildir
\begin{equation*}
\begin{split}p\in C^1[a,b], \quad p(x)>0, \quad q,{\varrho}\in C[a,b], \quad q(x)\in {{\mathbb  R}},
 \quad {\varrho}(x)>0, \quad x\in [a,b].\end{split}
\end{equation*}

\subsection{Skilgreining}
\label{\detokenize{Kafli03:skilgreining}}
Við segjum að virki \(L\) af Sturm\textendash{}Liouville\textendash{}gerð sé \sphinxstyleemphasis{reglulegur} ef
föllin \(p\), \(q\) og \({\varrho}\) uppfylla þessi skilyrði.


\subsection{Skilgreining}
\label{\detokenize{Kafli03:id2}}
Á rúmið \(C[a,b]\) skilgreinum við formið
\begin{equation*}
\begin{split}{{\langle u,v\rangle}} =\int_a^b u(x)\overline{v(x)}{\varrho}(x)\, dx, \qquad
 u,v\in C[a,b],\end{split}
\end{equation*}
og á rúmið \(C^1[a,b]\) skilgreinum við formið
\begin{equation*}
\begin{split}{{\langle u,v\rangle}}_L =\int_a^b \bigg(p(x)u{{^{\prime}}}(x)\overline{v{{^{\prime}}}(x)}
 +q(x)u(x)\overline{v(x)}\bigg) \, dx, \qquad
 u,v\in C^1[a,b].\end{split}
\end{equation*}
Bæði eru þessi form línuleg í fyrri breytistærðinni, en andlínuleg í
þeirri síðari. Það þýðir að
\begin{equation*}
\begin{split}\begin{aligned}
  {{\langle \alpha u+\beta v,w\rangle}} &=
 \alpha{{\langle u,v\rangle}} + \beta{{\langle u,w\rangle}},\\
 {{\langle u,\alpha v+\beta w\rangle}}&=\bar\alpha{{\langle u,v\rangle}} +\bar\beta
 {{\langle u,w\rangle}},\end{aligned}\end{split}
\end{equation*}
fyrir öll \(u,v\in C[a,b]\), \(\alpha,\beta\in {{\mathbb  C}}\). Fyrst \({\varrho}>0\), þá er
formið \({{\langle \cdot,\cdot\rangle}}\) innfeldi
og tilheyrandi staðal táknum við með,
\begin{equation*}
\begin{split}\|u\|= \sqrt{{{\langle u,u\rangle}}}.\end{split}
\end{equation*}
\begin{sphinxadmonition}{attention}{Athugið:}
Við segjum að formið sé innfeldi á vigurrúmi samfelldra falla á \([a,b]\), \(C[a,b]\), því það uppfyllir þær reiknireglur sem hið kunnuglega innfeldi endanlegra vigra uppfyllir. Við getum því unnið með það með sama hætti og gamla góða innfeldið.  Athugið einnig að þegar \(rho(x) = \frac{1}{b-a}\) fæst sama innfeldi og sami staðall og við skilgreindum á \(L^2\). Almennt jákvætt fall \(rho\) sem kemur fyrir í innfeldi af þessu tagi er oft kallað \sphinxstyleemphasis{vigt}.
\end{sphinxadmonition}

Við munum nú skoða hvernig setja má fram jaðarskilyrði af tiltekinni gerð og athugum svo eigingildisverkefnin sem þau skilgreina ásamt virkjanum \(L\) sem unnið er með. \sphinxstyleemphasis{Jaðargildisvirki} \(B\) er vörpun sem úthlutar samfellt deildanlegu falli \(u\in C^1[a,b]\) punkti \(Bu = (B_1 u , B_2)\) þar sem
\begin{equation*}
\begin{split}\begin{align*}
B_1 u&=\alpha_{11}u(a)+\alpha_{12}u{{^{\prime}}}(a)
+\beta_{11}u(b)+\beta_{12}u{{^{\prime}}}(b) \\
B_1 u&=\alpha_{11}u(a)+\alpha_{12}u{{^{\prime}}}(a)
+\beta_{11}u(b)+\beta_{12}u{{^{\prime}}}(b) \\
\end{align*}\end{split}
\end{equation*}
þar sem stuðlarnir \(\alpha_{jk}\) og \(\beta_{jk}\) eru
rauntölur. Við gerum ráð fyrir í hvorum virkjanna \(B_1\) og \(B_2\) sé að minnsta kosti einn stuðull frábrugðinn núlli.


\subsection{Skilgreining}
\label{\detokenize{Kafli03:id3}}
Rúmið \(C^2_B[a,b]\) er skilgreint sem mengi allra
\(u\in C^2[a,b]\) sem uppfylla óhliðruðu jaðarskilyrðin
\(Bu=0\).


\subsection{Skilgreining}
\label{\detokenize{Kafli03:id4}}
Við segjum að virkinn \(L\) sé \sphinxstyleemphasis{samhverfur} á \(C^2_B[a,b]\) eða
\sphinxstyleemphasis{samhverfur með tilliti til jaðarskilyrðanna} \(Bu=0\) ef
\begin{equation*}
\begin{split}{{\langle Lu,v\rangle}} ={{\langle u,Lv\rangle}}, \qquad u,v\in C^2_B[a,b].\end{split}
\end{equation*}

\subsection{Formúla Greens}
\label{\detokenize{Kafli03:formula-greens}}
Eftirfarandi formúla er kennd við Green
\begin{equation*}
\begin{split}\begin{aligned}
 {{\langle Lu,v\rangle}} -{{\langle u,Lv\rangle}}    &=p(b)\left|
 \begin{matrix} u(b) & u{{^{\prime}}}(b) \\ \bar v(b) &\bar v{{^{\prime}}}(b)
 \end{matrix}\right| -
 p(a)\left|
 \begin{matrix} u(a) & u{{^{\prime}}}(a) \\ \bar v(a) &\bar v{{^{\prime}}}(a)
 \end{matrix}\right|.\nonumber\end{aligned}\end{split}
\end{equation*}
Af formúlu Greens sést að virki er samhverfur þá og því aðeins
\begin{equation*}
\begin{split}p(b)\left|
\begin{matrix} u(b) & u{{^{\prime}}}(b) \\ \bar v(b) &\bar v{{^{\prime}}}(b)
\end{matrix}\right| =
p(a)\left|
\begin{matrix} u(a) & u{{^{\prime}}}(a) \\ \bar v(a) &\bar v{{^{\prime}}}(a)
\end{matrix}\right|\end{split}
\end{equation*}
fyrir öll \(u,v\in C^2_B[a,b]\).

Við höfum einkum áhuga á eftirfarandi tveimur tilfellum sem hægt er að sannfæra sig um að eru samhverf með því að nota skilyrðið úr formúlu Greens.


\subsection{Setning og skilgreining}
\label{\detokenize{Kafli03:setning-og-skilgreining}}
(i) Ef jaðarskilyrðin eru \sphinxstyleemphasis{aðskilin}, þ.e.a.s.
\begin{equation*}
\begin{split}B_1u=\alpha_1u(a)-\beta_1u{{^{\prime}}}(a), \qquad
 B_2u=\alpha_2u(b)+\beta_2u{{^{\prime}}}(b),\end{split}
\end{equation*}
þar sem \(\alpha_1, \beta_1, \alpha_2, \beta_2\in {{\mathbb  R}}\),
\((\alpha_1,\beta_1)\neq (0,0)\) og \((\alpha_2,\beta_2)\neq (0,0)\), þá er \(L\) samhverfur á \(C^2_B[a,b]\).

(ii) Ef \(p(a)=p(b)\) og jaðarskilyrðin eru \sphinxstyleemphasis{lotubundin}, þ.e.a.s.
\begin{equation*}
\begin{split}B_1u=u(a)-u(b), \qquad B_2u=u{{^{\prime}}}(a)-u{{^{\prime}}}(b),\end{split}
\end{equation*}
þá er \(L\) samhverfur á \(C^2_B[a,b]\).

Nú erum við reiðubúin að fjalla um eigingildisverkefni sem svara til virkja af Sturm-Liouville gerð með jaðarskilyrðum af þessu tagi.


\section{Eigingildisverkefni af Sturm\textendash{}Liouville\textendash{}gerð}
\label{\detokenize{Kafli03:eigingildisverkefni-af-sturmliouvilleger}}
Lítum á eigingildisverkefnið
\begin{equation*}
\begin{split}Lu= {\lambda} u , \qquad Bu=0,\end{split}
\end{equation*}
þar sem \(L\) er virki af Sturm\textendash{}Liouville\textendash{}gerð og
\(B\) er almennur jaðargildisvirki.

Línulega rúmið sem spannað er af öllum eiginföllum með tilliti til eigingildisins \({\lambda}\)
köllum við \textit{eiginrúmið} með tilliti til eigingildisins
\({\lambda}\) og við táknum það með \(E_{\lambda}\).


\subsection{Skilgreining}
\label{\detokenize{Kafli03:id5}}
Ef \(L\) er reglulegur virki af Sturm\textendash{}Liouville\textendash{}gerð, þá segjum við
að eigingildisverkefnið sé \sphinxstyleemphasis{reglulegt}.


\subsection{Setning}
\label{\detokenize{Kafli03:setning}}
Gerum ráð fyrir að virkinn \(L\) af Sturm\textendash{}Liouville\textendash{}gerð sé
samhverfur á \(C^2_B[a,b]\). Þá eru öll eigingildin rauntölur og
eiginföllin sem svara til ólíkra eigingilda eru innbyrðis hornrétt. Að auki má velja grunn í eiginrúminu \(E_\lambda\) sem samanstendur af raungildum föllum.


\subsection{Setning}
\label{\detokenize{Kafli03:id6}}
Öll eigingildin eru \(\geq 0\) í tilfellunum:

(i) \(q(x)\geq 0\) fyrir öll \(x\in [a,b]\), jaðarskilyrðin eru
aðskilin, \(B_1u=\alpha_1u(a)-\beta_1u{{^{\prime}}}(a)=0\),
\(B_2u=\alpha_2u(b)+\beta_2u{{^{\prime}}}(b)=0\),
\(\alpha_1\geq 0\), \(\beta_1\geq 0\), \(\alpha_2\geq 0\) og
\(\beta_2\geq 0\).

(ii) \(q(x)\geq 0\) fyrir öll \(x\in [a,b]\), \(p(a)=p(b)\)
og jaðarskilyrðin eru lotubundin, \(B_1u=u(a)-u(b)=0\) og
\(B_2u=u{{^{\prime}}}(a)-u{{^{\prime}}}(b)=0\).


\bigskip\hrule\bigskip


Eftirfarandi setning er meginniðurstaða þessarar umfjöllunar. Hún alhæfir það sem við höfum áður fjallað um með Fourier-röðum.


\subsection{Setning}
\label{\detokenize{Kafli03:id7}}
Gerum ráð fyrir að
\begin{equation*}
\begin{split}Lu={\lambda} u, \qquad Bu=0,\end{split}
\end{equation*}
sé reglulegt Sturm\textendash{}Liouville\textendash{}eigingildisverkefni og að \(L\) sé
samhverfur með tilliti til jaðarskilyrðanna \(Bu=0\). Þá er til
óendanleg runa \({\lambda}_0<{\lambda}_1<{\lambda}_2\cdots \to +{\infty}\) af eigingildum og tilsvarandi raungildum eiginföllum
\(u_0,u_1,u_2,\dots\), sem uppfylla
\begin{equation*}
\begin{split}{{\langle u_j,u_k\rangle}}=\begin{cases} 1, &j=k,\\0, &j\neq k,\end{cases}\end{split}
\end{equation*}
og sérhvert fall \(u\in C^2_B[a,b]\) er unnt að liða í eiginfallaröð
\begin{equation*}
\begin{split}u(x)=\sum\limits_{n=0}^{\infty} c_n(u)u_n(x), \qquad x\in [a,b],\end{split}
\end{equation*}
sem er samleitin í jöfnum mæli á \([a,b]\) og stuðlarnir eru gefnir
með formúlunni
\begin{equation*}
\begin{split}c_n(u)={{\langle u,u_n\rangle}}= \int_a^bu(x)u_n(x){\varrho}(x)\, dx.\end{split}
\end{equation*}

\subsection{Skilgreining}
\label{\detokenize{Kafli03:id8}}
Fyrir sérhvert heildanlegt fall \(f\) á \([a,b]\), þá
skilgreinum við \sphinxstyleemphasis{Fourier\textendash{}stuðul fallsins} \(f\)  \sphinxstyleemphasis{með tilliti til
eiginfallsins} \(u_n\) með
\begin{equation*}
\begin{split}c_n(f)= {{\langle f,u_n\rangle}} =\int_a^b f(x) u_n(x){\varrho}(x)\, dx\end{split}
\end{equation*}
og \sphinxstyleemphasis{eiginfallaröðina af} \(f\)  \sphinxstyleemphasis{með tilliti til eiginfallanna}
\((u_n)_{n=0}^{\infty}\) með
\begin{equation*}
\begin{split}\sum\limits_{n=0}^{\infty} c_n(f)u_n(x).\end{split}
\end{equation*}
\begin{sphinxadmonition}{attention}{Athugið:}
Við höfum einnig andhverfuformúlu fyrir eiginfallaraðir af föllum sem eru samfellt deildanleg á köflum sem er samhljóða andhverfuformúlu Fouriers.
\end{sphinxadmonition}


\section{Green-föll fyrir jaðargildisverkefni}
\label{\detokenize{Kafli03:green-foll-fyrir-jaargildisverkefni}}
Lítum á línulegan jaðargildisvirkja  \(B\)
á \([a,b]\) á forminu
\begin{equation*}
\begin{split}\begin{cases}
 B:C^{m-1}[a,b]\to {{\mathbb  C}}^m, \qquad Bu=(B_1u,\dots,B_mu),\\
 B_ju=\sum\limits_{l=1}^m \alpha_{jl}u^{(l-1)}(a)+
 \beta_{jl}u^{(l-1)}(b).
 \end{cases}\end{split}
\end{equation*}
Gerum ráð fyrir því að fyrir sérhvert \(j\) sé að minnsta kosti
ein talnanna \(\alpha_{jl}\), \(\beta_{jl}\),
\(l=1,\dots,m\) frábrugðin \(0\). Skilgreinum \(C^m_B[a,b]\)
sem rúm allra \(u\in C^m[a,b]\) sem uppfylla óhliðruðu
jaðarskilyrðin \(Bu=0\).


\subsection{Setning}
\label{\detokenize{Kafli03:id9}}
Látum \(P(x,D)=a_m(x)D^m+\cdots+a_1(x)D+a_0(x)\) vera afleiðuvirkja
á \([a,b]\) með samfellda stuðla, gerum ráð fyrir að
\(a_m(x)\neq 0\) fyrir öll \(x\in [a,b]\), látum
\(B:C^{m-1}[a,b]\to {{\mathbb  C}}^m\) vera jaðargildisvirkja og
gerum ráð fyrir að \({\lambda}=0\) sé ekki eigingildi \(P(x,D)\)
á \(C^m_B[a,b]\). Þá hefur jaðargildisverkefnið
\begin{equation*}
\begin{split}P(x,D)u=f(x), \qquad Bu=0,\end{split}
\end{equation*}
ótvírætt ákvarðaða lausn sem uppfyllir
\begin{equation*}
\begin{split}u(x) = \int_a^b G_B(x,{\xi})f({\xi})\, d{\xi},\end{split}
\end{equation*}
þar sem fallið \(G_B\) hefur eftirtalda eiginleika:

(i) \({\partial}_x^{k}G_B(x,{\xi})\) er samfellt á
\([a,b]\times [a,b]\) fyrir \(k=0,\dots,m-2\).

(ii)\({\partial}_x^{m-1}G_B(x,{\xi})\) er samfellt í öllum punktum
á \([a,b]\times [a,b]\) fyrir utan línuna \(x={\xi}\) og tekur
stökkið \(1/a_m({\xi})\) yfir hana.

(iii) \(P(x,D_x)G_B(x,{\xi})=0\) ef \(x\neq {\xi}\).

(iv) \(BG_B(\cdot,{\xi})=0\) ef \({\xi}\in ]a,b[\),
þ.e. \(G_B\) uppfyllir óhliðruð jaðarskilyrði, sem fall af fyrri
breytistærðinni.

Skilyrðin (i)-(iv) ákvarða fallið \(G_B\) ótvírætt.


\subsection{Setning}
\label{\detokenize{Kafli03:id10}}
Látum \(P(x,D)=a_2(x)D^2+a_1(x)D+a_0(x)\) vera annars stigs
afleiðuvirkja, þar sem \(a_2(x)\neq 0\) fyrir öll
\(x\in [a,b]\), og gerum ráð fyrir að jaðarskilyrðin séu aðskilin, þ.e.a.s.
\begin{equation*}
\begin{split}B_1u=\alpha_1u(a)-\beta_1u{{^{\prime}}}(a), \quad
 B_2u=\alpha_2u(b)+\beta_2u{{^{\prime}}}(b),\end{split}
\end{equation*}
og \((\alpha_1,\beta_1)\neq(0,0)\), \((\alpha_2,\beta_2)\neq (0,0)\). Gerum ráð fyrir að \(u_1\) og \(u_2\) myndi grunn í
núllrúmi virkjans og
\begin{equation*}
\begin{split}B_1u_1=0, \qquad B_2u_2=0.\end{split}
\end{equation*}
Þá er Green-fallið fyrir jaðargildisverkefnið
\begin{equation*}
\begin{split}P(x,D)u=f(x), \qquad Bu=0,\end{split}
\end{equation*}
gefið með formúlunni
\begin{equation*}
\begin{split}G_B(x,{\xi}) = \begin{cases} \dfrac{u_1({\xi})u_2(x)}
 {a_2({\xi})W(u_1,u_2)({\xi})}, &a\leq {\xi}\leq x\leq b,\\
  \dfrac{u_1(x)u_2({\xi})}
 {a_2({\xi})W(u_1,u_2)({\xi})}, &a\leq x\leq {\xi}\leq b,
 \end{cases}\end{split}
\end{equation*}
þar sem \(W(u_1,u_2)\) er Wronski-ákveða fallanna \(u_1\) og
\(u_2\).


\section{Eiginfallaliðun og Green\textendash{}föll}
\label{\detokenize{Kafli03:eiginfallaliun-og-greenfoll}}

\subsection{Reikniaðferð}
\label{\detokenize{Kafli03:reikniafer}}
Eftirfarandi losaralegu reikningar gera okkur kleift að finna Green-fall fyrir jaðargildisverkefnið
\begin{equation*}
\begin{split}Lu=f(x), \qquad x\in ]a,b[, \qquad Bu=0,\end{split}
\end{equation*}
þar sem
\begin{itemize}
\item {} 
\(L\) er virki af Sturm\textendash{}Liouville\textendash{}gerð

\item {} 
\(L\) er reglulegur og samhverfur með tilliti til jaðarskilyrðanna \(Bu=0\).

\end{itemize}

Nú fæst líkt og fyrir Fourier-raðir að
\begin{equation*}
\begin{split}u(x)=\sum\limits_{\substack{n=0 \\ \lambda_n\neq 0}}^{\infty} \dfrac {c_n(f)}{\lambda_n}u_n(x)\end{split}
\end{equation*}
er lausn ef röðin er nógu hratt samleitin þannig að víxla megi á diffrun og óendanlegri röð.

Ef \({\lambda}=0\) er eigingildi, þá gerum við ráð fyrir að \(f\) sé hornrétt á eiginrúmið \(E_0\).

Stingum inn formúlunni fyrir stuðlana \(c_n(f)\) og fáum
\begin{equation*}
\begin{split}\begin{aligned}
 u(x)&= \sum\limits_{n=0}^{\infty} \dfrac 1{\lambda_n}
 \bigg(\int_a^b f({\xi})u_n({\xi}){\varrho}({\xi})\, d{\xi}\bigg)
 u_n(x)\\
 &=\int_a^b{\varrho}({\xi})\bigg(\sum\limits_{n=0}^{\infty} \dfrac{u_n(x)u_n({\xi})}
 {\lambda_n}\bigg) f({\xi})\, d{\xi}.\nonumber\end{aligned}\end{split}
\end{equation*}
Green\textendash{}fallið fyrir jaðargildisverkefnið er ótvírætt ákvarðað, svo
\begin{equation*}
\begin{split}G_B(x,{\xi})={\varrho}({\xi})\sum\limits_{n=0}^{\infty}
 \dfrac{u_n(x)u_n({\xi})}{\lambda_n}.\end{split}
\end{equation*}

\section{Úrlausn hlutafleiðujafa með eiginfallaröðum}
\label{\detokenize{Kafli03:urlausn-hlutafleiujafa-me-eiginfallaroum}}
Við höldum nú áfram að fjalla um hvernig eigingildisverkefni koma við sögu í úrlausn hlutafleiðujafna. Við munum nálgast umfjöllunina með því að taka dæmi, sum þeirra kunnugleg en önnur ný. Það koma aðallega við sögu tvennskonar lausnaraðferðir
\begin{itemize}
\item {} 
Sett er fram lausnartilgáta á hlutafleiðujöfnu með hliðarskilyrðum í formi \sphinxstylestrong{eiginfallaraðar með tilliti til einnar breytistærðarinnar} þar sem stuðlarnir eru háðir hinni breytistærðinni.
\begin{itemize}
\item {} 
Eiginfallaröðin er valin þannig að hún innihaldi \sphinxstylestrong{eiginföll} þess hluta virkjans í verkefninu sem svarar til breytunnar sem liðað er með tilliti til og þannig að \sphinxstylestrong{eiginföllin uppfylli jaðarskilyrðin}.

\item {} 
Tilgátunni er stungið inn í hlutafleiðujöfnuna og gert ráð fyrir að víxla megi á óendanlegu röðinni og þeim afleiðum sem koma við sögu.

\item {} 
Þá fæst (hlut)afleiðujafna fyrir stuðlana ásamt hliðarskilyrðum sem mögulega má leysa.

\end{itemize}

\item {} 
\sphinxstylestrong{Aðskilnaði breytistærða er beitt} og þá fást eigingildisverkefni sem þarf að leysa og lausnir þeirra gefa fjölskyldu af ólíkum aðgreinanlegum lausnum, eina fyrir hvert eigingildi.  Lausn upphaflega verkefnisins má rita sem línulega samantekt af slíkum aðgreinanlegum lausnum.

\end{itemize}

Byrjum á að líta á dæmi um aðferðina sem líst er í fyrri punktinum.

Margar af mikilvægustu hlutafleiðujöfnum sem fengist er við til að mynda í eðlisfræði innihalda Laplace-virkjann og við byrjum á stuttri umfjöllun um Laplace-jöfnuna. Þegar Laplace-jafnan er leyst á gefnu mengi með fallsjaðarskilyrðum kallast verkefnið \sphinxstylestrong{Dirichlet-verkefnið}.


\subsection{Dirichlet-verkefnið á rétthyrningi}
\label{\detokenize{Kafli03:dirichlet-verkefni-a-retthyrningi}}
Lítum á verkefnið
\begin{equation*}
\begin{split}\begin{cases} \Delta u=0, &0<x<L, \ 0<y<M,\\
 u(x,0)=\varphi_1(x), \ u(x,M)=\varphi_2(x), &0<x<L,\\
 u(0,y)=\psi_1(y), \ u(L,y)=\psi_2(y), &0<y<M.
 \end{cases}\end{split}
\end{equation*}
\begin{figure}[htbp]
\centering
\capstart

\noindent\sphinxincludegraphics[width=0.650\linewidth]{{dirichlet1}.png}
\caption{Mynd: Dirichlet verkefnið á rétthyrningi.}\label{\detokenize{Kafli03:id11}}\end{figure}

Skiptum því í fjóra hluta
\begin{equation*}
\begin{split}\begin{cases} \Delta u_1=0,\\
 u_1(x,0)=\varphi_1(x), \ u_1(x,M)=0,\\
 u_1(0,y)=u_1(L,y)=0,
 \end{cases}\qquad
 \begin{cases} \Delta u_2=0,\\
 u_2(x,0)=0, u_2(x,M)=\varphi_2(x),\\
 u_2(0,y)=u_2(L,y)=0,
 \end{cases}\end{split}
\end{equation*}\begin{equation*}
\begin{split}\begin{cases} \Delta u_3=0,\\
 u_3(x,0)=u_3(x,M)=0,\\
 u_3(0,y)=\psi_1(y), \ u_3(L,y)=0,
 \end{cases} \qquad
 \begin{cases} \Delta u_4=0,\\
 u_4(x,0)=u_4(x,M)=0,\\
 u_4(0,y)=0, u_4(L,y)=\psi_2(y).
 \end{cases}\end{split}
\end{equation*}
Ef \(u_1\), \(u_2\), \(u_3\) og \(u_4\) eru lausnir þá er
\(u(x,y)=u_1(x,y)+u_2(x,y)+u_3(x,y)+u_4(x,y)\) lausn upphaflega verkefnis.

Nóg er að leysa verkefnið fyrir \(u_1\) því lausnina á hinum má skrifa niður út frá þeirri lausn.
\begin{itemize}
\item {} 
Vegna jaðarskilyrða \(u_1(0,y)=u_1(L,y)=0\) liðum við \(u_1(x,y)\) í Fourier-sínusröð í breytistærðinni \(x\), með stuðla sem eru háðir \(y\)

\end{itemize}
\begin{equation*}
\begin{split}u_1(x,y)=\sum\limits_{n=1}^\infty u_{1n}(y)\sin\big(n\pi x/L\big), \\\end{split}
\end{equation*}\begin{itemize}
\item {} 
Ákvörðum stuðlana \(u_{1n}(y)\), með því að víxla á óendanlegu röðinni og \(\Delta\) og stingum svo inn jaðarskilyrðunum.

\item {} 
Fáum þá að \(u_{1n}\) er lausn á jaðargildisverkefninu

\end{itemize}
\begin{equation*}
\begin{split}\begin{cases}
 u_{1n}{{^{\prime\prime}}}(y)-(n\pi/L)^2 u_{1n}(y)=0, &0<y<M,\\
 u_{1n}(0)=b_n(\varphi_1), \quad u_{1n}(M)=0.
 \end{cases}\end{split}
\end{equation*}
þar sem \(b_n(\varphi_1)\) er \(n\)-ti Fourier-sínus-stuðull \(\varphi_1\). Lausn þessa verkefnis er
\begin{equation*}
\begin{split}\begin{aligned}
 u_{1n}(y)&=b_n(\varphi_1)\cosh\big(n\pi y/L\big)- b_n(\varphi_1)
 \dfrac{\cosh\big(n\pi M/L\big)}{\sinh\big(n\pi M/L\big)}
 \sinh\big(n\pi y/L\big)\\
 &=b_n(\varphi_1)\dfrac
 {\sinh\big(n\pi M/L\big) \cosh\big(n\pi y/L\big)
 -\cosh\big(n\pi M/L\big) \sinh\big(n\pi y/L\big)}
 {\sinh\big(n\pi M/L\big)}\\
 &=b_n(\varphi_1)\dfrac
 {\sinh\big(n\pi (M-y)/L\big)}
 {\sinh\big(n\pi M/L\big)}.\end{aligned}\end{split}
\end{equation*}
Fáum svo \(u_2\) með því að skipta á
\(y\) og \(M-y\) og \(u_3\) og \(u_4\) með því að skipta á hlutverkum \(x\) og
\(y\). Lokaniðurstaðn er því
\begin{equation*}
\begin{split}\begin{aligned}
 u(x,y)&=\sum\limits_{n=1}^\infty
 b_n(\varphi_1)
 \dfrac{\sinh\big(n\pi(M-y)/L\big)}{\sinh\big(n\pi M/L\big)}
 \sin\big(n\pi x/L\big)\\
 &+\sum\limits_{n=1}^\infty
 b_n(\varphi_2)
 \dfrac{\sinh\big(n\pi y/L\big)}{\sinh\big(n\pi M/L\big)}
 \sin\big(n\pi x/L\big)\nonumber\\
 &+\sum\limits_{n=1}^\infty
 b_n(\psi_1)
 \dfrac{\sinh\big(n\pi (L-x)/M\big)}{\sinh\big(n\pi L/M\big)}
 \sin\big(n\pi y/M\big)\nonumber\\
 &+\sum\limits_{n=1}^\infty
 b_n(\psi_2)
 \dfrac{\sinh\big(n\pi x/M\big)}{\sinh\big(n\pi L/M\big)}
 \sin\big(n\pi y/M\big).\nonumber\end{aligned}\end{split}
\end{equation*}
\begin{sphinxadmonition}{attention}{Athugið:}
Það reyndist mikilvægt í þessari aðferð að föllin \(\sin(n{\pi}x/L)\) uppfylla gefnu jaðarskilyrðin og eru eiginföll \(\partial_x^2\).
\end{sphinxadmonition}


\subsection{Dirichlet-verkefnið á skífu}
\label{\detokenize{Kafli03:dirichlet-verkefni-a-skifu}}
Lítum á sama verkefni á hringskífu
\begin{equation*}
\begin{split}\begin{cases} \Delta u=
 \dfrac{\partial^2u}{\partial x^2}+
 \dfrac{\partial^2u}{\partial y^2}=0, &x^2+y^2<a^2,\\
 u(x,y)=\varphi(x,y), &x^2+y^2=a^2.
 \end{cases}\end{split}
\end{equation*}
Þar sem svæðið er skífa er eðlilegt að umrita verkefnið með því að nota \sphinxstylestrong{pólhnit}.
Laplace-virkjann er í pólhnitum
\begin{equation*}
\begin{split}\Delta = \dfrac 1r\dfrac{\partial}{\partial r}
 \bigg(r\dfrac{\partial }{\partial r}\bigg)
 +\dfrac 1{r^2}\dfrac{\partial^2 }{\partial\theta^2},\end{split}
\end{equation*}
og því má rita verkefnið á forminu
\begin{equation*}
\begin{split}\begin{cases}
 \dfrac 1r\dfrac{\partial}{\partial r}
 \bigg(r\dfrac{\partial v}{\partial r}\bigg)
 +\dfrac 1{r^2}\dfrac{\partial^2 v}{\partial\theta^2}=0, &r<a,
 \ {\theta}\in {{\mathbb  R}},\\
 v(a,\theta)={\psi}(\theta), &{\theta}\in {{\mathbb  R}}.
 \end{cases}\end{split}
\end{equation*}
með \(v(r,\theta) = u(x(r,\theta),y(r,\theta))\).

\begin{figure}[htbp]
\centering
\capstart

\noindent\sphinxincludegraphics[width=0.750\linewidth]{{dirichlet2}.png}
\caption{Mynd: Dirichlet verkefnið á skífu.}\label{\detokenize{Kafli03:id12}}\end{figure}
\begin{itemize}
\item {} 
Þar sem \(v\) og \({\psi}\) eru \(2\pi\)-lotubundin föll prófum við lausnartilgáta sem er Fourier-röðum með tilliti til \({\theta}\) með stuðlum sem geta verið háðir \(r\)

\end{itemize}
\begin{equation*}
\begin{split}v(r,\theta)=\sum\limits_{n=-\infty}^{+\infty}
 v_n(r)e^{in\theta}.\end{split}
\end{equation*}
og liðum \(\psi\) sömuleiðis í Fourier-röð
\begin{quote}
\begin{equation*}
\begin{split}{\psi}(\theta)=\sum\limits_{n=-\infty}^{+\infty}
{\psi}_n e^{in\theta}.\end{split}
\end{equation*}\end{quote}
\begin{itemize}
\item {} 
Ákvörðum stuðlana \(v_{n}(r)\), með því að víxla á óendanlegu röðinni fyrir \(v\) og \(\Delta\) og stingum svo inn jaðarskilyrðunum.

\item {} 
Fáum þá að \(v_{n}\) er lausn á jaðargildisverkefninu

\end{itemize}
\begin{equation*}
\begin{split}\begin{cases}
 r\dfrac d{dr}\bigg(r\dfrac{dv_n}{dr}\bigg)-n^2v_n=0, &r<a,\\
 v_n(a)={\psi}_n, \quad v_n(r) \text{ takmarkað ef } r\to 0.
 \end{cases}\end{split}
\end{equation*}
Þetta er Euler-jafna og því stingum við inn lausnartilgátu \(v_n(r)=r^\alpha\)
\begin{equation*}
\begin{split}r\dfrac d{dr}\bigg( r\dfrac d{dr}r^\alpha\bigg)=\alpha^2r^\alpha=
 n^2r^\alpha.\end{split}
\end{equation*}
og sjáum að \(\alpha=\pm n\). Almenn lausn
afleiðujöfnunar er því
\begin{equation*}
\begin{split}v_n(r)=
 \begin{cases}
 A_nr^{|n|}+B_nr^{-|n|}, &n\neq 0\\
 A_0+B_0\ln r, &n=0.
 \end{cases}\end{split}
\end{equation*}
Til þess að lausnin geti verið takmörkuð í \(r=0\), þá útilokum við liðina með neikvæðum veldisvísi og logrann. Skilyrðið
\(v_n(a)={\psi}_n\) gefur að \(A_n={\psi}_n/a^{|n|}\). Þar með
er lausnin fundin
\begin{equation*}
\begin{split}v(r,\theta)=\sum\limits_{n=-\infty}^{+\infty}
 \psi_n \bigg(\dfrac r a\bigg)^{|n|}e^{in\theta}.\end{split}
\end{equation*}
\begin{sphinxadmonition}{attention}{Athugið:}
Það reyndist mikilvægt í þessari aðferð að föllin \(e^{in\theta}\) eru eiginföll \(\partial_\theta^2\).
\end{sphinxadmonition}


\subsection{Varmaleiðnijafnan með tímaháðum jaðarskilyrðum}
\label{\detokenize{Kafli03:varmaleinijafnan-me-timahaum-jaarskilyrum}}
Reiknum hitastig í jarðvegi sem fall af tíma \(t\) og dýpi \(x\).

Hitastigið á yfirborði er gefið sem fall af tíma \(f(t)\) og gert ráð fyrir að það sé \(T\)-lotubundið fall (t.d.vegna árstíðasveiflna). Ritum
\begin{equation*}
\begin{split}f(t)=\sum\limits_{n=-\infty}^{+\infty}
 c_n(f)e^{in\omega t}, \qquad  \omega=2\pi/T.\end{split}
\end{equation*}
Setjum upp jaðargildisverkefnið
\begin{equation*}
\begin{split}\begin{cases}
 \dfrac{\partial u}{\partial t}-\kappa
 \dfrac{\partial^2 u}{\partial x^2}=0, &x>0, \ t\in {{\mathbb  R}},\\
 u(0,t)=f(t), &t\in {{\mathbb  R}},\\
 u(x,t) \text{ takmarkað ef } & x\to +\infty.
 \end{cases}\end{split}
\end{equation*}\begin{itemize}
\item {} 
Prófum lausn \(u(x,t)\) sem er \(T\)-lotubundið fall af \(t\) fyrir fast \(x\). Liðum það í \(u\) í Fourier-röð miðað við \(t\) með stuðla sem geta verið háðir \(x\)

\end{itemize}
\begin{equation*}
\begin{split}u(x,t)=\sum\limits_{n=-\infty}^{+\infty}
 u_n(x)e^{in\omega t}.\end{split}
\end{equation*}\begin{itemize}
\item {} 
Ákvörðum stuðlana \(u_{n}(x)\), með því að víxla á óendanlegu röðinni fyrir \(v\) og virkjanum \(\dfrac{\partial}{\partial t}-\kappa\dfrac{\partial^2}{\partial x^2}\) og stingum svo inn jaðarskilyrðunum.

\item {} 
Fáum þá að \(u_{n}\) er lausn á

\end{itemize}
\begin{equation*}
\begin{split}\begin{cases}
 u_n{{^{\prime\prime}}}(x)-\dfrac{in\omega}\kappa u_n(x)=0,\\
 u_n(0)=c_n(f),\\
 u_n(x) \text{ er takmarkað ef } x \to +\infty.
 \end{cases}\end{split}
\end{equation*}
Kennijafna afleiðujöfnunnar er
\begin{equation*}
\begin{split}\lambda^2-\dfrac{in\omega}\kappa=0\end{split}
\end{equation*}
og núllstöðvar hennar eru \(\lambda=\pm k_n\), þar sem
\begin{equation*}
\begin{split}k_n=
 \begin{cases}
 \bigg(\dfrac 1{\sqrt 2}+\dfrac i{\sqrt 2}\bigg)
 \sqrt{n\omega/\kappa}, &n>0,\\
 0, &n=0,\\
 \bigg(\dfrac 1{\sqrt 2}-\dfrac i{\sqrt 2}\bigg)
 \sqrt{|n|\omega/\kappa}, &n<0.
 \end{cases}\end{split}
\end{equation*}
Lausnin er því
\begin{equation*}
\begin{split}u_n(x)=\begin{cases}
 A_ne^{-k_nx}+B_ne^{k_nx}, &n\neq 0\\
 A_0+B_0x, &n=0.
 \end{cases}\end{split}
\end{equation*}
Til þess að lausnin haldist takmörkuð ef \(x\to +\infty\), þá
verður \(B_n=0\) að gilda fyrir öll \(n\). Jaðarskilyrðið
\(u_n(0)=c_n(f)\) gefur að \(A_n=c_n(f)\). Við höfum því að
\begin{equation*}
\begin{split}u_n(x)=c_n(f)e^{-\sqrt{|n|\omega/2\kappa}\, x}
 e^{-i{{\operatorname{sign}}}(n)\sqrt{|n|\omega/2\kappa}\, x},\end{split}
\end{equation*}
og þar með er lausnin fundin
\begin{equation*}
\begin{split}u(x,t)=\sum\limits_{n=-\infty}^{+\infty}
 c_n(f)e^{-\sqrt{|n|\omega/2\kappa}\, x}
 e^{i(n\omega t-{{\operatorname{sign}}}(n)\sqrt{|n|\omega/2\kappa}\, x)}.\end{split}
\end{equation*}
Við sjáum að sveifluvíddin og fasahliðrunin í liðnum
\(u_n(x)e^{in\omega t}\) í lausninni eru háð dýpi og tíðninni
\(n\omega\).


\section{Áfram um eigingildisverkefni - aðskilnaður breytistærða}
\label{\detokenize{Kafli03:afram-um-eigingildisverkefni-askilnaur-breytistaera}}
Í þessum síðustu greinum kaflans munum við fara í gegnum fleiri reikniaðferðir sem beita má við lausn hlutafleiðujafna. Við reynum að koma almennum hugmyndum til skila en styðjumst þó að mestu við ákveðin dæmi til að skýra hugmyndirnar.


\subsection{Dirichlet-verkefnið á rétthyrningi - aftur}
\label{\detokenize{Kafli03:dirichlet-verkefni-a-retthyrningi-aftur}}
Skoðum aftur Dirchlet-verkefnið á rétthyrningi. Lítum á jöfnu 2 af jöfnunum fjórum sem komu áður fyrir
\begin{equation*}
\begin{split}\begin{cases} \Delta u={\partial}_x^2u+{\partial}_y^2u=0, &0<x<L, \ 0<y<M,\\
 u(0,y)=u(L,y)=0, &0<y<M,\\
 u(x,0)=0, \ u(x,M)=\varphi(x), &0<x<L,\\
 \end{cases}\end{split}
\end{equation*}
þar sem \(\varphi\) er gefið fall á \([0,L]\).
\begin{itemize}
\item {} 
Leitum fyrst að öllum lausnum af gerðinni \(v(x,y)=X(x)Y(y)\) sem uppfylla jöfnuna og óhliðruðu jaðarskilyrðin.

\item {} 
Stingum næst \(v\) inn í hlutafleiðu jöfnuna og fáum

\end{itemize}
\begin{equation*}
\begin{split}X{{^{\prime\prime}}}(x)Y(y)+X(x)Y{{^{\prime\prime}}}(y)=0.\end{split}
\end{equation*}\begin{itemize}
\item {} 
Deilum í gegnum jöfnuna með \(X(x)Y(y)\) og fáum þá

\end{itemize}
\begin{equation*}
\begin{split}-\dfrac{X{{^{\prime\prime}}}(x)}{X(x)}=\dfrac{Y{{^{\prime\prime}}}(y)}{Y(y)}.\end{split}
\end{equation*}
Fallið vinstra megin er einungis háð \(x\), en fallið hægra megin er einungis háð \(y\). Því hlýtur hvor hlið að vera föst. Við
höfum því
\begin{equation*}
\begin{split}-X{{^{\prime\prime}}}(x)=\lambda X(x) \qquad \text{ og } \qquad Y{{^{\prime\prime}}}(y)=\lambda Y(y),\end{split}
\end{equation*}
þar sem \(\lambda\) er fasti.
\begin{itemize}
\item {} 
Lítum nú á jaðarskilyrðin

\end{itemize}
\begin{equation*}
\begin{split}X(0)Y(y)=X(L)Y(y)=0, \qquad X(x)Y(0)=0,\end{split}
\end{equation*}
og sjáum að \(X\) er lausn á eigingildisverkefninu
\begin{equation*}
\begin{split}-X{{^{\prime\prime}}}=\lambda X, \qquad X(0)=X(L)=0.\end{split}
\end{equation*}
Þetta höfum við leyst áður og fengum eigingildin
\(\lambda=\lambda_n=\big(n\pi/L\big)^2\), \(n=1,2,3,\dots\), og
tilsvarandi eiginföll
\begin{equation*}
\begin{split}X_n(x)=C_n \sin\big(n\pi x/L\big), \qquad n=1,2,3,\dots.\end{split}
\end{equation*}
Leysum næst
\begin{equation*}
\begin{split}Y{{^{\prime\prime}}}(y)=\big(n\pi/L\big)^2 Y(y), \qquad Y(0)=0.\end{split}
\end{equation*}
Þessi jafna hefur greinilega lausnina
\begin{equation*}
\begin{split}Y_n(y)=D_n \sinh\big(n\pi y/L\big), \qquad n=1,2,3,\dots.\end{split}
\end{equation*}
Nú eru allar lausnir á Laplace-jöfnunni af gerðinni
\(v(x,y)=X(x)Y(y)\) með óhliðruðu jaðarskilyrðunum gefnar með formúlunni
\begin{equation*}
\begin{split}v(x,y)=C_nD_n \sin\big(n\pi x/L\big)\sinh\big(n\pi y/L\big), \qquad
 n=1,2,3,\dots.\end{split}
\end{equation*}
Getum valið \(D_n=1\). Tökum óendanlega línulega samatekt af þessum lausnum
\begin{equation*}
\begin{split}u(x,y)=\sum\limits_{n=1}^\infty
 C_n\sin\big(n\pi x/L\big)\sinh\big(n\pi y/L\big).\end{split}
\end{equation*}
Síðasta jaðarskilyrðið,
\(u(x,M)=\varphi(x)\) er uppfyllt ef
\begin{equation*}
\begin{split}\begin{aligned}
 u(x,M)&= \sum\limits_{n=1}^\infty
 C_n \sin\big(n\pi x/L\big)\sinh\big(n\pi M/L\big)\\
 &= \sum\limits_{n=1}^\infty
 b_n(\varphi) \sin\big(n\pi x/L\big)=\varphi(x),\end{aligned}\end{split}
\end{equation*}
þar sem \(b_n(\varphi)\) er Fourier-sínusstuðull fallsins
\(\varphi\).

Samanburður á stuðlum gefur
\begin{equation*}
\begin{split}u(x,y)=\sum_{n=1}^\infty
 b_n(\varphi)\dfrac{\sinh\big(n\pi y/L\big)}{\sinh\big(n\pi
 M/L\big)} \sin\big(n\pi x/L\big).\end{split}
\end{equation*}

\subsection{Dirichlet-verkefnið á skífu - aftur}
\label{\detokenize{Kafli03:dirichlet-verkefni-a-skifu-aftur}}
Leysum aftur Dirichlet-verkefnið á hringskífu með aðskilnaði breytistærða,
\begin{equation*}
\begin{split}\begin{cases}
 \dfrac 1r\dfrac{\partial}{\partial r}
 \bigg(r\dfrac{\partial v}{\partial r}\bigg)
 +\dfrac 1{r^2}\dfrac{\partial^2 v}{\partial\theta^2}=0, &r<a,
 \ {\theta}\in {{\mathbb  R}},\\
 v(a,\theta)={\psi}(\theta), &{\theta}\in {{\mathbb  R}},
 \end{cases}\end{split}
\end{equation*}
þar sem föllin \(v\) og \(\psi\) eru \(2\pi\)-lotubundin í
\(\theta\).
\begin{itemize}
\item {} 
Leitum fyrst að öllum lausnum af gerðinni \(w(r,\theta)=R(r)\Theta(\theta)\).

\item {} 
Stingum tilgátunni inn í hlutafleiðujöfnuna og fáum

\end{itemize}
\begin{equation*}
\begin{split}r \big(r R{{^{\prime}}}(r)\big){{^{\prime}}}\Theta(\theta)
 +R(r)\Theta{{^{\prime\prime}}}(\theta)=0.\end{split}
\end{equation*}\begin{itemize}
\item {} 
Deilum í gegn með \(R(r)\Theta(\theta)\) og fáum

\end{itemize}
\begin{equation*}
\begin{split}r\big(r R{{^{\prime}}}(r)\big){{^{\prime}}}/R(r)
 =-\Theta {{^{\prime\prime}}}(\theta)/\Theta (\theta).\end{split}
\end{equation*}
Vinstri hliðin er eingöngu háð \(r\) en hægri hliðin er eingöngu háð \(\theta\). Þar með eru báðar hliðar jafnar fasta, segjum \(\lambda\). Fáum þá jöfnurnar
\begin{equation*}
\begin{split}-\Theta{{^{\prime\prime}}}(\theta)=\lambda\Theta(\theta),
 \qquad
 r\dfrac d{dr}\bigg(r\dfrac {d R}{dr}(r)\bigg)=\lambda {R(r)}.\end{split}
\end{equation*}\begin{itemize}
\item {} 
Almenn lausn á fyrri jöfnunni er

\end{itemize}
\begin{equation*}
\begin{split}\Theta(\theta)=\begin{cases}
 Ae^{i\beta\theta}+Be^{-i\beta\theta},  &\lambda=\beta^2\neq 0,\\
 A_0+B_0\theta, &\lambda=0.
 \end{cases}\end{split}
\end{equation*}
Þar sem fallið \(\Theta\) er \(2\pi\)-lotubundið fæst
að einu gildin sem \(\lambda\) getur tekið eru
\(\lambda=\lambda_n=n^2\), \(n=0,1,2,\dots\), og \(B_0=0\).
Þar með er
\begin{equation*}
\begin{split}\Theta(\theta)=\begin{cases}
 A_ne^{in\theta}+B_ne^{-in\theta},  &n=1,2,3,\dots,\\
 A_0, &\lambda=0.
 \end{cases}\end{split}
\end{equation*}\begin{itemize}
\item {} 
Lítum næst á afleiðujöfnuna fyrir \(R(r)\) með \(\lambda=n^2\). Þetta er Euler-jafna. Með því að leita að lausn af gerðinni \(R(r)=r^\alpha\) sjáum við að \(\alpha=\pm n\). Almenn lausn á seinni afleiðujöfnunni fyrir \(R(r)\) með \(\lambda=n^2\) er því

\end{itemize}
\begin{equation*}
\begin{split}R(r)=\begin{cases}
 C_nr^n+D_nr^{-n}, &n=1,2,3,\dots,\\
 C_0+D_0\ln r, &n=0.
 \end{cases}\end{split}
\end{equation*}
Þar sem lausnin verður að gilda í \(r=0\) þarf \(D_n=0\), \(n=0,1,2,\dots\). Þar með er
\begin{equation*}
\begin{split}R(r)=\begin{cases}
 C_nr^n, &n=1,2,3,\dots,\\
 C_0, &n=0.
 \end{cases}\end{split}
\end{equation*}\begin{itemize}
\item {} 
Allar lausnir á verkefninu af gerðinni \(w(r,\theta)=R(r)\Theta(\theta)\)  eru þá

\end{itemize}
\begin{equation*}
\begin{split}w(r,\theta)=
 C_nr^n\big(A_ne^{in\theta}+B_ne^{-in\theta}\big), \qquad n=0,1,2,\dots,\end{split}
\end{equation*}
þar sem \(A_n\), \(B_n\) og \(C_n\) eru fastar. Veljum \(C_n=1\).

Almenn lausn hlutafleiðujöfnunnar er línuleg samantekt þessara lausna
\begin{equation*}
\begin{split}v(r,\theta)=\sum\limits_{-\infty}^{+\infty}A_nr^{|n|}e^{in\theta},\end{split}
\end{equation*}
þar sem við höfum sett \(A_n=B_{-n}\) ef \(n<0\).
\begin{itemize}
\item {} 
Notum nú jaðarskilyrðið í \(r=a\),

\end{itemize}
\begin{equation*}
\begin{split}v(a,\theta)=\sum\limits_{-\infty}^{+\infty}A_na^{|n|}e^{in\theta}
 =\sum\limits_{-\infty}^{+\infty}c_n(\psi)a^{|n|}e^{in\theta}=\psi(\theta).\end{split}
\end{equation*}
Með samanburði á stuðlum fæst að \(A_n=c_n(\psi)/a^{|n|}\)
og þar með fæst
\begin{equation*}
\begin{split}v(r,\theta)=\sum\limits_{-\infty}^{+\infty}
 c_n(\psi)\bigg(\dfrac ra\bigg)^{|n|} e^{in\theta}.\end{split}
\end{equation*}

\section{Tvöfaldar Fourier-raðir}
\label{\detokenize{Kafli03:tvofaldar-fourier-rair}}
Látum \({\varphi}:\overline D\to {{\mathbb  C}}\) vera samfellt
deildanlegt á \(D=\{(x,y); 0<x<L, 0<y<M\}\) og samfellt á lokuninni
\(\overline  D\). Ef \({\varphi}\) er jafnt \(0\) á jaðrinum
\({\partial}D\), þá getum við liðað \({\varphi}\) í
Fourier-sínusröð með tilliti til \(y\)
\begin{equation*}
\begin{split}{\varphi}(x,y)= \sum\limits_{m=1}^{\infty}
 {\varphi}_m(x)\sin\big(m{\pi}y/M\big).\end{split}
\end{equation*}
Fallið \({\varphi}_m\) er samfellt deildanlegt og tekur gildið
\(0\) í \(x=0\) og \(x=L\), svo við getum einnig liðað það í
Fourier-sínusröð
\begin{equation*}
\begin{split}\varphi_m(x) = \sum_{n=1}^\infty b_{n,m} \sin(n\pi x/L).\end{split}
\end{equation*}
Höfum því framsetningu á \(\varphi\) með \sphinxstyleemphasis{tvöfaldri Fourier-röð}
\begin{equation*}
\begin{split}{\varphi}(x,y)=\sum\limits_{n=1}^{{\infty}}
 \sum\limits_{m=1}^{{\infty}} b_{n,m}
 \sin\big(n{\pi}x/L\big)\sin\big(m{\pi}y/M\big).\end{split}
\end{equation*}

\subsection{Dæmi - Rétthyrnd tromma}
\label{\detokenize{Kafli03:daemi-retthyrnd-tromma}}
Himna er strekkt á rétthyrndan ramma með hliðarlengdir \(L\) og \(M\) og sveiflast þar. Lóðrétt færsla hennar frá jafnvægi
í punkti \((x,y)\) á tíma \(t\) er táknuð með \(u(x,y,t)\) og uppfyllir tvívíðu bylgjujöfnuna. Ef staða og hraði
trommunnar eru gefin við tímann \(t=0\), þá er \(u\) lausn
verkefnisins
\begin{equation*}
\begin{split}\begin{cases}
 \dfrac{{\partial^2} u}{{\partial} t^2}
 -c^2\bigg(\dfrac{\partial^2u}{\partial x^2}
 +\dfrac{\partial^2u}{\partial y^2}\bigg)=0,
 &0<x<L, 0<y<M, t>0,\\
 u(0,y,t)=u(L,y,t)=0,
 &0<y<M, t>0,\\
 u(x,0,t)=u(x,M,t)=0,
 &0<x<L, t>0,\\
 u(x,y,0)=\varphi(x,y), \ {\partial}_tu(x,y,0)={\psi}(x,y),
 &0<x<L, 0<y<M.
 \end{cases}\end{split}
\end{equation*}
Lítum á lausnartilgátu á formi tvöfaldrar Fourier-raðar
\begin{equation*}
\begin{split}u(x,y,t)=\sum\limits_{n=1}^{{\infty}}
 \sum\limits_{m=1}^{{\infty}} u_{n,m}(t)
 \sin\big(n{\pi}x/L\big)\sin\big(m{\pi}y/M\big).\end{split}
\end{equation*}
Stingum henni inni í hlutafleiðujöfnuna og víxlum á hlutafleiðuvirkjanum og óendanlegu röðinni. Fáum þá jöfnuna
\begin{equation*}
\begin{split}u_{n,m}{{^{\prime\prime}}}(t)+c^2{\pi}^2(n^2/L^2+m^2/M^2)u_{n,m}=0,\end{split}
\end{equation*}
sem hefur almenna lausn
\begin{equation*}
\begin{split}u_{n,m}(t)=A_{n,m}\cos\big(\sqrt{n^2/L^2+m^2/M^2}\, {\pi}ct\big)
 +B_{n,m}\sin\big(\sqrt{n^2/L^2+m^2/M^2}\, {\pi}ct\big).\end{split}
\end{equation*}
Út frá upphafsskilyrðunum fæst
\begin{equation*}
\begin{split}A_{n,m}=b_{n,m}({\varphi}) \qquad \text{ og } \qquad
 B_{n,m}=\dfrac{b_{n,m}({\psi})}{\sqrt{n^2/L^2+m^2/M^2}\, {\pi}c}.\end{split}
\end{equation*}
Mögulegar tíðnir í sveiflunni eru því
\begin{equation*}
\begin{split}\{\tfrac \pi 2\sqrt{n^2/L^2+m^2/M^2}\, c; n,m=1,2,3,\dots\}.\end{split}
\end{equation*}
Lægsta tíðnin \(\frac \pi 2\sqrt{1/L^2+1/M^2}\, c\) nefnist
\sphinxstyleemphasis{grunntíðni} og hinar tíðnirnar nefnast \sphinxstyleemphasis{yfirtíðnir}. Yfirtíðnirnar eru ekki heiltölumargfeldi af grunntíðninni eins og gildir fyrir sveiflandi streng.  Þetta er skýringin á því hvers vegna trommur gefa ekki frá sér hreinan tón eins og strengir.


\section{Almennt um eiginfallaraðir}
\label{\detokenize{Kafli03:almennt-um-eiginfallarair}}
Við ljúkum umfjöllun þessa kafla á að taka dæmi sem undirstrikar það að lausnaraðferðirnar sem við höfum verið að beita þar sem eiginfallaraðir koma við sögu eru almennar og ekki bundnar við notkun hefbundinna Fourier-raða. Með þessa hugmynd að leiðarljósi er í sumum tilfellum hægt að komast langt í að skrifa niður lausn verkefnis án þess að þekkja eiginföllin og eigingildin.


\subsection{Dæmi - Alhæft varmaleiðniverkefni}
\label{\detokenize{Kafli03:daemi-alhaeft-varmaleiniverkefni}}
Látum \(P(x,D_x)\) vera afleiðuvirkja af Sturm-Liouville gerð. Látum \(u = u(x,t)\) vera fall af tveimur breytistærðum og \(B_1\) og \(B_2\) samhverfa jaðargildisvirkja. Skoðum verkefnið
\begin{equation*}
\begin{split}\begin{cases}
 \dfrac{\partial u}{\partial t}+P(x,\partial_x)u=f(x,t),
 &x\in ]a,b[, \ t>0,\\
 u(x,0)=\varphi(x), & x\in ]a,b[,\\
 B_1u(\cdot,t)=B_2u(\cdot,t)=0, &t>0.
 \end{cases}\end{split}
\end{equation*}
\(B_ju(\cdot,t)\) táknar að \(B_j\) verki með tilliti til fyrri breytistærðarinnar \(x\).

Gerum eftirfarandi lausnartilgátu
\begin{equation*}
\begin{split}u(x,t)=\sum\limits_{n=0}^{\infty} c_n(t)u_n(x),\end{split}
\end{equation*}
þar sem \(u_n(x)\) eru eiginföll virkjans \(P\) (ásamt jaðarskilyrðum) með tilsvarandi eigingildi eru \(\lambda_n\) og \(c_n(t)\) eru Fourier-stuðlar \(u(x,t)\) með tilliti til eiginfallanna.

Liðum föllin \(f\) og \(\varphi\) einnig í eiginfallaraðir
\begin{equation*}
\begin{split}f(x,t)=\sum\limits_{n=0}^{\infty} f_n(t)u_n(x), \qquad
 \varphi(x)=\sum\limits_{n=0}^{\infty} \varphi_nu_n(x).\end{split}
\end{equation*}
Stignum lausnatilgátunni inn í hlutafleiðujöfnuna og víxlum á hlutafleiðuvirkjanum og óendanlegu röðinni. Þá fæst
\begin{equation*}
\begin{split}\begin{aligned}
 \dfrac{{\partial}u}{{\partial} t}(x,t) +P(x,{\partial}_x)u(x,t)&= \sum\limits_{n=0}^{\infty}\bigg(
 c_n{{^{\prime}}}(t)+{\lambda}_nc_n(t)\bigg)u_n(x)\nonumber\\
 &=\sum\limits_{n=0}^{\infty} f_n(t)u_n(x)=f(x,t),\nonumber\\
 \end{aligned}\end{split}
\end{equation*}
ásamt upphafsskilyrðinu
\begin{equation*}
\begin{split} \begin{aligned}
 u(x,0)&=\sum\limits_{n=0}^{\infty} c_n(0)u_n(x)
=\sum\limits_{n=0}^{\infty} {\varphi}_nu_n(x)={\varphi}(x).\nonumber\end{aligned}\end{split}
\end{equation*}
Með því að bera saman stuðlana í jöfnunum fæst upphafsgildisverkefni fyrir \(c_n(t)\),
\begin{equation*}
\begin{split}\begin{cases}
 c_n{{^{\prime}}}(t)+{\lambda}_nc_n(t)=f_n(t),\\
 c_n(0)={\varphi}_n.
 \end{cases}\end{split}
\end{equation*}
Þetta er fyrsta stigs jafna með fastastuðla, svo
\begin{equation*}
\begin{split}c_n(t)={\varphi}_ne^{-{\lambda}_n t}+
 e^{-{\lambda}_n t}\int_0^te^{{\lambda}_n {\tau}}f_n({\tau})\, d{\tau}.\end{split}
\end{equation*}
Athugið að við gátum skrifað niður lausn án þess að þekkja eiginföllin \(u_n\) og tilsvarandi eigingildi \(\lambda_n\).


\chapter{Fourier-ummyndun}
\label{\detokenize{Kafli04:fourier-ummyndun}}\label{\detokenize{Kafli04::doc}}

\section{Fourier-ummyndun. Reiknireglur. Plancerel-jafnan}
\label{\detokenize{Kafli04:fourier-ummyndun-reiknireglur-plancerel-jafnan}}

\subsection{Skilgreining á \protect\(L^1(\mathbb R)\protect\)}
\label{\detokenize{Kafli04:skilgreining-a-l-1-mathbb-r}}
Við byrjum á að skilgreina rúm heildanlegra falla \(L^1(\mathbb R)\). Við táknum \(L^1(\mathbb R)\) mengi allra falla \(f\) þannig að \(|f|\) er heildanlegt á \(\mathbb R\).
\begin{equation*}
\begin{split}\int_{-\infty}^\infty |f(x)| dx < \infty\,.\end{split}
\end{equation*}
\(L^1(\mathbb R)\) er vigurrúm, af því að
\begin{enumerate}
\def\theenumi{\arabic{enumi}}
\def\labelenumi{\theenumi .}
\makeatletter\def\p@enumii{\p@enumi \theenumi .}\makeatother
\item {} 
Ef \(f \in L^1(\mathbb R)\) og \(g \in L^1(\mathbb R)\) þá er fallið \(f+g \in L^1(\mathbb R)\)

\end{enumerate}
\begin{equation*}
\begin{split}\int_{-\infty}^\infty |f(x)+g(x)| dx \le  \int_{-\infty}^\infty |f(x)| dx < \infty\, +\int_{-\infty}^\infty |g(x)| dx < \infty\,.\end{split}
\end{equation*}\begin{enumerate}
\def\theenumi{\arabic{enumi}}
\def\labelenumi{\theenumi .}
\makeatletter\def\p@enumii{\p@enumi \theenumi .}\makeatother
\setcounter{enumi}{1}
\item {} 
Ef \(f \in L^1(\mathbb R)\) þá er \(\alpha f \in L^1(\mathbb R)\), þar sem \(\alpha \in\mathbb R\)

\end{enumerate}
\begin{equation*}
\begin{split}\int_{-\infty}^\infty |\alpha f(x)| dx = |\alpha|\int_{-\infty}^\infty  |f(x)| dx <\infty\,.\end{split}
\end{equation*}

\subsection{Skilgreining á Fourier-ummyndun}
\label{\detokenize{Kafli04:skilgreining-a-fourier-ummyndun}}
Fyrir sérhvert fall \(f \in L^1(\mathbb R)\) skilgreinum við fallið
\begin{equation*}
\begin{split}\mathcal{F} f(k) = \int_{-\infty}^\infty e^{-i k x} f(x)dx\,, \qquad k \in\mathbb{R}\,.\end{split}
\end{equation*}
Við köllum fallið \(\mathcal{F} f\) Fourier-mynd fallsins \(f\) og við táknum hana með \(\mathcal{F}\{f\}\) eða \(\widehat{f}\).

Við köllum vörpun \(\mathcal{F}\) Fourier-ummyndun.
Hún er skilgreind á \(L^1(\mathbb R)\) og varpar falli \(f\in L^1(\mathbb R)\) á Fourier-mynd sína \(\mathcal{F} f\).

\begin{sphinxadmonition}{attention}{Athugið:}
Skilgreiningin á Fourier-ummyndun er ekki stöðluð.
\end{sphinxadmonition}


\subsection{Sýnidæmi}
\label{\detokenize{Kafli04:synidaemi}}
Skilgreinum fall \(f_a\) á eftifarandi hátt
\begin{equation*}
\begin{split}f_a (x) = \begin{cases} 1, \qquad &|x|<a\,, \\ 0, \qquad & |x|\ge a \,  \end{cases} \,, \qquad f_a : \mathbb{R}\to \mathbb{R}\,,\quad a>0\,.\end{split}
\end{equation*}
Við sjáum að \(f \in L^1(\mathbb R)\). Við reiknum nú Fourier-mynd fallsins \(f\)
\begin{equation*}
\begin{split}\widehat{f}_a(k)=\int_a^a e^{-i k x} dx= {e^{-i a k}-e^{i a k}\over -i k}=2 {\sin a k\over k}.\end{split}
\end{equation*}

\subsection{Reiknireglur}
\label{\detokenize{Kafli04:reiknireglur}}\label{\detokenize{Kafli04:example1}}
Við byrjum á að skoða reiknireglur fyrir Fourier-ummyndanir.
\begin{enumerate}
\def\theenumi{\arabic{enumi}}
\def\labelenumi{\theenumi .}
\makeatletter\def\p@enumii{\p@enumi \theenumi .}\makeatother
\item {} 
Látum \(f\) og \(g\) vera tvö föll í \(L^1(\mathbb R)\). Látum \(\alpha\) og \(\beta\) vera tvær tölur í \(\mathbb C\). Þá gildir

\end{enumerate}
\begin{equation*}
\begin{split}\mathcal{F}\left\{\alpha f+ \beta g\right\}(k) &=& \int_{-\infty}^\infty e^{-i k x}\left(\alpha f(x)+\beta g(x)\right)dx
= \alpha \int_{-\infty}^\infty e^{-i k x} f(x) dx+\beta \int_{-\infty}^\infty e^{-i k x} g(x) dx
\\ &=& \alpha\, \mathcal{F}\{f\}(k)+\beta\, \mathcal F\{g\}(k)\,,\end{split}
\end{equation*}
Þ.e.a.s. að Fourier-ummyndun er línuleg vörpun.
\begin{enumerate}
\def\theenumi{\arabic{enumi}}
\def\labelenumi{\theenumi .}
\makeatletter\def\p@enumii{\p@enumi \theenumi .}\makeatother
\setcounter{enumi}{1}
\item {} 
Látum \(f \in L^1(\mathbb R)\) og \(\alpha\in\mathbb R\smallsetminus\{0\}\). Þá gildir

\end{enumerate}
\begin{equation*}
\begin{split}\mathcal{F}\left\{f(\alpha x)\right\}(k) = {1\over |\alpha|} \mathcal F\{f(x)\}\left({k\over \alpha }\right)\,, \qquad k\in\mathbb R\,\end{split}
\end{equation*}
sem segir okkur hvernig Fourier-ummyndun breytist þegar \(x \to\alpha x\).
\begin{enumerate}
\def\theenumi{\arabic{enumi}}
\def\labelenumi{\theenumi .}
\makeatletter\def\p@enumii{\p@enumi \theenumi .}\makeatother
\setcounter{enumi}{2}
\item {} 
Látum  \(f \in L^1(\mathbb R)\) og \(\alpha\in\mathbb R\). Þá gildir

\end{enumerate}
\begin{equation*}
\begin{split}\mathcal{F}\left\{f(x-\alpha)\right\}(k) = e^{-i \alpha k} \mathcal F\{f(x)\}\left({k}\right)\,, \qquad k\in\mathbb R\,\end{split}
\end{equation*}
sem segir okkur hvernig Fourier-ummyndun breytist þegar  \(x \to x-\alpha\).
\begin{enumerate}
\def\theenumi{\arabic{enumi}}
\def\labelenumi{\theenumi .}
\makeatletter\def\p@enumii{\p@enumi \theenumi .}\makeatother
\setcounter{enumi}{3}
\item {} 
Látum  \(f \in L^1(\mathbb R)\) og \(\alpha\in\mathbb R\). Þá gildir

\end{enumerate}
\begin{equation*}
\begin{split}\mathcal{F}\left\{e^{i \alpha x}f(x)\right\}(k) = \mathcal F\{f(x)\}\left(k-\alpha\right)\,, \qquad k\in\mathbb R\,\end{split}
\end{equation*}
sem segir okkur hvernig Fourier-ummyndun breytist þegar  \(k \to k-\alpha\).
\begin{enumerate}
\def\theenumi{\arabic{enumi}}
\def\labelenumi{\theenumi .}
\makeatletter\def\p@enumii{\p@enumi \theenumi .}\makeatother
\setcounter{enumi}{4}
\item {} 
Látum \(f \in L^1(\mathbb R)\). Þá gildir

\end{enumerate}
\begin{equation*}
\begin{split}\mathcal{F}\overline{\left\{f(x)\right\}}(k) = \overline{\mathcal F\{f(x)\}\left(-k\right)}\,, \qquad k\in\mathbb R\,.\end{split}
\end{equation*}
Athugum að ef \(f \in L^1(\mathbb R)\) er raungilt, þ.e. \(f: \mathbb R\to\mathbb R\), þá gildir
\begin{equation*}
\begin{split}\mathcal{F}\left\{f(x)\right\}(k) = \overline{\mathcal F\{f(x)\}\left(-k\right)}\,, \qquad k\in\mathbb R\,.\end{split}
\end{equation*}\begin{enumerate}
\def\theenumi{\arabic{enumi}}
\def\labelenumi{\theenumi .}
\makeatletter\def\p@enumii{\p@enumi \theenumi .}\makeatother
\setcounter{enumi}{5}
\item {} 
Látum \(f \in L^1(\mathbb R)\) vera jafnstætt. Þá gildir

\end{enumerate}
\begin{equation*}
\begin{split}\mathcal{F}\left\{f(x)\right\}(k) = 2 \int_0^\infty \cos(k\, x) f(x) dx \,, \qquad k\in\mathbb R\,.\end{split}
\end{equation*}\begin{enumerate}
\def\theenumi{\arabic{enumi}}
\def\labelenumi{\theenumi .}
\makeatletter\def\p@enumii{\p@enumi \theenumi .}\makeatother
\setcounter{enumi}{6}
\item {} 
Látum \(f \in L^1(\mathbb R)\) vera oddstætt. Þá gildir

\end{enumerate}
\begin{equation*}
\begin{split}\mathcal{F}\left\{f(x)\right\}(k) = - 2 i \int_0^\infty \sin(k\, x) f(x) dx \,, \qquad k\in\mathbb R\,.\end{split}
\end{equation*}\begin{enumerate}
\def\theenumi{\arabic{enumi}}
\def\labelenumi{\theenumi .}
\makeatletter\def\p@enumii{\p@enumi \theenumi .}\makeatother
\setcounter{enumi}{7}
\item {} 
Látum \(f \in \mathcal{C}^1(\mathbb R)\). Gerum ráð fyrir að \(f\)  og \(f'\) séu í \(L^1(\mathbb R)\). Þá gildir

\end{enumerate}
\begin{equation*}
\begin{split}\mathcal{F}\left\{f'(x)\right\}(k)= i k \mathcal{F}\left\{f(x)\right\}(k)\,, \qquad k \in\mathbb R\,.\end{split}
\end{equation*}
Regla 8 tengir Fourier-mynd fallsins \(f\) og Fourier-mynd afleiðu þess \(f'\).

Ef \(f\in\mathcal{C}^m(\mathbb R)\)  og  \(f, f', \dots, f^{(m)} \in L^1(\mathbb R)\), þá gildir
\begin{equation*}
\begin{split}\mathcal{F}\left\{f^{(j)}(x)\right\}(k)= (i k)^j \mathcal{F}\left\{f(x)\right\}(k)\,, \qquad k \in\mathbb R\,, \quad j=0, 1, \dots\, m\,.\end{split}
\end{equation*}\begin{enumerate}
\def\theenumi{\arabic{enumi}}
\def\labelenumi{\theenumi .}
\makeatletter\def\p@enumii{\p@enumi \theenumi .}\makeatother
\setcounter{enumi}{8}
\item {} 
Gerum ráð fyrir að föll \(f\) og \(x f\) séu í \(L^1(\mathbb R)\). Þá gildir

\end{enumerate}
\begin{equation*}
\begin{split}\mathcal{F}\left\{x f(x)\right\}(k)= i \frac{d}{dk}\mathcal{F}\left\{f(x)\right\}(k)\,, \qquad k \in\mathbb R\,.\end{split}
\end{equation*}
Regla 9 segir okkur hver afleiða Fourier-myndar fallsins \(f\) er.

Gerum ráð fyrir að föll \(f, x f, \dots, x^j f\) séu í \(L^1(\mathbb R)\). Þá gildir
\begin{equation*}
\begin{split}\mathcal{F}\left\{x^j f(x)\right\}(k)= i^j \frac{d^j}{dk^j}\mathcal{F}\left\{f(x)\right\}(k)\,, \qquad k \in\mathbb R\,.\end{split}
\end{equation*}

\subsection{Sýnidæmi}
\label{\detokenize{Kafli04:rulesft}}\label{\detokenize{Kafli04:id1}}
Við skoðum núna dæmi um hvernig nota má reiknireglurnar til þess að reikna Fourier-mynd falla.

Athugum fall \(f(x)=e^{-a x^2/2}\) þar sem  \(a>0\). Fallið \(f\) uppfyllir afleiðujöfnu
\begin{equation*}
\begin{split}f'(x)+a x f(x)=0\,.\end{split}
\end{equation*}
Ef við reiknum Fourier-myndina af þessari jöfnu og notum reiknireglur 9, þá fáum við
\begin{equation*}
\begin{split}0= i k \widehat{f}(k)+i a {d\over dk}\widehat{f}(k)\,.\end{split}
\end{equation*}
Þetta er bara fyrsta stigs afleiðujafna fyrir Fourier-mynd fallsins \(f\), og lausnin er
\begin{equation*}
\begin{split}\widehat{f}(k)= C e^{-{k^2\over 2 a}}\,, \qquad C\in \mathbb{R}\,.\end{split}
\end{equation*}
Til þess að finna fastann \(C\), getum notað
\begin{equation*}
\begin{split}C=\widehat{f}(0)=\int_{-\infty}^{\infty} f(x) dx= \int_{-\infty}^{\infty} e^{-a x^2/2} dx= \sqrt{{2\pi}\over a}\,.\end{split}
\end{equation*}
Að lokum, fáum við
\begin{equation*}
\begin{split}\mathcal{F}(e^{-a x^2/2})(k) = \sqrt{{2\pi}\over a}e^{-{k^2\over 2 a}}\,.\end{split}
\end{equation*}

\subsection{Eiginleikar Fourier-myndar}
\label{\detokenize{Kafli04:eiginleikar-fourier-myndar}}
Nú viljum við skoða eiginleika Fourier-myndar. Gerum ráð fyrir að fall \(f\) sé t.d. samfellt eða diffranlegt og svo framvegis, hvaða eiginleika hefur Fourier-mynd fallsins \(f\)?

\sphinxstylestrong{Setning (Riemann-Lebesgue setning)}

Ef \(f\in L^1({{\mathbb R}})\), þá er \({{\cal F}}f\in C({{\mathbb R}})\) og
\begin{equation*}
\begin{split}\lim_{\xi\to \pm \infty}{{\cal F}}f(\xi)=0.\end{split}
\end{equation*}
Ef við táknum mengi falla sem eru samfelld og stefna á núll þegar breytan stefnir á óendanlegt með \(C_0({{\mathbb R}})=\{F\in C({{\mathbb R}})\,;\, \lim_{|\xi|\to +\infty}F(\xi)=0\}\), þá þýðir setningin að Fourier-ummyndun \(\mathcal F\) varpar rúminu \(L^1(\mathbb R)\) í \(C_0(\mathbb R)\).

\sphinxstylestrong{Setning}

Gerum ráð fyrir að fall \(f\in L^1(\mathbb R)\) og að \(f\) sé takmarkað. Gerum ráð fyrir að Fourier-mynd \({{\cal F}}f\) fallsins \(f\) sé jákvæð fyrir öll \(k\), þ.e. \({{\cal F}}f(k)\ge 0\). Þá er \({{\cal F}}\in L^1(\mathbb R)\).

Athugum að ef fall \(f\in L^1(\mathbb R)\) er takmarkað (þ.e. \(|f|\le M\)), þá er \(f\in L^2(\mathbb R)\) (af því að \(|f|^2\le M|f|\)).


\subsection{Plancerel-jafnan}
\label{\detokenize{Kafli04:plancerel-jafnan}}
Til þess að einfalda rithátt, táknum við hér Fourier-mynd falls \(f\) með \(\widehat f\).

Gerum ráð fyrir að \(f\in L^1(\mathbb R)\) og að \(f\) sé takmarkað.
Þá gildir
\begin{equation*}
\begin{split}\int_{-\infty}^\infty |f(x)|^2 dx= {1\over 2\pi} \int_{-\infty}^\infty |\widehat{f}(k)|^2 dk\,.\end{split}
\end{equation*}
Þetta er Plancherel-jafnan. Hún er alhæfing af Parseval-jöfnu fyrir Fourier-ummyndunina.


\section{Andhverfuformúla Fouriers. Afleiðujöfnur}
\label{\detokenize{Kafli04:andhverfuformula-fouriers-afleiujofnur}}

\subsection{Andhverfuformúla Fouriers}
\label{\detokenize{Kafli04:andhverfuformula-fouriers}}
Við viljum nú finna fall \(f\) ef við gerum ráð fyrir að Fourier-myndin \(\mathcal{F}f\) sé gefin. Við munum skoða og reikna út Fourier-myndina af Fourier-mynd falls \(f\), þ.e.a.s. \(\mathcal{F}(\mathcal{F}f)\). Hugmyndin að baki er að Fourier-myndin af Fourier-mynd fallsins \(f\) gefur fallið \(f\). Þetta er þó ekki svo einfalt. Fyrsta vandamál er að jafnvel þótt \(f\in L^1(\mathbb R)\) þýðir það ekki nauðsynlega að \(\mathcal{F}f\) sé í \(L^1(\mathbb R)\) (svo Fourier-mynd hennar er ekki endilega vel skilgreind).

Ef við gerum ráð fyrir að bæði föllin \(f\) og \(\mathcal{F}f\) séu í \(L^1(\mathbb R)\) og séu samfelld, þá er
\begin{equation*}
\begin{split}({{\cal F}}{{\cal F}}f)(x) &=
\int_{-\infty}^{+\infty}e^{-ix\xi} \widehat{f}(\xi) d\xi =
\int_{-\infty}^{+\infty}e^{-ix\xi}
\bigg(\int_{-\infty}^{+\infty}e^{-iy\xi}f(y) \, dy\bigg)\, d\xi\\
&=\int_{-\infty}^{+\infty}
\bigg(\int_{-\infty}^{+\infty}e^{-i(x+y)\xi}f(y) \, dy\bigg)\, d\xi\,.\end{split}
\end{equation*}
Vandamálið nú er að við getum ekki skipt á röð heildanna, við getum ekki heildað fyrst yfir \(\xi\) og svo yfir \(y\) af því að heildið \(\int_{-\infty}^{+\infty} e^{-i(x+y)\xi} d\xi\) er ekki samleitið. Til að leysa málið, stingum við falli \(e^{-\varepsilon|\xi|}\) inn í heildið og tökum síðan markgildi \(\varepsilon\to 0+\). Nú getum við reiknað út heildið að ofan og við fáum
\begin{equation*}
\begin{split}({{\cal F}}{{\cal F}}f)(x)&=\lim_{\varepsilon\to 0}
\int_{-\infty}^{+\infty}e^{-\varepsilon|\xi|}
\bigg(\int_{-\infty}^{+\infty}e^{-it\xi}f(t-x) \, dt\bigg)\, d\xi\\
&=\lim_{\varepsilon\to 0} \int_{-\infty}^{+\infty}f(t-x)
{{\cal F}}\{e^{-\varepsilon|\xi|}\}(t) \, dt\\
&=\lim_{\varepsilon\to 0} \int_{-\infty}^{+\infty}f(t-x)
{{\cal F}}\{e^{-|\xi|}\}(t/\varepsilon) \varepsilon^{-1}\, dt\\
&=\lim_{\varepsilon\to 0} \int_{-\infty}^{+\infty}f(\varepsilon t-x)
{{\cal F}}\{e^{-|\xi|}\}(t) \, dt\\
&=f(-x)\int_{-\infty}^{+\infty}\dfrac 2{1+t^2} \, dt= 2\pi f(-x).\end{split}
\end{equation*}
Að lokum getum við tekið þetta saman í eftirfarandi setningu

\sphinxstylestrong{Setning (Andhverfuformúla Fouriers)}

Gerum ráð fyrir að fall \(f\in L^1(\mathbb R)\cap \mathcal{C}(\mathbb R)\) og \(\widehat{f}\in L^1(\mathbb R)\cap \mathcal{C}(\mathbb R)\). Þá gildir
\begin{equation*}
\begin{split}f(x) =\dfrac 1{2\pi}\int_{-\infty}^{+\infty}e^{ix\xi}\widehat f(\xi) \,
d\xi = \dfrac 1{2\pi}({{\cal F}}{{\cal F}}f)(-x), \qquad x\in {{\mathbb  R}}.\end{split}
\end{equation*}
Setningin segir okkur að fallið \(f\) sé samfelld samantekt (superposition á ensku) af veldisvísisföllum \(e^{ix\xi}\). Hún alhæfir framsetningu á lotubundnum föllum með Fourier-röðum til falla sem eru ekki lotubundin.

\sphinxstylestrong{Fylgisetning}

Ef \(\widehat{f}=\widehat{g}\), þá er \(f=g\).

\sphinxstylestrong{Sýnidæmi}

Andhverfuformúlan getur verið mjög gagnleg til þess að reikna Fourier-mynd. Við sjáum þetta með dæmi.
Ef við viljum reikna Fourier-mynd falls \(f(x)={\sin a x\over x}\), getum við notað andhverfuformúlu Fouriers og sýnidæmi {\hyperref[\detokenize{Kafli04:example1}]{\sphinxcrossref{\DUrole{std,std-ref}{4.1.3}}}}, það er
\begin{equation*}
\begin{split}\mathcal{F}\left({\sin a x\over x}\right) = \begin{cases} \pi\,, \qquad &|\xi|<a\\ 0\,, \qquad & \text{annars} \end{cases}\,.\end{split}
\end{equation*}
Ef við viljum reikna Fourier-mynd fallsins \(f\) beint út frá skilgreiningu þess, er það erfitt!


\subsection{Földun og Fourier-ummyndun}
\label{\detokenize{Kafli04:foldun-og-fourier-ummyndun}}
\sphinxstylestrong{Skilgreining}

Látum \(f\) og \(g\) vera tvö föll á \(\mathbb  R\). Við skilgreinumn \sphinxstylestrong{földun} þeirra með
\begin{equation*}
\begin{split}f\ast g(x)= \int_{-\infty}^{+\infty}f(x-t)g(t) \, dt,\end{split}
\end{equation*}
fyrir öll \(x\in {{\mathbb  R}}\) þannig að heildið sé til.

\sphinxstylestrong{Eiginleikar}
\begin{enumerate}
\def\theenumi{\arabic{enumi}}
\def\labelenumi{\theenumi .}
\makeatletter\def\p@enumii{\p@enumi \theenumi .}\makeatother
\item {} 
Gerum ráð fyrir að heildið að ofan sé til, þá er

\end{enumerate}
\begin{equation*}
\begin{split}f\ast g(x)= \int_{-\infty}^{+\infty}f(x-t)g(t) \, dt = \int_{-\infty}^{\infty} f(s)g(x-s)ds = g \ast f(x)\,,\end{split}
\end{equation*}
þar sem við höfum notað \(s=x-t\).
\begin{enumerate}
\def\theenumi{\arabic{enumi}}
\def\labelenumi{\theenumi .}
\makeatletter\def\p@enumii{\p@enumi \theenumi .}\makeatother
\setcounter{enumi}{1}
\item {} 
Ef \(f\in L^1(\mathbb R)\) og \(g\) er takmarkað, þá er földun þeirra skilgreind á \({{\mathbb  R}}\).

\item {} 
Ef \(f\in L^1(\mathbb R)\) og líka \(g\in L^1(\mathbb R)\), þá er földunin vel skilgreind, og ennfremur gildir að \(f\ast g\) er í \(L^1(\mathbb R)\).

\item {} 
Földunin uppfyllir sömu reglur og venjulegt margfeldi uppfyllir:

\end{enumerate}
\begin{equation*}
\begin{split}& f\ast (\alpha g +\beta h)= \alpha (f\ast g)+\beta (f\ast h)\,, \quad \forall \alpha, \beta \in\mathbb{R}\,.
\\
& f\ast g = g\ast f\,,
\\
& f\ast (g\ast h)= (f\ast g)\ast h\,,\end{split}
\end{equation*}
þar sem \(f, g, h\) eru föll á \(\mathbb  R\), þ.a. földun þeirra sé vel skilgreind.
\begin{enumerate}
\def\theenumi{\arabic{enumi}}
\def\labelenumi{\theenumi .}
\makeatletter\def\p@enumii{\p@enumi \theenumi .}\makeatother
\setcounter{enumi}{4}
\item {} 
Gerum ráð fyrir að fall \(f\) sé diffranlegt og faldanir \(f\ast g\) og \(f'\ast g\) séu vel skilgreindar. Þá er \(f\ast g\) diffranlegt og \((f\ast g)'=f'\ast g\). Ef \(g\) er líka diffranlegt, þá gildir \((f\ast g)'= f\ast g'\).

\end{enumerate}

Við getum alhæft niðurstöðuna að ofan ef til dæmis fallið \(f\) er \(m\)-sinnum diffranlegt og \(f, f', \dots f^{(m)}\) eru takmörkuð, þá er \(f\ast g \in\mathcal{C}^m(\mathbb{R})\) og
\begin{equation*}
\begin{split}(f\ast g)^{(k)}(x)= (f^{(k)}\ast g)(x)\,, \qquad x\in\mathbb{R}\, \quad k=0, \dots, m.\end{split}
\end{equation*}
\sphinxstylestrong{Setning}

Frá eiginleika 3, fáum við eftirfarandi setningu
\begin{quote}

Ef \(f\in L^1(\mathbb R)\) og líka \(g\in L^1(\mathbb R)\), þá er földunin \(f\ast g\) í \(L^1(\mathbb R)\) og
\end{quote}
\begin{equation*}
\begin{split}{{\cal F}}\{f\ast g\}(\xi)={{\cal F}}f(\xi){{\cal F}}g(\xi), \qquad \xi\in {{\mathbb  R}}.\end{split}
\end{equation*}

\subsection{Afleiðujöfnur og Fourier-ummyndun}
\label{\detokenize{Kafli04:afleiujofnur-og-fourier-ummyndun}}
Við byrjum á að skoða afleiðujöfnu með fasta stuðla
\begin{equation*}
\begin{split}P(D)u=(a_mD^ m+\cdots+a_1D+a_0)u=f(x).\end{split}
\end{equation*}
Til þess að finna lausn á jöfnunni getum við notað Fourier-ummyndun, ef t.d. \(f\in L^1(\mathbb R)\). Munið eftir reiknireglu 8, ef við gerum ráð fyrir að \(u\) og afleiður þess séu í \(L^1(\mathbb R)\). Þá fáum við eftirfarandi niðurstöðu

\sphinxstylestrong{Setning}
\begin{quote}

Gerum ráð fyrir að \(f\in L^1(\mathbb R)\) og \(\widehat{f}\in L^1(\mathbb R)\).
Gerum ráð fyrir að \(P(i\xi)\neq 0\). Þá hefur afleiðujafnan (ref) lausn \(u\in L^1(\mathbb R)\cap \mathcal{C}^m (\mathbb R)\) sem gefin er með formúlunni
\end{quote}
\begin{equation*}
\begin{split}u(x)=\dfrac 1{2\pi}\int_{-\infty}^ {+\infty}
e^{ix\xi} \dfrac{\widehat f(\xi)}{P(i\xi)}\, d\xi, \qquad x\in {{\mathbb  R}}.\end{split}
\end{equation*}
Við sjáum að fallið \(u\) sem skilgreint er að ofan uppfyllir jöfuna
\begin{equation*}
\begin{split}P(D)u(x)&=\dfrac 1{2\pi}\int_{-\infty}^{+\infty}P(D_x)e^{ix\xi}
\dfrac{\widehat f(\xi)}{P(i\xi)}\, d\xi=
\dfrac 1{2\pi}\int_{-\infty}^{+\infty}P(i\xi)e^{ix\xi}
\dfrac{\widehat f(\xi)}{P(i\xi)}\, d\xi\\
&= \dfrac 1{2\pi}\int_{-\infty}^{+\infty}e^{ix\xi}
\widehat f(\xi)\, d\xi=f(x).\end{split}
\end{equation*}
\sphinxstylestrong{Afleiðujöfnur, Fourier-ummyndun og földun}

Gerum ráð fyrir að \(P(i\xi)\neq 0\) fyrir öll \(\xi\in\mathbb R\).
Ef við táknum andhverfu Fourier-mynd falls \({1\over P(i\xi)}\) (athugum að \({1\over P(i\xi)} \in L^1(\mathbb{R})\)) með
\begin{equation*}
\begin{split}E(x)= {1\over 2\pi} \int_{-\infty}^{\infty} {e^{i x \xi}\over P(i\xi)} d\xi\,, \qquad x\in\mathbb R\,,\end{split}
\end{equation*}
þá fæst
\begin{equation*}
\begin{split}E\ast f(x) &= \int_{-\infty}^{\infty} E(x-t) f(t)dt= \int_{-\infty}^{\infty}\left({1\over 2\pi} \int_{-\infty}^{\infty} {e^{i (x-t) \xi}\over P(i\xi)} d\xi\right)f(t) dt=\\
&={1\over 2\pi}\int_{-\infty}^{\infty}{e^{i x \xi}\over P(i\xi)}\left(\int_{-\infty}^\infty f(t) e^{-i \xi t} dt\right)d\xi= {1\over 2\pi}\int_{-\infty}^{\infty}{e^{ix\xi}\over P(i \xi)}\widehat f(\xi)\, d\xi =u(x).\end{split}
\end{equation*}
\sphinxstylestrong{Setning}

Gerum ráð fyrir að \(P\) sé margliða af stigi \(m\) með ólikar núllstöðvar \(\lambda_1, \dots, \lambda_{\ell}\) með margfeldni \(m_1, \dots, m_{\ell}\), að \(P(i\xi)\) hafi enga núllstöð á \(\mathbb{R}\), að \(Q\) sé margliða af stigi \(\le m-1\) og að stofnbrotaliðun á ræða fallinu \(Q/P\) sé gefin með
\begin{equation*}
\begin{split}\dfrac {Q(\zeta)}{P({\zeta})} =\sum\limits_{k=1}^\ell
\sum\limits_{j=1}^{m_k} \dfrac{A_{jk}}{({\zeta}-{\lambda}_k)^j}.\end{split}
\end{equation*}
Þá er andhverfa Fourier-mynd fallsins \({\xi}\mapsto Q(i\xi)/P(i{\xi})\) gefin með formúlunni
\begin{equation*}
\begin{split}f(x)&=
\sum_{\substack{{{\operatorname{Re\, }}}{\lambda}_k<0}}
\sum\limits_{j=1}^{m_k} A_{jk}
\tfrac 1{(j-1)!}H(x)x^{j-1}e^{{\lambda}_kx}\\
&-\sum_{\substack{{{\operatorname{Re\, }}}{\lambda}_k>0}}
\sum\limits_{j=1}^{m_k} A_{jk} \tfrac 1{(j-1)!} H(-x)x^{j-1}e^{{\lambda}_kx},
\qquad x\neq 0.\nonumber\end{split}
\end{equation*}
\sphinxstylestrong{Sýnidæmi}

Skoðum jöfnu
\begin{equation*}
\begin{split}-u{{^{\prime\prime}}}+\omega^ 2u=e^{-|x|}=f(x), \qquad \omega^ 2 \neq 1,
\qquad x\in {{\mathbb  R}}.\end{split}
\end{equation*}
Við sjáum að \(P(X)=-X^2+\omega^2\), og \(P(i\xi)=\xi^2+\omega^2\). Fourier-mynd fallsins \(e^{-|x|}=f(x)\) er \(\widehat f(\xi)={2 \over 1+\xi^ 2}\). Tökum Fourier-mynd jöfnunnar, þá fáum við
\begin{equation*}
\begin{split}\xi^ 2 \widehat u(\xi)+\omega^ 2 \widehat u(\xi) = \dfrac
2{1+\xi^ 2}, \qquad \xi\in {{\mathbb  R}}.\end{split}
\end{equation*}
Þá er
\begin{equation*}
\begin{split}\widehat u(\xi)=
\dfrac 2{(\omega^ 2+\xi^ 2)(1+\xi^ 2)} =\dfrac 1{1-\omega^ 2}\bigg(
\dfrac 1\omega {{\cal F}}\{e^{-\omega|x|}\}({\xi})-{{\cal F}}\{e^{-|x|}\}({\xi})
\bigg).\end{split}
\end{equation*}
Nú getum við notað andhverfuformúlu og þá fæst loks að
\begin{equation*}
\begin{split}u(x)= \dfrac 1{1-\omega^ 2}\bigg( \dfrac 1\omega e^{-\omega|x|} - e^{-|x|} \bigg).\end{split}
\end{equation*}

\section{Úrlausn á hlutafleiðujöfnum með Fourier-ummyndun}
\label{\detokenize{Kafli04:urlausn-a-hlutafleiujofnum-me-fourier-ummyndun}}

\subsection{Einvíða bylgjujafnan og d’Alembert-formúla}
\label{\detokenize{Kafli04:einvia-bylgjujafnan-og-d-alembert-formula}}
Við skoðum einvíðu bylgjujöfnuna
\begin{equation*}
\begin{split}\dfrac{\partial^2u}{\partial t^2}
-c^2\dfrac{\partial^2u}{\partial x^2}=0,\end{split}
\end{equation*}
þar sem fallið \(u(x,t)\) er skilgreint fyrir öll \(x \in \mathbb{R}\) og \(t \in \mathbb{R}\).
Leitum að slíkri lausn.

Skiptum um hnit með \(x = {\xi+\eta\over 2}\) og \(t = {\xi-\eta\over 2 c}\) og skrifum bylgjujöfnuna sem
\begin{equation*}
\begin{split}\partial^2_{t} u(t,x)- c^2 \partial^2_x u (t,x)= -4 \,c^2 \partial^2_{\xi\eta} v(\eta,\xi)\, \qquad v(\eta,\xi) = u(x(\eta,\xi),t(\eta,\xi)).\end{split}
\end{equation*}
Athugum að við notum að \(\partial^2_{t,x}=\partial^2_{x,t}\), sem gildir til dæmis ef lausnin er tvisvar sinnum samfellt deildanleg.

Almenn lausn á jöfnunni að ofan er \(u(\eta, \xi)=f(\xi)+g(\eta)\), þar sem föllin \(f(\xi)\), \(g(\eta)\) eru ótiltekin. Þá er
\begin{equation*}
\begin{split}u(x,t)=f(\xi(x,t))+g(\eta(x,t))= f(x+ct)+g(x-ct)\,.\end{split}
\end{equation*}
Þá fæst niðurstaðan:
\begin{description}
\item[{\sphinxstylestrong{Setning}}] \leavevmode
Sérhver lausn \(u\in C^2({{\mathbb R}}^2)\) á bylgjujöfnunni er af gerðinni \(u(x,t)=f(x+ct)+g(x-ct)\), þar sem \(f,g\in C^2({{\mathbb R}})\).
Ef \(u(x,t)=f_1(x+ct)+g_1(x-ct)\) er önnur slík framsetning á lausninni, þá er til fasti \(A\) þannig að \(f_1(x)=f(x)+A\) og \(g_1(x)=g(x)-A\).

\end{description}

Fyrir gefið \(t_0 > 0\), er graf fallsins \(g(x − ct_0)\) næstum því eins og graf fallsins \(g(x)\), eini munurinn er að grafið \(g(x − ct_0)\) er hliðrað um \(c t_0\) til hægri. Við túlkum því fallið \(g(x − ct)\) sem bylgju sem hreyfist til hægri með hraða \(c\) og köllum það \sphinxstyleemphasis{framáttarbylgju}. Á svipaðan hátt er graf fallsins \(f(x+ct_0)\) hliðrað um \(c t_0\) til vinstri, fallið \(f(x + ct)\) táknar bylgju sem hreyfist til vinstri með hraða \(c\) og kallast það \sphinxstyleemphasis{bakáttarbylgja}.

Við skoðum nú bylgjujöfnuna með upphafsskilyrðum, það er

\phantomsection\label{\detokenize{Kafli04:upphafbylgja}}\begin{equation*}
\begin{split}\begin{cases}
\dfrac{\partial^2u}{\partial t^2}
-c^2\dfrac{\partial^2u}{\partial x^2}=0, &x\in {{\mathbb  R}},\ t>0, \\
u(x,0)=\varphi(x), \quad \partial_tu(x,0)=\psi(x), &x\in {{\mathbb  R}}.
\end{cases}\end{split}
\end{equation*}
Við viljum finna lausn sem er í \(C^2({{\mathbb R}}^2)\), sem gefin er í setningunni að ofan. Þá þurfum við tengja \(f(x+ct), g(x-ct)\) við \(\varphi(x), \psi(x)\).
Niðurstaðan er

\sphinxstylestrong{Setning}
\begin{quote}

Upphafsgildisverkefnið {\hyperref[\detokenize{Kafli04:upphafbylgja}]{\sphinxcrossref{\DUrole{std,std-ref}{að ofan}}}} hefur ótvírætt ákvarðaða lausn sem gefin er með formúlunni
\end{quote}
\begin{equation*}
\begin{split}u(x,t)=\dfrac 12\big(\varphi(x+ct)+\varphi(x-ct)\big)
+\dfrac 1{2c}\int_{x-ct}^{x+ct}\psi({\xi})\, d{\xi}.\end{split}
\end{equation*}
Formúlan kallst d’Alembert-formúlan. Hún gefur almenna lausn í \(C^2({{\mathbb R}}^2)\) á upphafsgildisverkefninu.

\begin{DUlineblock}{0em}
\item[] 
\item[] 
\end{DUlineblock}


\begin{center}
\includegraphics[width=4cm,keepaspectratio=true]{polarggb.png}
\end{center}


\begin{DUlineblock}{0em}
\item[] 
\item[] 
\end{DUlineblock}


\subsection{Bylgjujafnan, Fourier-ummyndun og földun}
\label{\detokenize{Kafli04:bylgjujafnan-fourier-ummyndun-og-foldun}}
Við getum skrifað d’Alembert-formúluna sem földunarheildi: Skilgreinum fall \(E_t\) sem
\begin{equation*}
\begin{split}E_t(x)=E(x,t)= \begin{cases} 1/2c, &|x|\leq ct,\\ 0,
&|x|>ct.\end{cases}\end{split}
\end{equation*}
þá er
\begin{equation*}
\begin{split}&& \dfrac 1{2c}\int_{x-ct}^{x+ct}\psi(y)\, dy
=\int_{-\infty}^{+\infty}E_t(x-y)\psi(y)\, dy = \big(E_t\ast \psi\big)(x),
\\
&& \dfrac 12\big(\varphi(x+ct)+\varphi(x-ct)\big)
=\dfrac{\partial}{\partial t}\bigg(
\dfrac 1{2c}\int_{x-ct}^{x+ct}\varphi(y)\, dy
\bigg) =\dfrac{\partial}{\partial t} E_t\ast \varphi(x),\end{split}
\end{equation*}
og lausnin verður
\begin{equation*}
\begin{split}u(x,t)=\dfrac{\partial}{\partial t}\big( E_t\ast \varphi\big)(x)+
E_t\ast \psi(x).\end{split}
\end{equation*}

\bigskip\hrule\bigskip


Við viljum nú leiða þessa formúlu út með því að nota Fourier-ummyndun. Tökum Fourier-mynd af öllum liðum sem koma fyrir í upphafsgildisverkefninu. Fyrst þurfum við að finna Fourier-myndir \(\partial_t u\) og \(\partial_x u\).
\begin{equation*}
\begin{split}{{\cal F}}\{\partial_t^2 u\}(\xi,t)
=\int\limits_{-\infty}^{+\infty}e^{-ix\xi}\partial_t^2 u(x,t)\, dx
=\partial_t^2\int\limits_{-\infty}^{+\infty}e^{-ix\xi}u(x,t)\, dx
=\partial_t^2\widehat u(\xi,t).\end{split}
\end{equation*}\begin{equation*}
\begin{split}{{\cal F}}\{{\partial}_x^2u\}({\xi},t)
=\int\limits_{-\infty}^{+\infty}e^{-ix\xi}\partial_x^2u(x,t)\, dx
=(i{\xi})^2\widehat u({\xi},t)=-{\xi}^2\widehat u({\xi},t),\end{split}
\end{equation*}
þar sem við höfum notað reikniregluna 8 í {\hyperref[\detokenize{Kafli04:rulesft}]{\sphinxcrossref{\DUrole{std,std-ref}{4.1.4}}}}.

Að lokum verður upphafsgildisverkefnið
\begin{equation*}
\begin{split}\begin{cases}
{\partial}_t^2\widehat u({\xi},t)+c^2{\xi}^2
\widehat u({\xi},t)=0, &{\xi}\in {{\mathbb  R}},\ t>0,\\
\widehat u({\xi},0)=\widehat\varphi({\xi}), \quad {\partial}_t\widehat
u({\xi},t)=\widehat{\psi}({\xi}), &{\xi}\in {{\mathbb  R}}.
\end{cases}\end{split}
\end{equation*}
Athugum að \({\partial}_t^2\widehat u({\xi},t)+c^2{\xi}^2\widehat u({\xi},t)=0\) er annars stigs venjuleg afleiðujafna í \(t\), og \(\xi\) er bara fasti. Þá er lausnin
\begin{equation*}
\begin{split}\begin{aligned}
\widehat u({\xi},t)&=
\cos(ct{\xi})\widehat\varphi({\xi})
+\dfrac{\sin(ct{\xi})}{c{\xi}}\widehat{\psi}({\xi})
\end{aligned}\end{split}
\end{equation*}
En, ef við reiknum Fourier-myndin fallsins \(E_t\) sem við skilgreindum að ofan, þá er
\begin{equation*}
\begin{split}\begin{aligned}
\widehat E_t(\xi)&=\int_{-\infty}^{+\infty}e^{-ix\xi}E_t(x)\, dx
=\dfrac 1{2c}\int_{-ct}^{ct}e^{-ix\xi}\, dx\\
 &=\dfrac 1{2c}\bigg[\dfrac{e^{-ix\xi}}{-i\xi}\bigg]_{-ct}^{ct}
=\dfrac{\sin(ct\xi)}{c\xi}.\end{aligned}\end{split}
\end{equation*}
Það þýðir að við getum umritað lausnina sem
\begin{equation*}
\begin{split}\widehat u({\xi},t)=\dfrac{\partial}{\partial t}\widehat E_t({\xi})\widehat\varphi({\xi})+\widehat E_t({\xi})\widehat {\psi}({\xi}).\end{split}
\end{equation*}
og niðurstaðan fyrir \(u\) fæst svo með því að taka andhvefa Fourier-mynd og nota földunarreglur.

\sphinxstylestrong{Hliðraða bylgjujafnan}

Við skoðum
\begin{equation*}
\begin{split}\begin{cases}
\dfrac{\partial^2u}{\partial t^2}
-c^2\dfrac{\partial^2u}{\partial x^2}=f(x,t), &x\in {{\mathbb  R}}, \ t>0,\\
u(x,0)=\partial_tu(x,0)=0, &x\in {{\mathbb  R}},
\end{cases}\end{split}
\end{equation*}
Leitum að sérlausn á þessu verkefni. Eins og áður notum við Fourier-ummyndun og fáum
\begin{equation*}
\begin{split}\begin{cases}
{\partial}_t^2\widehat u({\xi},t)+c^2{\xi}^2
\widehat u({\xi},t)=\widehat f({\xi},t), &{\xi}\in {{\mathbb  R}},\ t>0,\\
\widehat u({\xi},0)={\partial}_t\widehat u({\xi},0)=0, &{\xi}\in {{\mathbb  R}}.
\end{cases}\end{split}
\end{equation*}
Green-fall afleiðuvirkjans \(D_t^2+c^2{\xi}^2\) er \(G_{\xi}(t,{\tau})=g({\xi},t-{\tau})=\sin(c(t-\tau){\xi})/c{\xi}\).
Athugum að \(g(ξ,t)=\widehat E_t({\xi})=\widehat E({\xi},t).\)

Þá er Fourier-mynd lausnarinnar á jöfnunni
\begin{equation*}
\begin{split}\widehat u({\xi},t)
=\int_0^t  g({\xi},t-\tau)\widehat f({\xi},\tau)\, d{\tau}
=\int_0^t\widehat E({\xi},t-\tau)\widehat f({\xi},\tau)\, d{\tau}.\end{split}
\end{equation*}
Til þess að finna \(u\) þurfum við að nota andhverfuformúlu Fouriers, þá er
\begin{equation*}
\begin{split}\begin{aligned}
u(x,t)&
=\dfrac 1{2{\pi}}\int_{-\infty}^{+\infty}e^{ix{\xi}}
\bigg(\int_0^t \widehat E({\xi},t-\tau)\widehat f({\xi},\tau)\, d{\tau}\bigg)
\, d{\xi}\label{15.4.3}\\
&=\int_0^t \bigg(\dfrac 1{2{\pi}}\int_{-\infty}^{+\infty}e^{ix{\xi}}
\widehat E({\xi},t-\tau)\widehat f({\xi},\tau)\, d{\xi}\bigg)
\, d{\tau}\nonumber\\
&=\int_0^t \int_{-\infty}^{+\infty}
E(x-y,t-\tau)f(y,\tau)\, dy\, d{\tau}.\nonumber\end{aligned}\end{split}
\end{equation*}
Ef við viljum skrifa þetta sem földunarheildi þurfum við að framlengja föllin fyrir öll \(t\). Við höfum \(E(x,t) = 0\) ef \(t < 0\) og ef við skilgreinum \(f(x,t) = 0\) fyrir \(t < 0\), þá fæst
\begin{equation*}
\begin{split}u(x,t) = E\ast f(x,t), \qquad x\in\mathbb{R}, t>0\end{split}
\end{equation*}
þar sem \(\ast\) stendur hér fyrir földun falla af tveimur breytistærðum sem er skilgreind með sambærilegum hætti og áður.

Þetta má einnig rita sem
\begin{equation*}
\begin{split}\begin{aligned}
u(x,t)
&=\int_0^t \int_{-\infty}^{+\infty} E(x-y,t-\tau)f(y,\tau)\, dy\,
d{\tau}\\
&=\dfrac 1{2c} \int_0^t \int_{x-c(t-\tau)}^{x+c(t-\tau)}
f(y,{\tau})\, dyd{\tau}\nonumber\\
&=\dfrac 1{2c}\iint\limits_{T(x,t)}f(y,{\tau})\, dyd{\tau},\nonumber\end{aligned}\end{split}
\end{equation*}
þar sem \(T(x,t)\) er þríhyrningurinn í \((y,\tau)\)-planinu með hornpunktana \((x, t), (x − ct, 0)\) og \((x + ct, 0)\). Þríhyrningurinn kallast {\color{red}\bfseries{}*}ákvörðunarsvæði punktins \((x,t)\).


\subsection{Varmaleiðnijafnan, Fourier-ummyndun og földun}
\label{\detokenize{Kafli04:varmaleinijafnan-fourier-ummyndun-og-foldun}}
Við lítum nú á varmaleiðnijöfnuna með upphafsskilyrðum

\phantomsection\label{\detokenize{Kafli04:upphafvarmi}}\begin{equation*}
\begin{split}\begin{cases}
\dfrac{{\partial}u}{\partial t}
-{\kappa}\dfrac{\partial^2u}{\partial x^2}=0, &x\in {{\mathbb  R}}, \ t>0,\\
u(x,0)={\varphi}(x), &x\in {{\mathbb  R}},
\end{cases}\end{split}
\end{equation*}
Eins og áður viljum við finna lausn með því að nota Fourier-ummmyndun. Tökum Fourier-mynd af öllum liðunum og þá fæst að Fourier-mynd fallsins \(u\) þarf að uppfylla
\begin{equation*}
\begin{split}\begin{cases}
\partial_t\widehat u({\xi},t)
+{\kappa}{\xi}^2\widehat u({\xi},t)=0, &{\xi}\in {{\mathbb  R}}, \ t>0,\\
\widehat u({\xi},0)=\widehat {\varphi}({\xi}), &{\xi}\in {{\mathbb  R}}.
\end{cases}\end{split}
\end{equation*}
Nú verður jafnan \(\partial_t\widehat u({\xi},t)+{\kappa}{\xi}^2\widehat u({\xi},t)=0\) einfaldlega fyrsta stigs afeiðujafna í \(t\), og lausn hennar er \(\widehat u({\xi},t)=e^{-{\kappa}t{\xi}^2}\widehat {\varphi}({\xi})\).

Við viljum finna lausn sem földunheildi. Athugum að
\begin{equation*}
\begin{split}e^{-{\kappa}t{\xi}^2} = \mathcal{F}(E)(\xi)\,
\qquad E(x,t)=E_t(x)=\begin{cases} \dfrac 1{\sqrt{4{\pi}{\kappa}t}}e^{-x^2/4{\kappa}t},
&x\in {{\mathbb  R}}, \ t>0,\\
0, &x\in {{\mathbb  R}}, \ t\leq 0.\end{cases}\end{split}
\end{equation*}
Þá er \(\widehat u({\xi},t)=e^{-{\kappa}t{\xi}^2}\widehat {\varphi}({\xi})= \widehat{E_t} ({\xi})\widehat {\varphi}({\xi})\).

Fallið \(E\) kallast \sphinxstyleemphasis{hitakjarni} eða \sphinxstyleemphasis{varmaleiðnikjarni}.

Til þess að skilja lausnina er gott að skoða eiginleika hitakjarnans \(E\):
\begin{equation*}
\begin{split}\begin{aligned}
1. & \lim\limits_{t\to 0+} E_t(x) =
\begin{cases} +{\infty}, &x=0,\\
0, &x\neq 0,\end{cases}
\\
2. &
\int_{-{\infty}}^{+{\infty}} E_t(x)\, dx
=\int_{-{\infty}}^{+{\infty}}
\dfrac 1{\sqrt{\pi}}e^{-y^2}\, dy = 1\,,\\
3. & \,({\partial}_t-{\kappa}{\partial}_x^2)E(x,t)=0, \qquad t>0\,.
\end{aligned}\end{split}
\end{equation*}
Þá getum við notað andhverfuformúlu Fouriers og við fáum:
\begin{equation*}
\begin{split}u(x,t)=E_t\ast {\varphi}(x)=\int_{-{\infty}}^{+{\infty}}
\dfrac 1{\sqrt{4{\pi}{\kappa}t}}e^{-(x-y)^2/4{\kappa}t}{\varphi}(y)\,
dy, \qquad t>0.\end{split}
\end{equation*}
Það er ekki erfitt að sjá að lausn  \(u(x,t)=E_t\ast {\varphi}(x)\) uppfyllir {\hyperref[\detokenize{Kafli04:upphafvarmi}]{\sphinxcrossref{\DUrole{std,std-ref}{upphafsgildisverkefnið að ofan}}}} með því að nota eingileika hitakjarnans \(E\):
\begin{equation*}
\begin{split}({\partial}_t-{\kappa}{\partial}_x^2)u(x,t)
=\int_{-{\infty}}^{+{\infty}}
({\partial}_t-{\kappa}{\partial}_x^2)E(x-y,t) {\varphi}(y)\, dy=0.\end{split}
\end{equation*}\begin{equation*}
\begin{split}\begin{aligned}
\lim\limits_{t\to 0+} u(x,t) &=
\lim\limits_{t\to 0+} E_t\ast {\varphi}(x)
\\
&=\int_{-{\infty}}^{+{\infty}}
\dfrac 1{\sqrt{\pi}}e^{-y^2}\lim\limits_{t\to 0+}
{\varphi}(x-\sqrt{4{\kappa}t}y)\, dy\\
&={\varphi}(x)\int_{-{\infty}}^{+{\infty}}
\dfrac 1{\sqrt{\pi}}e^{-y^2}\, dy ={\varphi}(x).\end{aligned}\end{split}
\end{equation*}
Þá skiljum við eftirfarandi setningu

\sphinxstylestrong{Setning}
\begin{quote}

Gerum ráð fyrir að \(\varphi\) sé samfellt og takmarkað fall á \({{\mathbb R}}\).  Þá hefur  {\hyperref[\detokenize{Kafli04:upphafvarmi}]{\sphinxcrossref{\DUrole{std,std-ref}{upphafsgildisverkefnið að ofan}}}} lausn \(u\) sem gefin er með formúlunni
\end{quote}
\begin{equation*}
\begin{split}u(x,t)=E_t\ast \varphi(x)=\int_{-\infty}^{+\infty}E_t(x-\xi)\varphi(\xi)\,
d\xi, \qquad x\in {{\mathbb  R}}, \ t>0,\end{split}
\end{equation*}
þar sem hitakjarninn er gefinn með formúlunni
\begin{equation*}
\begin{split}E(x,t)=E_t(x)=H(t) \dfrac
1{\sqrt{4{\pi}{\kappa}t}}e^{-x^2/4{\kappa}t},
\qquad (x,t)\neq (0,0).\end{split}
\end{equation*}
\sphinxstylestrong{Hliðraða varmaleiðnijafnan með óhliðruðum upphafsskilyrðum}

Við lítum nú á hliðruðu varmaleiðnijöfnuna með óhliðruðu upphafsskilyrði, þ.e.

\phantomsection\label{\detokenize{Kafli04:upphafvarmi2}}\begin{equation*}
\begin{split}\begin{cases}
\dfrac{{\partial}u}{\partial t}
-{\kappa}\dfrac{\partial^2u}{\partial x^2}=f(x,t), &x\in {{\mathbb  R}}, \ t>0,\\
u(x,0)=0, &x\in {{\mathbb  R}}.
\end{cases}\end{split}
\end{equation*}
Leitum að sérlausn á henni. Við tökum Fourier-myndina af öllum liðunum
\begin{equation*}
\begin{split}\begin{cases}
\partial_t\widehat u({\xi},t)
+{\kappa}{\xi}^2\widehat u({\xi},t)=\widehat f({\xi},t), &{\xi}\in {{\mathbb  R}}, \ t>0,\\
\widehat u({\xi},0)=0, &{\xi}\in {{\mathbb  R}}.
\end{cases}\end{split}
\end{equation*}
Skoðum jöfnuna að ofan: hún er fyrsta stigs hliðruð afleiðujafna í \(t\). Við getum notað Green-fall, og það er \(G_\xi(t,\tau)=e^{-\kappa(t-\tau)\xi^2}=\widehat E_{t-\tau}(\xi)\).

Eins og áður við skrifum við Fourier-mynd lausnar og eftir það tökum við andhverfu Fourier-myndina. Þá er
\begin{equation*}
\begin{split}\widehat u({\xi},t)=\int_0^te^{-{\kappa}(t-{\tau})x^2}\widehat
f({\xi},t)\, d{\tau} = \int_0^t\widehat E_{t-{\tau}}({\xi})\widehat
f({\xi},t)\, d{\tau}.\end{split}
\end{equation*}\begin{equation*}
\begin{split}\begin{aligned}
u(x,t)&=\dfrac 1{2{\pi}}\int_{-\infty}^{+\infty}e^{ix{\xi}}
\bigg(\int_0^t\widehat E_{t-{\tau}}({\xi})\widehat
f({\xi},{\tau})\, d{\tau} \bigg)\, d{\xi}\\
&=\int_0^t\bigg(
\dfrac 1{2{\pi}}\int_{-\infty}^{+\infty}e^{ix{\xi}}
\widehat E_{t-{\tau}}({\xi})\widehat
f({\xi},{\tau})\, d{\xi}\bigg)\, d{\tau}\\
&=\int_0^t \int_{-\infty}^{+\infty}
E(x-y,t-\tau)f(y,{\tau})\, dy d{\tau} \\
&= E\ast f(x,t) \end{aligned}\end{split}
\end{equation*}
Við fáum eftirfarandi niðurstöðu

\sphinxstylestrong{Setning}
\begin{quote}

Gerum ráð fyrir að \(f\) sé samfellt fall á opna efra hálfplaninu \(\{(x,t); t>0\}\), sé takmarkað á lokuninni \(\{(x,t); t\geq 0\}\) og taki gildið 0 á neðra hálfplaninu \(\{(x,t); t<0\}\). Gerum ráð fyrir að  \({\varphi}\) sé samfellt takmarkað fall á \({{\mathbb R}}\). Þá hefur {\hyperref[\detokenize{Kafli04:upphafvarmi2}]{\sphinxcrossref{\DUrole{std,std-ref}{upphafsgildisverkefnið að ofan}}}} ótvírætt ákvarðaða lausn \(u\), sem gefin er með formúlunni
\end{quote}
\begin{equation*}
\begin{split}u(x,t)=E_t\ast {\varphi}(x)+E\ast f(x,t), \qquad x\in {{\mathbb  R}},\ t>0,\end{split}
\end{equation*}
þar sem \(E\) táknar hitakjarnann, sem skilgreindur er með formúlunni
\begin{equation*}
\begin{split}E(x,t)=E_t(x)=H(t) \dfrac
1{\sqrt{4{\pi}{\kappa}t}}e^{-x^2/4{\kappa}t},
\qquad (x,t)\neq (0,0).\end{split}
\end{equation*}
\sphinxstylestrong{Hliðraða varmaleiðnijafnan með hliðruðu upphafsskilyrði}

Upphafsgildisverkefnið er núna
\begin{equation*}
\begin{split}\begin{cases}
\dfrac{{\partial}u}{\partial t}
-{\kappa}\Delta u=f(x,t), &x\in {{\mathbb  R}}^n, \ t>0,\\
u(x,0)=\lim\limits_{t\to 0+}u(x,t)={\varphi}(x), &x\in {{\mathbb  R}}^n,
\end{cases}\end{split}
\end{equation*}
Við gerum ráð fyrir að \(f\) sé samfellt fall á \(\{(x,t)\in {{\mathbb R}}^n\times {{\mathbb R}}; t\geq 0\}\), \({\varphi}\) sé samfellt fall á \({{\mathbb R}}^n\) og bæði \(f\) og \({\varphi}\) séu takmörkuð.

Hitakjarninn er
\begin{equation*}
\begin{split}E(x,t)=E_t(x)=H(t) \dfrac
1{\big(4{\pi}{\kappa}t\big)^{n/2}}e^{-x^2/4{\kappa}t},
\qquad x\in {{\mathbb  R}}^n,\ (x,t)\neq (0,0),\end{split}
\end{equation*}
og lausnin verður
\begin{equation*}
\begin{split}u(x,t)=E_t\ast {\varphi}(x)+E\ast f(x,t), \qquad x\in {{\mathbb  R}}^n, t>0.\end{split}
\end{equation*}

\section{Fourier-ummyndun og leifareikningur}
\label{\detokenize{Kafli04:fourier-ummyndun-og-leifareikningur}}
Við skoðum hér hvernig við getum reiknað Fourier-myndir og andhverfu þeirra með því að nota leifareikning.
Við byrjum á að setja fram fyrstu niðurstöðu fyrir Fourier-myndir.

Munið að við táknum með \(\mathcal O(X)\) mengi allra fágaðra falla á \(X\).


\subsection{Fourier-myndir reiknaðar með leifareikningi}
\label{\detokenize{Kafli04:fourier-myndir-reiknaar-me-leifareikningi}}
\sphinxstylestrong{Setning}
\begin{quote}

Látum fall \(f\in L^1({{\mathbb R}})\cap {{\cal O}}({{\mathbb C}}\setminus A)\), þar sem \(A\) er endanlegt mengi.
Gerum ráð fyrir að \(\lim\limits_{r\to \infty}\max_{|z|=r}|f(z)|=0\).
Táknum efra hálfplanið með \(H_+=\{z\in {{\mathbb C}}; {{\operatorname{Im\, }}}z>0\}\) og neðra hálfplanið með \(H_-=\{z\in {{\mathbb C}}; {{\operatorname{Im\, }}}z<0\}\).
Þá er
\end{quote}
\begin{equation*}
\begin{split}\widehat f(\xi) =
\begin{cases}  2\pi i\sum_{\alpha\in A\cap H_+}
{{\operatorname{Res}}}(e^{-iz\xi}f(z),\alpha), & \xi < 0,\\
-2\pi i\sum_{\alpha\in A\cap H_-}
{{\operatorname{Res}}}(e^{-iz\xi}f(z),\alpha),  & \xi > 0.
\end{cases}\end{split}
\end{equation*}
\sphinxstylestrong{Sýnidæmi}

Skoðum fall \(f(x)=1/(1+x^2), x\in {{\mathbb R}}.\) Athugum að fallið \(f\) er jafnstætt, svo samkvæmt reglu 6 er Fourier-mynd þess jafnstæð. Þá getum við reiknað Fourier-mynd fyrir \(\xi<0\), og eftir það framlengjt hana samkvæmt því. Fallið \(f\) hefur eitt skaut í \(x=i\) á \(H_+\) og \(\max_{|z|=r}|f(z)| \le {1\over r^2-1}\) sem stefnir á 0 þegar \(r\) stefnir á óendanlegt. Þá beitum við setningunni að ofan og fáum
\begin{equation*}
\begin{split}\widehat f(\xi) = 2\pi i
{{\operatorname{Res}}}\bigg(\dfrac{e^{-iz\xi}}{1+z^2},i\bigg)
=\pi e^{\xi}, \qquad \xi<0.\end{split}
\end{equation*}
Að lokum fæst
\begin{equation*}
\begin{split}\widehat f(\xi) =\pi e^{-|\xi|}, \qquad \xi\in {{\mathbb  R}}.\end{split}
\end{equation*}

\subsection{Andhverfar Fourier-myndir reiknaðar með leifareikningi}
\label{\detokenize{Kafli04:andhverfar-fourier-myndir-reiknaar-me-leifareikningi}}
Á svipaðan hátt höfum við niðurstöðu fyrir andhverfar  Fourier-myndir.

\sphinxstylestrong{Setning}
\begin{quote}

Gerum ráð fyrir því að \(f\in L^1({{\mathbb R}})\cap PC^1({{\mathbb R}})\), að það sé hægt að framlengja skilgreiningarsvæði Fourier\textendash{}myndarinnar \(\widehat f\), þannig að \(\widehat f\in {{\cal O}}({{\mathbb C}}\setminus A)\), þar sem mengið \(A\) er endanlegt og \(\lim\limits_{r\to +\infty}\max_{|\zeta|=r}|\widehat f(\zeta)|=0\). Þá er
\end{quote}
\begin{equation*}
\begin{split}\tfrac 12 (f(x+)+f(x-))=\begin{cases}
i\sum\limits_{\alpha\in A\cap H_+}{{\operatorname{Res}}}\big(e^{ix\zeta}\widehat
f(\zeta),\alpha\big), & x>0\\
-i\sum\limits_{\alpha\in A\cap H_-}{{\operatorname{Res}}}\big(e^{ix\zeta}\widehat
f(\zeta),\alpha\big), & x<0.
\end{cases}\end{split}
\end{equation*}
Athugum að ef fallið \(f\) er samfellt þá er \(\tfrac 12 (f(x+)+f(x-))= f(x)\).

\sphinxstylestrong{Sýnidæmi}

Lítum á \(\widehat f(\xi)=\xi/(\xi^2+4\xi+5)\). Fallið \(\widehat f\) hefur tvö skaut í \(zeta_1 = -2+i\in H_+\) og \(\zeta_2 = -2-i\in H_-\). Ennfremur er það hvorki jafnstætt né oddstætt, svo við þurfum að reikna báðar leifar:
\begin{equation*}
\begin{split}\begin{gathered}
i{{\operatorname{Res}}}\bigg( \dfrac{e^{ix\zeta}\zeta}{\zeta^2+4\zeta+5},-2+i\bigg)
= (-1+i/2)e^{-x-2ix},\\
-i{{\operatorname{Res}}}\bigg( \dfrac{e^{ix\zeta}\zeta}{\zeta^2+4\zeta+5},-2-i\bigg)
= (-1-i/2)e^{x-2ix}.\end{gathered}\end{split}
\end{equation*}
Að lokum fáum við samkvæmt setningunni að ofan
\begin{equation*}
\begin{split}f(x) =\begin{cases}
(-1+i/2)e^{-x-2ix}\,, \qquad & x>0\\
(-1-i/2)e^{x-2ix}\,, \qquad & x<0
\end{cases}\end{split}
\end{equation*}
sem við getum umskrifað líka sem \(f(x)=-(1-i{{\operatorname{sign}}}(x)/2)e^{-|x|-2ix}\), fyrir öll  \(x\in\mathbb{R}\).


\section{Laplace-ummyndun og leifareikningur}
\label{\detokenize{Kafli04:laplace-ummyndun-og-leifareikningur}}
Rifjum upp að
\begin{enumerate}
\def\theenumi{\arabic{enumi}}
\def\labelenumi{\theenumi .}
\makeatletter\def\p@enumii{\p@enumi \theenumi .}\makeatother
\item {} 
Ef fall \(f\) á \(\mathbb{R_+}\) er af veldisvísisgerð, þá eru til fastar \(M>0\) og \(c>0\) þ.a.

\end{enumerate}
\begin{equation}\label{equation:Kafli04:eq.expo}
\begin{split}|f(t)|\leq Me^{c t}, \qquad t\in {{\mathbb  R}}_+.\end{split}
\end{equation}\begin{enumerate}
\def\theenumi{\arabic{enumi}}
\def\labelenumi{\theenumi .}
\makeatletter\def\p@enumii{\p@enumi \theenumi .}\makeatother
\setcounter{enumi}{1}
\item {} 
Skilgreining á Laplace-mynd er

\end{enumerate}
\begin{equation}\label{equation:Kafli04:eq.test}
\begin{split}\mathcal L{f}(s)= \int_{0}^\infty e^{-s t} f(t) dt \,, \qquad s\in \mathbb C\,,\end{split}
\end{equation}
þar sem \(f\) er skilgreint á \(\mathbb{R_+}\) með gildi í \(\mathbb C\), og er heildanlegt á sérhverju lokuðu og takmörkuðu bili á \(\mathbb{R_+}\).


\subsection{Andhverfar Laplace-myndir}
\label{\detokenize{Kafli04:andhverfar-laplace-myndir}}\phantomsection\label{\detokenize{Kafli04:fouriermellin}}
\sphinxstylestrong{Setning: Andhverfuformúla Fourier-Mellin}
\begin{quote}

Gerum ráð fyrir að fall \(f:{{\mathbb R}}_+\to {{\mathbb C}}\) sé í \(PC^1 (\mathbb R)\) og sé af veldisvísisgerð \eqref{equation:Kafli04:eq.expo}. Þá gildir  um sérhvert \(\xi> c\) og sérhvert \(t> 0\) að
\end{quote}
\begin{equation*}
\begin{split}\begin{aligned}
\label{7.2.2}
\tfrac 12(f(t+)+f(t-))& = \lim_{R\to +\infty} \dfrac 1{2\pi}
\int_{-R}^{+R}e^{(\xi+i\eta)t}{{\cal L}}f(\xi+i\eta) \, d\eta \\
&= \lim_{R\to +\infty} \dfrac 1{2\pi i}
\int_{\xi-iR}^{\xi+iR}e^{\zeta t}{{\cal L}}f(\zeta) \, d\zeta. \nonumber\end{aligned}\end{split}
\end{equation*}
Ef \(\mathcal{L} f(\xi+i \eta)\in L^1(\mathbb R)\) sem fall af \(\eta\), þá er \(f\) samfellt í \(t\) og
\begin{equation*}
\begin{split}\begin{aligned}
f(t)&=  \dfrac 1{2\pi}
\int_{-\infty}^{+\infty}e^{(\xi+i\eta)t}{{\cal L}}f(\xi+i\eta) \, d\eta
\\
&= \dfrac 1{2\pi i}
\int_{\xi-i\infty}^{\xi+i\infty}e^{\zeta t}{{\cal L}}f(\zeta) \,
d\zeta. \nonumber\end{aligned}\end{split}
\end{equation*}
Athugum að \(\int_{\xi-i\infty}^{\xi+i\infty}\) og \(\int_{\xi-iR}^{\xi+iR}\) tákna heildi eftir línunni \(\{\xi+i \eta; \eta \in \mathbb{R}\}\).

\sphinxstylestrong{Setning}

Látum \(f\) og \(g\) vera tvö samfelld föll af veldisvísisgerð á \(\mathbb{R_+}\), og gerum ráð fyrir að \(\mathcal{L}f(\alpha_j)=\mathcal{L}g(\alpha_j)\), þar sem \(\{\alpha_j\}\) er runa af ólíkum punktum, \(\alpha_j\to\alpha, \operatorname{Re}\alpha_j>c, \operatorname{Re}\alpha>c\). Þá er \(f(t)=g(t)\).


\subsection{Andhverfar Laplace-myndir reiknaðar með leifareikningi}
\label{\detokenize{Kafli04:andhverfar-laplace-myndir-reiknaar-me-leifareikningi}}\phantomsection\label{\detokenize{Kafli04:laplaceres}}
Hér viljum við hafa praktiskar aðferðir til þess að reikna heildi í {\hyperref[\detokenize{Kafli04:fouriermellin}]{\sphinxcrossref{\DUrole{std,std-ref}{setningunni}}}}.
Við getum notað leifareikning og við byrjum á að skoða niðurstöðuna fyrst.

\(M_r\) táknar hálfhringinn sem stikaður er með \(\gamma_r(\theta)=\xi+i r e^{i\theta}, \theta\in [0, \pi]\).

\sphinxstylestrong{Setning}
\begin{quote}

Gerum ráð fyrir að fall \(f:{{\mathbb R}}_+\to {{\mathbb C}}\) sé í \(PC^1 (\mathbb R)\) og sé af veldisvísisgerð \eqref{equation:Kafli04:eq.expo}.
Gerum ráð fyrir að hægt sé að framlengja \(\mathcal{L}f\) yfir í fágað fall á \(\mathbb C\setminus A\), þar sem \(A\) er endanlegt mengi.
Ef \(\xi>c\) og \(\lim\limits_{r\to +\infty}\max_{\zeta\in M_r}|{{\cal L}}f(\zeta)|=0\),
þá er
\end{quote}
\begin{equation}\label{equation:Kafli04:eq.LaplaceRes}
\begin{split}\frac 12(f(t+)+f(t-))=
\sum_{\alpha\in A}{{\operatorname{Res}}}(e^{\zeta t}{{\cal L}}f(\zeta),\alpha)
\qquad t>0.\end{split}
\end{equation}
Ef \(f\) er samfellt, þá er
\begin{equation*}
\begin{split}f(t)= \sum_{\alpha\in A}{{\operatorname{Res}}}(e^{\zeta t}{{\cal L}}f(\zeta),\alpha).\end{split}
\end{equation*}

\subsection{Andhverfar Laplace-myndir og afleiðujöfnur}
\label{\detokenize{Kafli04:andhverfar-laplace-myndir-og-afleiujofnur}}
Við skoðum afleiðujöfnu með fastastuðla
\begin{equation*}
\begin{split}P(D)u=(D^m+a_{m-1}D^{m-1}+\cdots+a_1D+a_0)u=f(t).\end{split}
\end{equation*}
Til þess að leysa jöfnuna getum við notað Green-fallið \(G(t,\tau)=g(t-\tau)\).
Munið að Laplace-mynd fallsins \(g\) er gefin með
\begin{equation*}
\begin{split}{{\cal L}}g(\zeta)=\dfrac 1{P(\zeta)}.\end{split}
\end{equation*}
Nú samkvæmt {\hyperref[\detokenize{Kafli04:laplaceres}]{\sphinxcrossref{\DUrole{std,std-ref}{setningunni}}}}, getum við reiknað út Green-fallið \(g\) með formúlunni \eqref{equation:Kafli04:eq.LaplaceRes}. Þá er
\begin{equation}\label{equation:Kafli04:eq.GreenLaplace}
\begin{split}g(t)= \sum\limits_{\alpha\in{\cal N}(P)}
{{\operatorname{Res}}}\bigg( \dfrac {e^{t\zeta}}{P(\zeta)},\alpha\bigg).\end{split}
\end{equation}

\subsection{Sýnidæmi}
\label{\detokenize{Kafli04:id4}}
Lítum á afleiðujöfnu
\begin{equation*}
\begin{split}P(D)u=(D^4-2 D^3+2 D^2-2 D+1)u=f(t)\end{split}
\end{equation*}
með óhliðruðu upphafsskilyrðunum
\begin{equation*}
\begin{split}u(0)=\dots=u^{(3)}(0)=0.\end{split}
\end{equation*}
Við reiknum út Green-fallið með því að nota Laplace-ummyndun og leifareikning. Samkvæmt formúlunni \eqref{equation:Kafli04:eq.GreenLaplace},  þurfum við að reikna kennimargliðuna \(P(\zeta)\). Þá er
\begin{equation*}
\begin{split}P(\zeta)= \zeta^ 4-2\zeta^ 3+2\zeta^ 2-2\zeta+1 = (\zeta-1)^ 2(\zeta-i)(\zeta+i).\end{split}
\end{equation*}
\({1\over P(\zeta)}\) hefur skaut í \(1, i, -i\), og við fáum
\begin{equation*}
\begin{split}\begin{aligned}
g(t)&= \sum\limits_{\alpha=1,i,-i} {{\operatorname{Res}}}\bigg(\dfrac{e^{\zeta t}}
{(\zeta-1)^ 2(\zeta-i)(\zeta+i)},\alpha\bigg)\\
&= \left.\dfrac d{d\zeta} \dfrac{e^{\zeta t}}{(\zeta-i)(\zeta+i)}
\right|_{\zeta=1} + \dfrac{e^{it}}{(i-1)^ 2(2i)} +
\dfrac{e^{-it}}{(-i-1)^ 2(-2i)}\\
&= \left.\dfrac{te^{\zeta t}}{\zeta^ 2+1}\right|_{\zeta=1}
+\left.\dfrac{e^{\zeta t}(-2\zeta)}{(\zeta^ 2+1)^ 2}\right|_{\zeta=1}
+\dfrac{e^{it}}4+\dfrac{e^{-it}}4\\
&=\tfrac 12 te^ t -\tfrac 12 e^ t +\tfrac 12\cos t.\end{aligned}\end{split}
\end{equation*}
Að lokum er lausnin gefin með
\begin{equation*}
\begin{split}u(t) = \int_0^t G(t,\tau) f(\tau) d\tau = \int_0^t \left(\tfrac 12 (t-\tau)e^{(t-\tau)} -\tfrac 12 e^{(t-\tau)} +\tfrac 12\cos (t-\tau)\right) f(\tau) d\tau\end{split}
\end{equation*}

\chapter{Mismunaaðferðir}
\label{\detokenize{Kafli05:mismunaaferir}}\label{\detokenize{Kafli05::doc}}
Við byrjum á að skoða praktískar og tölulegar aðferðir til þess að nálga lausnir á afleiðujöfnum og hlutafleiðujöfnum með upphafs- og jaðarskilyrðum.

Fyrsta aðferðin sem við ætlum að fjalla um er \textit{mismunaaðferð}.
Það eru ýmsar útgáfur fyrir mismunaaðferðir, en aðalatriði mismunaaðferða er að þær umbreyta afleiðujöfnum (eða hlutafleiðujöfnum) í algebrujöfnuhneppi.


\section{Mismunaaðferð fyrir venjulegar afleiðujöfnur}
\label{\detokenize{Kafli05:mismunaafer-fyrir-venjulegar-afleiujofnur}}\phantomsection\label{\detokenize{Kafli05:ch-5-1}}
Við skoðum fyrst jaðargildisverkefni fyrir almenna annars stigs afleiðujöfnu á bili \([a,b]\):
\begin{equation}\label{equation:Kafli05:eq.ODE1}
\begin{split}\begin{cases}
  Lu=a_2u''+a_1u'+a_0u=f,& \text{ á } ]a,b[\\
  B_1u=\alpha_1u(a)-\beta_1u'(a)=\gamma_1,&(\alpha_1,\beta_1)\neq (0,0),\\
  B_2u=\alpha_2u(b)+\beta_2u'(b)=\gamma_2,&(\alpha_2,\beta_2)\neq (0,0).
  \end{cases}\end{split}
\end{equation}
Við gerum ráð fyrir að raungildu föllin \(a_0, a_1,a_2\) séu samfelld í \([a,b]\) og fallið \(f\) sé raungilt.

Við veljum skiptingu á bilinu \([a,b]\), þ.e.
\begin{equation*}
\begin{split}a=x_0<x_1<x_2<\cdots <x_N=b.\end{split}
\end{equation*}
Til þess að einfalda reikninga getum við valið að skipta bilinu í jafna hluta, þ.e.
\begin{equation*}
\begin{split}h:= {b-a\over N}, \qquad x_j=a+jh,  \quad j=0, \dots, N.\end{split}
\end{equation*}
Athugum að þá gildir \(x_{j-1}=x_j-h\) og \(x_{j+1}=x_j+h\).

Við finnum gildi afleiðujöfnunnar í punktunum \(x_j\):
\begin{equation*}
\begin{split}\begin{cases}
\alpha_1u(x_0)-\beta_1u'(x_0)=\gamma_1,\\
a_2(x_j)u''(x_j)+a_1(x_j)u'(x_j)+a_0(x_j)u(x_j)=f(x_j), \qquad
j=1,\dots,N-1,\\
\alpha_2u(x_N)+\beta_2u'(x_N)=\gamma_2.
\end{cases}\end{split}
\end{equation*}
Við þurfum að finna nálgun fyrir afleiður \(u'(x_j),u''(x_j)\).
Það er eðlilegt að nálga afleiður \(u'(x_j),u''(x_j)\) með samsvarandi mismunakvóta.

Við ætlum að nota eftirfarandi nálgun
\begin{enumerate}
\def\theenumi{\arabic{enumi}}
\def\labelenumi{\theenumi .}
\makeatletter\def\p@enumii{\p@enumi \theenumi .}\makeatother
\item {} 
Í vinstri endapunkti \(x_0=a\), notum við

\end{enumerate}
\begin{equation*}
\begin{split}u'(x)\approx \dfrac{u(x+h)-u(x)}h\end{split}
\end{equation*}\begin{enumerate}
\def\theenumi{\arabic{enumi}}
\def\labelenumi{\theenumi .}
\makeatletter\def\p@enumii{\p@enumi \theenumi .}\makeatother
\setcounter{enumi}{1}
\item {} 
Í innri punktum bilsins \(x_1, \dots, x_{N-1}\), notum við

\end{enumerate}
\begin{equation}\label{equation:Kafli05:eq.approxder}
\begin{split}u'(x)\approx \dfrac{u(x+h)-u(x-h)}{2h} \qquad \text{ og } \qquad
u''(x)\approx \dfrac{u(x-h)-2u(x)+u(x+h)}{h^2},\end{split}
\end{equation}\begin{enumerate}
\def\theenumi{\arabic{enumi}}
\def\labelenumi{\theenumi .}
\makeatletter\def\p@enumii{\p@enumi \theenumi .}\makeatother
\setcounter{enumi}{2}
\item {} 
Í hægri endapunkti \(x_N=b\), notum við

\end{enumerate}
\begin{equation*}
\begin{split}u'(x)\approx \dfrac{u(x)-u(x-h)}h,\end{split}
\end{equation*}
Hugmyndin er að umrita hneppi \eqref{equation:Kafli05:eq.ODE1} með nálgununum að ofan og finna lausnir \(u(x_j)\).

Af hverjum veljum við nálgunarformúlur \eqref{equation:Kafli05:eq.ODE1}?
Við metum skekkju í nálgunarformúlunum sem við höfum skrifað að ofan.
Gerum ráð fyrir að fall \(\varphi\in C^4(I)\) á bili \(I\), og \(x, x+h, x-h\) séu í \(I\). Þá er
\begin{equation*}
\begin{split}\begin{aligned}
\varphi(x+h)&=\varphi(x)+\varphi'(x)h+\tfrac 12 \varphi''(x)h^2
+\tfrac 16 \varphi'''(x)h^3+\tfrac 1{24}\varphi^{(4)}(\xi)h^4,\\
\varphi(x-h)&=\varphi(x)-\varphi'(x)h+\tfrac 12 \varphi''(x)h^2
-\tfrac 16 \varphi'''(x)h^3+\tfrac 1{24}\varphi^{(4)}(\eta) h^4,\end{aligned}\end{split}
\end{equation*}
þar sem \(\xi \in (x,x+h)\) og \(\eta \in (x-h,x)\).

Við berum saman mismunandi nálgunarformúlur fyrir fyrsta stigs afleiður:
\begin{equation*}
\begin{split}\begin{aligned}
\varphi'(x)&-\dfrac{\varphi(x+h)-\varphi(x)}h
=-\tfrac 12 \varphi''(x)h
-\tfrac 16 \varphi'''(x)h^2-\tfrac 1{24}\varphi^{(4)}(\xi)h^3,\\
\varphi'(x)&-\dfrac{\varphi(x)-\varphi(x-h)}h
=\tfrac 12 \varphi''(x)h
-\tfrac 16 \varphi'''(x)h^2+\tfrac 1{24}\varphi^{(4)}(\eta)h^3,\\
\varphi'(x)&-\dfrac{\varphi(x+h)-\varphi(x-h)}{2h}
=-\tfrac 1{12}\varphi'''(x)h^2-\tfrac 1{48}
\big(\varphi^{(4)}(\xi)-\varphi^{(4)}(\eta)\big) h^3.
\end{aligned}\end{split}
\end{equation*}
Fyrir annars stigs afleiður höfum við:
\begin{equation*}
\begin{split}\varphi''(x)-\dfrac{\varphi(x-h)-2\varphi(x)+\varphi(x+h)}{h^2}
=-\tfrac 1{24}
\big(\varphi^{(4)}(\xi)+\varphi^{(4)}(\eta)\big) h^2.\end{split}
\end{equation*}
Við sjáum að ef við notum nálgunarformúluna \eqref{equation:Kafli05:eq.approxder} þá er skekkjan af öðru stigi í \(h\). Ef við notum fyrstu tvær formúlurnar, þá er skekkjan af fyrsta stigi í \(h\). Það segir okkur að þegar \(h \ll 1\), þá stefnir skekkjan í nálgunarformúlunni sem við viljum nota hraðar á núll, sem er auðvitað miklu betra.

Til þess að einfalda rithátt skrifum við \(u_j=u(x_j)\)  og  \(f_j=f(x_j)\).
Ennfremur, til þess að leggja áherslu á að \(u_j=u(x_j)\) er óþekkt stærð sem við viljum reikna út, setjum við að lokum \(c_j\) í staðinn fyrir \(u_j=u(x_j)\).

\sphinxstylestrong{Mismunajafna í vinstri endapunkti}

Í vinstri endapunkti \(x_0=a\) höfum við
\begin{equation*}
\begin{split}\alpha_1u(x_0)-\beta_1u'(x_0)=\gamma_1.\end{split}
\end{equation*}
Við nálgum afleiðuna eins og að ofan, þá fáum við
\begin{equation*}
\begin{split}\alpha_1 u_0-\beta_1\dfrac{u_1-u_0}h\approx \gamma_1,\end{split}
\end{equation*}
og að lokum notum við \(c_j\)
\begin{equation*}
\begin{split}\alpha_1c_0-\beta_1\dfrac{c_1-c_0}h= \gamma_1.\end{split}
\end{equation*}
\sphinxstylestrong{Mismunajafna í innri punktum bilsins}

Í innri punktum bilsins \(x_1, \dots, x_{N-1}\), notum við nálgunarformúlur \eqref{equation:Kafli05:eq.approxder}, og að lokum fáum við
\begin{equation*}
\begin{split}a_2(x_j)\dfrac{u_{j-1}-2u_j+u_{j+1}}{h^2}+a_1(x_j)
\dfrac{u_{j+1}-u_{j-1}}{2h}+a_0(x_j)u_j\approx f_j,\end{split}
\end{equation*}
og fyrir \(c_j\)
\begin{equation*}
\begin{split}a_2(x_j)\dfrac{c_{j-1}-2c_j+c_{j+1}}{h^2}+a_1(x_j)
\dfrac{c_{j+1}-c_{j-1}}{2h}+a_0(x_j)c_j=f_j.\end{split}
\end{equation*}
\sphinxstylestrong{Mismunajafna í hægri endapunkti}

Í hægri endapunkti \(x_N=b\), notum við nálgarformúluna að ofan, þá fæst
\begin{equation*}
\begin{split}\alpha_2u_N+\beta_2\dfrac{u_N-u_{N-1}}h\approx \gamma_2,\end{split}
\end{equation*}
og fyrir \(c_{N-1}\) og \(c_N\) þá er
\begin{equation*}
\begin{split}\alpha_2c_N+\beta_2\dfrac{c_N-c_{N-1}}h = \gamma_2.\end{split}
\end{equation*}
\sphinxstylestrong{Hneppið}

Að lokum tökum við saman nálgunarjöfnurnar í \((N+1)\times(N+1)\) hneppi:
\begin{equation*}
\begin{split}\begin{cases}
\alpha_1c_0-\beta_1\dfrac{c_1-c_0}h= \gamma_1,\\
a_2(x_j)\dfrac{c_{j-1}-2c_j+c_{j+1}}{h^2}+a_1(x_j)
\dfrac{c_{j+1}-c_{j-1}}{2h}+a_0(x_j)c_j=f_j,\\
\alpha_2c_N+\beta_2\dfrac{c_N-c_{N-1}}h = \gamma_2,
\end{cases}\end{split}
\end{equation*}
sem við getum umskrifað sem
\begin{equation}\label{equation:Kafli05:eq.firstapprox}
\begin{split}\begin{cases}
\big(\alpha_1+\tfrac {\beta_1}h\big)c_0
-\tfrac {\beta_1}h c_1= \gamma_1,\\
\big(\tfrac{a_2(x_j)}{h^2}-\tfrac{a_1(x_j)}{2h}\big) c_{j-1}
+\big(-\tfrac{2a_2(x_j)}{h^2}+a_0(x_j)\big)c_j
+\big(\tfrac{a_2(x_j)}{h^2}+\tfrac{a_1(x_j)}{2h}\big) c_{j+1} =f_j, \qquad j=1, \dots, N-1,\\
-\tfrac{\beta_2}h c_{N-1}+\big(\alpha_2+\tfrac{\beta_2}h \big) c_N = \gamma_2.
\end{cases}\end{split}
\end{equation}

\subsection{Sýnidæmi}
\label{\detokenize{Kafli05:synidaemi}}\phantomsection\label{\detokenize{Kafli05:example-5-1-1}}
Við lítum á eftirfarandi jaðargildisverkefni:
\begin{equation*}
\begin{split}\begin{cases}
&\left(x^2+1\right) u''(x)-2 x u'(x)+2 u(x)=1,\qquad  x \in ]0,1[\\
&u'(0)+u(0)=0,\\
&u(1)=1.
\end{cases}\end{split}
\end{equation*}
Fyrst skiptum við bilinu jafnt í tvo hluta,  þ.e.
\begin{equation*}
\begin{split}x_0=0, \qquad x_1= 1/2 , \qquad x_2 = 1, \qquad N=2, \qquad h={1 \over 2},\end{split}
\end{equation*}
og við notum nálgunarjöfnurnar \eqref{equation:Kafli05:eq.firstapprox}, og þá fáum við
\begin{equation*}
\begin{split}\begin{cases}
& 2 c_1-c_0=0, \\
& c_2-1=0,\\
& 6 c_0-8 c_1+4 c_2=1
\end{cases}\end{split}
\end{equation*}
sem gefur okkur
\begin{equation*}
\begin{split}c_0= -{3\over 2}, \qquad c_1 = -{3\over 4}, \qquad c_2 =1.\end{split}
\end{equation*}
Ef við veljum \(N=4\), þ.e.
\begin{equation*}
\begin{split}x_0=0, \quad x_1= 1/4 , \quad x_2 = 1/2, \quad x_3 = 3/4,  \quad x_4 = 1, \qquad N=4, ~~h={1 \over 4},\end{split}
\end{equation*}
þá fáum við
\begin{equation*}
\begin{split}\begin{cases}
& 3 c_0 -4 c_1=0, \\
& c_4 =1,\\
& -18 c_0+32 c_1-16 c_2=-1, \\
& 22 c_1 -38 c_2+18 c_3=1, \\
& 28 c_2-48 c_3+22 c_4 =1.
\end{cases}\end{split}
\end{equation*}
Lausnin er
\begin{equation*}
\begin{split}c_0=-{5\over 6},~~ c_1=-{5\over 8}, ~~ c_2 = -{1 \over 4},~~ c_3 = {7\over 24}, ~~ c_4 = 1.\end{split}
\end{equation*}
\noindent{\hspace*{\fill}\sphinxincludegraphics[width=0.850\linewidth]{{fig-difference-method}.png}\hspace*{\fill}}

\sphinxstyleemphasis{Lausnir fyrir} \(N=4, 10, 100\) \sphinxstyleemphasis{og lausn} \(u(x)={1 \over 2} (2 x^2+x-1).\)

\noindent{\hspace*{\fill}\sphinxincludegraphics[width=0.850\linewidth]{{error-difference-method}.png}\hspace*{\fill}}

\sphinxstyleemphasis{Skekkjan} \(|u_j-c_j|\)  \sphinxstyleemphasis{í}  \(x=1/2\)  \sphinxstyleemphasis{sem fall af}  \(N\).


\section{Heildun yfir hlutbil}
\label{\detokenize{Kafli05:heildun-yfir-hlutbil}}\phantomsection\label{\detokenize{Kafli05:ch-5-2}}
Við skoðum nú tilfelli þegar afleiðuvirkinn er af Sturm-Liouville gerð, þ.e.
\begin{equation*}
\begin{split}L u= -(p u^\prime)^\prime+ q u,\end{split}
\end{equation*}
þar sem við gerum ráð fyrir að fallið \(p\) sé samfellt diffranlegt á bili \([a,b]\), og fallið \(p\) sé samfellt á \([a,b]\).

Jaðargildisverkefnið er af gerðinni
\begin{equation*}
\begin{split}\begin{cases}
Lu=-(pu')'+qu=f,& \text{á } ]a,b[\\
\alpha_1u(a)-\beta_1u'(a)=\gamma_1,\\
\alpha_2u(b)+\beta_2u'(b)=\gamma_2.
\end{cases}\end{split}
\end{equation*}
Látum \([c,d]\) vera hlutbil af bilinu \([a,b]\).
Við heildum jöfnuna yfir \(x\in [c, d]\), þá er
\begin{equation*}
\begin{split}p(c)u'(c)-p(d)u'(d)
+\int_c^d q(x)u(x)\, dx=\int_c^d f(x)\, dx.\end{split}
\end{equation*}
Við viljum finna nálgunarjöfnur fyrir jöfnuna að ofan.
Við notum mismunakvóta til þess að nálga afleiður.
Við veljum skiptingu
\begin{equation*}
\begin{split}x_0 < x_1 < \dots < x_N.\end{split}
\end{equation*}
Við nálgum heildi með Riemann-summu með miðpunktsnálgun:
\begin{equation*}
\begin{split}\int_{x_0}^{x_N} \varphi(x)dx= \sum\limits_{j=0}^{N-1} \varphi(m_j)h,\end{split}
\end{equation*}
þar sem \(m_j\) eru miðpunktar hlutbilanna
\begin{equation*}
\begin{split}& m_j= {x_j+ x_{j+1}\over 2}, \qquad j=0, \dots, N-1,\\
& h = {x_N - x_{0}\over N},\\
& x_j= x_0+ j h, \qquad j=0, \dots, N.\end{split}
\end{equation*}
Athugum að
\begin{equation*}
\begin{split}m_j= {x_j+ x_{j+1}\over 2}= x_j+ {1\over 2} h, \qquad j=0, \dots, N-1.\end{split}
\end{equation*}
Eins og við gerðum í {\hyperref[\detokenize{Kafli05:ch-5-1}]{\sphinxcrossref{\DUrole{std,std-ref}{5.1}}}}, þurfum við að skoða jaðarskilyrðin hvert í sínu lagi.

Til þess að einfalda rithátt, fyrir öll \(j\) táknum við
\begin{equation*}
\begin{split}& u_j=u(x_j), ~~~ f_j=f(x_j), ~~~ p_j=p(x_j),~~~ q_j=q(x_j), \\
& p_{j+\frac 12}=p(m_j).\end{split}
\end{equation*}
\sphinxstylestrong{Mismunajafna í vinstri endapunkti bilsins}

Við heildum yfir hlutbilið \([x_0, m_0]\).
Athugum að \(m_0-x_0=\frac 12 h\).
Við fáum
\begin{equation*}
\begin{split}p(x_0)u'(x_0)-p(m_0)u'(m_0)
+\int_{x_0}^{m_0} qu\, dx=\int_{x_0}^{m_0} f\, dx,\end{split}
\end{equation*}
sem gefur okkur
\begin{equation*}
\begin{split}u'(x_0)=\dfrac 1{p(x_0)}\bigg(p(m_0)u'(m_0)
-\int_{x_0}^{m_0}q(x)u(x)\, dx
+\int_{x_0}^{m_0}f(x)\, dx\bigg).\end{split}
\end{equation*}
Við nálgum heildið með
\begin{equation*}
\begin{split}&\int_{x_0}^{m_0}q(x)u(x) dx\approx \frac 12 h q(x_0) u(x_0) = \frac 12 h q_0 u_0, \\
& \int_{x_0}^{m_0}f(x) dx\approx \frac 12 h f(x_0)= \frac 12 h f_0.\end{split}
\end{equation*}
Við nálgum afleiðuna \(u'(m_0)\) með mismunakvóta:
\begin{equation*}
\begin{split}u'(m_0)\approx {u(x_1)-u(x_0)\over 2 h/2}=  {u_1-u_0\over h}.\end{split}
\end{equation*}
Þá er \(u'(x_0)\)
\begin{equation*}
\begin{split}u'(x_0)\approx\dfrac 1{p_0}\bigg(p_{\frac 12} {u_1-u_0\over h}
-\frac 12 h q_0 u_0
+\frac 12 h f_0\bigg).\end{split}
\end{equation*}
Nú notum við jaðarskilyrði í \(x_0\):
\begin{equation*}
\begin{split}\alpha_1u(a)-\beta_1u'(a)=\gamma_1\end{split}
\end{equation*}
sem er nú
\begin{equation*}
\begin{split}\alpha_1u_0-\dfrac {\beta_1}{p_0}\bigg(
p_{\frac 12}\dfrac{u_1-u_0}{h}
-\tfrac 12 h q_0u_0+ \tfrac 12 hf_0\bigg)\approx \gamma_1.\end{split}
\end{equation*}
Eins og áður setjum við \(c_0, c_1\) í staðinn fyrir \(u_0, u_1\) til þess að tákna óþekkta stærð:
\begin{equation*}
\begin{split}\alpha_1 c_0-\dfrac {\beta_1}{p_0}\bigg(
p_{\frac 12}\dfrac{c_1-c_0}{h}
-\tfrac 12 h q_0c_0+\tfrac 12 hf_0\bigg)=\gamma_1.\end{split}
\end{equation*}
\sphinxstylestrong{Mismunajafna í hægri endapunkti}

Við gerum svipað fyrir hlutbilið \([ m_{N-1}, x_N]\):
\begin{equation*}
\begin{split}u'(x_N)=\dfrac 1{p(x_N)}\bigg(p(m_{N-1})u'(m_{N-1})
+\int_{m_{N-1}}^{x_N}q(x)u(x)\, dx-
\int_{m_{N-1}}^{x_N}f(x)\, dx\bigg).\end{split}
\end{equation*}
Við nálgum heildið með
\begin{equation*}
\begin{split}& \int_{m_{N-1}}^{x_N}q(x)u(x)\, dx \approx \tfrac 12h
q_Nu_N, \\
& \int_{m_{N-1}}^{x_N}f(x)\, dx \approx \tfrac 12h f_N,\end{split}
\end{equation*}
og afleiðuna \(u'(x_{N-1})\) með formúlunni
\begin{equation*}
\begin{split}u'(m_{N-1}) \approx\dfrac{u_N-u_{N-1}}{h}.\end{split}
\end{equation*}
Þá getum við nálgað jaðarskilyrðið í \(x_N\)
\begin{equation*}
\begin{split}\alpha_2u(x_N)+\beta_2u'(x_N)=\gamma_2\end{split}
\end{equation*}
með
\begin{equation*}
\begin{split}\alpha_2u_N+\dfrac {\beta_2}{p_N}\bigg(
p_{N-\frac 12}\dfrac{u_N-u_{N-1}}{h}+\tfrac 12h
q_Nu_N -\tfrac 12h f_N\bigg)\approx\gamma_2.\end{split}
\end{equation*}
\sphinxstylestrong{Mismunajöfnur í innri punktum skiptingarinnar}

Nú skoðum við hlutbilin  \([m_{j−1}, m_j ]\) fyrir \(j=1, \dots, N-1\).
Athugum að \(x_j\) eru miðpunktar bilanna \([m_{j−1}, m_j ]\).

Við heildum jöfnuna yfir \([m_{j−1}, m_j ]\)
\begin{equation*}
\begin{split}p(m_{j-1})u'(m_{j-1})-p(m_j)u'(m_j)+\int_{m_{j-1}}^{m_j}q(x)u(x)\, dx
=\int_{m_{j-1}}^{m_j}f(x)\, dx.\end{split}
\end{equation*}
Við nálgum heildið með miðpunktsnálgun
\begin{equation*}
\begin{split}\int_{m_{j-1}}^{m_j}q(x)u(x)\, dx \approx h\ q_ju_j
\qquad \text{ og } \qquad
\int_{m_{j-1}}^{m_j}f(x)\, dx \approx h\ f_j,\end{split}
\end{equation*}
og afleiður með mismunakvótum
\begin{equation*}
\begin{split}u'(m_j)\approx \dfrac{u_{j+1}-u_j}h \qquad \text{ og } \qquad
u'(m_{j-1})\approx \dfrac{u_j-u_{j-1}}h.\end{split}
\end{equation*}
Þá verður jafnan
\begin{equation*}
\begin{split}p_{j-\frac 12}\dfrac{u_j-u_{j-1}}h
-p_{j+\frac 12}\dfrac{u_{j+1}-u_j}h
+h q_ju_j\approx hf_j,\end{split}
\end{equation*}
og þegar við stingum \(c_j\) í jöfnuna í staðinn fyrir \(u_j\), fáum við
\begin{equation*}
\begin{split}p_{j-\frac 12}\dfrac{c_{j}-c_{j-1}}{h^2}
-p_{j+\frac 12}\dfrac{c_{j+1}-c_j}{h^2}
+q_jc_j= f_j.\end{split}
\end{equation*}
\sphinxstylestrong{Nálgunarjöfnuhneppið}

Að lokum, er línulega jöfnuhneppið fyrir nálgunargildin
\begin{equation*}
\begin{split}\begin{cases}
\bigg(\alpha_1+\dfrac{\beta_1}{p_0}
\bigg(\dfrac{p_{\frac 12}}h+\tfrac 12 hq_0\bigg)\bigg)c_0
-\dfrac{\beta_1p_{\frac 12}}{p_0h}c_1
=\gamma_1+\tfrac 12 \dfrac{\beta_1hf_0}{p_0},
\\
-\dfrac{p_{j-\frac 12}}{h^2} c_{j-1}
+\bigg(\dfrac{p_{j-\frac 12}+p_{j+\frac 12}}{h^2}+q_j\bigg) c_j
-\dfrac{p_{j+\frac 12}}{h^2} c_{j+1}  =f_j, \qquad j=1, \dots, N-1,
\\
-\dfrac{\beta_2p_{N-\frac 12}}{p_Nh}c_{N-1}+
\bigg(\alpha_2+\dfrac{\beta_2}{p_N}\bigg(\dfrac{p_{N-\frac 12}}{h}+\tfrac
12 h q_N\bigg)\bigg)c_N
=\gamma_2+\tfrac 12\dfrac{\beta_2h}{p_N}f_N.
\end{cases}\end{split}
\end{equation*}

\subsection{Sýnidæmi}
\label{\detokenize{Kafli05:id1}}\phantomsection\label{\detokenize{Kafli05:example-5-2-1}}
Við endurtökum nú sýnidæmi {\hyperref[\detokenize{Kafli05:example-5-1-1}]{\sphinxcrossref{\DUrole{std,std-ref}{5.1.1}}}}.
Við sjáum að
\begin{equation*}
\begin{split}& p(x)=\frac{1}{x^2+1}, \quad q(x)=-\frac{2}{\left(x^2+1\right)^2},\quad \rho=-\frac{1}{\left(x^2+1\right)^2}, \\
& \mathcal L u = {1\over \rho} \left( -(p u^\prime)^\prime +q u\right) = \widetilde{f}, \\
& \Rightarrow \left( -(p u^\prime)^\prime +q u\right) = \rho \widetilde{f} := f,\end{split}
\end{equation*}
þar sem \(f(x)= -\frac{1}{\left(x^2+1\right)^2}\).
Þá er jaðargildisverkefnið af gerðinni
\begin{equation*}
\begin{split}\begin{cases}
u''(x)-\frac{2 x u'(x)}{x^2+1}+\frac{2 u(x)}{x^2+1}=\frac{1}{x^2+1},\\
u'(0)+u(0)=0,\\
u(1)=1.
\end{cases}\end{split}
\end{equation*}
Við notum nálgunarformúlurnar að ofan fyrir \(N=4\).
Athugum að
\begin{equation*}
\begin{split}& x_0= 0, ~~ x_1= \frac{1}{4}, ~~ x_2=\frac{1}{2}, ~~ x_3=\frac{3}{4}, ~~ x_4=1,\\
& m_0=\frac{1}{8}, ~~ m_1=\frac{3}{8}, ~~ m_2=\frac{5}{8}, ~~ m_3=\frac{7}{8}.\end{split}
\end{equation*}
Línulega hneppið er
\begin{equation*}
\begin{split}\begin{cases}
1398 c_0-2048 c_1=-65, \\
c_4=1, \\
84388 c_0-150038 c_1+75140 c_2=4745,\\
142400 c_1-246206 c_2+116800 c_3=6497,
\\
282500 c_2-484886 c_3+222500 c_4 =10057,
\end{cases}\end{split}
\end{equation*}
sem gefur okkur
\begin{equation*}
\begin{split}c_0= -0.4895, c_1=-0.3024, c_2= 0.0091,
c_3= 0.4434, c_4= 1.\end{split}
\end{equation*}
Er þetta betra en í dæmi {\hyperref[\detokenize{Kafli05:example-5-1-1}]{\sphinxcrossref{\DUrole{std,std-ref}{5.1.1}}}}? Skoðum myndina.

\noindent{\hspace*{\fill}\sphinxincludegraphics[width=0.850\linewidth]{{comparison-51vs52}.png}\hspace*{\fill}}


\subsection{Línuleg brúun og þúfugrunnföll}
\label{\detokenize{Kafli05:linuleg-bruun-og-ufugrunnfoll}}\phantomsection\label{\detokenize{Kafli05:ch-5-2-2}}
Við höfum reiknað út lausnir \(c_j\) á nálgunarformúlunni.
Hvernig getum við endurgert fallið \(u\) sem uppfyllir \(u_j \approx c_j\)?
Við getum notað línulega samantekt til að finna \sphinxstyleemphasis{nálgunarfall} \(v\in C[a,b]\), þ.a. \(v(x_j)= c_j\) og svo \(v_j \approx u_j\).

Fyrst skilgreinum við \sphinxstyleemphasis{þúfugrunnföllin} \(\varphi_j(x)\), fyrir \(j=0, \dots, N\), þ.a.
\begin{enumerate}
\def\theenumi{\arabic{enumi}}
\def\labelenumi{\theenumi .}
\makeatletter\def\p@enumii{\p@enumi \theenumi .}\makeatother
\item {} 
\(\varphi_j(x)\) eru samfelld,

\item {} 
\(\varphi_j(x_i)=\delta_{i,j}\).

\end{enumerate}

Við tökum t.d. jafna skiptingu á bilinu \([a,b]\)
\begin{equation*}
\begin{split}a=x_0<x_1<x_2<\cdots <x_N=b, \qquad x_j= a + j h, \qquad j=0, \dots, N,\end{split}
\end{equation*}
þá eru föllin \(\varphi_j(x)\) af gerðinni
\begin{equation*}
\begin{split}\begin{aligned}
\varphi_0(x)&=\begin{cases} \tfrac{x_1-x}{h}, &x\in [x_0,x_1],\\
0, &\text{annars},
\end{cases} &
\varphi'_0(x)&=\begin{cases} \tfrac{-1}{h}, &x\in ]x_0,x_1[,\\
0, &x\in {{\mathbb  R}}\setminus [x_0,x_1],
\end{cases}
\nonumber
\\
\varphi_j(x)&=\begin{cases} \tfrac{x-x_{j-1}}{h}, &x\in [x_{j-1},x_j[,\\
\tfrac{x_{j+1}-x}{h}, &x\in [x_j,x_{j+1}],\\
0, &\text{annars}.
\end{cases}&
\varphi'_j(x)&=\begin{cases} \tfrac{1}{h}, &x\in ]x_{j-1},x_j[,\\
\tfrac{-1}{h}, &x\in [x_j,x_{j+1}],\\
0, &x\in {{\mathbb  R}}\setminus [x_{j-1},x_{j+1}].
\end{cases}
\\
\varphi_N(x)&=\begin{cases} \tfrac{x-x_{N-1}}{h}, &x\in [x_{N-1},x_N],\\
0, &\text{annars}. \nonumber
\end{cases}&
\varphi'_N(x)&=\begin{cases} \tfrac{1}{h}, &x\in [x_{m-1},x_m],\\
0, &\text{annars}. \nonumber
\end{cases}\end{aligned}\end{split}
\end{equation*}
Athugum að föllin \(\varphi_j(x)\) eru samfelld á \([a,b]\).

\noindent{\hspace*{\fill}\sphinxincludegraphics[width=0.850\linewidth]{{tent-functions}.png}\hspace*{\fill}}

\sphinxstyleemphasis{Þúfugrunnföllin} \(\varphi_j(x)\) \sphinxstyleemphasis{fyrir dæmin} {\hyperref[\detokenize{Kafli05:example-5-1-1}]{\sphinxcrossref{\DUrole{std,std-ref}{5.1.1}}}} \sphinxstyleemphasis{og} {\hyperref[\detokenize{Kafli05:example-5-2-1}]{\sphinxcrossref{\DUrole{std,std-ref}{5.2.1}}}}. \sphinxstyleemphasis{Hér} \(N=4\).

Þá skilgreinum við nálgunarfall \(v\), með því að setja
\begin{equation*}
\begin{split}v(x)=c_0\varphi_0(x)+\cdots+c_N\varphi_N (x),\end{split}
\end{equation*}
þar sem \(c_j\) eru lausnir á nálgunarformúlunni.
Það er ljóst að \(v\) er samfellt á bilinu \([a,b]\), og
\begin{equation*}
\begin{split}v(x_j)= \sum_{i=0}^N c_i\varphi_i(x_j)= c_j.\end{split}
\end{equation*}
\noindent{\hspace*{\fill}\sphinxincludegraphics[width=0.850\linewidth]{{v-functions}.png}\hspace*{\fill}}

\sphinxstyleemphasis{Nálgunarföllin} \(v\) \sphinxstyleemphasis{fyrir dæmin} {\hyperref[\detokenize{Kafli05:example-5-1-1}]{\sphinxcrossref{\DUrole{std,std-ref}{5.1.1}}}} \sphinxstyleemphasis{og} {\hyperref[\detokenize{Kafli05:example-5-2-1}]{\sphinxcrossref{\DUrole{std,std-ref}{5.2.1}}}}. \sphinxstyleemphasis{Hér} \(N=4\).


\section{Mismunaaðferð fyrir hlutaafleiðujöfnur}
\label{\detokenize{Kafli05:mismunaafer-fyrir-hlutaafleiujofnur}}
Nú lítum við á verkefnið að nálga lausnir á hlutafleiðujöfnum með upphafs- og jaðarskilyrðum.

Látum \(D\) vera svæði í \(R^2\) og skoðum eftirfarandi jaðargildisverkefni
\begin{equation*}
\begin{split}\begin{cases}
Lu=-\nabla\cdot (p\nabla u)+qu=
-p\nabla^2u-\nabla p\cdot \nabla u+qu=f, \qquad \text{ á } D\\
\alpha u+\beta\dfrac{\partial u}{\partial n}
=\gamma, \qquad  \text{á } \ \partial D.
\end{cases}\end{split}
\end{equation*}
Við gerum ráð fyrir að \(p\in C^1(\mathbb R^2)\) og \(q, f\) séu samfelld á \(\mathbb R^2\). \(\alpha,\beta\) og \(\gamma\) séu gefin föll á jaðrinum  \(\partial D\) þ.a. \((\alpha, \beta)\neq (0,0)\) á \(\partial D\).

Látum \(\partial D_1\) vera þann hluta jaðarsins þ.a. \(\beta=0\), þá er \(u(x,y)=\gamma(x,y)/\alpha(x,y)\) fyrir öll \((x,y)\in \partial D_1\). Þá skiptum við jaðrinum í tvö sundurlæg mengi \(\partial D=\partial D_1\cup \partial D_2\), og við umskrifum jaðargildisverkenið sem
\begin{equation}\label{equation:Kafli05:eq.pde1}
\begin{split}\begin{cases}
Lu=-\nabla\cdot (p\nabla u)+qu=f, \quad &\text{á } D\\
u=\gamma,\quad &\text{á } \ \partial_1 D,\\
\alpha u+\beta\dfrac{\partial u}{\partial n}
=\gamma, \quad  &\text{á } \ \partial_2 D,
\end{cases}\end{split}
\end{equation}
þar sem höfum við táknað \(\gamma(x,y)/\alpha(x,y)\) með \(\gamma\) á \(\partial_1 D\) að ofan.

Til þess að nálga lausn \(u\) á jaðargildisverkefninu \eqref{equation:Kafli05:eq.pde1} þurfum við að undirbúa net á svipaðan hátt og við gerðum í {\hyperref[\detokenize{Kafli05:ch-5-1}]{\sphinxcrossref{\DUrole{std,std-ref}{5.1}}}}. Það sem er ólíkt hér er að við erum í \(\mathbb R^2\), þ.e.a.s. við þurfum að byggja skiptingu eftir \(x\) og \(y\).

\sphinxstylestrong{Net}

Gerum ráð fyrir að svæðið \(D\) sé innihaldið í rétthyrningi
\begin{equation*}
\begin{split}R=\{(x,y)\,;\, a\leq x\leq b, c\leq y\leq d\}.\end{split}
\end{equation*}
Við tökum \(h>0\) og við skiptum bilunum á eftirfarandi hátt
\begin{equation*}
\begin{split}x_m=a+mh \qquad \text{ og } \qquad y_n=c+nh, \qquad m,n\in {{\mathbb  N}\cup {0}}.\end{split}
\end{equation*}
\noindent{\hspace*{\fill}\sphinxincludegraphics[width=0.450\linewidth]{{ex-grid-rectangle}.png}\hspace*{\fill}}

\sphinxstyleemphasis{Dæmi af}  \(D \subset R \subset {\mathbb R^2}\).

Hnútpunktar í netinu eru \((x_i, y_k)\) og línurnar gegnum hnútpunktana eru stikaðar með
\begin{equation*}
\begin{split}{{\mathbb  R}}\ni t\mapsto (a+mh,t) \qquad \text{ og } \qquad
{{\mathbb  R}}\ni s\mapsto (s,c+nh).\end{split}
\end{equation*}
Við sjáum að línurnar skera \(\overline D =D\cup\partial D\) í punktinum \((x_i, y_k)\) , \(i=1,\dots, M_1\) og \(k=1,\dots,M_2\).

Við þurfum að velja hvernig við viljum raða punktunum sem eru í \(\overline D\). Til þess að einfalda nálgunarformúlur, er gott að raða punktunum á eftirfarandi hátt:

Setjum \(M = M_1 \times M_2\), og
\begin{itemize}
\item {} 
Allir hnútpunktar eru \((x_j, y_j)\) þar sem \(j=1, \dots, M\),

\item {} 
Hnútpunktar \((x_j, y_j)\) fyrir  \(j=1,+dots, N\leq M\) eru í  \(D\cup \partial D_2\),

\item {} 
Hnútpunktar \((x_j, y_j)\) fyrir \(j=N+1, \dots, M\) eru í  \(\partial D_1\).

\end{itemize}

Hugmyndin að baki þessu er að fallgildi  \(u\) eru þekkt á \(\partial D_1\) af því að \(u\) uppfyllir Dirichlet-jaðarskilyrði þar.

\sphinxstylestrong{Heildun yfir hlutsvæði}

Gerum ráð fyrir að hlutsvæði \(\Omega\) sé í \(D\), t.d. getur \(\Omega\) verið hlutrétthyrningur milli \((x_j,y_j)\) fyrir \(j=1, \dots 4\).

Við heildum jöfnuna yfir \(\Omega\) og við notum Gauss-setninguna sem gefur okkur
\begin{equation*}
\begin{split}\iint_\Omega \nabla\cdot\big(p\nabla u\big) \, dxdy
=\int_{\partial\Omega} p\dfrac{\partial u}{\partial n}\, ds,\end{split}
\end{equation*}
þá er jafnan í \eqref{equation:Kafli05:eq.pde1}
\begin{equation*}
\begin{split}-\int_{\partial\Omega } p\dfrac{\partial u}{\partial n}\, ds
+\iint_\Omega qu\, dxdy =\iint_\Omega f\, dxdy.\end{split}
\end{equation*}
Við ætlum að nota þessi heildi til þess að nálga lausn \(u\) í punktunum \((x_j, y_j)\) fyrir  \(j=1,\dots, N\).


\subsection{Nálgunarjafna}
\label{\detokenize{Kafli05:nalgunarjafna}}
Nú erum við tilbúin til að leiða út nálgunarjöfnur fyrir jaðargildisverkenið \eqref{equation:Kafli05:eq.pde1}. Eins og áður greinum við á milli innri punkta og jaðarpunkta.

\sphinxstylestrong{Nálgunarjafna í innri punkti}

Látum \((x_j, y_j)\) vera innri punkt í \(D\), og \(\Omega_j\) vera ferninginn með miðju í \((x_j, y_j)\) og kantlengdina \(h\).
Látum grannpunkta \((x_j, y_j)\) vera \((x_l,y_l), (x_r,y_r), (x_s,y_s)\) og \((x_t,y_t)\).
Við táknum með \(m_{j,k}\) miðpunkt línunnar milli \((x_j,y_j)\) og \((x_k,y_k)\), og með \(S_{j,k}\) hlið af \(\partial\Omega_j\) sem er milli \((x_j,y_j)\) og \((x_k,y_k)\).
Athugum að hér eru \(j, l, r\), og \(s\) gefin, á meðan lagt er saman yfir \(k=l,r,s, t\).

Við sjáum á myndinni hér:

\noindent{\hspace*{\fill}\sphinxincludegraphics[width=0.450\linewidth]{{internal-points}.png}\hspace*{\fill}}

\sphinxstyleemphasis{Dæmi af innri punktum}  \((x_j,y_j)\).

Þá er yfir \(\Omega_j\)
\begin{equation*}
\begin{split}\iint_{\Omega_j} \nabla\cdot\big(p\nabla u\big) \, dxdy
=\int_{\partial{\Omega}_j} p\dfrac{\partial u}{\partial n}\, ds =
\sum_{k=l,r,s,t} \int_{S_{j,k}} p\dfrac{\partial u}{\partial n}\, ds.\end{split}
\end{equation*}
Við nálgun heildið yfir \(S_{j,k}\) með því að finna gildi fallsins \(p\) og afleiðunnar \({\partial u\over \partial n}\) í miðpunktum \(m_{j,k}\), þá er
\begin{equation*}
\begin{split}\int_{S_{j,k}}p\dfrac{\partial u}{\partial n} \, ds
\approx p(m_{j,k})\dfrac{\partial u}{\partial n}(m_{j,k}) h
\approx p(m_{j,k})\dfrac{u(x_k,y_k)-u(x_j,y_j)}{h} h.\end{split}
\end{equation*}
Liðirnir sem eftir standa eru nálgaðir með
\begin{equation*}
\begin{split}\iint_{\Omega_j}q u \, dxdy \approx q(x_j,y_j)u(x_j,y_j)\,  h^2,\end{split}
\end{equation*}\begin{equation*}
\begin{split}\iint_{\Omega_j}f \, dxdy \approx f(x_j,y_j)\,  h^2.\end{split}
\end{equation*}
Til þess að einfalda rithátt, setjum við \(u_k=u(x_k,y_k), q_k=q(x_k,y_k), f_k=f(x_k,y_k)\) og \(p_{j,k}=p(m_{j,k})\), og að lokum fáum við að jafnan \eqref{equation:Kafli05:eq.pde1} er nálguð með
\begin{equation*}
\begin{split}&&-\int_{\partial\Omega } p\dfrac{\partial u}{\partial n}\, ds
+\iint_\Omega qu\, dxdy =\iint_\Omega f\, dxdy\\
&&
\Rightarrow ~~
-p_{j,l}(u_l-u_j)
-p_{j,r}(u_r-u_j)
-p_{j,s}(u_s-u_j)
-p_{j,t}(u_t-u_j)
+q_ju_jh^2\approx f_jh^2.\end{split}
\end{equation*}
Við notum \(c_k\) í staðinn fyrir \(u_k\) fyrir \(k=j,l,r,s,t\), og þá er nálgunarformúlan fyrir innri punktana
\begin{equation}\label{equation:Kafli05:eq.approx1}
\begin{split}\begin{gathered}
\big(h^{-2}\big({p_{j,l}+p_{j,r}+p_{j,s}+p_{j,t}}\big)+q_j\big)c_j
\\
-h^{-2}p_{j,l}c_l
-h^{-2}p_{j,r}c_r
-h^{-2}p_{j,s}c_s
-h^{-2}p_{j,t}c_t
=f_j.
\end{gathered}\end{split}
\end{equation}
Í sértilfellinu þegar \(p=1\), þá fáum við
\begin{equation}\label{equation:Kafli05:eq.approx1s}
\begin{split}\big(4{h^{-2}}+q_j\big)c_j
-{h^{-2}}c_l
-{h^{-2}}c_r
-{h^{-2}}c_s
-{h^{-2}}c_t
=f_j.\end{split}
\end{equation}
\sphinxstylestrong{Nálgunarjafna í jaðarpunkti}

Látum \((x_j,y_j)\) vera jaðarpunkt í  \(\partial D_2\). Látum grannpunkta \((x_j, y_j)\) vera \((x_l,y_l), (x_s,y_s)\) og \((x_t,y_t)\).
Við gerum ráð fyrir að línan \(S_j\) milli miðpunktanna \(m_{j,s}\) og \(m_{j,t}\) sé í \(\partial D_2\).
Athugum að hér er \(k=s, l, t\), og hér er flatarmál svæðisins \(\Omega_j\) \(h^2/2\).

Við sjáum á myndinni hér:

\noindent{\hspace*{\fill}\sphinxincludegraphics[width=0.450\linewidth]{{boundary-points}.png}\hspace*{\fill}}

\sphinxstyleemphasis{Dæmi af jaðarpunktum}  \((x_j,y_j)\).

Við höfum
\begin{equation*}
\begin{split}\iint_{\Omega_j} \nabla\cdot\big(p\nabla u\big) \, dxdy=
\int_{\partial \Omega_j} p\dfrac{\partial u}{\partial n}\, ds
=\sum_{k=l,s,t} \int_{S_{j,k}} p\dfrac{\partial u}{\partial n}\, ds
+\int_{S_{j}} p\dfrac{\partial u}{\partial n}\, ds.\end{split}
\end{equation*}
Heildið yfir \(S_j\) er
\begin{equation*}
\begin{split}\int_{S_{j}} p\dfrac{\partial u}{\partial n}\, ds \approx
p_j \dfrac{\partial u}{\partial n}(x_j,y_j) \approx p_j {\gamma_j-\alpha_j u_j\over \beta_j} h,\end{split}
\end{equation*}
þar sem við höfum notað jaðarskilyrði á \(\partial D_2\) fyrir \(\dfrac{\partial u}{\partial n}(x_j,y_j)\), og táknað \(\gamma(x_j,y_j), \alpha(x_j,y_j), \beta(x_j,x_j)\) með \(\gamma_j,\alpha_j, \beta_j\).

Heildið yfir \(S_{j,l}\) er
\begin{equation*}
\begin{split}\int_{S_{j,l}} p\dfrac{\partial u}{\partial n} \, ds
\approx p_{j,l}\dfrac {u_l-u_j} h h,\end{split}
\end{equation*}
en heildin yfir \(S_{j,s}\) og \(S_{j,t}\) eru gefin með
\begin{equation*}
\begin{split}\int_{S_{j,s}} p\dfrac{\partial u}{\partial n} \, ds
\approx p_{j,s}\dfrac{u_s-u_j}h\tfrac 12 h
\qquad \text{ og }  \qquad
\int_{S_{j,t}} p\dfrac{\partial u}{\partial n} \, ds
\approx p_{j,t}\dfrac{u_t-u_j}h\tfrac 12 h.\end{split}
\end{equation*}
Að lokum þurfum við að nálga heildið með föllum \(f\) og \(q\)
\begin{equation*}
\begin{split}\iint_{\Omega_j} q u\, dxdy
\approx q_ju_j \tfrac 12 h^2
\quad \text{ og } \quad
\iint_{\Omega_j} f\, dxdy
\approx f_j \tfrac 12 h^2.\end{split}
\end{equation*}
Að lokum að lokum setjum við saman alla liði og þá er
\begin{equation}\label{equation:Kafli05:eq.approx2}
\begin{split}\begin{gathered}
\big(2h^{-2}p_{j,l}+h^{-2}p_{j,s}+h^{-2}p_{j,t}
+2h^{-1}p_j\tfrac{\alpha_j}{\beta_j}+q_j\big)c_j
\\
-2h^{-2}p_{j,l}c_l-h^{-2}p_{j,s}c_s-h^{-2}p_{j,t}c_t
=f_j+2h^{-1}p_j\tfrac{\gamma_j}{\beta_j}.
\end{gathered}\end{split}
\end{equation}
þar sem við höfum skrifað \(c_j\) í staðinn fyrir \(u_j\).

Í sértilfellinu þegar \(p=1\), þá fáum við
\begin{equation}\label{equation:Kafli05:eq.approx2s}
\begin{split}\big(4h^{-2}
+q_j+2h^{-1}\tfrac{\alpha_j}{\beta_j}\big)c_j
-2h^{-2}c_l-h^{-2}c_s-h^{-2}c_t
=f_j+2h^{-1}\tfrac{\gamma_j}{\beta_j}\end{split}
\end{equation}
\sphinxstylestrong{Samantekt}

Til þess að nálga lausnagildi á jaðargildisverkefninu \eqref{equation:Kafli05:eq.pde1} getum við leyst eftirfarandi  jöfnuhneppi:

Fyrir \((x_j, y_j)\in D\):
\begin{equation*}
\begin{split}\begin{gathered}
\big(h^{-2}\big({p_{j,l}+p_{j,r}+p_{j,s}+p_{j,t}}\big)+q_j\big)c_j
\\
-h^{-2}p_{j,l}c_l
-h^{-2}p_{j,r}c_r
-h^{-2}p_{j,s}c_s
-h^{-2}p_{j,t}c_t
=f_j.\end{gathered}\end{split}
\end{equation*}
Fyrir \((x_j, y_j)\in \partial D_2\):
\begin{equation*}
\begin{split}\begin{gathered}
\big(2h^{-2}p_{j,l}+h^{-2}p_{j,s}+h^{-2}p_{j,t}
+2h^{-1}p_j\tfrac{\alpha_j}{\beta_j}+q_j\big)c_j
\\
-2h^{-2}p_{j,l}c_l-h^{-2}p_{j,s}c_s-h^{-2}p_{j,t}c_t
=f_j+2h^{-1}p_j\tfrac{\gamma_j}{\beta_j}.\end{gathered}\end{split}
\end{equation*}
Fyrir \((x_j, y_j)\in \partial D_1\):
\begin{equation*}
\begin{split}c_j= \gamma_j \,.\end{split}
\end{equation*}

\subsection{Sýnidæmi: Dirichlet-verkefni á ferningi}
\label{\detokenize{Kafli05:synidaemi-dirichlet-verkefni-a-ferningi}}
Látum \(D\) vera  \(D=\{(x,y)\,;\, 0<x<1, 0<y<1\}\). Við viljum  leysa Dirichlet-verkefnið yfir \(D\):
\begin{equation*}
\begin{split}\begin{cases}
-\nabla^2u+qu=f,&\text{í } D,\\
u(x,y)=\gamma(x,y),& (x,y)\in \partial D.
\end{cases}\end{split}
\end{equation*}
Hér erum við í sértilfellinu þegar \(p=1\), og \(\partial D_2\) er ekki til.

Setjum \(h={1\over 3}\). Við röðum punktum þ.a. punktar \((x_j,y_j)\) fyrir \(j=1, \dots, 4\) eru innri punktar og \((x_j,y_j)\) fyrir \(j=5,\dots, 16\) eru jaðarpunktar. Við sjáum þetta á eftirfarandi mynd

\noindent{\hspace*{\fill}\sphinxincludegraphics[width=0.450\linewidth]{{Dirichlet-example}.png}\hspace*{\fill}}

Fyrir punkta \((x_j,y_j)\) með \(j=1, \dots, 4\), notum við nálgunarjöfnur \eqref{equation:Kafli05:eq.approx1s}, og þá er
\begin{equation*}
\begin{split}\begin{aligned}
(36+q_1)c_1-9c_2-9c_3-9c_6-9c_9&=f_1,\\
(36+q_2)c_2-9c_1-9c_4-9c_7-9c_{10}&=f_2,\\
(36+q_3)c_3-9c_1-9c_4-9c_{11}-9c_{14}&=f_3,\\
(36+q_4)c_4-9c_2-9c_3-9c_{12}-9c_{15}&=f_4.\\\end{aligned}\end{split}
\end{equation*}
Fyrir punkta \((x_j,y_j)\) með  \(j=5,\dots, 16\), notum við nálgunarjöfnur \eqref{equation:Kafli05:eq.approx2s}, og þá er
\begin{equation*}
\begin{split}c_j = \gamma_j.\end{split}
\end{equation*}
Það þýðir að við getum leyst jöfnuhneppið fyrir \(c_1, \dots, c_4\) og þá höfum við
\begin{equation*}
\begin{split}\left[  \begin{matrix}
36+q_1&-9&-9& 0\\
-9&36+q_2&0&-9\\
-9&0&36+q_3&-9\\
0&-9&-9&36+q_4\\
\end{matrix}\right]
\left[
\begin{matrix}
c_1\\c_2\\c_3\\c_4
\end{matrix}
\right]
=\left[
\begin{matrix}
f_1+9\gamma_6+9\gamma_9,\\
f_2+9\gamma_7+9\gamma_{10},\\
f_3+9\gamma_{11}+9\gamma_{14},\\
f_4+9\gamma_{12}+9\gamma_{15}.\\
\end{matrix}
\right].\end{split}
\end{equation*}

\subsection{Sýnidæmi: Jaðargildisverkefni á ferningi með blandað jaðarskilyrði}
\label{\detokenize{Kafli05:synidaemi-jaargildisverkefni-a-ferningi-me-blanda-jaarskilyri}}
Lítum nú á jaðargildisverkefni á \(D=\{(x,y)\,;\, 0<x<1, 0<y<1\}\)
\begin{equation*}
\begin{split}\begin{cases}
-\nabla^2u+qu=f,&\text{í } D,\\
u(x,y)=\gamma(x,y),& (x,y)\in \partial D_1,\\
\alpha(x,y)u(x,y)+\dfrac{\partial u}{\partial n}(x,y)
=\gamma(x,y),& (x,y)\in \partial D_2,
\end{cases}\end{split}
\end{equation*}
þar sem
\begin{equation*}
\begin{split}\begin{aligned}
\partial D_1  &= \{(x,y)\,;\, 0<x<1, 0=y, y=1,~~ \text{og} \quad 0<y<1, x=0 \},\\
\partial D_2 &= \{(x,y)\,;\, 0<y<1, x=1\}.
\end{aligned}\end{split}
\end{equation*}
Athugum að hér er \(p=1\) og \(\beta=1\).
Setjum \(h={1\over 3}\), og röðum punktum þ.a. \((x_j,y_j)\) fyrir \(j=1,\dots, 6\) eru í \(\partial D_2\) og \((x_j,y_j)\) fyrir \(j=7, \dots, 16\) eru í \(\partial D_1\). Við sjáum þetta á eftirfarandi mynd

\noindent{\hspace*{\fill}\sphinxincludegraphics[width=0.450\linewidth]{{mixed-example}.png}\hspace*{\fill}}

Fyrir punkta \((x_j,y_j)\) með  \(j=7,\dots, 16\), þá er
\begin{equation*}
\begin{split}c_j = \gamma_j.\end{split}
\end{equation*}
Fyrir punkta \((x_j,y_j)\) með \(j=1,2,4,5\) notum við nálgunarjöfnur \eqref{equation:Kafli05:eq.approx1s}, fyrir punkta \((x_j,y_j)\) með \(j=3,6\) notum við nálgunarjöfnur \eqref{equation:Kafli05:eq.approx2s}, og þá er
\begin{equation*}
\begin{split}\begin{aligned}
(36+q_1)c_1-9c_2-9c_4-9c_8-9c_{11}&=f_1,\\
(36+q_2)c_2-9c_1-9c_3-9c_5-9c_{9}&=f_2,\\
(36+q_3+6\alpha_3)c_3-18c_2-9c_6-9c_{10}&=f_3+6\gamma_3,\\
(36+q_4)c_4-9c_1-9c_5-9c_{12}-9c_{14}&=f_4.\\
(36+q_5)c_5-9c_2-9c_4-9c_{6}-9c_{15}&=f_5.\\
(36+q_6+6\alpha_6)c_6-9c_3-18c_{5}-9c_{16}&=f_6+6\gamma_6.\\
\end{aligned}\end{split}
\end{equation*}
Við getum skrifað þetta með fylkjajöfnu
\begin{equation*}
\begin{split}\begin{gathered}
\left[  \begin{matrix}
36+q_1&-9&0&-9& 0& 0\\
-9&36+q_2&-9&0&-9&0\\
0&-18&36+q_3+6\alpha_3&0&0&-9\\
-9&0&0&36+q_4&-9&0\\
0&-9&0&-9&36+q_5&-9\\
0&0&-9&0&-18&36+q_6+6\alpha_6\\
\end{matrix}\right]
\left[
\begin{matrix}
c_1\\c_2\\c_3\\c_4 \\ c_5 \\ c_6
\end{matrix}
\right]\\
=\left[
\begin{matrix}
f_1+9\gamma_8+9\gamma_{11}\\
f_2+9\gamma_{9}\\
f_3+9\gamma_{10}+6\gamma_3\\
f_4+9\gamma_{12}+9\gamma_{14}\\
f_5+9\gamma_{15}\\
f_6+9\gamma_{16}+6\gamma_6\\
\end{matrix}
\right].\end{gathered}\end{split}
\end{equation*}

\section{Almenn mismunaaðferð á rétthyrningi}
\label{\detokenize{Kafli05:almenn-mismunaafer-a-retthyrningi}}
Við lítum nú á almennt jaðargildisverkefni
\begin{equation}\label{equation:Kafli05:eq.exercise}
\begin{split}\begin{cases}
-\nabla\cdot\big(p\nabla u\big)+qu=f, &\text{í } D,\\
\alpha  u + \beta \dfrac{\partial u}{\partial n} = \gamma,&
\text{á } \partial D,
\end{cases}\end{split}
\end{equation}
á rétthyrningi \(D\) in \(R^2\)
\begin{equation*}
\begin{split}D = \ ]a,b[\times ]c,d[ \ = \{(x,y)\, ;\, a<x<b, c<y<d\}.\end{split}
\end{equation*}
Við gerum ráð fyrir að \((\alpha,\beta ) \neq (0,0)\) fyrir \((x,y) \in \partial D\).

Athugum að í hornunum geta föllin \(\alpha, \beta\) og \(\gamma\) verið ósamfelld og afleiðan \(\dfrac{\partial u}{\partial n}\) getur einnig verið ekki vel skilgreind.

Nú skiptum við rétthyrningnum \(D\) í reglulegt net með kantlengdina \(h\).
Setjum \(N=(b-a)/h\) og \(M=(d-c)/h\), þ.a. \(N\) og \(M\) eru  náttúrlegar tölur.

Hvernig getum við tölusett netpunkta í \(D\)?

\sphinxstylestrong{Hnit netpunkta í} \((x,y)\) \sphinxstylestrong{plani}

Það þýðir að við veljum skiptingu á bilinu í jafna hluta, þ.e.
\begin{equation*}
\begin{split}a=x_1<x_2<x_3<⋯<x_{(N+1)}=b, \qquad c=y_1<y_2<y_3<⋯<y_{(M+1)}=d,\end{split}
\end{equation*}
og
\begin{equation*}
\begin{split}x_j=a+(j-1) h, \quad j=1, \dots, N+1\,,\qquad y_k=a+(k-1) h, \quad k=1, \dots, M+1.\end{split}
\end{equation*}
Athugum að hér notum við \(j, k\) sem byrja frá 1 (í staðinn fyrir 0), af því að við höfum í huga að Matlab (eða Octave, eða Mathematica, …) byrja í 1 í tölusetningu vigra.
\begin{enumerate}
\def\theenumi{\arabic{enumi}}
\def\labelenumi{\theenumi .}
\makeatletter\def\p@enumii{\p@enumi \theenumi .}\makeatother
\item {} 
\sphinxstylestrong{Tvívíð tölusetning netpunkta}

\end{enumerate}

Hér setjum við \(1\le j \le N+1\) og \(1 \le k \le M+1\), og við tölusetjum netpunktana með \((x_j, y_k)\).
\begin{enumerate}
\def\theenumi{\arabic{enumi}}
\def\labelenumi{\theenumi .}
\makeatletter\def\p@enumii{\p@enumi \theenumi .}\makeatother
\setcounter{enumi}{1}
\item {} 
\sphinxstylestrong{Einföld tölusetning netpunkta}

\end{enumerate}

Hér  tölusetjum við netpunktana með \((x_i, y_i)\) þar sem \(i= j+(k-1)(N+1)\). Til dæmis setjum við  \(k=1\) og skoðum punkta með \(i= j+(k-1)(N+1)=j\), þar sem \(j=1, \dots, N+1\). Síðan veljum við \(k=2\) og skoðum punkta með \(i= j+(k-1)(N+1)=j+(N+1)\), þar sem \(j=1, \dots, N+1\), og svo framvegis.

\sphinxstylestrong{Uppbygging forrits}

Til þess að byggja upp jöfnuhneppi \(A{\mathbf c}={\mathbf b}\) sem nálgar gildi lausna \(u(x_j,y_k)\) á \eqref{equation:Kafli05:eq.exercise}, þurfum við líta á eftirfarandi tilvik:
\begin{enumerate}
\def\theenumi{\arabic{enumi}}
\def\labelenumi{\theenumi .}
\makeatletter\def\p@enumii{\p@enumi \theenumi .}\makeatother
\item {} 
Innri punktar

\end{enumerate}
\begin{quote}

Skoðum innri punkta \((x_j,y_k)\)  þar sem \(1<j<N+1\) og \(1<k<M+1\). Við notum nálgunarjöfnuna \eqref{equation:Kafli05:eq.approx2} hér, og við fáum fylkjastök \(a_{i,i-\ell}, a_{i,i-1}, a_{i,i+1}, a_{i,i+\ell}\) og \(a_{i,i}\), þar sem \(\ell=N+1\). Í hægri hlið hneppisins er \(f_i=f(x_j,y_k)\).
\end{quote}
\begin{enumerate}
\def\theenumi{\arabic{enumi}}
\def\labelenumi{\theenumi .}
\makeatletter\def\p@enumii{\p@enumi \theenumi .}\makeatother
\setcounter{enumi}{1}
\item {} 
Punktar á jöðrum en ekki hornpunktar

\end{enumerate}
\begin{quote}

Skoðum punkta á jaðrinum \(\partial D\) sem eru ekki hornpunktar. Ef \(\beta(x_j, y_k)=0\), þá gildir að
\end{quote}
\begin{equation*}
\begin{split}c_i=\gamma(x_j,y_k)/\alpha(x_j,y_k).\end{split}
\end{equation*}
Það þýðir að jöfnuhneppið hér er \(a_{i,i}=1\) og í hægri hliðinni \(b_i=\gamma(x_j,y_k)/\alpha(x_j,y_k)\).
\begin{quote}

Ef \(\beta(x_j,y_k))\neq 0\), þá þurfum við nota  nálgunarjöfnuna \eqref{equation:Kafli05:eq.approx2}.
\end{quote}
\begin{enumerate}
\def\theenumi{\arabic{enumi}}
\def\labelenumi{\theenumi .}
\makeatletter\def\p@enumii{\p@enumi \theenumi .}\makeatother
\setcounter{enumi}{2}
\item {} 
Hornpunktar

\end{enumerate}
\begin{quote}

Ef \(\beta(x_j, y_k)=0\),  vitum við eins og áður hvaða gildi fallið \(u\) hefur í punktunum.

Ef \(\beta(x_j,y_k))\neq 0\), þurfum við að fara varlega, af því að föllin \(\alpha\), \(\beta\) og \(\gamma\) geta verið ósamfelld. Gott er að sjá fyrir sér hvað er að gerast í hornpunktinum, og skrifa upp mismunajöfnur.
\end{quote}
\begin{enumerate}
\def\theenumi{\arabic{enumi}}
\def\labelenumi{\theenumi .}
\makeatletter\def\p@enumii{\p@enumi \theenumi .}\makeatother
\setcounter{enumi}{3}
\item {} 
Gefin gildi í einstaka netpunktum

\end{enumerate}
\begin{quote}

Ef við viljum (eða vitum) að fallið \(u\) tekur gefið gildi \(U_s\) í einhverjum netpunktum, með \(s=1, \dots, \mu\). Við tölusetjum slíka punkta með \(i\). Þá gildir að einu stök fylkisins \(A\) í línu \(i\) sem eru ekki núll eru \(a_{i,i}=1\). Á hægri hliðinni höfum við \(b_i=U_s\).
\end{quote}

Þegar við höfum reiknað út fylkið \(A\) og vigurinn \(b\), þá getum við notað t.d. Matlab, (eða Octave eða Mathematica eða Maple…), til þess að fá nágunargildin \(c_i\).
T.d. getum við skrifað í Matlab:
\begin{quote}

S=sparse(A); c=Sb;
\end{quote}

Það er líka gott að teikna graf lausnarinnar.
T.d. getum við skrifað í Matlab:
\begin{quote}

surf(x,y,W’),
\end{quote}
\begin{description}
\item[{og jafnhæðarlínur hennar má teikna með}] \leavevmode
contour(x,y,W’).

\end{description}


\chapter{Bútaaðferðir}
\label{\detokenize{Kafli06:butaaferir}}\label{\detokenize{Kafli06::doc}}
Í þessum kafla fjöllum við um \textit{bútaaðferð}. Eins og fyrir mismunaaðferðir eru til ólíkar útgáfur og við ætlum að skoða einföldustu tilfellin. Ennfremur,  ætlum við að læra bútaaðferðir fyrir jaðargildisverkefni í \(\mathbb R\) og \(\mathbb{R}^2\).


\section{Hlutheildun, innfeldi og tvílínulegt form}
\label{\detokenize{Kafli06:hlutheildun-innfeldi-og-tvilinulegt-form}}

\subsection{Jaðargildisverkefni í \protect\(\mathbb R\protect\)}
\label{\detokenize{Kafli06:jaargildisverkefni-i-mathbb-r}}\phantomsection\label{\detokenize{Kafli06:ch-6-1-1}}
Jaðargildisverkefnin í \(\mathbb R\) sem við viljum leysa eru
\begin{equation}\label{equation:Kafli06:eq.system1}
\begin{split}\begin{cases}
Lu=-(pu')'+qu=f,& \text{ á } ]a,b[,\\
B_1u=\alpha_1u(a)-\beta_1u'(a)=\gamma_1,&(\alpha_1,\beta_1)\neq (0,0),\\
B_2u=\alpha_2u(b)+\beta_2u'(b)=\gamma_2,&(\alpha_2,\beta_2)\neq (0,0).
\end{cases}\end{split}
\end{equation}
Þá er  afleiðuvirkinn af Sturm-Liouville gerð, og við gerum ráð fyrir að \(p\) sé samfellt diffranlegt á bili \([a,b]\) og \(q\) sé samfellt á  \([a,b]\).

Við skilgreinum \(V\) sem mengi raungildra falla sem eru samfelld og samfellt diffranleg á köflum á bilinu  \([a,b]\), þ.e.a.s.
\begin{equation*}
\begin{split}V=\{ \varphi: [a,b]\subset{\mathbb  R}\to {{\mathbb  R}}~; \varphi \in C[a,b]\cap PC^1[a,b] \}.\end{split}
\end{equation*}
Athugum að
\begin{enumerate}
\def\theenumi{\arabic{enumi}}
\def\labelenumi{\theenumi .}
\makeatletter\def\p@enumii{\p@enumi \theenumi .}\makeatother
\item {} 
\(\varphi^\prime\) er heildanlegt fall á bilinu \([a,b]\) og undirstöðusetningin gildir í \(V\), þ.e.a.s.

\end{enumerate}
\begin{equation*}
\begin{split}\varphi(x)=\varphi(c)+\int_c^x\varphi'(t)\, dt, \qquad
x,c\in [a,b],\end{split}
\end{equation*}
af því að enda þótt \(\varphi\) sé ekki diffranlegt t.d. í \(x_\ell\), þá er \(\varphi\) í \(PC^1[a,b]\), þ.e.a.s. markgildi frá vinstri og hægri af afleiðunni \(\varphi^\prime\) eru til í \(x_\ell\),þ.e. \(\varphi^\prime(x_\ell+)\) og  \(\varphi^\prime(x_\ell-)\) eru til og eru endanleg. Með öðrum orðum getum við alltaf skilgreint afleiðuna í \(x_\ell\) með
\begin{equation*}
\begin{split}\varphi^\prime(x_\ell)=\tfrac 12\big(\varphi^\prime(x_\ell+)+\varphi^\prime(x_\ell-)\big).\end{split}
\end{equation*}\begin{enumerate}
\def\theenumi{\arabic{enumi}}
\def\labelenumi{\theenumi .}
\makeatletter\def\p@enumii{\p@enumi \theenumi .}\makeatother
\setcounter{enumi}{1}
\item {} 
Ennfremur gildir hlutheildun, þ.e. ef \(\varphi, \psi \in V\) þá er

\end{enumerate}
\begin{equation*}
\begin{split}-\int_c^d\psi'(x)\varphi(x)\, dx
=\int_c^d\psi(x)\varphi'(x) \, dx
-\big[\psi(x)\varphi(x)\big]_c^d
\qquad c,d\in [a,b].\end{split}
\end{equation*}
Munið að í \DUrole{xref,std,std-ref}{3.2.2} skilgreindum við
\begin{enumerate}
\def\theenumi{\Roman{enumi}}
\def\labelenumi{\theenumi .}
\makeatletter\def\p@enumii{\p@enumi \theenumi .}\makeatother
\item {} 
Innfeldi

\end{enumerate}

Fyrir  tvö raungild heildanleg föll \(\varphi\) og \(\psi\) á bilinu \([a,b]\), þá er innfeldi þeirra skilgreint með
\begin{equation*}
\begin{split}{{\langle \varphi,\psi\rangle}}=\int_a^b \varphi(x)\psi(x)\, dx
=\int_a^b \varphi \psi \, dx.\end{split}
\end{equation*}
Ljóst er að innfeldi er vel skilgreint fyrir föllin í \(V\).
\begin{enumerate}
\def\theenumi{\Roman{enumi}}
\def\labelenumi{\theenumi .}
\makeatletter\def\p@enumii{\p@enumi \theenumi .}\makeatother
\setcounter{enumi}{1}
\item {} 
Tvílínulegt form sem \(L\) gefur af sér

\end{enumerate}

Látum \(L\) vera afleiðuvirkja af Sturm-Liouville gerð. Við skilgreinum tvílínulega formið sem \(L\) gefur af sér með
\begin{equation*}
\begin{split}{{\langle \varphi,\psi\rangle}}_L=\int_a^b\big(p\varphi' \psi'
+q\varphi \psi\big)\, dx,
\qquad \varphi, \psi \in V.\end{split}
\end{equation*}
Nú ætlum við að nota þetta til þess að undirbúa nálgunarformúlur fyrir \eqref{equation:Kafli06:eq.system1}.
Við tökum \(\varphi\in V\) og \(v\in \mathcal C^2([a,b])\) og við reiknum eftirfarandi innfeldi út
\begin{equation*}
\begin{split}{{\langle Lv,\varphi\rangle}} &=& -\int_a^b {d\over dx}\left(p(x){dv\over dx} \right)\varphi dx+ \int_a^b q(x) v(x) \varphi(x) dx
\\
    &=&
-\big[pv'\varphi\big]_a^b+ \int_a^b\big(pv'\varphi'+qv\varphi \big) \, dx.\end{split}
\end{equation*}
Við sjáum að það er
\begin{equation*}
\begin{split}{{\langle Lv,\varphi\rangle}}={{\langle v,\varphi\rangle}}_L-\big[pv'\varphi\big]_a^b.\end{split}
\end{equation*}\begin{enumerate}
\def\theenumi{\arabic{enumi}}
\def\labelenumi{\theenumi .}
\makeatletter\def\p@enumii{\p@enumi \theenumi .}\makeatother
\item {} 
Ef við gerum ráð fyrir að \(\varphi\) uppfylli eftirfarandi jaðarskilyrði

\end{enumerate}
\begin{equation*}
\begin{split}\varphi(a)=\varphi(b)=0\end{split}
\end{equation*}
þá verður innfeldið \(\langle Lv,\varphi\rangle\)
\begin{equation*}
\begin{split}{{\langle Lv,\varphi\rangle}}={{\langle v,\varphi\rangle}}_L\,.\end{split}
\end{equation*}\begin{enumerate}
\def\theenumi{\arabic{enumi}}
\def\labelenumi{\theenumi .}
\makeatletter\def\p@enumii{\p@enumi \theenumi .}\makeatother
\setcounter{enumi}{1}
\item {} 
Ef við gerum ráð fyrir að \(v=u\) sé \sphinxstylestrong{lausn á afleiðujöfnunni} \eqref{equation:Kafli06:eq.system1}, þá er \(Lu=f\) og innfeldið verður

\end{enumerate}
\begin{equation}\label{equation:Kafli06:eq.cond1}
\begin{split}{{\langle f,\varphi\rangle}}={{\langle u,\varphi\rangle}}_L, \qquad \varphi\in V, ~ \varphi(a)=\varphi(b)=0\,.\end{split}
\end{equation}

\subsection{Jaðargildisverkefnin í \protect\(\mathbb{R}^2\protect\)}
\label{\detokenize{Kafli06:jaargildisverkefnin-i-mathbb-r-2}}\phantomsection\label{\detokenize{Kafli06:ch-6-1-2}}
Við viljum halda áfram á svipaðan hátt í \(\mathbb{R}^2\). Nú er jaðargildisverkefnið í \(D\subset \mathbb{R}^2\)
\begin{equation}\label{equation:Kafli06:eq.system2}
\begin{split}\begin{cases}
Lu=-\nabla\cdot (p\nabla u)+qu=f, \qquad \text{ á } D,\\
\alpha u+\beta\dfrac{\partial u}{\partial n}
=\gamma, \qquad  \text{á } \ \partial D,
\end{cases}\end{split}
\end{equation}
og \(p\in C^1(D)\), \(q\) of \(f\) eru samfelld á \(D\). Athugum að \(p, q, f\) eru föll á \(D\subset \mathbb{R}^2\), og \(\gamma, \alpha, \beta\) eru föll á \(\partial D\subset \mathbb{R}^2\).

Athugum
\begin{enumerate}
\def\theenumi{\arabic{enumi}}
\def\labelenumi{\theenumi .}
\makeatletter\def\p@enumii{\p@enumi \theenumi .}\makeatother
\item {} 
Leibniz reglan í \(\mathbb{R}^n\)

\end{enumerate}
\begin{equation*}
\begin{split}\nabla\cdot \big(\varphi p\nabla u\big)=
\big(\nabla\cdot(p\nabla u)\big)\varphi
+p\nabla u  \cdot \nabla \varphi,\end{split}
\end{equation*}
af því að
\begin{equation*}
\begin{split}\nabla\cdot (\varphi {\mathbf V})=(\nabla\cdot {\mathbf V})\varphi  + {\mathbf V} \cdot \nabla  \varphi,\end{split}
\end{equation*}
og hér \(\mathbf V= p\nabla u\).
\begin{enumerate}
\def\theenumi{\arabic{enumi}}
\def\labelenumi{\theenumi .}
\makeatletter\def\p@enumii{\p@enumi \theenumi .}\makeatother
\setcounter{enumi}{1}
\item {} 
Gauss setning

\end{enumerate}
\begin{equation*}
\begin{split}\iint_D \nabla \cdot (\varphi p\nabla u)\, dA
=\int_{\partial D} p\dfrac{\partial u}{\partial n}\varphi \, ds\end{split}
\end{equation*}\begin{enumerate}
\def\theenumi{\arabic{enumi}}
\def\labelenumi{\theenumi .}
\makeatletter\def\p@enumii{\p@enumi \theenumi .}\makeatother
\setcounter{enumi}{2}
\item {} 
Hlutheildun í \(\mathbb{R}^2\)

\end{enumerate}

Við sjáum að úr 1. og 2. fáum við
\begin{equation*}
\begin{split}-\iint\limits_D\nabla\cdot \big( p\nabla  u\big) \varphi \, dA=
-\int\limits_{\partial D} p\dfrac{\partial u}{\partial n}\varphi\, ds
+\iint\limits_D p\nabla  u\cdot \nabla  \varphi\, dA.\end{split}
\end{equation*}
Munið að
\begin{enumerate}
\def\theenumi{\arabic{enumi}}
\def\labelenumi{\theenumi .}
\makeatletter\def\p@enumii{\p@enumi \theenumi .}\makeatother
\item {} 
Innfeldi

\end{enumerate}

Gerum ráð fyrir að \(\varphi\) og \(\psi\) séu tvö raungild heildanleg föll á \(\bar D= D \cap \partial D\), þá er innfeldi þeirra
\begin{equation*}
\begin{split}{{\langle \varphi,\psi\rangle}}=
\iint_D \varphi(x,y)\psi(x,y)\, dxdy
=\iint_D \varphi\psi \, dA.\end{split}
\end{equation*}\begin{enumerate}
\def\theenumi{\arabic{enumi}}
\def\labelenumi{\theenumi .}
\makeatletter\def\p@enumii{\p@enumi \theenumi .}\makeatother
\setcounter{enumi}{1}
\item {} 
Tvílínulegt form sem \(L\) gefur af sér

\end{enumerate}

Látum \(L\) vera hlutafleiðuvirkja eins og í verkefninu \eqref{equation:Kafli06:eq.system1}, og gerum ráð fyrir að \(\varphi\) og \(\psi\) séu þ.a. fyrsta stigs hlutafleiður þeirra séu vel skilgreindar og takmarkaðar á \(D\). Þá skilgreinum við tvílínulega formið með
\begin{equation*}
\begin{split}{{\langle \varphi,\psi\rangle}}_L=\iint\limits_D\big(p\, \nabla  \varphi\cdot \nabla
\psi +q\, \varphi\psi\big)\, dA.\end{split}
\end{equation*}
Við skoðum nú innfeldi milli \(Lv\) og \(\phi\), þar sem \(L\) er virkinn í \eqref{equation:Kafli06:eq.system2}. Við gerum ráð fyrir að \(v\in C^2(\overline D)\).
Þá er
\begin{equation*}
\begin{split}\begin{aligned}
\langle L v, \varphi\rangle=\iint_D \big(Lv\big) \varphi\, dA
 &=\iint_D\big( p\,  \nabla v\cdot \nabla \varphi+q v\varphi\big)  \,
 dA-\int_{\partial D}p\dfrac{\partial v}{\partial n} \varphi \, ds\\
 &={{\langle v,\varphi\rangle}}_L
 -\int_{\partial D}p\dfrac{\partial v}{\partial n} \varphi \, ds.\end{aligned}\end{split}
\end{equation*}
Við sjáum að
\begin{enumerate}
\def\theenumi{\arabic{enumi}}
\def\labelenumi{\theenumi .}
\makeatletter\def\p@enumii{\p@enumi \theenumi .}\makeatother
\item {} 
Ef \(\varphi\) er núll á jaðrinum \(\partial D\), þá er

\end{enumerate}
\begin{equation*}
\begin{split}\langle L v, \varphi\rangle={{\langle v,\varphi\rangle}}_L\end{split}
\end{equation*}\begin{enumerate}
\def\theenumi{\arabic{enumi}}
\def\labelenumi{\theenumi .}
\makeatletter\def\p@enumii{\p@enumi \theenumi .}\makeatother
\setcounter{enumi}{1}
\item {} 
Ef \(v=u\) er lausn á jaðarverkefni \eqref{equation:Kafli06:eq.system2}, þá gildir

\end{enumerate}
\begin{equation}\label{equation:Kafli06:eq.cond2}
\begin{split}{{\langle u,\varphi\rangle}}_L={{\langle f,\varphi\rangle}}, \qquad \varphi\in C^1(\overline
D), \quad \varphi=0 \text{ á } \partial D.\end{split}
\end{equation}

\section{Aðferð Galerkins fyrir Dirichlet-verkefnið}
\label{\detokenize{Kafli06:afer-galerkins-fyrir-dirichlet-verkefni}}

\subsection{Galerkin-aðferðir í einni vídd fyrir Dirichlet-verkefni}
\label{\detokenize{Kafli06:galerkin-aferir-i-einni-vidd-fyrir-dirichlet-verkefni}}\phantomsection\label{\detokenize{Kafli06:ch-6-2-1}}
Við lítum á jaðargildisverkefnið \eqref{equation:Kafli06:eq.system1} í sértilfellinu þegar \(\beta_1 =\beta_2=0\), þ.e.a.s. við höfum Dirichlet-verkefni:
\begin{equation}\label{equation:Kafli06:eq.diri1}
\begin{split}\begin{cases}
Lu=-(pu')'+qu=f,& \text{ á } ]a,b[,\\
u(a)=\gamma_1/\alpha_1, \quad  u(b)=\gamma_2/\alpha_2.
\end{cases}\end{split}
\end{equation}
Aðalatriðið í Galerkin-aðferð er að smíða nálgunarfall \(v(x)\) fyrir lausn \(u\) á Dirichlet-verkefninu að ofan á eftirfarandi hátt
\begin{equation*}
\begin{split}v(x)=\psi_0(x)+c_1\varphi_1(x)+\cdots+c_N\varphi_N(x),\end{split}
\end{equation*}
þar sem
\begin{enumerate}
\def\theenumi{\arabic{enumi}}
\def\labelenumi{\theenumi .}
\makeatletter\def\p@enumii{\p@enumi \theenumi .}\makeatother
\item {} 
fallið \(\psi_0(x)\) er valið þ.a. það uppfyllir jaðarskilyrðin í \eqref{equation:Kafli06:eq.diri1}, þ.e.a.s.

\end{enumerate}
\begin{equation*}
\begin{split}\psi_0(a)=\gamma_1/\alpha_1, \qquad  \psi_0(b)=\gamma_2/\alpha_2,\end{split}
\end{equation*}\begin{enumerate}
\def\theenumi{\arabic{enumi}}
\def\labelenumi{\theenumi .}
\makeatletter\def\p@enumii{\p@enumi \theenumi .}\makeatother
\setcounter{enumi}{1}
\item {} 
föllin \(\varphi_1,\dots,\varphi_N\) eru valin þ.a. þau uppfylla óhliðruðu jaðarskilyrðin, þ.e.a.s.

\end{enumerate}
\begin{equation*}
\begin{split}\varphi_j(a)=\varphi_j(b)=0,  \qquad j=1, \dots, N,\end{split}
\end{equation*}\begin{enumerate}
\def\theenumi{\arabic{enumi}}
\def\labelenumi{\theenumi .}
\makeatletter\def\p@enumii{\p@enumi \theenumi .}\makeatother
\setcounter{enumi}{2}
\item {} 
stuðlanir \(c_1, \dots, c_N\) eru óþekktir, og markmiðið er að reikna þá út.

\end{enumerate}

Það er ljóst að nálgunarfallið \(v\) uppfyllir jaðarskilyrðin í \eqref{equation:Kafli06:eq.diri1} \sphinxstyleemphasis{by construction}, þ.e.a.s.
\begin{equation*}
\begin{split}v(a)=\gamma_1/\alpha_1, \qquad  v(b)=\gamma_2/\alpha_2.\end{split}
\end{equation*}
Hvernig getum við fundið nálgunargildi \(c_1, \dots, c_N\)?
Við krefjumst að \(v\) uppfylli jöfnu \eqref{equation:Kafli06:eq.cond1}, þá er
\begin{equation*}
\begin{split}{{\langle v,\varphi_j\rangle}}_L={{\langle f,\varphi_j\rangle}}, \qquad j=1,2,\dots,N.\end{split}
\end{equation*}
Við sjáum að þetta er jafngilt því að
\begin{equation*}
\begin{split}{{\langle \psi_0,\varphi_j\rangle}}_L+\sum_{k=1}^Nc_k{{\langle \varphi_k,\varphi_j\rangle}}_L
={{\langle f,\varphi_j\rangle}}, \qquad j=1,\dots,N.\end{split}
\end{equation*}
Nú höfum við \(N\times N\) jöfnuhneppi fyrir \(N\) nálgunargildi, af því að
\begin{equation}\label{equation:Kafli06:eq.matrixG1d}
\begin{split}\begin{bmatrix}
\langle \varphi_1, \varphi_1\rangle_L & \langle \varphi_1, \varphi_2\rangle_L & \dots &\langle \varphi_1, \varphi_N\rangle_L \\
\langle \varphi_2, \varphi_1\rangle_L & \langle \varphi_2, \varphi_2\rangle_L & \dots &\langle \varphi_2, \varphi_N\rangle_L \\
\vdots & \vdots &\ddots &\vdots \\
\langle \varphi_N, \varphi_1\rangle_L & \langle \varphi_N, \varphi_2\rangle_L & \dots & \langle \varphi_N, \varphi_N\rangle_L
\end{bmatrix}
\begin{bmatrix}
c_1 \\ c_2 \\ \vdots \\c_N
\end{bmatrix} =
\begin{bmatrix}
-\langle \psi_0, \varphi_1\rangle_L +\langle f, \varphi_1\rangle \\
-\langle \psi_0, \varphi_2\rangle_L +\langle  f, \varphi_2\rangle\\ \vdots \\ -\langle \psi_0, \varphi_N\rangle_L +\langle  f,\varphi_N\rangle
\end{bmatrix}.\end{split}
\end{equation}
Almennt, ef afleiðuvirki er línulegur, þá er hneppið að ofan línulegt.

\sphinxstylestrong{Þýðing og sambandið við jaðargildisverkefnin}

Við sjáum að hugmyndin að baki aðferð Galerkins er frekar ólik m.v. mismunaaðferð.
Í mismunaaðferðum fáum við algebrujöfnuhneppi úr afleiðujöfnum með því að nálga afleiður með mismunakvótum.
Hér fáum við algebrujöfnuhneppi með því þess að krefjast þess að nálgunarfall uppfylli \sphinxstyleemphasis{veika framsetningu} afleiðujöfnunnar, sem er \eqref{equation:Kafli06:eq.cond1}.

Munið að \({{\langle v,\varphi_j\rangle}}_L=\langle L v, \varphi_j\rangle\), þá segir jafnan \eqref{equation:Kafli06:eq.cond1} okkur að
\begin{equation*}
\begin{split}{{\langle (L v-f),\varphi_j\rangle}}=0.\end{split}
\end{equation*}
Ef \(u\) er nákvæm lausn á jöfnunni \eqref{equation:Kafli06:eq.diri1}, það þýðir að \(Lu=f\), svo \((L v-f)\) er mismunur milli nálgunarfallsins \(v\) og lausnarinnar \(u\), og við krefjumst þess að mismunur þeirra sé  \sphinxstylestrong{þverstæður} m.t.t. fallanna \(\varphi_j\) sem við notum til þess að smiða nálgunarfallið \(v\).

Af hverju? Aðalatriðið er að mismunurinn er lágmarkaður ef hann er þverstæður m.t.t. plansins sem er spannað af \(\varphi_j, ~j=1, \dots, N\), þ.e.
\begin{equation*}
\begin{split}{{\langle L(v-u),\varphi_j\rangle}}=0.\end{split}
\end{equation*}
\begin{sphinxadmonition}{attention}{Athugið:}
Föllin \(\varphi_j~j=1, \dots, N\) þurfa að vera línulega óháð! Annars hefur fylkið í \eqref{equation:Kafli06:eq.matrixG1d} ekki max stétt!
\end{sphinxadmonition}


\subsection{Galerkin-aðferðir í tveimur víddum fyrir Dirichlet-verkefni}
\label{\detokenize{Kafli06:galerkin-aferir-i-tveimur-viddum-fyrir-dirichlet-verkefni}}
Við lítum á jaðargildisverkefnið \eqref{equation:Kafli06:eq.system2} í sértilfellinu þegar \(\beta_1 =\beta_2=0\) á \(\partial D\), þ.e
\begin{equation}\label{equation:Kafli06:eq.diri2}
\begin{split}\begin{cases}
Lu=-\nabla\cdot (p\nabla u)+qu=f, \qquad \text{ á } D,\\
u=\gamma/\alpha, \qquad  \text{á } \ \partial D.
\end{cases}\end{split}
\end{equation}
Við höldum áfram á svipaðan hátt, og við skilgreinum nálgunarfall \(v\)
\begin{equation*}
\begin{split}v(x,y)=\psi_0(x,y)+c_1\varphi_1(x,y)+\cdots+c_N\varphi_N(x,y),
(x,y) \in \bar D,\end{split}
\end{equation*}
þ.a.
\begin{enumerate}
\def\theenumi{\arabic{enumi}}
\def\labelenumi{\theenumi .}
\makeatletter\def\p@enumii{\p@enumi \theenumi .}\makeatother
\item {} 
fall \(\psi_0(x,y)\) uppfyllir eftirfarandi jaðarskilyrði

\end{enumerate}
\begin{equation*}
\begin{split}\psi_0(x,y)=\gamma(x,y)/\alpha(x,y), \qquad (x,y) \in \partial D\end{split}
\end{equation*}\begin{enumerate}
\def\theenumi{\arabic{enumi}}
\def\labelenumi{\theenumi .}
\makeatletter\def\p@enumii{\p@enumi \theenumi .}\makeatother
\setcounter{enumi}{1}
\item {} 
föllin \(\phi_j~j=1, \dots, N\) uppfylla eftirfarandi jaðarskilyrði

\end{enumerate}
\begin{equation*}
\begin{split}\varphi_j(x,y)=0,  \qquad (x,y) \in \partial D, \qquad j=1, \dots, N,\end{split}
\end{equation*}
Það er ljóst að nálgunarfallið  uppfyllir a.m.k. jaðarskilyrðin í \eqref{equation:Kafli06:eq.diri2}.
Eins og áður er markmiðið  að reikna  stuðlana \(c_j\), og til þess að ákvarða þá notum við skilyrði \eqref{equation:Kafli06:eq.cond2},
\begin{equation*}
\begin{split}\langle v, \varphi_j\rangle_L = \langle f , \varphi_j\rangle \,, \qquad j=1, \dots, N,\end{split}
\end{equation*}
sem gefur okkur \(N\) skilyrði fyrir \(c_j\)
\begin{equation*}
\begin{split}{{\langle \psi_0,\varphi_j\rangle}}_L+\sum_{k=1}^Nc_k{{\langle \varphi_k,\varphi_j\rangle}}_L
={{\langle f,\varphi_j\rangle}}, \qquad j=1,\dots,N.\end{split}
\end{equation*}
Eins og áður getum við skrifað \(N \times N\) hneppi, þ.a. \([A]\vec{c} =\vec{b}\), þar sem
\begin{equation*}
\begin{split}A_{jk}={{\langle \varphi_k,\varphi_j\rangle}}_L
={{\langle \varphi_j,\varphi_k\rangle}}_L, \qquad j,k=1,\dots,N,\end{split}
\end{equation*}
og
\begin{equation*}
\begin{split}b_j={{\langle f,\varphi_j\rangle}}-{{\langle \psi_0,\varphi_j\rangle}}_L, \qquad j=1,\dots,N.\end{split}
\end{equation*}
Formlega höfum við sömu hneppi eins og í \(\mathbb R\). En nú erum við í \(\mathbb{R}^2\), þ.e.a.s. innfeldið og tvílínulega formið innihalda tvöfalt heildi (yfir \(x, y\)), sjáið {\hyperref[\detokenize{Kafli06:ch-6-1-2}]{\sphinxcrossref{\DUrole{std,std-ref}{6.1.2}}}}.


\section{Bútaaðferð í einni vídd}
\label{\detokenize{Kafli06:butaafer-i-einni-vidd}}
Hér beinum við  athygli okkar að jaðargildisverkefni í einni vídd þar sem við veljum þúfugrunnföllin til þess að nálga lausn.

Almennt er jaðargildisverkefnið gefið með \eqref{equation:Kafli06:eq.system1}.
Við veljum skiptingu á bili \([a,b]\), þ.e.
\begin{equation*}
\begin{split}a=x_0<x_1<\cdots<x_N=b, ~~~ h:= (b-a)/N, ~~ x_j= a+j h, ~~j=0, \dots, N.\end{split}
\end{equation*}
Munið líka að miðpunktar eru gefnir með
\begin{equation*}
\begin{split}m_j= x_j+ h,~~~~j=0, \dots, N-1.\end{split}
\end{equation*}
Munið að þúfugrunnföllin eru skilgreind þ.a. \(\varphi_j(x_i)=\delta_{ij}\), sjáið {\hyperref[\detokenize{Kafli05:ch-5-2-2}]{\sphinxcrossref{\DUrole{std,std-ref}{5.2.2}}}}.
Sérstaklega, þýðir það að þúfugrunnföllin eru í \(V\), og að \(\varphi_0(a)=1\) og \(\varphi_N(b)=1\).


\subsection{Blönduð jaðarskilyrði í báðum endapunktum}
\label{\detokenize{Kafli06:blondu-jaarskilyri-i-baum-endapunktum}}
Við gerum ráð fyrir að \(\beta_1\neq 0\) og \(\beta_2\neq 0\).

Við skilgreinum nálgunarfallið
\begin{equation*}
\begin{split}v(x)=c_0\varphi_0(x)+\cdots+c_N\varphi_N(x).\end{split}
\end{equation*}
Munið kafla {\hyperref[\detokenize{Kafli06:ch-6-1-2}]{\sphinxcrossref{\DUrole{std,std-ref}{6.1.2}}}}, almennt höfum við
\begin{equation}\label{equation:Kafli06:eq.form1dgenv1}
\begin{split}{{\langle u,\varphi\rangle}}_L + p(a)u'(a)\varphi(a)-p(b)u'(b)\varphi(b)
= {{\langle f,\varphi\rangle}}, \qquad \varphi\in V.\end{split}
\end{equation}
Við sjáum núna að \(\varphi(a), \varphi(b)\) eru ekki núll almennt, svo við þurfum að skoða jaðarliði líka.
Fyrst notum við jaðarskilyrði í \eqref{equation:Kafli06:eq.system1}, þá fáum við
\begin{equation*}
\begin{split}{{\langle u,\varphi\rangle}}_L  +
\dfrac {p(a)}{\beta_1}(\alpha_1u(a)-\gamma_1)\varphi(a)
+\dfrac{p(b)}{\beta_2}(\alpha_2u(b)-\gamma_2)\varphi(b)
={{\langle f,\varphi\rangle}}.\end{split}
\end{equation*}
Nú stingum við í jöfnuna að nálgunarfallið er gefið með samantekt af þúfugrunnföllum og notum \(\varphi_j\) í staðinn fyrir \(\varphi\). Þá er fyrir \(j=0, \dots, N\)
\begin{equation*}
\begin{split}\sum_{i=0}^N c_i{{\langle \varphi_i ,\varphi_j\rangle}}_L  +
\dfrac {p(a)}{\beta_1}(\alpha_1 \sum_{i=0}^N c_i \varphi_i(a)-\gamma_1)\varphi_j(a)
+\dfrac{p(b)}{\beta_2}(\alpha_2\sum_{i=0}^N c_i \varphi_i(b)-\gamma_2)\varphi_j(b)
={{\langle f,\varphi_j\rangle}}.\end{split}
\end{equation*}
Það er gagnlegt að skrifa nálgunarformúlur á fylkjaformi, þ.e.a.s.
\begin{equation*}
\begin{split}A{\mathbf c}={\mathbf b}, ~~~\text{þar sem}~~~ A=\big(a_{jk}\big)_{j,k=0}^N.\end{split}
\end{equation*}
Stök fylkisins \(A\) eru gefin með
\begin{equation*}
\begin{split}a_{ji}= {{\langle \varphi_i ,\varphi_j\rangle}}_L+ \dfrac {p(a)\alpha_1}{\beta_1}\varphi_i(a)\varphi_j(a)+\dfrac {p(b)\alpha_2}{\beta_2}\varphi_i(b)\varphi_j(b),~~~i,j=0, \dots, N,\end{split}
\end{equation*}
og stuðlar vigursins \(\mathbf b\) eru gefnir með
\begin{equation*}
\begin{split}b_j
={{\langle f,\varphi_j\rangle}}+\dfrac {p(a)\gamma_1}{\beta_1}\varphi_j(a)
+\dfrac{p(b)\gamma_2}{\beta_2}\varphi_j(b), ~~~j=0, \dots, N.\end{split}
\end{equation*}
Við viljum skoða jöfnuhneppið nánar. Munið

T.d. fyrir \(j=0\) þurfum við bara að reikna eftirfarandi stök
\begin{equation*}
\begin{split}\begin{aligned}
a_{00}&=\int_{x_0}^{x_1}\big(p(\varphi_0')^2+q\varphi_0^2\big)\, dx+
\dfrac{p(a)\alpha_1}{\beta_1}
\\
a_{01}&=
\int_{x_0}^{x_1}\big(p\varphi_0'\varphi_1'+q\varphi_0\varphi_1\big)\,
dx
\end{aligned}\end{split}
\end{equation*}
af því að \(\varphi_0\) hefur stoð á bili \([x_0,x_1]\), \(\varphi_j\) með \(j=1, \dots, N-1\) er ekki núll bara yfir bilið \([x_{j-1},x_{j+1}]\) og \(\varphi_N\) er ekki núll á bili \([x_{N-1},x_{N}]\).

Lítum nú a stuðla hægri hliðarinnar, þá er
\begin{equation*}
\begin{split}b_0=\int_{x_0}^{x_1}f\varphi_0\, dx+\dfrac{p(a)\gamma_1}{\beta_1}
\approx \dfrac{h f(m_0)}2+\dfrac{p(a)\gamma_1}{\beta_1},\end{split}
\end{equation*}
af því að \(\varphi_0(a)=1\) og \(\varphi_0(b)=0\).

Nú viljum við nálga heildið að ofan, við getum haldið áfram eins og áður, t.d.
\begin{equation*}
\begin{split}\begin{aligned}
a_{00}&=\int_{x_0}^{x_1}\big(p(\varphi_0')^2+q\varphi_0^2\big)\, dx+
\dfrac{p(a)\alpha_1}{\beta_1}
\approx \dfrac{p(m_0)}{h}+\dfrac{h q(m_0)}3+\dfrac{p(a)\alpha_1}{\beta_1}
\\
a_{01}&=
\int_{x_0}^{x_1}\big(p\varphi_0'\varphi_1'+q\varphi_0\varphi_1\big)\,
dx
\approx -\dfrac{p(m_0)}{h}+\dfrac{h q(m_0)}6.
\\
b_0&=\int_{x_0}^{x_1}f\varphi_0\, dx+\dfrac{p(a)\gamma_1}{\beta_1}
\approx \dfrac{h f(m_0)}6+\dfrac{p(a)\gamma_1}{\beta_1}.\end{aligned}\end{split}
\end{equation*}
Fyrir \(j=1, \dots, N-1\) þurfum við að reikna stökin \(a_{jj-1},a_{jj},a_{jj+1}\) og líka \(\mathbf{b}_j\). Við notum sömu nálgun fyrir heildið, þá er
\begin{equation*}
\begin{split}\begin{aligned}
a_{j,j-1}&=\int_{x_{j-1}}^{x_j}
\big( p\varphi_{j-1}'\varphi_j'+q\varphi_{j-1}\varphi_j\big)\, dx
\approx -\dfrac{p(m_{j-1})}{h}+\dfrac{h q(m_{j-1})}6,\\
a_{j,j}&=\int_{x_{j-1}}^{x_{j+1}}
\big( p(\varphi_j')^2+q \varphi_j^2\big)\, dx
\approx \dfrac{p(m_{j-1})}{h}+\dfrac{p(m_j)}{h}
+\dfrac{(q(m_{j-1})+ q(m_j))h}3,\\
a_{j,j+1}&=\int_{x_j}^{x_{j+1}}
\big( p\varphi_j'\varphi_{j+1}'+q \varphi_j\varphi_{j+1}\big)\, dx
\approx -\dfrac{p(m_j)}{h}
+\dfrac{h q(m_j)}6,\\
b_j&=\int_{x_{j-1}}^{x_{j+1}}f\varphi_j\, dx
\approx \dfrac{h (f(m_{j-1})+f(m_j))}2.\end{aligned}\end{split}
\end{equation*}
Að lokum þurfum við að skoða \(j=N\), nú höfum við að \(\varphi_N(b)=1\), þá fáum við
\begin{equation*}
\begin{split}\begin{aligned}
a_{N,N-1}&=\int_{x_{N-1}}^{x_N}
\big( p\varphi_{N-1}'\varphi_N'+q\varphi_{N-1}\varphi_N\big)\, dx
\approx -\dfrac{p(m_{N-1})}{h}+\dfrac{h q(m_{N-1})}6,\\
a_{NN}&=\int_{x_{N-1}}^{x_{N}}
\big( p\big(\varphi_N'\big)^2+q\varphi_N^2\big)\, dx
+\dfrac{p(b)\alpha_2}{\beta_2}
\approx \dfrac{p(m_{N-1})}{h}
+\dfrac{h q(m_{N-1})}3+\dfrac{p(b)\alpha_2}{\beta_2},\\
b_N&=\int_{x_{N-1}}^{x_{N}}f\varphi_N\, dx+\dfrac{p(b)\gamma_2}{\beta_2}
\approx \dfrac{h f(m_{N-1})}2+\dfrac{p(b)\gamma_2}{\beta_2}.\end{aligned}\end{split}
\end{equation*}

\subsection{Fallsjaðarskilyrði}
\label{\detokenize{Kafli06:fallsjaarskilyri}}
Lítum á jaðargildisverkefnið \eqref{equation:Kafli06:eq.system1}.

Við gerum ráð fyrir að \(\beta_1=0\), þ.e.a.s. að við höfum Dirchlet jaðarskilyrði í vinstri endapunktinum, þ.e. \(u(a)=\gamma_1/\alpha_1\).

Þá  setjum við \(c_0=\gamma_1/\alpha_1\), svo að nálgunarfallið \(v\) tekur gildi \(\gamma_1/\alpha_1\) í punktinum \(a\).
Það þýðir að fyrir \(j=0\) setjum við
\begin{equation*}
\begin{split}a_{00}=1, \quad a_{0j}=0, \ j=1,\dots,N, \ b_0=\gamma_1/\alpha_1,\end{split}
\end{equation*}
og jöfnuhneppið er eins og áður.

Ef við höfum Dirchlet jaðarskilyrði í hægri endapunktinum, þ.e.a.s. að \(\beta_2=0\), þá veljum við \(c_N=\gamma_2/\alpha_2\), svo að nálgunarfallið uppfyllir rétt jaðarskilyrði í \(b\).
Þess vegna setjum við
\begin{equation*}
\begin{split}a_{NN}=1, \quad a_{Nj}=0, \ j=0,\dots,N-1, \ b_N=\gamma_2/\alpha_2.\end{split}
\end{equation*}

\section{Aðferð Galerkins með almennum jaðarskilyrðum}
\label{\detokenize{Kafli06:afer-galerkins-me-almennum-jaarskilyrum}}
Við lítum á jaðargildisverkefnið \eqref{equation:Kafli06:eq.system1} og \eqref{equation:Kafli06:eq.system2}. Hér viljum við ekki tilgreina grunn fyrir nálgunarfall, en  ætlum frekar að ákvarða skilyrði og nálgunarformúlur almennt.

Við skilgreinum \sphinxstyleemphasis{veika framsetningu á jaðargildisverkefnunum} með formúlu
\begin{equation}\label{equation:Kafli06:eq.weakform1d}
\begin{split}{{\langle u,\varphi\rangle}}_{L,B}={{\langle f,\varphi\rangle}}+T_B(\varphi), \qquad
\varphi\in V_B,\end{split}
\end{equation}
þar sem
\begin{enumerate}
\def\theenumi{\arabic{enumi}}
\def\labelenumi{\theenumi .}
\makeatletter\def\p@enumii{\p@enumi \theenumi .}\makeatother
\item {} 
\((\psi,\varphi)\mapsto {{\langle \psi,\varphi\rangle}}_{L,B}\) er tvílínulegt form sem er bæði háð virkjanum \(L\) og jaðarskilyrðunum \(B\),

\item {} 
\(\varphi\mapsto T_B(\varphi)\) er línulegt form sem er háð jaðarskilyrðunum \(B\),

\item {} 
\(V_B\) er mengi af föllum, sem skilgreint er út frá jaðarskilyrðunum.

\end{enumerate}

Við veljum \(\psi_0\) þ.a. fallið uppfylli viðeigandi jaðarskilyrði, og eftir það veljum við \(\varphi_1,\dots,\varphi_N\in V_B\) og krefjumst þess að nálgunarfallið \(v=\psi_0+c_1\varphi_1+\cdots+\varphi_N\) uppfylli línulega jöfnuhneppið \eqref{equation:Kafli06:eq.weakform1d}.

Þá er almennt
\begin{equation*}
\begin{split}{{\langle v,\varphi_j\rangle}}_{L,B}={{\langle f,\varphi_j\rangle}}+T_B(\varphi_j),
\qquad j=1,\dots,N.\end{split}
\end{equation*}
Á fylkjaformi höfum við
\begin{equation*}
\begin{split}a_{jk} &=&{{\langle \varphi_k,\varphi_j\rangle}}_{L,B}
={{\langle \varphi_j,\varphi_k\rangle}}_{L,B}, \qquad j,k=1,\dots,N,\\
b_j &=& {{\langle f,\varphi_j\rangle}}+T_B(\varphi_j)-{{\langle \psi_0,\varphi_j\rangle}}_{L,B},
\qquad j=1,\dots,N,\end{split}
\end{equation*}
sem gefur okkur jöfnuhneppið á fylkjaformi:
\begin{equation*}
\begin{split}A{\mathbf c}={\mathbf b}, ~~~\text{þar sem}~~~ A=\big(a_{jk}\big)_{j,k=1}^N.\end{split}
\end{equation*}
Þá höfum við \(N\) algebrujöfnur fyrir \(N\) nálgunargildi \(c_j, ~j=1, \dots, N\), og við getum reiknað þau út.


\subsection{Í einni vídd}
\label{\detokenize{Kafli06:i-einni-vidd}}
Við skoðum nú \eqref{equation:Kafli06:eq.weakform1d} í ólíkum tilfellum.
Munið samkvæmt kafla {\hyperref[\detokenize{Kafli06:ch-6-1-2}]{\sphinxcrossref{\DUrole{std,std-ref}{6.1.2}}}}, höfum við almennt
\begin{equation}\label{equation:Kafli06:eq.form1dgen}
\begin{split}{{\langle u,\varphi\rangle}}_L + p(a)u'(a)\varphi(a)-p(b)u'(b)\varphi(b)
= {{\langle f,\varphi\rangle}}, \qquad \varphi\in V.\end{split}
\end{equation}\begin{enumerate}
\def\theenumi{\Roman{enumi}}
\def\labelenumi{\theenumi .}
\makeatletter\def\p@enumii{\p@enumi \theenumi .}\makeatother
\item {} 
\sphinxstylestrong{Dirichlet-jaðarskilyrði}

\end{enumerate}

Þá er verkefnið eins og \eqref{equation:Kafli06:eq.diri1} sem við  fjölluðum um í {\hyperref[\detokenize{Kafli06:ch-6-2-1}]{\sphinxcrossref{\DUrole{std,std-ref}{6.2.1}}}}. Þá veljum við \(\psi_0\) þ.a. \(\psi_0(a)=\gamma_1/\alpha_1\) og \(\psi_0(b)=\gamma_2/\alpha_2\).

Hér skilgreinum við mengi falla
\begin{equation*}
\begin{split}V_B=\{\varphi\in V\,;\, \varphi(a)=\varphi(b)=0\},\end{split}
\end{equation*}
og þá er
\begin{equation*}
\begin{split}{{\langle u,\varphi\rangle}}_L={{\langle f,\varphi\rangle}},\end{split}
\end{equation*}
sem segir okkur að
\begin{equation*}
\begin{split}{{\langle \varphi,\psi\rangle}}_{L,B}={{\langle \varphi,\psi\rangle}}_L ~~~\text{og}~~~ T_B(\varphi)=0 ~~~ \varphi,\psi\in V_B.\end{split}
\end{equation*}\begin{enumerate}
\def\theenumi{\Roman{enumi}}
\def\labelenumi{\theenumi .}
\makeatletter\def\p@enumii{\p@enumi \theenumi .}\makeatother
\setcounter{enumi}{1}
\item {} 
\sphinxstylestrong{Dirichlet jaðarskilyrði í vinstri endapunkti}

\end{enumerate}

Lítum á
\begin{equation*}
\begin{split}\begin{cases}
Lu=-(pu')'+qu=f, \\
B_1u=\alpha_1u(a)=\gamma_1,  \\
B_2u=\alpha_2u(b)+{\beta}_2u'(b)=\gamma_2, \quad \beta_2\neq 0.
\end{cases}\end{split}
\end{equation*}
Nú tökum við
\begin{equation*}
\begin{split}V_B=\{\varphi\in V\,;\, \varphi(a)=0\},\end{split}
\end{equation*}
og notum jaðarskilyrði í hægri endapunkti til þess að einfalda tvílínulega formið \eqref{equation:Kafli06:eq.form1dgen}, þ.e.a.s.
\begin{equation*}
\begin{split}{{\langle u,\varphi\rangle}}_L
+\dfrac{p(b)\alpha_2}{\beta_2}u(b)\varphi(b)
={{\langle f,\varphi\rangle}}
+\dfrac{p(b)\gamma_2}{\beta_2}\varphi(b), ~~~ \varphi\in V_B.\end{split}
\end{equation*}
Ef við berum jöfnuna að ofan saman við jöfnu \eqref{equation:Kafli06:eq.weakform1d},  sjáum við að
\begin{equation*}
\begin{split}{{\langle \varphi,\psi\rangle}}_{L,B}={{\langle \varphi,\psi\rangle}}_L
+\dfrac{p(b)\alpha_2}{\beta_2}\varphi(b)\psi(b),
\quad \text{ og } \quad
T_B(\varphi)=
\dfrac{p(b)\gamma_2}{\beta_2}\varphi(b),
\qquad \varphi,\psi\in V_B.\end{split}
\end{equation*}\begin{enumerate}
\def\theenumi{\Roman{enumi}}
\def\labelenumi{\theenumi .}
\makeatletter\def\p@enumii{\p@enumi \theenumi .}\makeatother
\setcounter{enumi}{2}
\item {} 
\sphinxstylestrong{Dirichlet jaðarskilyrði í hægri endapunkti}

\end{enumerate}

Lítum á
\begin{equation*}
\begin{split}\begin{cases}
Lu=-(pu')'+qu=f, \\
B_1u={\alpha}_1u(a)-\beta_1u'(a)=\gamma_1, \quad \beta_1\neq 0 \\
B_2u=\alpha_2u(b)=\gamma_2,
\end{cases}\end{split}
\end{equation*}
og höldum áfram eins og áður. Við skilgreinum
\begin{equation*}
\begin{split}V_B=\{\varphi\in V \,;\, \varphi(b)=0 \},\end{split}
\end{equation*}
og með því að nota jaðarskilyrði verður formið \eqref{equation:Kafli06:eq.form1dgen}
\begin{equation*}
\begin{split}{{\langle u,\varphi\rangle}}_L
+\dfrac {p(a)\alpha_1}{\beta_1}u(a)\varphi(a)
={{\langle f,\varphi\rangle}}
+\dfrac {p(a)\gamma_1}{\beta_1}\varphi(a), \qquad \varphi\in V_B.\end{split}
\end{equation*}
Á svipaðan hátt berum við jöfnuna að ofan saman við \eqref{equation:Kafli06:eq.weakform1d}, og sjáum að hér gildir
\begin{equation*}
\begin{split}{{\langle \varphi,\psi\rangle}}_{L,B}={{\langle \varphi,\psi\rangle}}_L
+\dfrac {p(a)\alpha_1}{\beta_1}\varphi(a)\psi(a)
\quad \text{ og } \quad
T_B(\varphi)=\dfrac {p(a)\gamma_1}{\beta_1}\varphi(a),
\qquad \varphi,\psi\in V_B.\end{split}
\end{equation*}\begin{enumerate}
\def\theenumi{\Roman{enumi}}
\def\labelenumi{\theenumi .}
\makeatletter\def\p@enumii{\p@enumi \theenumi .}\makeatother
\setcounter{enumi}{3}
\item {} 
\sphinxstylestrong{Blönduð jaðarskilyrði í báðum endapunktum}

\end{enumerate}

Jaðargildisverkefnið er
\begin{equation*}
\begin{split}\begin{cases}
Lu=-(pu')'+qu=f, \\
B_1u={\alpha}_1u(a)-\beta_1u'(a)=\gamma_1, \quad \beta_1\neq 0, \\
B_2u=\alpha_2u(b)+{\beta}_2u'(b)=\gamma_2, \quad \beta_2\neq 0.
\end{cases}\end{split}
\end{equation*}
Ef \(\beta_1\neq 0\) og \(\beta_2\neq 0\), tökum við  \(\psi_0\) sem núllfallið, þá er nálgunarfallið gefið með
\begin{equation*}
\begin{split}v(x)=c_1\varphi_1(x)+\cdots+c_N\varphi_N(x),
~~~~x \in [a,b].\end{split}
\end{equation*}
Við getum notað jaðarskilyrðin til þess að einfalda tvílínulega formið \eqref{equation:Kafli06:eq.form1dgen}, þ.e.a.s.
\begin{equation*}
\begin{split}{{\langle u,\varphi\rangle}}_L  +
\dfrac {p(a)}{\beta_1}(\alpha_1u(a)-\gamma_1)\varphi(a)
+\dfrac{p(b)}{\beta_2}(\alpha_2u(b)-\gamma_2)\varphi(b)
={{\langle f,\varphi\rangle}}.\end{split}
\end{equation*}
Ef við berum jöfnuna að ofan saman við jöfnu \eqref{equation:Kafli06:eq.weakform1d},  skiljum við nú hvað \(T_B\) er og restin, þ.e.a.s mengi fallanna er
\begin{equation*}
\begin{split}V_B=V,\end{split}
\end{equation*}
línulega formið \(T_B\) er gefið með
\begin{equation*}
\begin{split}T_B(\varphi)=\dfrac {p(a)\gamma_1}{\beta_1}\varphi(a)
+\dfrac{p(b)\gamma_2}{\beta_2}\varphi(b), \qquad \varphi \in V_B,\end{split}
\end{equation*}
og  tvílínulega formið \({{\langle \varphi,\psi\rangle}}_{L,B}\) er gefið með
\begin{equation*}
\begin{split}{{\langle \varphi,\psi\rangle}}_{L,B}={{\langle \varphi,\psi\rangle}}_L
+\dfrac {p(a)\alpha_1}{\beta_1}\varphi(a)\psi(a)
+\dfrac{p(b)\alpha_2}{\beta_2}\varphi(b)\psi(b),
\qquad \varphi,\psi\in V_B.\end{split}
\end{equation*}

\subsection{Í tveimur víddum}
\label{\detokenize{Kafli06:i-tveimur-viddum}}
Við viljum skoða veiku framsetninguna \eqref{equation:Kafli06:eq.weakform1d} fyrir jaðargildisverkefni í \(\mathbb{R}^2\).
Fyrst er gagnlegt að skrifa jaðargildisverkefnið sem
\begin{equation*}
\begin{split}\begin{cases}
Lu=-\nabla\cdot (p\nabla u)+qu=f, \quad &\text{á } D\\
u=\gamma,\quad &\text{á } \ \partial D_1,\\
\alpha u+\beta\dfrac{\partial u}{\partial n}
=\gamma, \quad  &\text{á } \ \partial D_2,
\end{cases}\end{split}
\end{equation*}
þar sem
\begin{equation*}
\begin{split}\partial_1D=\{(x,y)\in \partial D\,;\, \beta(x,y)=0\}
\qquad \text{ og } \qquad
\partial_2D=\{(x,y)\in \partial D\,;\, \beta(x,y)\neq 0\},\end{split}
\end{equation*}
og \(\partial D=\partial_1D\cup \partial_2 D\) (munið \DUrole{xref,std,std-ref}{5.3}). Við gerum alltaf ráð’fyrir að \(p\in C^1\) og \(q, f\) séu samfelld á \(\bar D\subset\mathbb{R}^2\).

Við höldum áfram eins og áður, þ.e.a.s.
\begin{enumerate}
\def\theenumi{\arabic{enumi}}
\def\labelenumi{\theenumi .}
\makeatletter\def\p@enumii{\p@enumi \theenumi .}\makeatother
\item {} 
Fyrst veljum við fallið \(\psi_0\) þ.a. \(\psi_0(x,y) = \gamma(x,y)\) fyrir öll \((x,y)\in\partial D_1\).

\item {} 
Eftir það, veljum við föllin \(\varphi\) þ.a. \(\varphi(x,y)=0\) fyrir \((x,y)\in\partial D_1\). Það þýðir að við veljum

\end{enumerate}
\begin{equation*}
\begin{split}V_B=\{ \varphi\in \mathcal{C}^2(\mathbb R)~~: ~~ \varphi(x,y)=0, ~~ (x,y)\in \partial D_1 \}.\end{split}
\end{equation*}\begin{enumerate}
\def\theenumi{\arabic{enumi}}
\def\labelenumi{\theenumi .}
\makeatletter\def\p@enumii{\p@enumi \theenumi .}\makeatother
\setcounter{enumi}{2}
\item {} 
Að lokum skilgreinum við nálgunarfallið með \(v=\psi_0+c_1\varphi_1+\cdots+c_N\varphi_N\) og við krefjumst þess að \(v\) uppfylli veiku framsetninguna \eqref{equation:Kafli06:eq.weakform1d}.

\end{enumerate}

Við sjáum nú hvað framsetningin \eqref{equation:Kafli06:eq.weakform1d} gefur okkur í \(\mathbb{R}^2\).
Munið að í kafla {\hyperref[\detokenize{Kafli06:ch-6-1-2}]{\sphinxcrossref{\DUrole{std,std-ref}{6.1.2}}}} reiknuðum við að
\begin{equation*}
\begin{split}\langle L u, \varphi\rangle= \langle u, \varphi\rangle_L - \int_{\partial D} p \dfrac{\partial u}{\partial n} \varphi ds\,,\end{split}
\end{equation*}
en nú tökum við \(\varphi \in V_B\) og jaðarinn er \(\partial D=\partial_1D\cup \partial_2 D\), þá getum við skrifað
\begin{equation*}
\begin{split}\langle L u, \varphi\rangle= \langle u, \varphi\rangle_L - \int_{\partial D_2} p \dfrac{\partial u}{\partial n} \varphi ds\ =
\langle u, \varphi\rangle_L - \int_{\partial D_2} p \dfrac{\gamma -\alpha u}{\beta} \varphi ds\,,\end{split}
\end{equation*}
þar sem í síðasta skrefi höfum við notað jaðarskilyrði í \(\partial D_2\). Nú erum við búin að skrifa niður veiku framsetninguna \eqref{equation:Kafli06:eq.weakform1d} fyrir nálgunarfallið \(v\) í \(\mathbb{R}^2\), þá er
\begin{equation}\label{equation:Kafli06:eq.weakformR2}
\begin{split}\langle v, \varphi\rangle_L + \int_{\partial D_2} p \dfrac{\alpha v}{\beta} \varphi ds = \langle f, \varphi\rangle + \int_{\partial D_2} p \dfrac{\gamma}{\beta} \varphi ds, \qquad \varphi\in V_B.\end{split}
\end{equation}
Við berum formúluna \eqref{equation:Kafli06:eq.weakformR2} saman við almennu stæðuna \eqref{equation:Kafli06:eq.weakform1d}, og við sjáum að hér höfum við
\begin{equation*}
\begin{split}{\langle \varphi,\psi\rangle}_{L,B}=
{{\langle \varphi,\psi\rangle}}_L+\int_{\partial_2D}\dfrac{p\alpha}\beta \varphi\psi\, ds\qquad \varphi,\psi\in V_B,\end{split}
\end{equation*}
og
\begin{equation*}
\begin{split}T_B(\varphi)=\int_{\partial_2D}\dfrac{p\gamma}\beta \varphi\, ds,
\qquad \varphi,\psi\in V_B.\end{split}
\end{equation*}

\subsection{Sýnidæmi}
\label{\detokenize{Kafli06:synidaemi}}
Lítum á eftirfarandi jaðargildisverkefni
\begin{equation}\label{equation:Kafli06:eq.example2d}
\begin{split}\begin{cases}
-\nabla^2 u= -\dfrac{\partial^2 u}{\partial x^2 }-\dfrac{\partial^2 u}{\partial y^2 }=1 &\text{á } \ D,\\
u(x,0)=1-x, &0<x<1,\\
\dfrac{\partial u}{\partial n}(0,y)=1-y, &0<y<1,\\
\dfrac{\partial u}{\partial n}(x,1-x)+u(x,1-x)=0, &0<x<1,
\end{cases}\end{split}
\end{equation}
þar sem \(D\) er
\begin{equation*}
\begin{split}D=\{(x,y)\, ;\, 0<x<1, 0<y<1-x\}.\end{split}
\end{equation*}
Hér höfum við að
\begin{equation*}
\begin{split}&&\partial D_1 =\{(x,0)\, ;\, 0\leq x\leq 1\},\\
&&\partial D_2 =\{(0,y)\, ;\, 0<y\leq 1\}\cup \{(x,1-x)\, ;\, 0< x<1\}.\end{split}
\end{equation*}
Við viljum nota aðferð Galerkins til þess að ákvarða nálgunarlausn af gerðinni
\begin{equation*}
\begin{split}v(x,y)=a+bx+cy+dxy.\end{split}
\end{equation*}
Við byrjum á að skoða Dirichlet skilyrði í \(\partial D_1\), og við veljum fallið \(\psi_0\) þ.a. \(\psi_0(x,0)=1-x\), fyrir \(x\in [0,1]\).
Þá getum við valið
\begin{equation*}
\begin{split}\psi_0(x,y)=1-x, \qquad (x,y)\in \bar D.\end{split}
\end{equation*}
Nú veljum við \(\varphi\) þ.a. \(\varphi(x,0)=0\), fyrir \(x\in [0,1]\), þ.e.a.s.
\begin{equation*}
\begin{split}V_B=\{ \varphi\in C^2(\bar D)~~: ~~\varphi(x,y)=0 \quad (x,y)\in \partial D_1 \}.\end{split}
\end{equation*}
Við þurfum að velja \(\varphi\), en með þetta val á fallinu \(\psi_0\), er það jafngilt að setja \(a=1\) og \(b=-1\). Það vantar bara að velja föll \(\varphi_1, \varphi_2\), sem þurfa að vera núll á jaðrinum \(\partial D_1\).
Við sjáum að einliður \(y\) og \(x y\) eru núll á jaðrinum \(\partial D_1\), þá getum við tekið
\begin{equation*}
\begin{split}\varphi_1 (x,y)=y , \qquad \varphi_2(x,y)= x y.\end{split}
\end{equation*}
Við beitum \eqref{equation:Kafli06:eq.weakformR2}, en fyrst skoðum við jaðarliði í \eqref{equation:Kafli06:eq.weakformR2}.
Athugum að \(p(x,y)=1\),  \(\alpha(0,y)=0\) fyrir \(y\in ]0,1]\), og \(\gamma(x,1-x)=0\) fyrir \(x\in ]0,1[\), þá er
\begin{equation*}
\begin{split}&& \int_{\partial D_2} p \dfrac{\alpha v}{\beta} \varphi ds=
\sqrt 2 \int_0^1 v(x,1-x) \varphi(x,1-x)dx, \\
&& \int_{\partial D_2} p \dfrac{\gamma}{\beta} \varphi ds= \int_0^1 (1-y)\varphi(0,y)dy.\end{split}
\end{equation*}
Athugum að
\begin{equation*}
\begin{split}\nabla \psi_0(x,y)=(-1,0)^T, \quad \nabla \varphi_1(x,y)= (0,1)^T , \quad \nabla \varphi_2(x,y)=(y,x)^T.\end{split}
\end{equation*}
Fyrir \(\varphi_1\) verður veika framsetningin \eqref{equation:Kafli06:eq.weakformR2}
\begin{equation*}
\begin{split}&&\int_D \nabla v \cdot \nabla \varphi_1 dA + \sqrt 2 \int_0^1 v(x,1-x) \varphi_1(x,1-x)dx= \int_D \varphi_1 dA+ \int_0^1 (1-y)\varphi_1(0,y)dy,
\\
&& c_1 \int_D dA+ c_2 \int_D x dA +\sqrt 2 \int_0^1 \left(1+c_1+c_2 x\right)(1-x)^2dx =\int_D y dA + \int_0^1 (1-y)y dy,\end{split}
\end{equation*}
sem gefur okkur
\begin{equation*}
\begin{split}c_1(\tfrac 12 +\tfrac{\sqrt 2}{3})+c_2(\tfrac 16 +\tfrac{\sqrt 2}{12})=(\tfrac 13 +\tfrac{\sqrt 2}{3}).\end{split}
\end{equation*}
Við höldum áfram á svipaðan hátt fyrir \(j=2\), þá er
\begin{equation*}
\begin{split}&&\int_D \nabla v \cdot \nabla \varphi_2 dA + \sqrt 2 \int_0^1 v(x,1-x) \varphi_2(x,1-x)dx= \int_D \varphi_2 dA+ \int_0^1 (1-y)\varphi_2(0,y)dy,
\\
&& \int_D (-y)dA+c_1 \int_D x dA+ c_2 \int_D (x^2+y^2) dA +\sqrt 2 \int_0^1 \left(1+c_1+c_2 x\right)x(1-x)^2dx =\int_D x\, y dA,\end{split}
\end{equation*}
sem gefur okkur
\begin{equation*}
\begin{split}c_1(\tfrac 16 +\tfrac{\sqrt 2}{12})+c_2(\tfrac 16 +\tfrac{\sqrt 2}{30})=(\tfrac{5}{24} -\tfrac{\sqrt 2}{12}).\end{split}
\end{equation*}
Að lokum fáum við
\begin{equation*}
\begin{split}c_1=-0.4360,  ~~~~c_2=1.0034,\end{split}
\end{equation*}
þá er nálgunarfallið gefið með
\begin{equation*}
\begin{split}v(x,y)=1-x-0.4360\, y+1.0034\, xy.\end{split}
\end{equation*}

\section{Bútaaðferð í tveimur víddum}
\label{\detokenize{Kafli06:butaafer-i-tveimur-viddum}}
Við ætlum að líta á jaðargildisverkefni \eqref{equation:Kafli06:eq.system2}, og hér við viljum nota aðferð Galerkins þar sem svæðinu \(\bar D\) er skipt í sammengi lokaðra þríhyrninga og nálgunarfallið er línuleg samantekt af þúfugrunnföllum.


\subsection{Net með þríhyrningum}
\label{\detokenize{Kafli06:net-me-rihyrningum}}\phantomsection\label{\detokenize{Kafli06:ch-6-5-1}}
Við skiptum svæðinu \(\bar D\)  í þríhyrninga, eins og í myndunum að neðan.

\noindent{\hspace*{\fill}\sphinxincludegraphics[width=0.250\linewidth]{{disk-with-trianglegrid}.png}\hspace*{\fill}}

\sphinxstyleemphasis{Hálfri skífu skipt í þríhyrninga.}

\noindent{\hspace*{\fill}\sphinxincludegraphics[width=0.650\linewidth]{{triangle-grid}.png}\hspace*{\fill}}

\sphinxstyleemphasis{Rétthyrningi skipt í þríhyrninga. Hér er} \(N=4\) \sphinxstyleemphasis{og} \(M=2\).

Við skoðum dæmi með rétthyrningnum \(D\)
\begin{equation*}
\begin{split}D=\{ (x,y)\in \mathbb{R}^2, ~~ a<x<b, ~~ c<y<d\}\,.\end{split}
\end{equation*}
Þar höfum við skiptingu á \(x\)-ás
\begin{equation*}
\begin{split}a=x_1 < x_2 < \dots < x_N= b, \qquad x_j= a+ (j-1)h \,,~~ j=1, \dots, N+1\,,\end{split}
\end{equation*}
þar sem \(h=(b-a)/N\), og skiptingu á \(y\)-ás
\begin{equation*}
\begin{split}c=y_1 < y_2 < \dots < y_M= d, \qquad y_p= c+ (p-1)k \,,~~ p=1, \dots, M+1\,,\end{split}
\end{equation*}
þar sem \(k=(d-c)/M\). Hornpunktar \((x_j, y_p)\) þríhyrninganna eru allir í \(\bar D\).
Við veljum að raða punktunum eins og í myndinni, þ.e.a.s. við notum vörpun
\begin{equation*}
\begin{split}\sigma: (j, p) \to \alpha= \sigma(j,p)= j+(p-1)(N+1)\,, ~~ j=1, \dots, N+1\,,~~p=1, \dots, M+1\,,\end{split}
\end{equation*}
svo er \(\alpha=1, \dots, (M+1)(N+1)\).

Sérhverjum þríhyrningi er lýst sem mengi
\begin{equation*}
\begin{split}T_{A,B,C}=\{(x,y)=(1-s-t)(x_A,y_A)+s(x_B,y_B)+t(x_{C},y_{C})
\,;\, s,t\in [0,1], s+t\leq 1\},\end{split}
\end{equation*}
þar sem \((x_A,y_A), (x_B,y_B)\) og \((x_C,y_C)\) eru hornpunktar þríhyrningsins. Á myndinni sjáum við  t.d. þríhyrninginn með hornpunkta 1, 2, 6, við táknum hann með \(T_{1,2,6}\).

\begin{sphinxadmonition}{attention}{Athugið:}
Röð punktanna skiptir máli hér! Við röðum punktunum \sphinxstyleemphasis{rangsælis} eftir jaðri þríhyrningsins.
\end{sphinxadmonition}

Athugum líka að
\begin{equation*}
\begin{split}T_{1,2,6}=T_{6,1,2}=T_{2,6,1}.\end{split}
\end{equation*}
Það er gagnlegt að skoða \sphinxstyleemphasis{einingarþríhyrning} með hornpunkta \((0,0), (1,0)\) og \((0,1)\). Við táknum hann með \(E\) og þá er
\begin{equation*}
\begin{split}E=\{(s,t)\,;\, s,t\in [0,1], s+t\leq 1\}.\end{split}
\end{equation*}
Þá getum við notað vörpun \(t_{A,B,C}\) til þess að varpa einingarþríhyrningnum í þríhyrninginn \(T_{A,B,C}\), þá er
\begin{equation*}
\begin{split}t_{A,B,C}: ~&E \to T_{A,B,C}\\
& (s,t) \mapsto (x,y)=(1-s-t)(x_A,y_A)+s(x_B,y_B)+t(x_{C},y_{C}).\end{split}
\end{equation*}
Við getum umritað vörpunina á fylkjaform á eftirfarandi hátt
\begin{equation}\label{equation:Kafli06:eq.maptriangle}
\begin{split}\left[\begin{matrix} x \\ y  \end{matrix}\right]
=
\left[\begin{matrix} x_A \\ y_A  \end{matrix}\right]+
\left[\begin{matrix}   x_B-x_A & x_C-x_A
\\ y_B-y_A & y_C-y_A
\end{matrix}\right]
\left[\begin{matrix}   s\\ t \end{matrix}\right].\end{split}
\end{equation}
Athugum að vörpunin er gagntæk, og andhverfan \(t^{-1}_{A,B,C}\) er gefin með
\begin{equation*}
\begin{split}t^{-1}_{A,B,C}: ~&T_{A,B,C} \to E\\
& \left[\begin{matrix}   x\\ y \end{matrix}\right] \mapsto  \left[\begin{matrix}   s\\ t \end{matrix}\right]
=
\frac{1}{d}\left[\begin{matrix}   y_C-y_A & -(x_C-x_A)
\\ -(y_B-y_A) & x_B-x_A
\end{matrix}\right]
\left[\begin{matrix}   x-x_A\\ y-y_A \end{matrix}\right],\end{split}
\end{equation*}
þar sem \(d\) er ákveða fylksins í \eqref{equation:Kafli06:eq.maptriangle}.

Seinna munum við nota flatarmál þríhyrningsins \(T_{A,B,C}\) og massamiðju \(M_{A,B,C}\), og þau eru gefin með
\begin{equation*}
\begin{split}&& area(T_{A,B,C})= \frac{|d|}{2}, \\
&& M_{A,B,C}=\tfrac 13\big((x_A,y_A)+(x_B,y_B)+(x_C,y_C)\big).\end{split}
\end{equation*}

\subsection{Þúfugrunnföll}
\label{\detokenize{Kafli06:ufugrunnfoll}}
Við ætlum að nota þúfugrunnföll til þess að nálga lausn á \eqref{equation:Kafli06:eq.system2}. Við skilgreinum þúfugrunnföll á \(\bar D\) á eftirfarandi hátt:
\begin{equation*}
\begin{split}&&\varphi_A: T_{A,B,C} \to [0,1] \,, \text{þ.a.}~~
\varphi(x_A,y_A)=1\,, ~~\varphi(x_B,y_B)=\varphi(x_C,y_C)=0 \,
\\
&& \varphi_A ~~\text{er samfellt og línulegt}.\end{split}
\end{equation*}
Við sjáum í dæmi að neðan graf fallsins \(\varphi_3\) fyrir einingarþríhyrninginn \(E_{1,2,3}\).
Það er ljóst að graf fallsins \(\varphi_A\) er plan í \(\mathbb{R}^3\) sem tengir punktana
\begin{equation*}
\begin{split}(x_A,y_A,1), ~~~ (x_B,y_B,0), ~~~ (x_C,y_C,0).\end{split}
\end{equation*}
\noindent{\hspace*{\fill}\sphinxincludegraphics[width=0.450\linewidth]{{example-phi-triangle}.png}\hspace*{\fill}}

\sphinxstyleemphasis{Dæmi um graf fallsins} \(\varphi_3\) \sphinxstyleemphasis{skilgreint yfir einingarþríhyrninginn} \(E_{1,2,3}\).

Við skilgreinum fall \(\varphi_E\) eins og grunnfallið á einingarþríhyrningnum \(E\) sem tekur gildið 1 í punktinum \((0,0)\). Þá er
\begin{equation*}
\begin{split}\varphi_E(s,t)=1-s-t, \qquad (s,t)\in E,\end{split}
\end{equation*}
og við fáum \(\varphi_E(0,0)=1\) og \(\varphi_E(1,0)=\varphi_E(0,1)=0\).

Hvernig getum við smíðað fallið \(\varphi_A\) alment? Við notum vörpunina \(t_{A,B,C}\), þ.e.a.s. við vörpum þríhyrningnum \(T_{A,B,C}\) í einingarþríhyrninginn \(E\) og við lesum úr því \(\varphi_E\), þ.e.
\begin{equation*}
\begin{split}\varphi_A(x,y)=\varphi_E(t_{A,B,C}^{-1}(x,y)).\end{split}
\end{equation*}
Ef við viljum t.d. skrifa niður \(\varphi_A\), þá er
\begin{equation}\label{equation:Kafli06:eq.defvarphiA}
\begin{split}\varphi_A(x,y)=\tfrac 1d \left((x_C-x_B)y-(y_C-y_B)x+x_B y_C-x_C y_B\right),\end{split}
\end{equation}
og það er ljóst að \(\varphi_A(x_B,y_B)=\varphi_A(x_C,y_C)=0\) og \(\varphi_A(x_A,y_A)=1\).

Athugum að fallið \(\varphi_B\) á \(T_{A,B,C}\), sem er skilgreint eins og \(\varphi_B(x_A,y_A)=\varphi_B(x_C,y_C)=0\) og \(\varphi_B(x_B,y_B)=1\), er gefið með
\begin{equation}\label{equation:Kafli06:eq.defvarphiB}
\begin{split}\varphi_B(x,y)=\tfrac 1d \left((x_A-x_C)y-(y_A-y_C)x+x_C y_A-x_A y_C\right).\end{split}
\end{equation}
Það er hjálplegt að skoða einginleika fallanna \(\varphi_A\), af því að við ætlum að nota þá til þess að reikna út veiku framsetningu jaðargildisverkefnisins.

\sphinxstylestrong{Eiginleikar þúfugrunnfallanna}

Fyrst ætlum við að skoða eiginleika þúfugrunnfallanna sem við munum nota seinna.
Við lítum á \(\varphi_A\) og \(\varphi_B\) sem eru skilgreind á \eqref{equation:Kafli06:eq.defvarphiA} og \eqref{equation:Kafli06:eq.defvarphiB}.
\begin{enumerate}
\def\theenumi{\arabic{enumi}}
\def\labelenumi{\theenumi .}
\makeatletter\def\p@enumii{\p@enumi \theenumi .}\makeatother
\item {} 
Stigull fallsins \(\varphi_A\) er gefinn með

\end{enumerate}
\begin{equation*}
\begin{split}\nabla\varphi_A(x,y)={1\over d}\left(-(y_C-y_B),(x_C-x_B)\right).\end{split}
\end{equation*}\begin{enumerate}
\def\theenumi{\arabic{enumi}}
\def\labelenumi{\theenumi .}
\makeatletter\def\p@enumii{\p@enumi \theenumi .}\makeatother
\setcounter{enumi}{1}
\item {} 
Þá er eftirfarandi heildi gefið með

\end{enumerate}
\begin{equation*}
\begin{split}\int_{T_{A,B,C}}\nabla\varphi_A \cdot \nabla\varphi_A \,dx dy={1\over 2 |d|} \left((x_C-x_B)^2+(y_C-y_B)^2\right).\end{split}
\end{equation*}\begin{enumerate}
\def\theenumi{\arabic{enumi}}
\def\labelenumi{\theenumi .}
\makeatletter\def\p@enumii{\p@enumi \theenumi .}\makeatother
\setcounter{enumi}{2}
\item {} 
Fyrir eftirfarandi heildi, fáum við

\end{enumerate}
\begin{equation*}
\begin{split}\int_{T_{A,B,C}}\nabla\varphi_A \cdot \nabla\varphi_B\, dx dy={1\over 2 |d|} \left(-(x_B-x_C)(x_A-x_C)-(y_B-y_C)(y_A-y_C)\right).\end{split}
\end{equation*}\begin{enumerate}
\def\theenumi{\arabic{enumi}}
\def\labelenumi{\theenumi .}
\makeatletter\def\p@enumii{\p@enumi \theenumi .}\makeatother
\setcounter{enumi}{3}
\item {} 
Athugum að

\end{enumerate}
\begin{equation*}
\begin{split}\varphi_A(M_{C,B,C})=\varphi_A\left(\tfrac 13(x_A+x_B+x_C),\tfrac 13(y_A+y_B+y_C)\right)=\tfrac 13.\end{split}
\end{equation*}
\sphinxstylestrong{Ritháttur í kennslubókinni}

Við getum notað sama rithátt og í kennslubókinni, þá skilgreinum við eftirfarandi \sphinxstyleemphasis{hliðarvigra}
\begin{equation*}
\begin{split}{\mathbf l}_A=(x_C-x_B,y_C-y_B), \qquad
{\mathbf l}_B=(x_A-x_C,y_A-y_C), \qquad
{\mathbf l}_C=(x_B-x_A,y_B-y_A).\end{split}
\end{equation*}
Við sjáum að hliðarvigrarnir liggja á mótlægum hliðum \(T_{A,B,C}\) við hornpunkta númer \(A, B\) og \(C\) miðað við rangsælis umferðarstefnu eftir jaðrinum.

Við snúum hliðarvigrunum um \(\pi/2\) réttsælis og þá fáum við
\begin{equation*}
\begin{split}{\mathbf l}_A^R=(y_C-y_B,-x_C+x_B,), \qquad
{\mathbf l}_B^R=(y_A-y_C,-x_A+x_C,), \qquad
{\mathbf l}_C^R=(y_B-y_A,-x_B+x_A,).\end{split}
\end{equation*}
Vigrarnir \({\mathbf l}_A^R, {\mathbf l}_B^R, {\mathbf l}_C^R\) eru hornréttir á hliðarnar á móti hornum númer \(A, B\) og \(C\) og snúa í stefnu ytri þvervigurs.
Sjáið mynd fyrir einingarþríhyrninginn \(E\).

\noindent{\hspace*{\fill}\sphinxincludegraphics[width=0.650\linewidth]{{boundary-vectors}.png}\hspace*{\fill}}

\sphinxstyleemphasis{Hliðarvigrarnir}  \({\mathbf l}_A, {\mathbf l}_B, {\mathbf l}_C\)  \sphinxstyleemphasis{(rautt, blátt og grænt) til vinstri og vigrarnir} \({\mathbf l}_A^R, {\mathbf l}_B^R, {\mathbf l}_C^R\)  \sphinxstyleemphasis{til hægri (rautt, blátt og grænt).}

Þá getum við notað hliðarvigrana til þess að skrifa niður eiginleika þúfugrunnfallanna, þ.e.
\begin{enumerate}
\def\theenumi{\arabic{enumi}}
\def\labelenumi{\theenumi .}
\makeatletter\def\p@enumii{\p@enumi \theenumi .}\makeatother
\item {} 
Stigull fallsins \(\varphi_A\) er gefinn með

\end{enumerate}
\begin{equation*}
\begin{split}\nabla\varphi_A(x,y)=-{{\mathbf l}_A^R\over d},\end{split}
\end{equation*}
og líka fyrir föllin \(\varphi_B, \varphi_C\),
\begin{equation*}
\begin{split}\nabla\varphi_B(x,y)=-{{\mathbf l}_B^R\over d},\qquad
\nabla\varphi_C(x,y)=-{{\mathbf l}_C^R\over d}.\end{split}
\end{equation*}\begin{enumerate}
\def\theenumi{\arabic{enumi}}
\def\labelenumi{\theenumi .}
\makeatletter\def\p@enumii{\p@enumi \theenumi .}\makeatother
\setcounter{enumi}{1}
\item {} 
Innfeldi stiglanna er gefið með

\end{enumerate}
\begin{equation*}
\begin{split}\nabla\varphi_\alpha(x,y)\cdot \nabla\varphi_\beta(x,y)={{\mathbf l}_\alpha^R\cdot {\mathbf l}_\beta^R\over d^2}={{\mathbf l}_\alpha\cdot {\mathbf l}_\beta\over d^2},
\qquad \alpha, \beta=A, B, C.\end{split}
\end{equation*}

\subsection{Go ahead!}
\label{\detokenize{Kafli06:go-ahead}}
\noindent{\hspace*{\fill}\sphinxincludegraphics[width=0.650\linewidth]{{triangle-grid}.png}\hspace*{\fill}}

\sphinxstyleemphasis{Rétthyrningi skipt í þríhyrninga. Hér er} \(N=4\) \sphinxstyleemphasis{og} \(M=2\).

Munið að við viljum finna lausn á eftirfarandi jaðargildisverkefni
\begin{equation*}
\begin{split}\begin{cases}
Lu=-\nabla\cdot (p\nabla u)+qu=f, \quad &\text{á } D\\
u={\gamma\over \alpha},\quad &\text{á } \ \partial D_1,\\
\alpha u+\beta\dfrac{\partial u}{\partial n}
=\gamma, \quad  &\text{á } \ \partial D_2,
\end{cases}\end{split}
\end{equation*}
þar sem
\begin{equation*}
\begin{split}\partial_1D=\{(x,y)\in \partial D\,;\, \beta(x,y)=0\}
\qquad \text{ og } \qquad
\partial_2D=\{(x,y)\in \partial D\,;\, \beta(x,y)\neq 0\},\end{split}
\end{equation*}
og \(\partial D=\partial_1D\cup \partial_2 D\) (munið \DUrole{xref,std,std-ref}{5.3}). Við gerum alltaf ráð fyrir að \(p\in C^1\) og \(q, f\) séu samfelld á \(\bar D\subset\mathbb{R}^2\).

Við táknum með
\begin{enumerate}
\def\theenumi{\arabic{enumi}}
\def\labelenumi{\theenumi .}
\makeatletter\def\p@enumii{\p@enumi \theenumi .}\makeatother
\item {} 
\(S\) sammengi þríhyrninganna á svæðinu \(\bar D\),

\item {} 
\(Q\) mengi talna sem svara til punktanna á \(\partial D_1\) sem uppfylla Dirichlet jaðarskilyrði,

\item {} 
\(R\) mengi talna sem svara til punktanna á \(\partial D_2 \cup D\),

\item {} 
\(P\) fjölda allra punkta, athugum að \(P=(N+1)(M+1)\),

\item {} 
\(\partial S_1\) sammengi línustrika sem tengja hornpunkta á \(\partial D_1\) (athugum að ef t.d. \(\bar D\) er rétthyrningur, þá er \(\partial S_1=\partial D_1\))

\item {} 
\(\partial S_2\) sammengi línustrika sem nálga \(\partial D_2\).

\end{enumerate}

Við skilgreinum nálgunarfallið sem
\begin{equation*}
\begin{split}v(x,y)=\psi_0(x,y)+\sum_{\alpha\in R} c_\alpha\varphi_\alpha(x,y), \qquad (x,y)\in S.\end{split}
\end{equation*}
Við veljum fallið \(\psi_0\) þ.a. það hefur gildi \({\gamma\over \alpha}\) á jaðrinum \(\partial D_1\), þá er
\begin{equation*}
\begin{split}\psi_0(x,y)=\sum_{\alpha\in Q} {\gamma_\alpha\over \alpha_\alpha} \varphi_\alpha(x,y),\end{split}
\end{equation*}
þar sem við höfum táknað \(\gamma(x_i, y_j)\) með \(\gamma_\alpha\) og munið að \(\alpha=\sigma(i,j)=i+(N+1)(j-1)\) og \((x_i,y_j)\in \partial D_1\).
Athugum að  \({\gamma_\alpha\over \alpha_\alpha}\) er bara rauntala.

Þá gildir fyrir \(\varphi_\beta\) með \(\beta\in R\)
\begin{equation*}
\begin{split}\langle \mathcal L u, \varphi_\beta\rangle=
\langle u, \varphi_\beta \rangle_L -\int_{\partial S_2} p\dfrac{\partial u}{\partial n} \varphi_\beta ds=
\langle u, \varphi_\beta \rangle_L +\int_{\partial S_2} p\frac{\alpha}{\beta} u\varphi_\beta ds-\int_{\partial S_2} p\frac{\gamma}{\beta} \varphi_\beta ds.\end{split}
\end{equation*}
Veika framsetningin er þá
\begin{equation*}
\begin{split}\langle u, \varphi_\beta \rangle_L +\int_{\partial S_2} p\frac{\alpha}{\beta} u\varphi_\beta ds= \langle f, \varphi_\beta\rangle+ \int_{\partial S_2} p\frac{\gamma}{\beta} \varphi_\beta ds, \qquad \beta\in R.\end{split}
\end{equation*}
Nú erum við búin og getum reiknað út veiku framsetninguna fyrir nálgunarfallið og \(\varphi_\beta\) með \(\beta\in R\).

Á vinstri hliðinni höfum við
\begin{equation*}
\begin{split}\langle v, \varphi_\beta \rangle_L =
\sum_{\alpha\in R} c_\alpha \int_S \left(p \nabla \varphi_\alpha \cdot \nabla \varphi_\beta +q \varphi_\alpha \varphi_\beta\right) dA
+\sum_{\alpha\in Q} {\gamma_\alpha\over \alpha_\alpha} \int_S \left(p \nabla \varphi_\alpha \cdot \nabla \varphi_\beta +q \varphi_\alpha \varphi_\beta\right) dA,\end{split}
\end{equation*}\begin{equation*}
\begin{split}\int_{\partial S_2} p\frac{\alpha}{\beta} v\varphi_\beta ds=
\sum_{\alpha\in R} c_\alpha\int_{\partial S_2} p\frac{\alpha}{\beta} \varphi_\alpha \varphi_\beta ds+
\sum_{\alpha\in Q} {\gamma_\alpha\over \alpha_\alpha} \int_{\partial S_2}p\frac{\alpha}{\beta} \varphi_\alpha \varphi_\beta ds\,.\end{split}
\end{equation*}
Á hægri hliðinni höfum við
\begin{equation*}
\begin{split}&&\langle f, \varphi_\beta\rangle= \int_S f\varphi_\beta dA
\\
&& \int_{\partial S_2} p\frac{\gamma}{\beta} \varphi_\beta ds\,.\end{split}
\end{equation*}
Á fylkjaformi \(A{\mathbf c}={\mathbf b}\) er
\begin{equation*}
\begin{split}&& a_{\beta, \alpha}= \int_S \left(p \nabla \varphi_\alpha \cdot \nabla \varphi_\beta +q \varphi_\alpha \varphi_\beta\right) dA+\int_{\partial S_2} p\frac{\alpha}{\beta} \varphi_\alpha \varphi_\beta ds,
\\
&& b_\beta = \int_S f\varphi_\beta dA+\int_{\partial S_2} p\frac{\gamma}{\beta} \varphi_\beta ds-
\sum_{\alpha\in Q} {\gamma_\alpha\over \alpha_\alpha} \left(\int_S \left(p \nabla \varphi_\alpha \cdot \nabla \varphi_\beta +q \varphi_\alpha \varphi_\beta\right) dA+\int_{\partial S_2}p\frac{\alpha}{\beta} \varphi_\alpha \varphi_\beta ds\right)\,.\end{split}
\end{equation*}
Nú þurfum við að reikna út heildin að ofan. Við nálgum þau með því að nota reglu „miðpunktanna“, það þýðir að fyrir sérhvert samfellt fall \(\psi\) nálgum við heildi yfir þríyrning \(T_{A,B,C}\) á eftirfarandi hátt
\begin{equation*}
\begin{split}\int_{T_{A,B,C}} \psi(x,y)dA\approx \psi\left(\tfrac{x_A+x_B+x_C}{3},\tfrac{y_A+y_B+y_C}{3}\right) area(T_{A,B,C}) ={|d|\over 2}\psi(M_{A,B,C})\,\end{split}
\end{equation*}
þar sem \(M_{A,B,C}\) er massmiðja þríhyrningsins \(T_{A,B,C}\).

Við skoðum ýmsa liði.
\begin{enumerate}
\def\theenumi{\arabic{enumi}}
\def\labelenumi{\theenumi .}
\makeatletter\def\p@enumii{\p@enumi \theenumi .}\makeatother
\item {} 
Í \(b_\beta\) höfum við

\end{enumerate}
\begin{equation*}
\begin{split}\int_S f\varphi_\beta dA\approx \sum_{T_{\beta}}f(M_{(\beta)}){|d|\over 6},\end{split}
\end{equation*}
af því að \(\varphi_\beta(M_{(\beta)})=\tfrac 13\). Athugum að summan hér þýðir að við þurfum að summa bara yfir þríhyrninga sem hafa punkt \(\beta\) fyrir hornpunkt (munið skilgreinguna á þúfugrunnföllum).
\begin{enumerate}
\def\theenumi{\arabic{enumi}}
\def\labelenumi{\theenumi .}
\makeatletter\def\p@enumii{\p@enumi \theenumi .}\makeatother
\setcounter{enumi}{1}
\item {} 
Í \(b_\beta\) og í \(a_{\beta \alpha}\) höfum við

\end{enumerate}
\begin{equation*}
\begin{split}\int_S p \nabla \varphi_\alpha \cdot \nabla \varphi_\beta dA \approx
\sum_{T_{\beta}} p(M_{(\beta)}){{\mathbf l}_\alpha^R\cdot {\mathbf l}_\beta^R\over 2|d|}=\sum_{T_{\beta}} p(M_{(\beta)}){{\mathbf l}_\alpha\cdot {\mathbf l}_\beta\over 2|d|},\end{split}
\end{equation*}
þar sem summan er yfir þríhyrninga sem hafa punkt \(\beta\) fyrir hornpunkt. Munið að innfeldi \(\nabla \varphi_\alpha \cdot \nabla \varphi_\beta\) er ekki núll aðeins ef \(\alpha\) og \(\beta\) eru tveir hornpunktar \(T_{\beta}\).
\begin{enumerate}
\def\theenumi{\arabic{enumi}}
\def\labelenumi{\theenumi .}
\makeatletter\def\p@enumii{\p@enumi \theenumi .}\makeatother
\setcounter{enumi}{2}
\item {} 
Í \(b_\beta\) og í \(a_{\beta \alpha}\) höfum við

\end{enumerate}
\begin{equation*}
\begin{split}\int_S q \varphi_\alpha \varphi_\beta dA \approx \sum_{T_{\beta}} q(M_{(\beta)}){|d|\over 18},\end{split}
\end{equation*}
af því að \(\varphi_\beta(M_{(\beta)})=\tfrac 13\). Aftur, við summun yfir þríhyrninga sem hafa punkt \(\beta\) fyrir hornpunkt.

\begin{sphinxadmonition}{attention}{Athugið:}
Í kennslubókinni er heildið að ofan nálgað á eftirfarandi hátt
\begin{equation*}
\begin{split}\int_{T_{A,B,C}} \psi(x,y)\, dA
\approx \tfrac{|d|}{6}\big(\psi_{A,B}+\psi_{B,C}+\psi_{C,A}\big),\end{split}
\end{equation*}
þar sem \(\psi\) er samfellt fall, \(\psi_{A,B}, \psi_{B,C}\) og \(\psi_{C,A}\) tákna gildi fallsins \(\psi\) í miðpunktum hliðanna \(AB, BC\) og \(CA\).
\end{sphinxadmonition}

Af 2. og 3. leiðir að
\begin{equation*}
\begin{split}\int_S \left(p \nabla \varphi_\alpha \cdot \nabla \varphi_\beta + q \varphi_\alpha \varphi_\beta\right) dA \approx
\sum_{T_{\beta}}\left(p(M_{(\beta)}){{\mathbf l}_\alpha\cdot {\mathbf l}_\beta\over 2|d|}+q(M_{(\beta)}){|d|\over 18}\right).\end{split}
\end{equation*}
\begin{sphinxadmonition}{attention}{Athugið:}
Í kennslubókinni er heildið að ofan nálgað á eftirfarandi hátt
\begin{equation*}
\begin{split}\int_{S}\big(p\nabla \varphi_\alpha\cdot \nabla \varphi_\beta
+q\varphi_\alpha\varphi_\beta\big)\, dA
\approx
\sum_{T_{\beta}}\dfrac {p(M_{(\beta)})}{2 |d|}{{\mathbf l}_\alpha\cdot {\mathbf l}_\beta}
+\sum_{T_{\beta}}\begin{cases}
\tfrac 1{12} q(M_{(\beta)})\, |d|,& \alpha=\beta,\\
\tfrac 1{24} q(M_{(\beta)})\, |d|,& \alpha\neq \beta.
\end{cases}\end{split}
\end{equation*}\end{sphinxadmonition}
\begin{enumerate}
\def\theenumi{\arabic{enumi}}
\def\labelenumi{\theenumi .}
\makeatletter\def\p@enumii{\p@enumi \theenumi .}\makeatother
\setcounter{enumi}{3}
\item {} 
Í síðasta skrefi þurfum við að nálga jaðarheildin yfir \(\partial S_2\) í veiku framsetningunni. Við notum reglu Simpsons, sem segir að

\end{enumerate}
\begin{equation*}
\begin{split}\int_a^b f(t)\, dt\approx \tfrac 16\big(f(a)+4f(\tfrac
12(a+b))+f(b)\big)(b-a)\end{split}
\end{equation*}
Við táknum opna línustrikið sem liggur í \(\partial S_2\) milli punkta \(A\) og \(B\) með \(S_{A,B}\),  lengd þess með \(|S_{A,B}|\), og miðpunkt þess með \(m_{A,B}\).

Athugum að
\begin{equation*}
\begin{split}\varphi_A(m_{A,B})=\varphi_B(m_{A,B})=\tfrac 12\qquad
\varphi_C(m_{A,B})=0.\end{split}
\end{equation*}
Þá fáum við
\begin{equation*}
\begin{split}\begin{gathered}
\int_{S_{A,B}} \psi\,
\varphi_\alpha^2\, ds
\approx \tfrac 16\big(\psi(x_\alpha,y_\alpha)+ \psi(m_{A,B})\big)
|S_{A,B}|,  \qquad \alpha=A,B,
\\
\int_{S_{A,B}} \psi\,
\varphi_A\varphi_B\, ds
\approx \tfrac 16 \psi(m_{A,B}) |S_{A,B}|, \\
\int_{S_{A,B}} \psi \,
\varphi_\alpha\, ds
\approx \tfrac 16\big(\psi(x_\alpha,y_\alpha)+2\psi(m_{A,B})\big) |S_{A,B}|
\qquad \alpha=A,B.\end{gathered}\end{split}
\end{equation*}
Að lokum fáum við fyrir jaðarheildin
\begin{equation*}
\begin{split}\begin{gathered}
\int_{S_{A,B}}\dfrac{p \alpha}{\beta}
\varphi_\alpha^2\, ds
\approx \dfrac 16\bigg(
p(x_\alpha,y_\alpha)
\dfrac{\alpha(x_\alpha,y_\alpha)}
{\beta(x_\alpha,y_\alpha)}
+p(m_{A,B})
\dfrac{\alpha(x_A,y_A)+\alpha(x_B,y_B)}
{\beta(x_A,y_A)+\beta(x_B,y_B)}
\bigg) |S_{A,B}|, \quad \alpha=A,B.
\end{gathered}\end{split}
\end{equation*}\begin{equation*}
\begin{split}\int_{S_{A,B}}\dfrac{p \alpha}{ \beta}
\varphi_A\varphi_B\, ds
\approx \frac 16
p(m_{A,B}) \dfrac{\alpha(x_A,y_A)+\alpha(x_B,y_B)}
{\beta(x_A,y_A)+\beta(x_B,y_B)} |S_{A,B}|.\end{split}
\end{equation*}\begin{equation*}
\begin{split}\int_{S_{A,B}}\dfrac{p \gamma}{\beta}
\varphi_\alpha\, ds
\approx \dfrac 16\bigg(
p(x_\alpha,y_\alpha)
\dfrac{\gamma(x_\alpha,y_\alpha)}
{\beta(x_\alpha,y_\alpha)}
+2
p(m_{A,B})
\dfrac{\gamma(x_A,y_A)+\gamma(x_B,y_B)}
{\beta(x_A,y_A)+\beta(x_B,y_B)}
\bigg) |S_{A,B}|, \quad j=A,B.\end{split}
\end{equation*}
Til þess að reikna út jaðarheildi yfir \(\partial S_2\) þurfum við að summa yfir öll línustrikin sem liggja í \(\partial S_2\).


\subsection{Sýnidæmi}
\label{\detokenize{Kafli06:id1}}
\noindent{\hspace*{\fill}\sphinxincludegraphics[width=0.650\linewidth]{{triangle-grid-numbers}.png}\hspace*{\fill}}

\sphinxstyleemphasis{Rétthyrningi skipt í þríhyrninga. Hér er} \(N=4\) \sphinxstyleemphasis{og} \(M=2\).

Við lítum á Dirichlet jaðarskilyrði, þ.e.
\begin{equation*}
\begin{split}\begin{cases}
-\nabla^2 u +q u= f \qquad D\\
u=\gamma \qquad \partial D,
\end{cases}\end{split}
\end{equation*}
þar sem svæði \(D\) er \(D=\{(x,y)\in \mathbb{R}^2~~: x\in]a, b[, ~~ y\in ]c, d[\}\). Við notum net eins og á myndinni að ofan, eins og við gerðum í {\hyperref[\detokenize{Kafli06:ch-6-5-1}]{\sphinxcrossref{\DUrole{std,std-ref}{6.5.1}}}}.

Hornpunktar \(1,2,3,4,5,6,10,11,12,13,14\) og 15 eru í \(\partial D_1\). Innri punktar eru 7,8 og 9.

Skoðum \(\beta=8\). Það eru 6 þríhyrningar sem hafa \(\beta=8\) fyrir hornpunkt. Það þýðir að þegar við reiknum \(a_{\beta=8,\alpha}\), eru einu stök fylkisins sem eru ekki núll þau sem hafa \(\alpha=7,3,4,9,13,12\).

Þegar við skiptum bilinu í jafna hluta, eins og við gerðum í {\hyperref[\detokenize{Kafli06:ch-6-5-1}]{\sphinxcrossref{\DUrole{std,std-ref}{6.5.1}}}}, þá er
\begin{equation*}
\begin{split}area(T)={|d|\over 2} ={h k\over 2},\end{split}
\end{equation*}
og
\begin{equation*}
\begin{split}\int_S \nabla \phi_\beta\cdot \nabla \phi_\beta dA=
{1\over 2 h k}\left(2 h^2+2 k^2 +2 (k^2+h^2)\right)=
{2\over h k}\left(k^2+h^2\right),\end{split}
\end{equation*}\begin{equation*}
\begin{split}\int_S q \phi_\beta  \phi_\beta dA= {h k\over 18} \left(q(M_1)+q(M_2)+q(M_3)+q(M_4)+q(M_5)+q(M_6)\right),\end{split}
\end{equation*}
þar sem \(M_i\) eru miðjupunktar fyrir 6 þríhyrninga sem hafa \(\beta\) fyrir hornpunkt.

Þetta gefur fyrir \(\beta=8\)
\begin{equation*}
\begin{split}a_{\beta,\beta}=\tfrac{2}{h k}(k^2+h^2) +{h k\over 18} \left(q(M_1)+q(M_2)+q(M_3)+q(M_4)+q(M_5)+q(M_6)\right).\end{split}
\end{equation*}
Fyrir \(\alpha\neq \beta=8\), fáum við
\begin{enumerate}
\def\theenumi{\arabic{enumi}}
\def\labelenumi{\theenumi .}
\makeatletter\def\p@enumii{\p@enumi \theenumi .}\makeatother
\item {} 
\end{enumerate}
\begin{equation*}
\begin{split}\int_S \nabla \phi_\alpha\cdot \nabla \phi_\beta dA=0  \quad \alpha=4, 12\end{split}
\end{equation*}
af því að hliðarvigrarnir \(\mathbf{l}_\alpha,\mathbf{l}_\beta\) eru hornréttir. Ennfremur höfum við
\begin{equation*}
\begin{split}\int_S q \phi_\alpha \phi_\beta dA={hk\over 18}(q(M_{2(5)})+q(M_{3(6)}))  \quad \alpha=4, 12,\end{split}
\end{equation*}
og það gefur
\begin{equation*}
\begin{split}a_{\beta,\alpha}={hk\over 18}(q(M_{2(5)})+q(M_{3(6)})) \qquad \text{með}\qquad \alpha=\sigma(j+1,p-1),  \alpha=\sigma(j-1,p+1).\end{split}
\end{equation*}
Athugum að við notum \(\beta=\sigma(j,p)\).
\begin{enumerate}
\def\theenumi{\arabic{enumi}}
\def\labelenumi{\theenumi .}
\makeatletter\def\p@enumii{\p@enumi \theenumi .}\makeatother
\setcounter{enumi}{1}
\item {} 
\end{enumerate}
\begin{equation*}
\begin{split}\int_S \nabla \phi_\alpha\cdot \nabla \phi_\beta dA=-{h^2\over 2 h k} 2=-{h\over k}  \quad \alpha=3, 13,\end{split}
\end{equation*}
af því að hliðarvigurinn \(\mathbf{l}_\alpha\) er láréttur, og það eru tveir þríhyrningar sem hafa \(\alpha, \beta\) fyrir hornpunkta.

Ennfremur höfum við
\begin{equation*}
\begin{split}\int_S q \phi_\alpha \phi_\beta dA={hk\over 18}(q(M_{1(4)})+q(M_{2(5)}))  \quad \alpha=3, 13,\end{split}
\end{equation*}
og það gefur
\begin{equation*}
\begin{split}a_{\beta,\alpha}=-{h\over k}+{hk\over 18}(q(M_{1(4)})+q(M_{2(5)}))  \qquad \text{með}\qquad \alpha=\sigma(j,p-1),\alpha=\sigma(j,p+1) .\end{split}
\end{equation*}\begin{enumerate}
\def\theenumi{\arabic{enumi}}
\def\labelenumi{\theenumi .}
\makeatletter\def\p@enumii{\p@enumi \theenumi .}\makeatother
\setcounter{enumi}{2}
\item {} 
\end{enumerate}
\begin{equation*}
\begin{split}\int_S \nabla \phi_\alpha\cdot \nabla \phi_\beta dA=-{k^2\over 2 h k} 2=-{k\over h}  \quad \alpha=7, 9,\end{split}
\end{equation*}
af því að hliðarvigurinn \(\mathbf{l}_\alpha\) er lóðréttur, og það eru tveir þríhyrningar sem hafa \(\alpha, \beta\) fyrir hornpunkta.

Ennfremur höfum við
\begin{equation*}
\begin{split}\int_S q \phi_\alpha \phi_\beta dA={hk\over 18}(q(M_{6(4)})+q(M_{1(3)}))  \quad \alpha=7, 9,\end{split}
\end{equation*}
og það gefur
\begin{equation*}
\begin{split}a_{\beta,\alpha}=-{k\over h} +{hk\over 18}(q(M_{6(4)})+q(M_{1(3)})) \qquad \text{með}\qquad \alpha=\sigma(j-1,p),\alpha=\sigma(j+1,p).\end{split}
\end{equation*}
Skoðum vigurinn \(\mathbf b\), þá er
\begin{equation*}
\begin{split}b_\beta=\langle f, \varphi_\beta\rangle= \frac{h k}{6} \left(f(M_1)+f(M_2)+f(M_3)+f(M_4)+f(M_5)+f(M_6)\right).\end{split}
\end{equation*}
Við þurfum að endurtaka aðferðina fyrir \(\beta=7,8,9\).

Fyrir jaðarpunkta þurfum við að setja
\begin{equation*}
\begin{split}a_{\beta,\beta }=1,  \qquad b_\beta=\gamma(x_j, y_p),\end{split}
\end{equation*}
þar sem \(\beta=\sigma(j,p)\).


\chapter{Viðauki}
\label{\detokenize{vidauki:viauki}}\label{\detokenize{vidauki::doc}}

\section{Kennsluáætlun}
\label{\detokenize{vidauki:kennsluaaetlun}}
Með fyrirvara um smávægilegar breytingar. Engir dæmatímar eða vinnustofur eru í fyrstu viku.


\begin{savenotes}\sphinxatlongtablestart\begin{longtable}{|*{3}{\X{1}{3}|}}
\hline
\sphinxstyletheadfamily 
Dagsetning
&\sphinxstyletheadfamily 
Efni
&\sphinxstyletheadfamily 
Lesefni
\\
\hline
\endfirsthead

\multicolumn{3}{c}%
{\makebox[0pt]{\sphinxtablecontinued{\tablename\ \thetable{} -- framhald frá fyrri síðu}}}\\
\hline
\sphinxstyletheadfamily 
Dagsetning
&\sphinxstyletheadfamily 
Efni
&\sphinxstyletheadfamily 
Lesefni
\\
\hline
\endhead

\hline
\multicolumn{3}{r}{\makebox[0pt][r]{\sphinxtablecontinued{Framhald á næstu síðu}}}\\
\endfoot

\endlastfoot

07.01.19.
&\begin{enumerate}
\def\theenumi{\arabic{enumi}}
\def\labelenumi{\theenumi .}
\makeatletter\def\p@enumii{\p@enumi \theenumi .}\makeatother
\item {} 
Dæmi um hlutafleiðujöfnur í eðlisfræði.

\end{enumerate}
&
11.1-11.2.
\\
\hline
09.01.19.
&
2. Hliðarskilyrði. Vel framsett verkefni.
Fyrsta stigs jöfnur.
&
11.3-11.4,12.1-12.2.
\\
\hline
14.01.19.
&\begin{enumerate}
\def\theenumi{\arabic{enumi}}
\def\labelenumi{\theenumi .}
\makeatletter\def\p@enumii{\p@enumi \theenumi .}\makeatother
\setcounter{enumi}{2}
\item {} 
Fourier-raðir.

\end{enumerate}
&
13.1-13.3.
\\
\hline
16.01.19.
&\begin{enumerate}
\def\theenumi{\arabic{enumi}}
\def\labelenumi{\theenumi .}
\makeatletter\def\p@enumii{\p@enumi \theenumi .}\makeatother
\setcounter{enumi}{3}
\item {} 
Samleitni Fourier-raða.

\end{enumerate}
&
13.4-13.6.
\\
\hline
18.01.19.
&
\sphinxstylestrong{Skiladæmi 1.}
&\\
\hline
21.01.19.
&
5. Úrlausn hlutafleiðujafa með
Fourier-röðum.
&
13.7-13.8.
\\
\hline
23.01.19.
&
6. Eigingildisverkefni.
Aðskilnaður breytistærða.
&
14.1-14.3
\\
\hline
25.01.19.
&
\sphinxstylestrong{Skiladæmi 2.}
&\\
\hline
28.01.19.
&\begin{enumerate}
\def\theenumi{\arabic{enumi}}
\def\labelenumi{\theenumi .}
\makeatletter\def\p@enumii{\p@enumi \theenumi .}\makeatother
\setcounter{enumi}{6}
\item {} 
Virkjar af Sturm-Liouville-gerð.

\end{enumerate}
&
14.4-14.7
\\
\hline
30.01.19.
&
8. Úrlausn hlutafleiðujafa
með eiginfallaröðum.
&
15.1-4.
\\
\hline
01.02.19.
&
\sphinxstylestrong{Skiladæmi 3.}
&\\
\hline
04.02.19.
&
9. Áfram um eigingildisverkefni
- aðskilnaður breytistærða.
&
15.5-7
\\
\hline
06.02.19.
&
10. Fourier-ummyndun. Reiknireglur.
Plancerel-jafnan.
&
16.1-2.
\\
\hline
08.02.19.
&
\sphinxstylestrong{Skiladæmi 4.}
&\\
\hline
11.02.19.
&
11. Andhverfuformúla Fouriers.
Afleiðujöfnur.
&
16.3-6
\\
\hline
13.02.19.
&
12. Úrlausn hlutafleiðajafa
með Fourier-ummyndun.
&
18.1-4,19.1-2
\\
\hline
Ákveðið síðar
&
\sphinxstylestrong{Próf 1}
&\\
\hline
18.02.19.
&\begin{enumerate}
\def\theenumi{\arabic{enumi}}
\def\labelenumi{\theenumi .}
\makeatletter\def\p@enumii{\p@enumi \theenumi .}\makeatother
\setcounter{enumi}{12}
\item {} 
Fourier-ummyndun og leifareikningur.

\end{enumerate}
&
16.7
\\
\hline
20.02.19.
&\begin{enumerate}
\def\theenumi{\arabic{enumi}}
\def\labelenumi{\theenumi .}
\makeatletter\def\p@enumii{\p@enumi \theenumi .}\makeatother
\setcounter{enumi}{13}
\item {} 
Laplace-ummyndun og leifareikningur.

\end{enumerate}
&
16.8-9.
\\
\hline
22.02.19.
&
\sphinxstylestrong{Skiladæmi 5.}
&\\
\hline
25.02.19.
&\begin{enumerate}
\def\theenumi{\arabic{enumi}}
\def\labelenumi{\theenumi .}
\makeatletter\def\p@enumii{\p@enumi \theenumi .}\makeatother
\setcounter{enumi}{14}
\item {} 
Fourier-ummyndun. Laplace-ummyndun.

\end{enumerate}
&
16.7-9.
\\
\hline
27.02.19.
&
16. Mismunaaðferð fyrir venjulegar
afleiðujöfnur.
&
21.1-2.
\\
\hline
01.03.19.
&
\sphinxstylestrong{Skiladæmi 6.}
&\\
\hline
04.03.19.
&\begin{enumerate}
\def\theenumi{\arabic{enumi}}
\def\labelenumi{\theenumi .}
\makeatletter\def\p@enumii{\p@enumi \theenumi .}\makeatother
\setcounter{enumi}{16}
\item {} 
Heildun yfir hlutbil.

\end{enumerate}
&
21.3.
\\
\hline
06.03.19.
&
18. Mismunaaðferð fyrir
hlutaafleiðujöfnur.
&
21.5.
\\
\hline
08.03.19.
&
\sphinxstylestrong{Skiladæmi 7.}
&\\
\hline
11.03.19.
&
19. Almenn mismunaaðferð á
rétthyrningi.
&
21.6.
\\
\hline
13.03.19.
&
20. Hlutheildun, innfeldi og
tvílínulegt form.
&
22.1-2.
\\
\hline
15.03.19.
&
\sphinxstylestrong{Skiladæmi 8.}
&\\
\hline
18.03.19.
&
21. Aðferð Galerkins fyrir
Dirichlet-verkefnið.
&
22.3.
\\
\hline
20.03.19.
&
22.Bútaaðferð í einni vídd.
&
22.5
\\
\hline
Ákveðið síðar
&
\sphinxstylestrong{Próf 2}
&\\
\hline
25.03.19.
&
23. Aðferð Galerkins með almennum
jaðarskilyrðum.
&
22.4.
\\
\hline&
\sphinxstylestrong{Umræður um heimaverkefni}
&\\
\hline
27.03.19.
&
24. Aðferð Galerkins með almennum
jaðarskilyrðum.
&
22.4
\\
\hline&
\sphinxstylestrong{Umræður um heimaverkefni}
&\\
\hline
01.04.19.
&\begin{enumerate}
\def\theenumi{\arabic{enumi}}
\def\labelenumi{\theenumi .}
\makeatletter\def\p@enumii{\p@enumi \theenumi .}\makeatother
\setcounter{enumi}{24}
\item {} 
Bútaaðferð í tveimur víddum.

\end{enumerate}
&
22.6.
\\
\hline
03.04.19.
&\begin{enumerate}
\def\theenumi{\arabic{enumi}}
\def\labelenumi{\theenumi .}
\makeatletter\def\p@enumii{\p@enumi \theenumi .}\makeatother
\setcounter{enumi}{25}
\item {} 
Bútaaðferð í tveimur víddum.

\end{enumerate}
&
22.6.
\\
\hline
08.04.19.
&\begin{enumerate}
\def\theenumi{\arabic{enumi}}
\def\labelenumi{\theenumi .}
\makeatletter\def\p@enumii{\p@enumi \theenumi .}\makeatother
\setcounter{enumi}{26}
\item {} 
Dæmatími

\end{enumerate}
&\\
\hline
10.04.19.
&
28. Upprifjun og undirbúningur fyrir
Próf. Valin prófdæmi.
&\\
\hline&
\sphinxstylestrong{Skil á heimaverkefni}
&\\
\hline
\end{longtable}\sphinxatlongtableend\end{savenotes}

Í dálkinum \sphinxstylestrong{Lesefni} er vísað
í kennslubók Ragnars Sigurðssonar.
\begin{itemize}
\item {} 
\DUrole{xref,std,std-ref}{genindex}

\item {} 
\sphinxhref{stae401.pdf}{Pdf-útgáfa}

\end{itemize}



\renewcommand{\indexname}{Atriðaskrá}
\printindex
\end{document}