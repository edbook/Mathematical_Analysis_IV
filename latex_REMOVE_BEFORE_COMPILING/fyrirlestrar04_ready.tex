
%
%Allir pakkar sem þarf að nota.
%
\usepackage[utf8]{inputenc}
\usepackage[T1]{fontenc}
\usepackage[icelandic]{babel}
\usepackage{amsmath}
\usepackage{amssymb}
\usepackage{pictex}
\usepackage{epsfig,psfrag}
\usepackage{makeidx}
%\selectlanguage{icelandic}
%----------------------------

%
\hoffset=-0.4truecm
\voffset=-1truecm
\textwidth=16truecm 
%\textwidth=12truecm 
\textheight=23truecm
\evensidemargin=0truecm
%
% Gömlu gildin á bókinni 
%
%\voffset 1.4truecm
%\hoffset .25truecm
%\vsize  16.0truecm
%\hsize  15truecm
%
%
% Skilgreiningar á ýmsum skipunum.
%
%\newcommand{\Sb}{
%$$
%\sum_{\footnotesize\begin{array}{l} j=1 \\ j\neq k \end{array}}
%$$
%}
\newcommand{\bolddot}{{\mathbf \cdot}}
\newcommand{\C}{{\mathbb  C}}
\newcommand{\Cn}{{\mathbb  C\sp n}}
\newcommand{\crn}{{{\mathbb  C\mathbb  R^n}}}
\newcommand{\R}{{\mathbb  R}}
\newcommand{\Rn}{{\mathbb  R\sp n}}
\newcommand{\Rnn}{{\mathbb  R\sp{n\times n}}}
\newcommand{\Z}{{\mathbb  Z}}
\newcommand{\N}{{\mathbb  N}}
\renewcommand{\P}{{\mathbb  P}}
\newcommand{\Q}{{\mathbb  Q}}
\newcommand{\K}{{\mathbb  K}}
\newcommand{\U}{{\mathbb  U}}
\newcommand{\D}{{\mathbb  D}}
\newcommand{\T}{{\mathbb  T}}
\newcommand{\A}{{\cal A}}
\newcommand{\E}{{\cal E}}
\newcommand{\F}{{\cal F}}
\renewcommand{\H}{{\cal H}}
\renewcommand{\L}{{\cal L}}
\newcommand{\M}{{\cal M}}
\renewcommand{\O}{{\cal O}}
\renewcommand{\S}{{\cal S}}
\newcommand{\dash}{{\sp{\prime}}}
\newcommand{\ddash}{{\sp{\prime\prime}}}
\newcommand{\tdash}{{\sp{\prime\prime\prime}}}
\newcommand{\set }[1]{{\{#1\}}}
\newcommand{\scalar}[2]{{\langle#1,#2\rangle}}
\newcommand{\arccot}{{\operatorname{arccot}}}
\newcommand{\arccoth}{{\operatorname{arccoth}}}
\newcommand{\arccosh}{{\operatorname{arccosh}}}
\newcommand{\arcsinh}{{\operatorname{arcsinh}}}
\newcommand{\arctanh}{{\operatorname{arctanh}}}
\newcommand{\Log}{{\operatorname{Log}}}
\newcommand{\Arg}{{\operatorname{Arg}}}
\newcommand{\grad}{{\operatorname{grad}}}
\newcommand{\graf}{{\operatorname{graf}}}
\renewcommand{\div}{{\operatorname{div}}}
\newcommand{\rot}{{\operatorname{rot}}}
\newcommand{\curl}{{\operatorname{curl}}}
\renewcommand{\Im}{{\operatorname{Im\, }}}
\renewcommand{\Re}{{\operatorname{Re\, }}}
\newcommand{\Res}{{\operatorname{Res}}}
\newcommand{\vp}{{\operatorname{vp}}}
\newcommand{\mynd}[1]{{{\operatorname{mynd}(#1)}}}
\newcommand{\dbar}{{{\overline\partial}}}
\newcommand{\inv}{{\operatorname{inv}}}
\newcommand{\sign}{{\operatorname{sign}}}
\newcommand{\trace}{{\operatorname{trace}}}
\newcommand{\conv}{{\operatorname{conv}}}
\newcommand{\Span}{{\operatorname{Sp}}}
\newcommand{\stig}{{\operatorname{stig}}}
\newcommand{\Exp}{{\operatorname{Exp}}}
\newcommand{\diag}{{\operatorname{diag}}}
\newcommand{\adj}{{\operatorname{adj}}}
\newcommand{\erf}{{\operatorname{erf}}}
\newcommand{\erfc}{{\operatorname{erfc}}}
\newcommand{\Lloc}{{L_{\text{loc}}\sp 1}}
\newcommand{\boldcdot}{{\mathbb \cdot}}
%\newcommand{\Cinf0}[1]{{C_0\sp{\infty}(#1)}}
\newcommand{\supp}{{\text{supp}\, }}
\newcommand{\chsupp}{{\text{ch supp}\, }}
\newcommand{\singsupp}{{\text{sing supp}\, }}
\newcommand{\SL}[1]{{\dfrac {1}{\varrho} 
\bigg(-\dfrac d{dx}\bigg(p\dfrac {d#1}{dx}\bigg)+q#1\bigg)}}
\newcommand{\SLL}[1]{-\dfrac d{dx}\bigg(p\dfrac {d#1}{dx}\bigg)+q#1}
\newcommand{\Laplace}[1]{\dfrac{\partial^2 #1}{\partial x^2}+\dfrac{\partial^2 #1}{\partial y^2}}
\newcommand{\polh}[1]{{\widehat #1_{\C^n}}}
\newcommand{\tilv}{{}}
%
\renewcommand{\chaptername}{Kafli}
%
% Númering á formæulum.
%
\numberwithin{equation}{section}
%
%  Innsetning á myndum.
%
\def\figura#1#2{
\vbox{\centerline{
\input #1
}
\centerline{#2}
}\medskip}
\def\vfigura#1#2{
\setbox0\vbox{{
\input #1
}}
\setbox1\vbox{\hbox{\box0}\hbox{{\obeylines #2}}}
\dimen0 = -\ht1
\advance\dimen0 by-\dp1
\dimen1 = \wd1
\dimen2 = -\dimen0
\divide\dimen2 by\baselineskip
\count100 = 1
\advance\count100 by\dimen2
\advance\count100 by1
\box1
\hangindent\dimen1
\hangafter=-\count100
\vskip\dimen0
}
%
%  Setningar, skilgreiningar, o.s.frv. 
%
\newtheorem{setning+}           {Setning}      [section]
\newtheorem{skilgreining+}  [setning+]  {Skilgreining}
\newtheorem{setningogskilgreining+}  [setning+]  {Setning og
skilgreining}
\newtheorem{hjalparsetning+}  [setning+]  {Hjálparsetning}
\newtheorem{fylgisetning+}  [setning+]  {Fylgisetning}
\newtheorem{synidaemi+}  [setning+]  {Sýnidæmi}
\newtheorem{forrit+}  [setning+]  {Forrit}

\newcommand{\tx}[1]{{\rm({\it #1}). \ }}

\newenvironment{se}{\begin{setning+}\sl}{\hfill$\square$\end{setning+}\rm}
\newenvironment{sex}{\begin{setning+}\sl}{\hfill$\blacksquare$\end{setning+}\rm}
\newenvironment{sk}{\begin{skilgreining+}\rm}{\hfill$\square$\end{skilgreining+}\rm}
\newenvironment{sesk}{\begin{setningogskilgreining+}\rm}{\hfill$\square$\end{setningogskilgreining+}\rm}
\newenvironment{hs}{\begin{hjalparsetning+}\sl}{\hfill$\square$\end{hjalparsetning+}\rm}
\newenvironment{fs}{\begin{fylgisetning+}\sl}{\hfill$\square$\end{fylgisetning+}\rm}
\newenvironment{sy}{\begin{synidaemi+}\rm}{\hfill$\square$\end{synidaemi+}\rm}
\newenvironment{fo}{\begin{forrit+}\rm}{\hfill\end{forrit+}\rm}
\newenvironment{so}{\medbreak\noindent{\it Sönnun:}\rm}{\hfill$\blacksquare$\rm}
\newenvironment{sotx}[1]{\medbreak\noindent{\it #1:}\rm}{\hfill$\blacksquare$\rm}
\newcounter{daemateljari}
\newcommand{\aefing}{\section{Æfingardæmi} \setcounter{daemateljari}{1}}
\newcommand{\daemi}{
{\medskip\noindent{\bf \thedaemateljari.}}
\addtocounter{daemateljari}{1}
}

%\def\aefing{{\large\bf\bigskip\bigskip\noindent Æfingardæmi}}
%\def\daemi#1{\medskip\noindent{\bf #1.}}
\def\svar#1{\smallskip\noindent{\bf #1.} \ }
\def\lausn#1{\smallskip\noindent{\bf #1.} \ }
\def\ugrein#1{\medbreak\noindent{\bf #1.} }
\newcommand{\samantekt}{\noindent{\bf Samantekt.} }
%\newcommand{\proclaimbox}{\hfill$\square$}

\chapter {LEIFAREIKNINGUR}
 
%\kaflahaus{leifareikningur}

\section{Samleitnar Laurent-raðir}  

Nú ætlum við að skoða föll sem eru ekki endilega fáguð á tiltekinni
hringskífu, heldur á svæðir þar sem búið er að skera út einn punkt
eða lokaða hringskífu.

\subsection*{Hringkragar}

Mengi af gerðinni 
$$A(\alpha,\varrho_1,\varrho_2)=\set{z\in \C;
\varrho_1<|z-\alpha|<\varrho_2}
$$ 
þar sem
$0\leq\varrho_1<\varrho_2\leq +\infty$
kallast {\it opinn hringkragi\index{hringkragi}
\index{hringkragi!opinn}\index{opinn hringkragi}}
með miðju í $\alpha$, {\it innri geisla\index{hringkragi!innri
geisli}\index{innri geisli hringkraga}}
$\varrho_1$, og {\it ytri geisla\index{hringkragi!ytri
geisli}\index{ytri geisli hringkraga} $\varrho_2$},
og mengi af gerðinni 
$$\bar A(\alpha,\varrho_1,\varrho_2)=\set{z\in \C;
\varrho_1\leq|z-\alpha|\leq\varrho_2}
$$ þar sem
$0<\varrho_1<\varrho_2\leq +\infty$
kallast {\it lokaður hringkragi\index{hringkragi!lokaður}\index{lokaður
hringkragi}} með miðju í $\alpha$, {\it innri geisla}
$\varrho_1$, og {\it ytri geisla $\varrho_2$}.

\figura{fig097}{{\small Mynd: Hringkragi}}

\subsection*{Laurent-setningin}
 
Fágað fall á skífu er hægt að setja fram með veldaröð.  
Ef fall er fágað á hringkraga þá koma til sögunnar neikvæð veldi:

\begin{se}\tx{Laurent\index{Laurent-setningin}\index{setning!Laurent}}
Látum $X$ vera opið hlutmengi af $\C$ og gerum ráð fyrir að
$A(\alpha,\varrho_1,\varrho_2)\subset X$.  Ef
$f\in \O(X)$, þá er unnt að skrifa $f$ sem
 \begin{equation*}f(z)=\sum_{n=-\infty}\sp{+\infty}a_n(z-\alpha)\sp n, \qquad z\in
A(\alpha,\varrho_1,\varrho_2).
\label{10.5.1}
 \end{equation*}
Stuðlar raðarinnar $a_n$ eru gefnir með formúlunni
 \begin{equation*}a_n=\dfrac 1{2\pi i}\int_{\partial S(\alpha,r)} \dfrac{f(\zeta)}
{(\zeta-\alpha)\sp{n+1}} \, d\zeta,
\label{10.5.2}
 \end{equation*}
og $r$ getur verið hvaða tala sem er á bilinu
$]\varrho_1,\varrho_2[$.  Röðin
$$\sum_{n=0}\sp{+\infty}a_n(z-\alpha)\sp n$$ 
er samleitin ef
$|z-\alpha|<\varrho_2$ og  röðin
$$\sum_{n=-\infty}\sp{-1}a_n(z-\alpha)\sp n$$ 
er samleitin ef
$|z-\alpha|>\varrho_1$.

\end{se}

\subsection*{Laurent-raðir}

\begin{sk}
Röð af gerðinni 
 $$\sum_{n=-\infty}\sp{+\infty}a_n(z-\alpha)\sp n
 $$
kallast {\it Laurent-röð\index{laurent-röð}}.  {\it Innri samleitnigeisli
\index{innri samleitnigeisli Laurent-raðar}\index{Laurent-röð!innri
samleitnigeisli}\index{Laurent-röð!samleitni}\index{samleitin Laurent-röð}}
raðarinnar $\varrho_1$ er skilgreindur sem neðra mark yfir
$\varrho=|z-\alpha|$ þannig að
$$ \sum_{n=-\infty}\sp{-1} a_n(z-{\alpha})\sp n $$
er samleitin, {\it ytri samleitnigeisli\index{Laurent-röð!ytri
samleitnigeisli}\index{ytri samleitnigeisli Laurent-raðar}} raðarinnar
$\varrho_2$ er skilgreindur sem efra mark yfir öll $\varrho=|z-\alpha|$
þannig að
$$ \sum_{n=0}\sp{+\infty}a_n(z-{\alpha})\sp n $$
er samleitin. 
Ef $\varrho_1<\varrho_2$ þá segjum við að Laurent-röðin
sé {\it samleitin}.    Stuðullinn $a_{-1}$ kallast {\it
leif\index{Laurent-röð!leif}\index{leif}\index{leif!Laurent-raðar}}
Laurent-raðarinnar 
og röðin 
$$\sum_{n=-\infty}\sp{-1}a_n(z-{\alpha})\sp n
$$ 
kallast {\it
höfuðhluti\index{höfuðhluti!Laurent-raðar}\index{Laurent-röð!höfuðhluti}}
hennar.
\end{sk}


Ef Laurent-röð $\sum_{n=-\infty}\sp{+\infty}a_n(z-\alpha)\sp n$ 
er samleitin og $\varrho_1$ og $\varrho_2$ tákna
innri og ytri samleitnigeisla hennar, þá skilgreinir hún fágað fall 
á hringkraganum
$A(\alpha,\varrho_1,\varrho_2)$ með formúlunni
 $$f(z)=\sum_{n=-\infty}\sp{+\infty}a_n(z-\alpha)\sp n.
 $$
Hugsum okkur nú að $\gamma$ sé lokaður vegur sem liggur í
$A(\alpha,\varrho_1,\varrho_2)$  og lítum á heildið
 \begin{equation*}\int_{\gamma} f(z)\, dz=
\sum_{n=-\infty}\sp{+\infty} a_n
\int_{\gamma} (z-\alpha)\sp n\, dz.
\label{10.5.3}
 \end{equation*}
Hér höfum við notfært okkur að röðin er samleitin í jöfnum mæli á
veginum $\gamma$ til þess að flytja heildið inn fyrir summutáknið. Nú
athugum við að allir liðirnir í summunni hafa stofnfall nema sá með
númerið $n=-1$.  Þar með er
 $$\int_{\gamma} f(z)\, dz=
a_{-1}
\int_{\gamma} \dfrac {dz}{z-\alpha}.
 $$
Ef nú $\gamma$ er einfaldur lokaður vegur, sem stikar jaðarinn
$\partial\Omega$ á svæðinu $\Omega$ í jákvæða stefnu, þá segir
Cauchy-formúlan að síðasta heildið sé $2\pi i$ ef $\alpha$ er inni í
svæðinu, en Cauchy-setningin segir að það sé $0$ ef $\alpha$ er utan
þess.  Þar með er
 \begin{equation*}\int_\gamma f(z) \, dz =\begin{cases}
2\pi i\, a_{-1}, &\alpha\in \Omega,\\
0, & \alpha\not\in \Omega.\end{cases}
\label{10.5.4}
 \end{equation*}
Í tilfellinu að $A(\alpha,\varrho_1,\varrho_2)\subset
S(\alpha,\varrho_2)\subset X$, þ.e.~þegar fallið 
$f$ er fágað á svæði sem inniheldur alla hringskífuna
$S(\alpha,\varrho_2)$, þá eru föllin
 $$\zeta\mapsto \dfrac
{f(\zeta)}{(\zeta-\alpha)\sp{n+1}}=(\zeta-\alpha)\sp{-n-1}f({\zeta}), 
 $$
fáguð í $S(\alpha,\varrho_2)$ fyrir öll $n<0$.  Cauchy-setninginn
segir okkur þá að $a_n=0$ ef $n<0$ og Cauchy-formúlan fyrir afleiður
gefur okkur
 $$a_n=\dfrac{f\sp{(n)}(\alpha)}{n!}, \qquad n\geq 0.
 $$
Ef $A(\alpha,\varrho_1,\varrho_2)\subset S(\alpha,\varrho_2)\subset X$,
þá  þýðir þetta sem sagt að Laurent-röð fallsins $f$ í ${\alpha}$ 
sé  Taylor-röð þess. 


\section{Einangraðir sérstöðupunktar}

\subsection*{Einangraðir punktar og dreifð mengi}

Látum nú $A$ vera hlutmengi í $\C$.  Rifjum það upp að punktur
$\alpha\in A$ kallast {\it einangraður punktur\index{einangraður
punktur}\index{einangraður
sérstöðupunktur}\index{sérstöðupunktur!einangraður}} í $A$ ef til er
$\varepsilon>0$ þannig að $S\sp *(\alpha,\varepsilon)\cap A=\varnothing$,
þ.e.a.s.~$\alpha$ er eini punkturinn í $A$ sem liggur í opnu skífunni
$S(\alpha,\varepsilon)$. Við segjum að mengið $A$ sé
{\it dreift\index{dreift mengi}} ef sérhver
punktur í því er einangraður. 


\subsection*{Höfuðhluti og leif}

Látum $X$ vera opið mengi,
$f\in \O(X)$ og $\alpha$ vera einangraðan sér\-stöðu\-punkt  fágaða
fallsins $f$. Samkvæmt Laurent-setningunni getum við skrifað 
$$
f(z)= \sum_{n=-\infty}\sp{+\infty}a_n(z-\alpha)\sp n, \qquad z\in 
S\sp *(\alpha,\varepsilon)=A(\alpha,0,\varepsilon),
$$
þar sem stuðlarnir eru ótvírætt ákvarðaðir.  Við köllum þessa röð
{\it Laurent-röð fágaða fallsins\index{Laurent-röð!fágaðs falls}
$f$ í punktinum}
$\alpha$, við köllum höfuðhluta raðarinnar {\it höfuðhluta  fágaða
fallsins\index{höfuðhluti!fágaðs falls} $f$ í punktinum} $\alpha$ og
við köllum leif raðarinnar {\it leif fallsins\index{Laurent-röð!leif}
\index{leif!falls} $f$ í punktinum} $\alpha$
og við táknum hana með
$$ \Res (f,\alpha). $$


\subsection*{Afmáanlegir sérstöðupunktar}

Einangraður
sérstöðupunktur ${\alpha}$ fágaða fallsins $f$ 
er sagður vera {\it afmáanlegur\index{afmáanlegur
sérstöðupunktur}\index{einangraður
sérstöðupunktur!afmáanlegur}\index{sérstöðupunktur!afmáanlegur}},
ef til er $r>0$ og $g\in \O(S({\alpha},r))$ þannig að
$S^*({\alpha},r)\subset X$ og $f(z)=g(z)$ fyrir öll
$z\in S^*({\alpha},r)$.


\begin{se}\tx{Riemann\index{Riemann-setningin}\index{setning!Riemann}}
Ef $\alpha$ er einangraður sérstöðupunktur fágaða
fallsins $f$, og  $\lim_{z\to \alpha}(z-\alpha)f(z)= 0$,
þá er $\alpha$ afmáanlegur sérstöðupunktur
\end{se}


\subsection*{Skaut}

\begin{sk}  Látum $f$ vera fágað fall á opnu mengi $X$ og $\alpha$ vera
einangraðan sérstöðupunkt fallsins $f$. 
Við segjum að $\alpha$ sé {\it skaut af stigi\index{einangraður
sérstöðupunktur!skaut}\index{sérstöðupunktur!skaut}}
$m>0$, ef til er fágað fall $g\in \O(U)$, þar sem $U$ er grennd um
$\alpha$, þannig að $g(\alpha)\neq 0$ og 
 $$f(z)=\dfrac{g(z)}{(z-\alpha)\sp m}, \qquad z\in U\setminus\set\alpha.
 $$
\end{sk}

Skautin einkennast af:


\begin{se} Fall $f$ hefur skaut í $\alpha$ ef og aðeins ef
$|f(z)|\to +\infty$ ef $z\to \alpha$.

{}
\end{se}

Hugsum okkur nú að fallið $f$  hafi skaut í punktinum $\alpha$
af stigi $m$. Þá er fallið sett fram með Laurent-röð af gerðinni
 $$f(z)=\sum\limits_{n=-m}^{+\infty} a_n(z-\alpha)^n,
 $$
í grennd um $\alpha$.  Ef höfuðhlutinn er táknaður með $h(z)$, þá er
$\alpha$ afmáanlegur sérstöðupunktur mismunarins
 $$f(z)-h(z) =f(z)-\sum\limits_{n=-m}^{-1} a_n(z-\alpha)^n 
= \sum\limits_{n=0}^\infty a_n(z-\alpha)^n. 
 $$


\subsection*{Stofnbrotaliðun}

{Á}ður en við segjum skilið við skautin, þá skulum við víkja ögn að
stofnbrotaliðun\index{stofnbrotaliðun}.  Við höfum gengum út frá
því sem vísum hlut, að það væri alltaf hægt að liða rætt fall í
stofnbrot\index{stofnbrot}. Nú skulum við sanna þetta og leiða út
formúlurnar fyrir stuðlunum í stofnbrotaliðuninni.

Látum $R=P/Q$ vera rætt fall og gerum ráð fyrir að $\stig P<\stig Q$.
Látum $\alpha_1,\dots,\alpha_k$ vera ólíkar núllstöðvar $Q$, látum
$m_1,\dots,m_k$ vera margfeldni þeirra og setjum $m=\stig Q=m_1+\cdots+m_k$.
Þá er greinilegt að fallið $R$ hefur skaut af stigi $\leq m_j$ í
$\alpha_j$ og ef við látum 
 $$h_j(z)=\dfrac{A_{j,0}}{(z-\alpha_j)^{m_j}}+\cdots+
\dfrac{A_{j,m_j-1}}{(z-\alpha_j)}
 $$
tákna höfuðhluta fallsins $R$ í punktinum $\alpha_j$, þá hefur fallið
 $$f(z)= R(z)-h_1(z)-\cdots-h_k(z)
 $$
afmáanlega sérstöðupunkta í $\alpha_1,\dots,\alpha_k$.  Við setjum
$f(\alpha_j)=\lim_{z\to \alpha_j}f(z)$, og fáum  að $f\in
\O(\C)$.  Fyrst $\stig P <\stig Q$, þá sjáum við að  fallið sem
stendur hægra megin jafnaðarmerkisins stefnir á $0$ ef $|z|\to
+\infty$.    Setning Liouville segir okkur nú að $f$ sé núllfallið.
Þar með er
 $$R(z)=h_1(z)+\cdots+h_k(z).
 $$
Stuðlarnir í stofnbrotaliðuninni fást nú með því að 
reikna liðina í veldaröð fallanna $(z-\alpha_j)^{m_j}R(z)$
í punktunum $\alpha_j$, þeir eru gefnir með formúlunni
$$ A_{j,\ell}=\left.\dfrac 1{\ell!}
\bigg(\dfrac {d}{dz}\bigg)^{\ell}\bigg(
\dfrac{P(z)}{q_j(z)}\bigg)\right|_{z=\alpha_j}, \qquad \ell=0,\dots,m_j-1,
$$
þar sem $q_j(z)=Q(z)/(z-\alpha_j)^{m_j}$.

\subsection*{Verulegir sérstöðupunktar}

\begin{sk}
Einangraður sérstöðupunktur fágaða fallsins $f$ kallast {\it
verulegur sérstöðupunktur\index{einangraður
sérstöðupunktur!verulegur}\index{sérstöðupunktur!verulegur}\index{verulegur
sérstöðupunktur}}, 
ef hann er hvorki afmáanlegur sérstöðu\-punktur né skaut.  

\end{sk}

Hegðun fágaðra falla í grennd um verulega sérstöðupunkta er lýst með:

\begin{se}\tx{Casorati-Weierstrass\index{Casorati-Weierstrass-setningin}
\index{setning!Casorati-Weierstrass}}
Gerum ráð fyrir að $\alpha$ sé
verulegur sérstöðupunktur fallsins $f$.  Ef $\beta\in \C$,
$\varepsilon>0$ og $\delta>0$, þá er til $z\in S(\alpha,\delta)$
þannig að $f(z)\in S(\beta,\varepsilon)$.
\end{se}

\section{Leifasetningin }

\noindent
Við sáum í síðasta kafla hvernig hægt er að hagnýta
Cauchy-formúluna  og Cauchy-formúluna fyrir afleiður 
til þess að reikna út ákveðin heildi.  
Við ætlum nú að beita  Cauchy-setningunni til þess að 
alhæfa þessar formúlur fyrir heildi yfir lokaða vegi.
Við höfum séð að það er einstaklega auðvelt að
reikna út vegheildi  af föllum, sem gefin eru með samleitnum
Laurent-röðum yfir lokaða vegi, því við getum alltaf heildað röðina
lið fyrir lið og allir liðirnir hafa stofnfall nema sá með númerið
$-1$.  

\begin{se}\tx{Leifasetningin\index{leifasetningin}}  
Látum $X$ vera opið hlutmengi í $\C$ og látum $\Omega$ vera opið
hlutmengi af $X$ sem uppfyllir sömu forsendur og í
Cauchy-setningunni.
Látum $A$ vera dreift hlutmengi af $X$ sem sker ekki jaðarinn
$\partial\Omega$ á $\Omega$.  Ef $f\in \O(X\setminus A)$, þá er
 \begin{equation*}
\int_{\partial\Omega}f(z)\, dz = 2\pi i \sum_{\alpha\in \Omega\cap A}
\Res(f,\alpha).
\label{10.7.1}
 \end{equation*}
\end{se}

\medskip
Leifasetningin hefur mikla hagnýta þýðingu við útreikninga á ákveðnum
heild\-um.  Við gerum þeim hagnýtingum skil í næsta kafla, en það 
sem eftir er þessa kafla ætlum við að halda áfram að fjalla um ýmsar
afleiðingar af Cauchy-setningunni.




\section{Útreikningur á leifum}


\subsection*{Cauchy-formúla og leifasetning}


Látum $X$ vera opið hlutmengi af $\C$ og $\Omega$ vera opið hlutmengi
af $X$, þannig að jaðarinn $\partial\Omega$ af $\Omega$ sé einnig
innihaldinn í $X$.  Við hugsum okkur jafnframt að $\partial\Omega$ sé
stikaður af endanlega mörgum vegum $\gamma_1,\dots,\gamma_N$, sem
skerast aðeins í endapunktum, og að þeir stiki $\partial\Omega$ í
jákvæða stefnu, sem þýðir að svæðið sé vinstra megin við snertilínuna
í punkti $\gamma_j(t)$, ef horft er í stefnu $\gamma_j\dash(t)$.  Hér
höfum við verið að telja upp hluta af forsendum Cauchy--setningarinnar.  Til
viðbótar gerum við ráð fyrir að $A$ sé dreift hlutmengi af $X$  og að
$f\in \A(X\setminus A)$.  Þá eru allir punktarnir í $A$ einangraðir
sérstöðupunktar fallsins $f$ og leifasetningin segir okkur að
 \begin{equation*}\int _{\partial {\Omega}} f(\zeta)\, d\zeta =2\pi i
\sum\limits_{\alpha\in A\cap \Omega}
\Res (f,\alpha).\label{11.1.1}
 \end{equation*}
Ef $A\cap \Omega=\varnothing$, þá er summan sett $0$, eins og alltaf
þegar summa yfir tóma mengið er tekin.  Þetta er í fullu samræmi við
Cauchy--setninguna, því í þessu tilfelli er $f$ fágað í grennd um
$\overline\Omega=\partial\Omega\cup \Omega$ og þá er heildið í
vinstri hliðinni jafnt $0$.   Cauchy--formúlan er líka sértilfelli af
leifasetningunni, því ef $z\in \Omega$ og $\Omega\cap A=\varnothing$,
þá hefur fallið $\zeta\mapsto f(\zeta)/(\zeta-z)$ eitt skaut $z$ af
 stigi $\leq 1$ í $\Omega$ og leifasetningin segir okkur að
 $$\dfrac 1{2\pi i}\int_{\partial\Omega} \dfrac{f(\zeta)}{\zeta-z}\,
d\zeta = \Res \bigg( \dfrac{f(\zeta)}{\zeta-z},z\bigg)=f(z).
 $$

\subsection*{Leif í einföldu skauti}

 {Á}ður en við snúum okkur að því að
beita leifasetningunni til að leysa ákveðin dæmi, 
þá skulum við huga að því, hvernig farið er að
því að reikna út leif  $\Res(f,\alpha)$ fallsins $f$ í einangraða
sérstöðupunktinum $\alpha$.  Samkvæmt skilgreiningu er
$\Res(f,\alpha)=a_{-1}$, þar sem 
 \begin{equation*}f(z)=\sum\limits_{n=-\infty}\sp{+\infty}a_n(z-\alpha)\sp n, \qquad
z\in S\sp *(\alpha,\varepsilon),\label{11.1.2}
 \end{equation*}
er framsetning á $f$ með Laurent--röð.  Ef við höfum skaut af stigi
$1$ í punktinum $\alpha$, þá eru allir stuðlarnir $a_n=0$, $n<-1$, í
Laurent--röðinnni og við fáum
 $$(z-\alpha)f(z)=\sum\limits_{n=-1}\sp{+\infty} a_n(z-\alpha)\sp{n+1} =
\sum\limits_{n=0}\sp{+\infty} a_{n-1}(z-\alpha)\sp{n}.
 $$
Af þessari formúlu leiðir síðan
 \begin{equation*}\Res(f,\alpha)=\lim_{z\to \alpha}(z-\alpha)f(z).\label{11.1.3}
 \end{equation*}


\subsection*{Leif í skauti af stigi $m>1$}


Við skulum gera ráð fyrir að $f$ hafi
skaut af stigi $m>0$ í punktinum $\alpha$. Samkvæmt skilgreiningu er
þá til fágað fall $g$ í 
grennd $U$ um $\alpha$ þannig að $g(\alpha)\neq 0$ og
$f(z)=g(z)/(z-\alpha)\sp m$, $z\in 
U\setminus \set \alpha$.  Við sjáum sambandið milli stuðlanna
$b_n$ í Taylor--röð fallsins $g$ í punktinum $\alpha$ og stuðlanna
$a_n$ í Laurent röð fallsins $f$, út frá formúlunni
 $$f(z)=(z-\alpha)\sp{-m}\sum_{n=0}\sp{+\infty}b_n(z-\alpha)\sp n=
\sum_{n=0}\sp{+\infty}b_n(z-\alpha)\sp {n-m}=
\sum_{n=-m}\sp{+\infty}b_{n+m}(z-\alpha)\sp {n},
 $$
sem gefur okkur
 \begin{equation*}\Res(f,\alpha)=a_{-1}=b_{m-1}=\dfrac{g\sp{(m-1)}(\alpha)}{(m-1)!}.\label{11.1.4}
 \end{equation*}
Sértilfellið að $\alpha$ sé skaut af fyrsta stigi, sem við skrifuðum
upp í  (\ref{11.1.3}), er einfaldast, 
 \begin{equation*}\Res(f,\alpha)= g(\alpha), \qquad\qquad m=1.
\label{11.1.5}
 \end{equation*}

\subsection*{Cauchy-formúla fyrir afleiður og leifasetning}

Cauchy--formúlan fyrir afleiður er einnig sértilfelli af
leifasetningunni, því ef $A\cap \Omega=\varnothing$ og $z\in \Omega$
þá hefur fallið $\zeta\mapsto f(\zeta)/(\zeta-z)^{n+1}$ skaut af
stigi $\leq n+1$ og samkvæmt (\ref{11.1.4}) er 
 $$\dfrac{n!}{2\pi i}
\int_{\partial\Omega}\dfrac{f(\zeta)}{(\zeta-z)^{n+1}}\, d\zeta = 
{n!} \Res\bigg(\dfrac{f(\zeta)}{(\zeta-z)^{n+1}},z\bigg) =
f^{(n)}(z).
 $$

\subsection*{Leif af kvóta tveggja falla}


Nú skulum  við hugsa okkur að $f$ hafi skaut af stigi  $m$ í
$\alpha$ og að $f$ sé gefið í 
grennd um $\alpha$ sem $f(z)=g(z)/h(z)$, þar sem $g(\alpha)\neq
0$ og $h(\alpha)=0$.  Þá getum við skrifað $h(z)=(z-\alpha)^mh_1(z)$ þar
sem $h_1(z)$ er fágað í grennd um $\alpha$ og
$h_1(\alpha)=h^{(m)}(\alpha)/m!\neq 0$. Ef  $f$ hefur skaut af fyrsta
stigi, þá er leifin 
 \begin{equation*}\Res(f,\alpha)= \lim_{z\to \alpha}(z-\alpha) f(z)
=\lim_{z\to \alpha} 
\dfrac{(z-\alpha)g(z)}{h(z)-h(\alpha)}=\dfrac{g(\alpha)}{h\dash(\alpha)}.
\label{11.1.6}
 \end{equation*}
Þetta segir okkur, að 
formúlan sem við leiddum út í setningu \tilv 3.3.6, 
er ekkert annað en sértilfelli af leifasetningunni, því þar gerðum við 
ráð fyrir að núllstöðvar $\alpha_1,\dots,\alpha_m$ margliðunnar $Q$
væru einfaldar og því gefur leifasetningin
 $$\int_{\partial\Omega}\dfrac{f(\zeta)}{Q(\zeta)}\, d\zeta
=  2\pi i\sum_{\alpha_j\in \Omega}
\Res\bigg(\dfrac{f(\zeta)}{Q(\zeta)}, \alpha_j\bigg) 
=  2\pi i\sum_{\alpha_j\in \Omega} \dfrac{f(\alpha_j)}{Q\dash(\alpha_j)}.
 $$
Ef $f(z)=g(z)/h(z)$, þar sem $g(\alpha)\neq
0$ og $h$ hefur núllstöð af stigi  $m>1$ og við skrifum
$h(z)=(z-{\alpha})^mh_1(z)$, þá er
 \begin{equation*}\Res(f,\alpha)=\dfrac 1{(m-1)!}\cdot
\left.\dfrac {d^{m-1}}{dz^{m-1}}\bigg(\dfrac
{g(z)}{h_1(z)}\bigg)\right|_{z=\alpha}. \label{11.1.7}
 \end{equation*}


\subsection*{Leifar reiknaðar út frá stuðlum í veldaröðum}

Höldum nú áfram með útreikning okkar á leifum, gerum ráð fyrir að
$f=g/h$ og  
 $$f(z)=\sum\limits_{n=-m}^{\infty}a_n(z-\alpha)^n, \quad
g(z)=\sum\limits_{n=k}^{\infty}b_n(z-\alpha)^n, \quad
h(z)=\sum\limits_{n=l}^{\infty}c_n(z-\alpha)^n, 
 $$
hugsum okkur að stuðlarnir $b_n$, $c_n$ séu gefnir, $c_l\neq 0$,
$b_k\neq 0$  og að við
viljum reikna út leifina $\Res(f,\alpha)=a_{-1}$.  
Taylor--röð  $g$ er þá gefin sem
margfeldi af Laurent--röð $f$ og Taylor--röð $h$,
 $$
\sum\limits_{n=-m}^{\infty}a_n(z-\alpha)^n
\sum\limits_{n=l}^{\infty}c_n(z-\alpha)^n=
\sum\limits_{n=k}^{\infty}b_n(z-\alpha)^n.
$$ 
Þetta segir okkur að $-m+l=k$ og að við fáum sambandið milli
stuðlanna með því að margfalda saman raðirnar í vinstri hliðinni
\begin{gather*}
a_{-m}c_l=b_k,\\
a_{-m+1}c_l+a_{-m}c_{l+1}=b_{k+1},\\
a_{-m+2}c_l+a_{-m+1}c_{l+1}+a_{-m}c_{l+2}=b_{k+2},\\
\qquad \vdots\qquad\qquad\qquad\vdots\\
a_{-2}c_l+a_{-3}c_{l+1}+\cdots+a_{-m}c_{l+m-2}=b_{k+m-2}\\
a_{-1}c_l+a_{-2}c_{l+1}+\cdots+a_{-m}c_{l+m-1}=b_{k+m-1}.
\end{gather*}
Fyrst $c_l\neq 0$, þá fáum við $m$ skrefa rakningarformúlu fyrir $a_{-m},
a_{-m+1},\dots, a_{-1}$ og í síðasta skrefinu er leif $f$ í $\alpha$
fundin,
\begin{align*}
a_{-m}&=c_l^{-1}b_k,\\
a_{-m+1}&=c_l^{-1}\big(b_{k+1}
-a_{-m}c_{l+1}\big),\\
a_{-m+2}&=c_l^{-1}\big(b_{k+2}
-a_{-m+1}c_{l+1}-a_{-m}c_{l+2}\big),\\
&\qquad \vdots\qquad\qquad\qquad\vdots\\
a_{-2}&=c_l^{-1}\big(b_{k+m-2}
-a_{-3}c_{l+1}-\cdots-a_{-m}c_{l+m-2}\big)\\
\Res(f,\alpha)=a_{-1}&=c_l^{-1}\big(
b_{k+m-1}-a_{-2}c_{l+1}-\cdots-a_{-m}c_{l+m-1}\big).\label{11.1.8}
\end{align*}
Ef engin af aðferðunum, sem við höfum verið að fjalla um hér, dugir
til að finna leifina þá er ekkert annað að gera en að reikna hana út frá
formúlunni sem við leiddum út í Laurent--setningunni,
$$
\Res(f,\alpha) = \dfrac 1{2\pi i}\int_{\partial
S(\alpha,\varepsilon)} f(\zeta)\, d\zeta,
$$
þar sem við veljum geislann $\varepsilon$ í hringnum nógu lítinn.


\section{Heildi yfir einingarhringinn} 


\noindent
Við skulum gera ráð fyrir að $f$ sé fall af
tveimur breytistærðum $(x,y)$ og að $f$ sé skilgreint í grennd um
einingarhringinn, $x\sp 2+y\sp 2=1$.  Við fáum nú endurbót á
aðferðinni, sem við leiddum út eftir setningu \tilv 3.3.6.  Eins og þar
athugum við, að ef $z$ er á einingarhringnum, $z=e\sp{i\theta}$, þá
er 
\begin{gather*}
\cos\theta=\dfrac 12(e\sp{i\theta}+e\sp{-i\theta})
=\dfrac12(z+\dfrac 1z)=\dfrac{z\sp 2+1}{2z},\\ 
\sin\theta=\dfrac 1{2i}(e\sp{i\theta}-e\sp{-i\theta})
=\dfrac1{2i}(z-\dfrac 1z)=\dfrac{z\sp 2-1}{2iz},\\ 
dz=ie\sp{i\theta}d\theta, \qquad d\theta=\dfrac 1{iz}dz.
\end{gather*}
Við getum því reiknað  heildið út með leifareikningi
\begin{align*}
\int_0\sp {2\pi}f(\cos\theta,\sin
\theta)\, d\theta &=
\int_{\partial S(0,1)}f\big(\dfrac{z\sp 2+1}{2z},\dfrac{z\sp 2-1}{2iz}\big)
\dfrac 1{iz}\, dz\\
&=2\pi i \sum_{\alpha\in A\cap S(0,1)} \Res\bigg(
f\big(\dfrac{z\sp 2+1}{2z},\dfrac{z\sp 2-1}{2iz}\big)\dfrac 1{iz},\alpha
\bigg),
\end{align*}
ef til er opin grennd $X$ um lokuðu einingarskífuna $\overline S(0,1)$
og dreift mengi $A$ þannig að fallið $z\mapsto f\big({(z\sp
2+1)}/{(2z)},{(z\sp 2-1)}/{(2iz)}\big)/(iz)$ sé fágað á $X\setminus
A$. 



\section{Heildi yfir raunásinn}

\noindent
Nú ætlum við að snúa okkur að heildum af gerðinni
 \begin{equation*}I=\int_{-\infty}\sp{+\infty}f(x) \, dx \label{11.3.1}
 \end{equation*}
þar sem fallið $f$ er fágað í grennd um $\R$.
Hugsum okkur fyrst að $f\in \O(\C\setminus A)$, þar sem $A$ er dreift
mengi. Aðferðin  byggir á því að athuga að 
 $$I=\lim_{r\to +\infty}\int_{-r}\sp r f(x)\, dx,
 $$
ef heildið (\ref{11.3.1}) er samleitið.
Leifasetningin gefur okkur þá
 $$\int_{-r}\sp{r}f(x)\, dx +\int_{\gamma_r}f(z)\, dz =
2\pi i\sum_{\alpha\in A\cap \Omega_r}\Res(f,\alpha)
 $$
og jafnframt 
 $$\int_{-r}\sp{r}f(x)\, dx +\int_{\beta_r}f(z)\, dz =
-2\pi i\sum_{\alpha\in A\cap \widetilde\Omega_r}\Res(f,\alpha),
 $$
þar sem $\Omega_r$ og $\widetilde\Omega_r$ eru hálfskífurnar á
myndinni.

\figura{fig101}{Mynd: Hálfskífur í efra og neðra hálfplani}

\noindent
Ef unnt er að sýna fram á að önnur hvor summan í hægri hliðunum hafi
markgildi ef $r\to +\infty$ og að tilsvarandi vegheildi
 $$\int_{\gamma_r}f(z)\, dz \qquad \text{ eða }
\qquad \int_{\beta_r}f(z)\, dz 
 $$
stefni á núll, þá verður
 $$I=\int_{-\infty}\sp{+\infty}f(x)\, dx =
2\pi i\sum_{\alpha\in A\cap H_+}\Res\big(f,\alpha)
 $$
eða
 $$I=\int_{-\infty}\sp{+\infty}f(x)\, dx =
-2\pi i\sum_{\alpha\in A\cap H_-}\Res\big(f,\alpha)
 $$
þar sem $H_+=\set{z\in \C; \Im z>0}$ táknar efra hálfplanið
og $H_-=\set{z\in \C; \Im z<0}$ táknar neðra hálfplanið.

Lítum nú á tilfellið að $f(x)=P(x)/Q(x)$ sé rætt fall, að $P$ og $Q$ séu
margliður  með $\stig\, P\leq \stig\, Q-2$, og að $Q$ hafi engar
núllstöðvar á $\R$.  Auðvelt er að sannfæra sig um að til er fasti
$C$ þannig að
 $$
|f(z)|\leq \dfrac C{r\sp 2},
 $$
ef $|z|=r$ og $r$ er það stórt að allar núllstöðvar $Q$ liggja í
$S(0,r-1)$.  Lengd veganna $\gamma_r$ og $\beta_r$ er $\pi r$, svo
við fáum 
 $$|\int_{\gamma_r}f(z)\, dz|\leq \pi C/r\to 0, \qquad r\to +\infty,
 $$
og sama mat fæst fyrir heildið af $f(z)$ yfir $\beta_r$.  Niðurstaðan
verður því að 
 $$\int\limits_{-\infty}\sp {+\infty} f(x)\, dx =
2\pi i\sum_{\alpha\in {\cal N}(Q)\cap H_+}\Res\big(f,\alpha)
=-2\pi i\sum_{\alpha\in {\cal N}(Q)\cap H_-}\Res\big(f,\alpha),
 $$
þar sem ${\cal N}(Q)$ er núllstöðvamengi $Q$. 
