
%
%Allir pakkar sem þarf að nota.
%
\usepackage[utf8]{inputenc}
\usepackage[T1]{fontenc}
\usepackage[icelandic]{babel}
\usepackage{amsmath}
\usepackage{amssymb}
\usepackage{pictex}
\usepackage{epsfig,psfrag}
\usepackage{makeidx}
%\selectlanguage{icelandic}
%----------------------------

%
\hoffset=-0.4truecm
\voffset=-1truecm
\textwidth=16truecm 
%\textwidth=12truecm 
\textheight=23truecm
\evensidemargin=0truecm
%
% Gömlu gildin á bókinni 
%
%\voffset 1.4truecm
%\hoffset .25truecm
%\vsize  16.0truecm
%\hsize  15truecm
%
%
% Skilgreiningar á ýmsum skipunum.
%
%\newcommand{\Sb}{
%$$
%\sum_{\footnotesize\begin{array}{l} j=1 \\ j\neq k \end{array}}
%$$
%}
\newcommand{\bolddot}{{\mathbf \cdot}}
\newcommand{\C}{{\mathbb  C}}
\newcommand{\Cn}{{\mathbb  C\sp n}}
\newcommand{\crn}{{{\mathbb  C\mathbb  R^n}}}
\newcommand{\R}{{\mathbb  R}}
\newcommand{\Rn}{{\mathbb  R\sp n}}
\newcommand{\Rnn}{{\mathbb  R\sp{n\times n}}}
\newcommand{\Z}{{\mathbb  Z}}
\newcommand{\N}{{\mathbb  N}}
\renewcommand{\P}{{\mathbb  P}}
\newcommand{\Q}{{\mathbb  Q}}
\newcommand{\K}{{\mathbb  K}}
\newcommand{\U}{{\mathbb  U}}
\newcommand{\D}{{\mathbb  D}}
\newcommand{\T}{{\mathbb  T}}
\newcommand{\A}{{\cal A}}
\newcommand{\E}{{\cal E}}
\newcommand{\F}{{\cal F}}
\renewcommand{\H}{{\cal H}}
\renewcommand{\L}{{\cal L}}
\newcommand{\M}{{\cal M}}
\renewcommand{\O}{{\cal O}}
\renewcommand{\S}{{\cal S}}
\newcommand{\dash}{{\sp{\prime}}}
\newcommand{\ddash}{{\sp{\prime\prime}}}
\newcommand{\tdash}{{\sp{\prime\prime\prime}}}
\newcommand{\set }[1]{{\{#1\}}}
\newcommand{\scalar}[2]{{\langle#1,#2\rangle}}
\newcommand{\arccot}{{\operatorname{arccot}}}
\newcommand{\arccoth}{{\operatorname{arccoth}}}
\newcommand{\arccosh}{{\operatorname{arccosh}}}
\newcommand{\arcsinh}{{\operatorname{arcsinh}}}
\newcommand{\arctanh}{{\operatorname{arctanh}}}
\newcommand{\Log}{{\operatorname{Log}}}
\newcommand{\Arg}{{\operatorname{Arg}}}
\newcommand{\grad}{{\operatorname{grad}}}
\newcommand{\graf}{{\operatorname{graf}}}
\renewcommand{\div}{{\operatorname{div}}}
\newcommand{\rot}{{\operatorname{rot}}}
\newcommand{\curl}{{\operatorname{curl}}}
\renewcommand{\Im}{{\operatorname{Im\, }}}
\renewcommand{\Re}{{\operatorname{Re\, }}}
\newcommand{\Res}{{\operatorname{Res}}}
\newcommand{\vp}{{\operatorname{vp}}}
\newcommand{\mynd}[1]{{{\operatorname{mynd}(#1)}}}
\newcommand{\dbar}{{{\overline\partial}}}
\newcommand{\inv}{{\operatorname{inv}}}
\newcommand{\sign}{{\operatorname{sign}}}
\newcommand{\trace}{{\operatorname{trace}}}
\newcommand{\conv}{{\operatorname{conv}}}
\newcommand{\Span}{{\operatorname{Sp}}}
\newcommand{\stig}{{\operatorname{stig}}}
\newcommand{\Exp}{{\operatorname{Exp}}}
\newcommand{\diag}{{\operatorname{diag}}}
\newcommand{\adj}{{\operatorname{adj}}}
\newcommand{\erf}{{\operatorname{erf}}}
\newcommand{\erfc}{{\operatorname{erfc}}}
\newcommand{\Lloc}{{L_{\text{loc}}\sp 1}}
\newcommand{\boldcdot}{{\mathbb \cdot}}
%\newcommand{\Cinf0}[1]{{C_0\sp{\infty}(#1)}}
\newcommand{\supp}{{\text{supp}\, }}
\newcommand{\chsupp}{{\text{ch supp}\, }}
\newcommand{\singsupp}{{\text{sing supp}\, }}
\newcommand{\SL}[1]{{\dfrac {1}{\varrho} 
\bigg(-\dfrac d{dx}\bigg(p\dfrac {d#1}{dx}\bigg)+q#1\bigg)}}
\newcommand{\SLL}[1]{-\dfrac d{dx}\bigg(p\dfrac {d#1}{dx}\bigg)+q#1}
\newcommand{\Laplace}[1]{\dfrac{\partial^2 #1}{\partial x^2}+\dfrac{\partial^2 #1}{\partial y^2}}
\newcommand{\polh}[1]{{\widehat #1_{\C^n}}}
\newcommand{\tilv}{{}}
%
\renewcommand{\chaptername}{Kafli}
%
% Númering á formæulum.
%
\numberwithin{equation}{section}
%
%  Innsetning á myndum.
%
\def\figura#1#2{
\vbox{\centerline{
\input #1
}
\centerline{#2}
}\medskip}
\def\vfigura#1#2{
\setbox0\vbox{{
\input #1
}}
\setbox1\vbox{\hbox{\box0}\hbox{{\obeylines #2}}}
\dimen0 = -\ht1
\advance\dimen0 by-\dp1
\dimen1 = \wd1
\dimen2 = -\dimen0
\divide\dimen2 by\baselineskip
\count100 = 1
\advance\count100 by\dimen2
\advance\count100 by1
\box1
\hangindent\dimen1
\hangafter=-\count100
\vskip\dimen0
}
%
%  Setningar, skilgreiningar, o.s.frv. 
%
\newtheorem{setning+}           {Setning}      [section]
\newtheorem{skilgreining+}  [setning+]  {Skilgreining}
\newtheorem{setningogskilgreining+}  [setning+]  {Setning og
skilgreining}
\newtheorem{hjalparsetning+}  [setning+]  {Hjálparsetning}
\newtheorem{fylgisetning+}  [setning+]  {Fylgisetning}
\newtheorem{synidaemi+}  [setning+]  {Sýnidæmi}
\newtheorem{forrit+}  [setning+]  {Forrit}

\newcommand{\tx}[1]{{\rm({\it #1}). \ }}

\newenvironment{se}{\begin{setning+}\sl}{\hfill$\square$\end{setning+}\rm}
\newenvironment{sex}{\begin{setning+}\sl}{\hfill$\blacksquare$\end{setning+}\rm}
\newenvironment{sk}{\begin{skilgreining+}\rm}{\hfill$\square$\end{skilgreining+}\rm}
\newenvironment{sesk}{\begin{setningogskilgreining+}\rm}{\hfill$\square$\end{setningogskilgreining+}\rm}
\newenvironment{hs}{\begin{hjalparsetning+}\sl}{\hfill$\square$\end{hjalparsetning+}\rm}
\newenvironment{fs}{\begin{fylgisetning+}\sl}{\hfill$\square$\end{fylgisetning+}\rm}
\newenvironment{sy}{\begin{synidaemi+}\rm}{\hfill$\square$\end{synidaemi+}\rm}
\newenvironment{fo}{\begin{forrit+}\rm}{\hfill\end{forrit+}\rm}
\newenvironment{so}{\medbreak\noindent{\it Sönnun:}\rm}{\hfill$\blacksquare$\rm}
\newenvironment{sotx}[1]{\medbreak\noindent{\it #1:}\rm}{\hfill$\blacksquare$\rm}
\newcounter{daemateljari}
\newcommand{\aefing}{\section{Æfingardæmi} \setcounter{daemateljari}{1}}
\newcommand{\daemi}{
{\medskip\noindent{\bf \thedaemateljari.}}
\addtocounter{daemateljari}{1}
}

%\def\aefing{{\large\bf\bigskip\bigskip\noindent Æfingardæmi}}
%\def\daemi#1{\medskip\noindent{\bf #1.}}
\def\svar#1{\smallskip\noindent{\bf #1.} \ }
\def\lausn#1{\smallskip\noindent{\bf #1.} \ }
\def\ugrein#1{\medbreak\noindent{\bf #1.} }
\newcommand{\samantekt}{\noindent{\bf Samantekt.} }
%\newcommand{\proclaimbox}{\hfill$\square$}

\chapter {CAUCHY-SETNINGIN OG CAUCHY-FORMÚLAN}
 

\section
{Vegheildun\index{heildi}\index{heildi!vegheildi}\index{vegheildun}}

\subsection*{Ferlar og vegir}


Ferill í  $\C$ er myndmengi samfellds falls $\gamma:[a,b]\to \C$, þar
sem gefin er stefnan frá {\it
upphafspunkti\index{ferill!upphafspunktur}\index{upphafspunktur
ferils}} $u_\gamma=\gamma(a)$
til {\it lokapunkts\index{ferill!lokapunktur}\index{lokapunktur
ferils}} $e_\gamma=\gamma(b)$ ferilsins.  

Ef $u_\gamma=e_\gamma$, þá segjum við að ferilinn sé {\it
lokaður\index{lokaður ferill}\index{ferill}\index{ferill!lokaður}}. 

Við
segjum að ferillinn sé {\it einfaldur\index{einfaldur
ferill}\index{ferill}\index{ferill!einfaldur}}  ef
$\gamma(t_1)\neq \gamma(t_2)$ fyrir $t_1\neq t_2$, með hugsanlegri
undantekningu að $\gamma(a)=\gamma(b)$.   Að ferillinn sé einfaldur
þýðir nákvæmlega að hann skeri ekki sjálfan sig, hugsanlega með þeirri
undantekningu að upphafs- og lokapunkturinn sé sá sami. 


Þó svo að
ferillinn sé myndmengi samfellda fallsins $\gamma$, þá lítum við
oft svo á að ferilinn sé fallið sjálft og köllum hann þá {\it
stikaferil}.    Stöku sinnum viljum við þó
gera greinarmun á þessu tvennu og þá notum við táknið 
$\mynd \gamma=\set{\gamma(t); t\in [a,b]}$ til að
tákna ferilinn og segjum að ferillinn sé {\it stikaður} með 
$\gamma$. 

\figura{fig091}{{\small Mynd: Vegir.}}

\noindent
Ferill sem er samfellt deildanlegur á köflum kallast {\it
vegur\index{ferill!vegur}\index{vegur!ferill}\index{vegur}}.

Þetta þýðir að til er skipting $a=t_0<t_1\cdots<t_n=b$ á bilinu
$[a,b]$ þannig að $\gamma$ sé samfellt deildanlegt á opnu bilunum
$]t_{j-1},t_j[$, $j=1,\dots, n$ og að í endapunktum\index{endapunktur
ferils}\index{ferill!endapunktur} bilanna séu bæði
hægri og vinstri afleiða til,
\begin{gather*}
\lim_{h\to 0+}\dfrac{\gamma(t_j+h)-\gamma(t_j)}h, \qquad
j=0,\dots,n-1,\\ 
\lim_{h\to 0-}\dfrac{\gamma(t_j+h)-\gamma(t_j)}h, \qquad
j=1,\dots,n.
\end{gather*}

\subsection*{Lengd vega}


Ef  $\gamma$ er vegur, þá er unnt að skilgreina lengd
hans\index{lengd!vegs}\index{vegur!lengd} sem 
$$
L(\gamma)=\lim \sum_{j=1}\sp N |\gamma(\tau_j)-\gamma(\tau_{j-1})|,
$$
þar sem markgildið er tekið þegar fínleiki skiptingarinnar
$a=\tau_0<\tau_1<\cdots<\tau_N=b$ stefnir á núll.  Með því að líta á
hægri hliðina í þessari jöfnu sem
Riemann-summu, þá fáum við
 $$L(\gamma)=\int_a\sp b |\gamma\dash(t)| \, dt.
 $$
Heildið er vel skilgreint, því $\gamma$ er samfellt deildanlegt á
köflum  og því er afleiðan samfelld alls staðar nema í endanlega
mörgum punktum, en í grennd um þá punkta er $\gamma\dash$ takmarkað.  

\subsection*{Heildi með tilliti til bogalengdar}


Látum nú $C$ vera  veg og $f$ vera samfellt fall á $C$.  Við
skilgreinum  heildið af $f$ yfir $C$ með tilliti til
bogalengdar\index{bogalengd}\index{heildi!m.t.t. bogalengdar} sem 
 $$\int_Cf \, ds = \int_a\sp b f(\gamma(t)) |\gamma\dash(t)|\, dt.
 $$
Við notum líka táknin
$$\int_\gamma f\, ds,  \quad \int_C f\, |dz| \quad \text { og }
\quad \int_\gamma f\, |dz|
$$
fyrir þetta heildi, ef vegurinn $C$ er stikaður með $\gamma$.  Við
sjáum að heildið er óháð stikuninni, því ef við stikum veginn með
$\gamma_1=\gamma\circ h$, þar sem $h:[c,d]\to [a,b]$, þá  fáum við
\begin{align*}
\int_c\sp d f(\gamma_1(t)) |\gamma_1\dash(t)|\, dt&=
\int_c\sp d f(\gamma(h(t))) |\gamma\dash(h(t))h\dash(t)|\, dt\\
&=
\int_a\sp bf(\gamma(\tau))|\gamma\dash(\tau)| \, d\tau.
\end{align*}
{Á} þessari sömu formúlu sjáum við að heildið er jafnframt óháð
stefnunni á veginum.  

\subsection*{Vegheildi}

Ef $f$ er skrifað sem fall af $z=x+iy$ og
$\gamma(t)=\alpha(t)+i\beta(t)$, þá skilgreinum við heildið af  $f$
yfir $C$ með tilliti til $x$ sem 
 $$\int_C f \, dx =\int_\gamma f\, dx = \int_a\sp b f(\gamma(t))
\alpha\dash(t) \, dt
 $$
og heildið af $f$ yfir $C$  með tilliti til $y$ sem
 $$\int_C f \, dy =\int_\gamma f\, dy = \int_a\sp b f(\gamma(t))
\beta\dash(t) \, dt.
 $$
Ef $f$ og $g$ eru samfelld föll á $C$, þá setjum við 
 $$\int_C f\, dx +g \, dy = \int_C f\, dx + \int_C g\, dy.
 $$
Eðlilegt er að skilgreina heildið af $f$ yfir veginn
$C$ með tilliti til $z$ og $\bar z$ með formúlunum
\begin{gather*}
\int_C f\, dz =\int_\gamma f\, dz= \int_\gamma f\, dx +if\, dy =
\int_a^bf(\gamma(t))\gamma\dash(t) \, dt,\\
\int_C f\, d\bar z =\int_\gamma f\, d\bar z= \int_\gamma
f\, dx -if\, dy = 
\int_a^bf(\gamma(t))\overline{\gamma\dash(t)} \, dt.
\end{gather*}
Við athugum nú að öll þessi heildi eru háð stefnunni á $C$.  


\subsection*{Heildi yfir öfugan veg}

Við
skilgreinum {\it öfuga veginn\index{vegur!öfugur}\index{öfugur vegur}}
$\gamma_-$ við $\gamma$ með formúlunni 
 $$\gamma_-(t)=\gamma(a+b-t), \qquad t\in [a,b].
 $$
Öfugi vegurinn ${\gamma}_-$ við ${\gamma}$ stikar sama mengi og
${\gamma}$, en farið er yfir mengið í öfuga stefnu,
þ.e.~$u_{\gamma_-}=e_{\gamma}$ og $e_{\gamma_-}=u_{\gamma}$.
Við fáum því     
\begin{align*}
\int_a\sp b f(\gamma_-(t))\alpha_-\dash(t) \, dt &=
\int_a\sp b f(\gamma(a+b-t))(-\alpha\dash(a+b-t)) \, dt \\
&=
-\int_a\sp b f(\gamma(t))\alpha\dash(t) \, dt.
\end{align*}
Þar með er
 $$\int_{\gamma_-}f\, dx = -\int_\gamma f\, dx,
 $$
og á sama hátt fáum við 
 $$
\int_{\gamma_-}f\, dy = -\int_\gamma f\, dy, \quad 
\int_{\gamma_-}f\, dz = -\int_\gamma f\, dz \quad \text{ og } \quad
\int_{\gamma_-}f\, d\bar z = -\int_\gamma f\, d\bar z.
 $$


\subsection*{Mat á heildum}

Við þurfum oft  að meta heildi og notfærum okkur þá
oftast formúluna
 $$
| \int_C f(z)\, dz| \leq \int_\gamma |f(z)|\, |dz|\leq \max_{z\in C}
|f(z)| \int_\gamma |dz|= \max_{z\in C}|f(z)|L(C).
 $$


\subsection*{Heildi yfir línustrik og hringboga }

Mikilvægustu vegheildin, sem við þurfum að reikna út, eru tekin yfir
hringboga\index{hringbogi} og línustrik\index{línustrik}.  Við skulum
líta á stikanir á þessum ferlum.  Ef $\alpha$ og $\beta$ eru tveir punktar í
$\C$, þá látum við $\scalar \alpha \beta$ tákna línustrikið á milli
þeirra.  Það er gefið með stikuninni
 $$
\gamma:[0,1]\to \C, \qquad \gamma(t)=(1-t)\alpha+t\beta, \qquad
\gamma\dash(t)= (\beta-\alpha), \qquad t\in [0,1],
 $$
og þar með er 
 $$\int_{\scalar{\alpha}{\beta}} f \, dz = (\beta-\alpha)\int_0\sp 1
f((1-t)\alpha+t\beta)\, dt.
 $$
Boginn af hringnum $\partial S(\alpha,r)$, sem liggur milli
horngildanna $t=a$ og $t=b$, $b-a\leq 2{\pi}$, er stikaður með
 $$\gamma:[a,b]\to \C, \qquad
\gamma(t)= \alpha+re\sp{it}, \qquad
\gamma\dash(t)= ire\sp{it}, \qquad t\in [a,b],
 $$
og við fáum 
 $$
\int_\gamma f \, dz = \int_a\sp b f(\alpha+re\sp{it})ire\sp{it}\, dt
= ir \int_a\sp b f(\alpha+re\sp{it})e\sp{it}\, dt.
 $$


Auðvelt er að sýna fram á, að opið mengi $X$ er svæði þá og því aðeins að
hægt sé að tengja sérhverja tvo punkta saman með vegi, sem samanstendur
af línustrikum. Einnig er auðvelt að sýna að alltaf sé hægt að velja ferilinn
einfaldan og strikin samsíða hnitaásunum. 

\subsection*{Stofnföll}

Undirstöðusetning stærðfræðigreiningarinnar gefur okkur

\begin{se} \label{se:10.1.3} Gerum ráð fyrir að $X$ sé opið mengi og $f\in C(X)$. Ef $f$ hefur
stofnfall\index{stofnfall} $F$, þ.e.a.s.~ef til er fall $F\in \O(X)$
þannig að
$F\dash=f$ þá er
 $$\int_\gamma f(z)\, dz = F(e_\gamma)-F(u_\gamma)
 $$
fyrir sérhvern veg $\gamma$ í $X$.  Sérstaklega gildir
 $$\int_\gamma f(z)\, dz = 0
 $$
fyrir sérhvern lokaðan veg $\gamma$ í $X$.  Ef $X$ er svæði og 
$f\dash(z)=0$ fyrir öll $z\in X$, þá er $f$ fastafall.
\end{se}


\section{Green-setningin}

Við látum $X$ vera opið hlutmengi af $\C$, $\Omega$ vera opið hlutmengi
af $X$ þannig að jaðarinn $\partial\Omega$ á $\Omega$ sé í $X$ og gerum
ráð fyrir  að $\partial\Omega$  sé einfaldur lokaður vegur sem stikaður
er í {\it jákvæða stefnu\index{jákvæð stefna}} miðað við $\Omega$.
Þetta þýðir að 
$$ \partial\Omega=\mynd {\gamma}=\set{\gamma(t); t\in [a,b]} $$
þar sem $\gamma(a)=\gamma(b)$, $\gamma(t_1)\neq \gamma(t_2)$ ef
$t_1\neq t_2$, $t_1,t_2\in ]a,b[$ og í punktum þar sem $\gamma$ er
deildanlegt, þá liggur $\Omega$ á vinstri hönd ef staðið er í 
$\gamma(t)$  og horft er í stefnu $\gamma\dash(t)$.

\figura{fig092}{{\small Mynd:  Stikun á jaðri }}

\noindent Þá segir Green-setningin að um sérhver $f,g\in C\sp 1(X)$ gildi
 $$\int_{\partial\Omega} f\, dx +g \, dy =\iint_\Omega(\partial_x
g-\partial_y f)\, dxdy.
 $$
Þegar þessi regla hefur verið sönnuð fyrir raungild föll $f$ og $g$,
þá er alveg ljóst að hún gildir fyrir tvinnföll einnig, því við tökum
heildin af raun- og þverhlutum fyrir hvort um sig.

Green-setningin  gildir fyrir almennari svæði en þetta, nefnilega svæði
$\Omega$ þar sem jaðarinn $\partial\Omega$ samanstendur af einföldum
vegum $\gamma_j:[a_j,b_j]\to \C$, $j=1,\dots,N$, 
sem skerast einungis í endapunktum og
hafa jákvæða stefnu miðað við $\Omega$.  Þetta þýðir að 
 $$\partial\Omega=\bigcup\limits_{j=1}^N\operatorname{mynd} (\gamma_j)=
\bigcup\limits_{j=1}^N \set{\gamma_j(t); t\in [a_j,b_j]},
 $$
og að í punktunum $\gamma_j(t)$, þar sem vegirnir eru deildanlegir, er
$\Omega$ á vinstri hönd ef staðið er í 
$\gamma(t)$  og horft er í stefnu $\gamma\dash(t)$.

\figura{fig093}{{\small Mynd: Stikun á jaðri sem myndaður er
úr fjórum vegum}}

\noindent
Við skilgreinum heildið  af $f$  með
tilliti til $x$ og $g$ með tilliti til $y$ yfir jaðarinn
$\partial\Omega$ með formúlunni
 $$\int_{\partial\Omega}f\, dx + g\, dy =\sum_{j=1}^N \int_{\gamma_j}f
\, dx + g\, dy
 $$
og Green-setningin fær þá sama form og áður
 $$\int_{\partial\Omega}f\, dx+g\, dy =\iint_\Omega
(\partial_xg-\partial_yf)\, dxdy.
 $$


 
\section {Cauchy-setningin og Cauchy-formúlan}

\subsection*{Cauchy-setningin}

Nú skulum við gera ráð fyrir því að $X$ sé opið hlutmengi í $\C$ og
að $\Omega\subset X$ uppfylli forsendur Green-setningarinnar.  Við
tökum $f\in C\sp 1(X)$, $f(z)=u(x,y)+iv(x,y)$, $z=x+iy$, þar sem $u$ og
$v$ eru raun-  og  þverhluti $f$.
Þá gefur Green-setningin,
\begin{align*}
\int_{\partial\Omega} f\, dz 
&=\int_{\partial\Omega} (u+iv)\, (dx+idy)\label{10.2.1}\\
&=\int_{\partial\Omega} u\, dx - v\, dy
+i\int_{\partial\Omega} v\, dx + u\, dy\nonumber\\
&=\iint_{\Omega}\big(-\partial_x v-\partial_y u\big) \, dxdy
+i\iint_{\Omega}\big(\partial_x u-\partial_y v\big) \, dxdy\nonumber\\
&=\iint_{\Omega}i\big(\partial_x u+i\partial_x v\big)-
\big(\partial_y u+i\partial_y v \big) \, dxdy \nonumber\\
&=i\iint_{\Omega}\big(\partial_x f+i\partial_y f\big) \, dxdy.
=2i\iint_{\Omega}\partial_{\bar z} f \, dxdy.\nonumber
\end{align*}
Nú erum við komin að undirstöðusetningu tvinnfallagreiningarinnar:

\begin{se}\tx{Cauchy-setningin\index{Cauchy!setningin}
\index{Cauchy}\index{setning!Cauchy}}
Látum $X$ vera opið hlutmengi af $\C$, $\Omega\subset X$ vera opið,
þannig að $\partial\Omega\subset X$ og gerum ráð fyrir að
$\partial\Omega$ samanstandi af endanlega mörgum vegum, sem skerast
aðeins í endapunktum og hafa jákvæða stefnu miðað við ${\Omega}$.
Ef $f\in C^1(X)$, þá er
\begin{equation*}\int_{\partial\Omega}f\, dz = 
2i\iint_{\Omega}\partial_{\bar z} f \, dxdy.
\label{10.2.2}
\end{equation*}
Ef $f\in \O(X)$, þá er
\begin{equation*}
\int_{\partial\Omega}f\, dz = 0.
\label{10.2.3}
\end{equation*}
\end{se}



\subsection*{Stjörnusvæði}


Á sumum tegundum svæða fáum við miklu almennari útgáfu af
Cauchy-setningunni en hér hefur verið sönnuð: 

\begin{sk}
Opið mengi $X$ kallast {\it stjörnusvæði með tilliti til punktsins }
$\alpha\in X$, ef línustrikið $\scalar \alpha z$ er innihaldið í $X$
fyrir sérhvert $z\in X$.  Við segjum að $X$ sé {\it
stjörnusvæði\index{stjörnusvæði}} ef
það er stjörnusvæði með tilliti til einhvers punkts í $X$.
\end{sk}

\figura{fig094}{Mynd: Dæmi  um stjörnusvæði.}

\begin{se} \label{se:10.2.3}
Ef $X$ er stjörnusvæði með tilliti til punktsins $\alpha$, 
þá hefur sérhvert $f\in \O(X)$ stofnfall, sem
gefið er með formúlunni
\begin{equation*}\label{10.2.4a}
F(z)=\int_{\scalar \alpha z} f(\zeta)\, d\zeta, \qquad z\in X.
\end{equation*}
og
þar með gildir 
\begin{equation*}\int_\gamma f\, dz =0
\label{10.2.4}
 \end{equation*}
fyrir sérhvern lokaðan veg $\gamma$ í $X$.
\end{se}

\subsection*{Cauchy-formúlan}


Nú ætlum við að beita Cauchy setningunni á fallið 
$\zeta\mapsto f(\zeta)/(\zeta-z)$ þar sem $z$ er punktur í
svæðinu $\Omega$.  Þá fæst:

\begin{se}\tx{Cauchy-formúlan\index{Cauchy!formúlan}\index{Cauchy}}
Gerum ráð fyrir sömu forsendum og í Cauchy -setn\-ing\-unni.  Ef
$f\in C^1(X)$, þá gildir um sérhvert $z\in \Omega$ að
\begin{equation*}
f(z)=\dfrac 1{2 \pi i}\int_{\partial\Omega}\dfrac
{f(\zeta)}{\zeta-z}\, d\zeta -\dfrac 1{2\pi}\iint_{\Omega}
\dfrac{(\partial_\xi+i\partial_\eta)f(\zeta)}
{\zeta-z}\, d\xi d\eta, \label{10.2.5}
\end{equation*}
þar sem breytan í heildinu er ${\zeta}={\xi}+i\eta$.
Ef $f\in \O(X)$, þá er 
\begin{equation*}
f(z)=\dfrac 1{2 \pi i}\int_{\partial\Omega}\dfrac
{f(\zeta)}{\zeta-z}\, d\zeta.\label{10.2.6}
\end{equation*}
\end{se}


\subsection*{Meðalgildissetning}

Í sértilfellinu að $\Omega$ sé hringskífa, þá gefur Cauchy-formúlan:

\begin{se}\tx{Meðalgildissetning\index{meðalgildissetning}}  
Látum $X$ vera opið mengi í $\C$, $f\in
\O(X)$,  $z\in X$
og gerum ráð fyrir að $\overline S(z,r)\subset X$.  
Þá  gildir
 $$f(z)=\dfrac 1{2\pi} \int_0\sp{2\pi}f(z+re\sp{it})\, dt.
 $$
\end{se}


Setningin segir okkur að meðalgildi fágaðs falls yfir jaðar
hringskífu er jafnt gildi fallsins í miðpunkti skífunnar.


\subsection*{Útreikningur á heildum}

Nú skulum við kanna, hvernig hægt er að beita Cauchy-formúlunni til
þess að reikna út ýmis ákveðin heildi.  Til undirbúnings á því hugsum
við okkur að forsendurnar í Cauchy-setningunni séu uppfylltar og að
$Q(z)$ sé margliða af stigi $m$ með einfaldar núllstöðvar
$\alpha_1,\dots,\alpha_m$ og að engin þeirra liggi á
$\partial\Omega$.  Við skrifum upp stofnbrotaliðun á $1/Q(z)$, sem
við fjölluðum um í grein \tilv 1.5, og fáum 
 $$\dfrac 1{Q(z)} = \dfrac 1{Q\dash(\alpha_1)(z-\alpha_1)}+\cdots+
\dfrac 1{Q\dash(\alpha_m)(z-\alpha_m)}.
 $$
Þá getum við liðað heildið
 $$\int_{\partial\Omega}\dfrac{f(z)}{Q(z)}\, dz =
\dfrac 1{Q\dash(\alpha_1)}\int_{\partial\Omega}\dfrac{f(z)}{z-\alpha_1}\,
dz 
+\cdots+
\dfrac 1{Q\dash(\alpha_m)}\int_{\partial\Omega}\dfrac{f(z)}{z-\alpha_m}\,
dz. 
 $$
Ef $\alpha_j\in \Omega$, þá gefur Cauchy-formúlan
$$
\int_{\partial\Omega}\dfrac{f(z)}{z-\alpha_j}\,
dz  = 2\pi i f(\alpha_j).
$$
Ef aftur á móti $\alpha_j\not\in\Omega$, þá er fallið
$f(z)/(z-\alpha_j)$ fágað í grennd um $\Omega\cup\partial\Omega$, svo
Cauchy-setningin segir okkur að heildi þess með tilliti til $z$ yfir
$\partial\Omega$ sé $0$.  Niðurstaða þessa útreiknings er því:

\begin{se}\label{set10.2.6}
Gerum ráð fyrir að forsendur Cauchy-setningarinnar séu uppfylltar og
að $Q$ sé margliða með einfaldar núllstöðvar
$\alpha_1,\dots,\alpha_m$ og að engin þeirra liggi á
$\partial\Omega$.  Þá er
 \begin{equation*}\int_{\partial\Omega} \dfrac{f(z)}{Q(z)} \, dz =
2\pi i\sum_{\alpha_j\in \Omega}
\dfrac{f(\alpha_j)}{Q\dash(\alpha_j)}.
\label{10.2.10}
 \end{equation*}
\end{se}

\subsection*{Heildi yfir hring}

Látum nú $R(x,y)=p(x,y)/q(x,y)$ vera rætt fall af tveimur
raunbreytistærð\-um og gerum ráð fyrir að $q(x,y)\neq 0$ ef $x\sp
2+y\sp 2=1$.  Lítum á heildið 
 \begin{equation*}\int_0\sp{2\pi} R(\cos\theta,\sin \theta) \, d\theta.\label{10.2.11}
 \end{equation*}
Við athugum að ef $z$  er á einingarhringnum og við skrifum
$z=e\sp{i\theta}$, þá er 
\begin{gather*}
\cos\theta=\dfrac 12(e\sp{i\theta}+e\sp{-i\theta})
=\dfrac12(z+\dfrac 1z)=\dfrac{z\sp 2+1}{2z},\label{10.2.12}\\ 
\sin\theta=\dfrac 1{2i}(e\sp{i\theta}-e\sp{-i\theta})
=\dfrac1{2i}(z-\dfrac 1z)=\dfrac{z\sp 2-1}{2iz},\label{10.2.13}\\ 
dz=ie\sp{i\theta}d\theta, \qquad d\theta=\dfrac 1{iz}dz.\label{10.2.14}
\end{gather*}
Heldið sem við viljum reikna er vegheildi
$$
\int_0\sp {2\pi}R(\cos\theta,\sin
\theta)\, d\theta =
\int_{\partial S(0,1)}R\big(\dfrac{z\sp 2+1}{2z},\dfrac{z\sp 2-1}{2iz}\big)
\dfrac 1{iz}\, dz.
$$
Það er alltaf hægt að umrita heildisstofninn í síðasta heildinu yfir
á $f(z)/Q(z)$, þar sem $Q$ er margliða. Í því tilfelli að $Q$ hefur 
einungis einfaldar núllstöðvar, þá getum við beitt síðustu  setningu.


\subsection*{Heildi yfir rauntalnalínuna}


Nú skulum við líta á heildi af gerðinni 
 $$\int_{-\infty}\sp{+\infty}\dfrac{f(x)}{Q(x)}\, dx,
 $$
þar sem $f$ er fágað í grennd um $\R\cup H_+$, þar sem $H_+=\set{z\in
\C; \Im z>0}$ táknar efra hálfplanið og $Q(z)$ er margliða sem hefur
einungis einfaldar núllstöðvar í efra hálfplaninu og engar
núllstöðvar á $\R$.  Nú lítum við á svæðið
$\Omega_r=\set{z\in \C; \Im z>0, |z|<r}$, sem er hálf hringskífa.
Jaðar hennar samanstendur af línustrikinu $\scalar {-r}r$ og
hálfhringnum $\gamma_r(t)=re\sp{it}$, $t\in [0,\pi]$. Ef við veljum nú
$r$ það stórt að allar núllstöðvar $Q$ í $H_+$ séu 
innihaldnar í $\Omega_r$, þá gefur
síðasta setning okkur að 
 $$
\int_{\partial\Omega_r} \dfrac {f(z)}{Q(z)}\, dz
=\int_{-r}\sp r \dfrac {f(x)}{Q(x)}\, dx +
\int_{\gamma_r}\dfrac{f(z)}{Q(z)}\, dz = 
2\pi i\sum_{\alpha_j\in H_+} \dfrac{f(\alpha_j)}{Q\dash(\alpha_j)}.
 $$

\figura{fig096}{{\small Mynd: Lokuð hálfskífa}}

\noindent
Ef heildið yfir ${\gamma}_r$ stefnir á $0$ þegar $r\to+{\infty}$,  
þá fæst niðurstaðan
 $$\int_{-\infty}\sp {+\infty} \dfrac {f(x)}{Q(x)}\, dx=
2\pi i\sum_{\alpha_j\in H_+} \dfrac{f(\alpha_j)}{Q\dash(\alpha_j)}.
 $$

\section{Cauchy-formúlan fyrir afleiður}



\subsection*{Cauchy-formúlan fyrir afleiður}

Hugsum okkur nú að forsendur Cauchy-setningarinnar séu uppfylltar og
að $\partial\Omega$ sé stikað af vegunum
$\gamma_j:[a_j,b_j]\to \C$, $j=1,\dots,N$.  Ef við beitum
Cauchy-formúlunni og skrifum upp stikunina á heildinu, þá fæst
 $$f(x+iy)=\dfrac 1{2\pi i}\sum_{j=1}\sp N \int_{a_j}\sp {b_j}
\dfrac {f(\gamma_j(t))}{\gamma_j(t)-x-iy}\gamma_j\dash(t)\, dt, 
\qquad f\in \O(X).
 $$
Nú er heildisstofninn óendanlega oft deildanlegt fall af $(x,y)$ á
$\Omega$, samfelldur á köflum sem fall af $t$ á $[a_j,b_j]$ og þar að
auki fágað fall af $z=x+iy$.  Við
megum því deilda fallið $f$ með því að taka afleiður undir heildið,
 \begin{equation*}f\dash(z)=\partial_xf(z)=
\dfrac 1{2\pi i}\sum_{j=1}\sp N \int_{a_j}\sp {b_j}
\dfrac {f(\gamma_j(t))}{(\gamma_j(t)-x-iy)\sp 2}\gamma_j\dash(t)\, dt
\label{10.3.1}
 \end{equation*}
fyrir öll $z\in \Omega$.  {Á} þessari formúlu sjáum við síðan að
$f\dash$ er fágað fall og að við megum beita hlutafleiðunum undir
heildið og fáum að afleiðan $f\ddash$ af $f\dash$ er
 \begin{equation*}f\ddash(z)=\partial_x\sp 2f(z)=
\dfrac 2{2\pi i}\sum_{j=1}\sp N \int_{a_j}\sp {b_j}
\dfrac {f(\gamma_j(t))}{(\gamma_j(t)-x-iy)\sp 3}\gamma_j\dash(t)\, dt.
\label{10.3.2}
 \end{equation*}
Með því að velja $\Omega$ sem opnar skífur sem þekja $X$, þá fáum við
að $f\in C\sp{\infty}(X)$ og að allar afleiður af $f$ eru fáguð
föll.  Þegar við fjölluðum um Taylor-raðir í setningu
\ref{se:2.3.7}, þá skilgreindum við  hærri $\C$-afleiður $f\sp{(n)}$ af $f$ með
 $$f\sp{(0)}=f, \qquad f\sp{(n)}=\big(f\sp{(n-1)})\dash, \quad n\geq 1.
 $$
Með þrepun fáum við nú:
\begin{se}\label{set10.3.1}\tx{Cauchy-formúlan fyrir
afleiður\index{Cauchy!formúlan fyrir afleiður}\index{Cauchy}}  
Látum $X$ og $\Omega$ vera eins og í Cauchy-setningunni
og tökum $z\in \Omega$.  Þá er sérhvert $f$ í $\O(X)$
óendanlega oft deildanlegt á $X$, allar hlutafleiður af $f$ eru
fáguð föll og
 \begin{equation*}f\sp{(n)}(z)=
\dfrac {n!}{2\pi i}\int_{\partial\Omega}
\dfrac {f(\zeta)}{(\zeta-z)\sp {n+1}}\, d\zeta.
\label{10.3.3}
 \end{equation*}
\end{se}


\subsection*{Cauchy-ójöfnur}



Með því að skrifa $\Omega$ sem hringskífu, þá fáum við:

\begin{fs}\tx{Cauchy-ójöfnur\index{Cauchy!ójöfnur}}\index{Cauchy}  Ef
$X$ er opið hlutmengi af $\C$, $\bar S(\alpha,\varrho)\subset X$,
$f\in \O(X)$ og $|f(z)|\leq M$ fyrir öll $z\in \partial
S(\alpha,\varrho)$, þá er $$ |f\sp{(n)}(\alpha)|\leq
Mn!/\varrho\sp n. $$
\end{fs}


\subsection*{Setning Morera}


Eftirfarandi setning er andhverfa  Cauchy-setningarinnar: 

\begin{se}\tx{Morera\index{Morera-setningin}\index{setning!Morera}}  
Látum $X$ vera opið mengi í $\C$,  $f\in C(X)$ og
gerum ráð fyrir að  
 $$\int_{\partial\Omega} f\, dz =0
 $$
fyrir sérhvert þríhyrningssvæði $\Omega$ þannig að $\Omega\cup
\partial \Omega\subset X$.  Þá er $f\in \O(X)$.
\end{se}
Ein áhugaverð afleiðing af Cauchy-ójöfnunum er:


\subsection*{Setning Liurville}




\begin{se}\tx{Liouville\index{Liouville setningin}
\index{setning!Liouville}} Látum $f\in \O(\C)$ og gerum ráð fyrir að
$f$ sé takmarkað fall.  Þá er $f$ fasti.  
\end{se}
\begin{so} 
Gerum ráð fyrir að $|f(z)|\leq M$ fyrir öll $z\in \C$.
Látum $z\in\C$ og $\varrho>0$.  Ójöfnur Cauchy gefa 
 $|f\dash(z)|\leq  M/\varrho$.
Við látum $\varrho\to +\infty$ og fáum að $f\dash(z)=0$.
Þar með er $f$ fastafall.
\end{so}



\subsection*{Undirstöðusetning algebrunnar}


\begin{se}
\tx{Undirstöðusetning algebrunnar\index{undirstöðusetning algebrunnar}}
Sérhver margliða af af stigi $n\geq 1$ með stuðla í $\C$ hefur núllstöð í $\C$.  
\end{se}



\section{Samleitni í jöfnum mæli}


Í útreikningum okkar  þurfum við oft að  vita
hvort formúlur eins og  
\begin{gather*}
\lim_{t\to \alpha}\lim_{n\to+\infty}f_n(t)=
\lim_{n\to+\infty}\lim_{t\to \alpha}f_n(t),\\
\lim_{t\to \alpha}\sum_{n=0}^\infty f_n(t)=\sum_{n=0}^\infty
\lim_{t\to \alpha}f_n(t)\\
\lim_{n\to\infty}\ \int_a \sp b f_n(t)  \, dt=
\int_a \sp b \lim_{n\to\infty}\
f_n(t) \, dt,\\
\sum_{n=0}^{\infty}\ \int_a \sp b f_n(t)  \, dt=
\int_a \sp b \bigg(\sum_{n=0}^{\infty}\
f_n(t)\bigg) \, dt,\\
\dfrac d{dt} \lim_{n\to \infty} f_n(t) =\lim_{n\to\infty}
\dfrac d{dt} f_n(t),\\
\dfrac d{dt} \sum_{n=0}^{\infty} f_n(t) =\sum_{n=0}^{\infty}
\dfrac d{dt} f_n(t),
\end{gather*}
gildi, þar sem $\set{f_n}$  er runa af föllum sem skilgreind eru á
bilinu $[a,b]$.  Eins getum við þurft að vita hvort markfall
samleitinnar fallarunu sé samfellt eða deildanlegt.  Við ætlum nú að
fjalla almennt um skilyrði á rununa $\set{f_n}$ sem tryggja að þessar
formúlur gildi.

\subsection*{Skilgreiningar og einfaldar
afleiðingar þeirra}

Við byrjum á því að rifja upp skilgreininguna á samleitni fallaruna.
Látum  $A$ vera mengi og $\set{ f_n}$ vera runu af föllum $f_n:A\to
\C$. Við segjum að runan $\set{f_n}$ stefni á fallið $f$, og táknum
það með 
 $$\lim_{n\to\infty}f_n=f \qquad \text{og} \qquad
f_n\to f,
 $$
ef talnarunan $\set{f_n(a)}$ stefnir á $f(a)$ fyrir öll $a\in A$.
Þetta þýðir að fyrir sérhvert $a\in A$ og sérhvert $\varepsilon>0$ er
til $N>0$ þannig að 
 $$|f_n(a)-f(a)|<\varepsilon, \qquad \text{fyrir öll} \quad n\geq N.
 $$
Talan $N$ getur verið háð bæði $a$ og $\varepsilon$,
$N=N(a,\varepsilon)$.  Ef hægt er að velja töluna $N$ 
{\it óháð} $a$, þá segjum við að fallarunan $\set{f_n}$ stefni á
fallið $f$ {\it í jöfnum mæli á } $A$:

\begin{sk}
Látum $A$ vera mengi og $\set{f_n}$ vera runu af föllum á $A$ með
gildi í $\C$.  Við segjum að $\set{f_n}$ stefni á fallið $f$ {\it í jöfnum
mæli} á $A$, ef fyrir sérhvert $\varepsilon>0$ er til $N$ þannig að
$$|f_n(a)-f(a)|<\varepsilon, \qquad 
\text{fyrir öll $n\geq N$ og öll $a\in A$.}
$$
Við segjum að $\set{f_n}$  sé {\it samleitin í jöfnum mæli á $A$}, 
ef til er fall
$f$ þannig að $\set{f_n}$ stefni á $f$ í jöfnum mæli á $A$.  Við
segjum að fallaröðin $\sum_{k=0}\sp\infty f_k$ sé {\it samleitin í jöfnum
mæli} ef runan af hlutsummum $\set{\sum_{k=0}\sp n f_k}$ er samleitin
í jöfnum mæli.  Ef fallaröðin $\sum_{k=0}\sp\infty |f_k|$
er samleitin í jöfnum mæli á $A$, þá segjum við að
$\sum_{k=0}\sp\infty f_k$  sé {\it alsamleitin í jöfnum
mæli\index{alsamleitinn í jöfnu mæli}} á
menginu $A$.
\end{sk}

\medskip
Í því tilfelli að $f_n$ og $f$ eru raungild föll 
má einnig orða skilgreininguna svo, að fyrir sérhvert $\varepsilon>0$
sé til $N=N(\varepsilon)$, þannig fyrir öll $n\geq N$ er graf fallsins 
$f_n$ innihaldið í menginu 
$$\set{(a,y); a\in A, f(a)-\varepsilon<y<f(a)+\varepsilon}.
$$

\begin{sy} 
Myndin sýnir runu $\set{f_n}$, $f_n:\R\to \R$, sem stefnir á
núllfallið $f$, en gerir það ekki í jöfnum mæli, því
$|f_n(1/n)-f(1/n)|=1$ fyrir öll $n$.  

\figura{figA1}{{\small Mynd: $f_n\to 0$, {\it ekki} í jöfnum mæli}}
\end{sy}


\subsection*{Samleitnipróf Weierstrass}

Við höfum  samanburðarpróf fyrir samleitni í jöfnum mæli:

\begin{se}\tx{Weierstrass--próf}  Gerum ráð
fyrir að $\sum_{k=0}\sp 
\infty f_k$ sé fallaröð á menginu $A$,  $\sum_{k=0}\sp
\infty M_k$ sé samleitin röð af jákvæðum rauntölum og 
$$0\leq |f_k(a)| \leq M_k \qquad \text{fyrir öll  $k\geq 1$ og öll $
a\in A$.}
$$
Þá er röðin $\sum_{k=0}\sp \infty f_k$ samleitin í jöfnum mæli á
$A$.
\end{se}


\subsection*{Setning Abels}

\begin{se}\tx{Abel\index{Abel-setningin}\index{setning!Abel}}
Ef $\sum_{n=0}^\infty a_nz^n$ er veldaröð með samleitnigeisla
$\varrho$, þá er hún samleitin í jöfnum mæli á sérhverri hringskífu
með miðju í $0$ og geisla $r<\varrho$. 
\end{se}

\subsection*{Samleitni í jöfnum mæli og samfelldni}

Nú ætlum við að kanna formúluna
 \begin{equation*}\lim_{t\to \alpha}\lim_{n\to+\infty}f_n(t)=
\lim_{n\to+\infty}\lim_{t\to \alpha}f_n(t). \label{B.2.1}
 \end{equation*}

\begin{se}\label{se:3.4.5}
 Látum $A$ vera hlutmengi af $\R\sp m$ og $\set{f_n}$ vera runu
af samfelldum föllum sem stefnir á fallið $f$ í jöfnum mæli á $A$.  Þá er
$f$ samfellt.
\end{se}


\begin{fs} 
Látum $A$ vera hlutmengi af $\R\sp m$ og $\sum_{k=0}\sp\infty
f_k$ vera röð af samfelldum föllum  sem er samleitin í jöfnum mæli á $A$.
Þá er 
 $$\lim_{x\to a} \sum\limits_{k=0}\sp\infty f_k(x) =
 \sum\limits_{k=0}\sp\infty \lim_{x\to a}f_k(x), \qquad \text{fyrir
öll $a\in A$.} 
 $$
\end{fs}



\subsection*{Samleitni í jöfnum mæli og heildun}

Næsta viðfangsefni er formúlan
\begin{equation*}
\lim_{n\to+\infty}\int_a \sp b f_n(t)  \, dt=
\int_a \sp b \lim_{n\to+\infty}\
f_n(t) \, dt.\label{B.3.1}
\end{equation*}

\begin{se}\label{se:B.3.1}
 Gerum ráð fyrir að   $\set{f_n}$ sé runa af  heildanlegum föllum á
$[a,b]$, að $f_n\to f$ í jöfnum mæli á bilinu $[a,b]$.  Setjum 
 $$g_n(x)=\int_a\sp x f_n(t)\, dt, \qquad
\text{og } \qquad g(x)=\int_a\sp x f(t)\, dt.
 $$
Þá stefnir $g_n$ á $ g$ í jöfnum mæli á $[a,b]$.
\end{se}

\medskip
Hliðstæða þessarar setningar fyrir raðir er:


\begin{fs} 
Gerum ráð fyrir að   $\set{f_k}$ sé runa af heildanlegum föllum
 á bilinu $[a,b]$  og
að röðin  $\sum_{k=0}\sp \infty f_k$ sé samleitin
í jöfnum mæli á bilinu $[a,b]$.  Þá er
 $$\int_a\sp x \sum_{k=0}\sp \infty f_k(t)\, dt
= \sum_{k=0}\sp \infty \int_a\sp x  f_k(t)\, dt, \qquad x\in [a,b].
 $$
\end{fs}


Með því að skipta á stærðinni $(b-a)$ og rúmmáli mengisins $A\subset
\R\sp m$ í sönnuninni á setninguni hér fyrir framan, fáum við með sömu
röksemdarfærslu: 

\begin{se}  Látum $A$ vera takmarkað hlutmengi í
$\R\sp m$ og $\set{f_n}$ vera runu af heildanlegum föllum á $A$.  Ef
$f_n\to f$ í jöfnum mæli á $A$, þá er 
 $$\lim\limits_{n\to +\infty} \int_A f_n(x)\, dx = \int_Af(x)\, dx.
 $$
\end{se}


Hliðstæðar setningar gilda einnig um vegheildi með tilliti til
bogalengdar og heildi yfir svæði með endanlegt flatarmál.

\begin{se}  Látum $X$ vera opið hlutmengi í
$\C$ og $\set{f_n}$ vera runu af samfelldum föllum á $X$.  Ef
$f_n\to f$ í jöfnum mæli á sérhverju lokuðu og takmörkuðu hlutmengi í 
$X$ og $\gamma$ er vegur í 
$X$,  þá er 
 $$\lim\limits_{n\to +\infty} \int_\gamma f_n(z)\, dz = 
\int_\gamma f(z)\, dz.
 $$
\end{se}



\subsection*{Samleitni í jöfnum mæli og deildun}

Nú snúum við okkur að formúlunni
\begin{equation*}
\dfrac d{dt} \lim_{n\to \infty} f_n(t) =\lim_{n\to\infty}
\dfrac d{dt} f_n(t).\label{B.4.1}
\end{equation*}

\begin{se}
Látum $\set{f_n}$ vera runu af föllum í  $C\sp 1([a,b])$,
 gerum ráð fyrir að runan $\set{f_n\dash}$ sé  samleitin í jöfnum mæli á
$[a,b]$ og að til sé $c\in [a,b]$ þannig runan $\set{f_n(c)}$ sé samleitin.
Þá er stefnir $\set{f_n}$ á fall $f\in C\sp 1([a,b])$ í jöfnum mæli á
$[a,b]$ og 
 $$f\dash(x)=\lim_{n\to \infty}f_n\dash(x), \qquad x\in [a,b].
 $$
\end{se}


\medskip
Með þrepun fáum við hliðstæða setningu fyrir hærri afleiður:

\begin{fs}
Látum $\set{f_n}$ vera runu af föllum í $C^m([a,b])$ og gerum
ráð fyrir því að runurnar $\set{f_n^{(k)}}$, $0\leq k\leq m$, séu
samleitnar í jöfnum mæli á $[a,b]$ og táknum markgildi $\set{f_n}$ með $f$.
Þá er $f\in C^m([a,b])$ og 
 $$f^{(k)}(t)=\lim_{n\to +\infty} f_n^{(k)}(t), \qquad t\in [a,b].
 $$
\end{fs}

\medskip
Raðaútgáfan ef þessari setningu er:

\begin{fs}
Látum $\sum_{n=0}^\infty f_n$ vera röð með liði $f_n$ í $C^m([a,b])$ og gerum
ráð fyrir því að raðirnar  $\sum_{n=0}^\infty {f_n^{(k)}}$, $0\leq k\leq m$, séu
samleitnar í jöfnum mæli á $[a,b]$ og setjum
$f=\sum_{n=0}^\infty {f_n}$.  Þá er $f\in C^m([a,b])$ og 
 $$f^{(k)}(t)=\sum_{n=0}^{\infty} f_n^{(k)}(t), \qquad t\in [a,b].
 $$
\end{fs}


\subsection*{Runur af fáguðum föllum}

Ef $\set{f_n}$ er runa af samfelldum föllum á opnu mengi $X$, sem er
samleitin í jöfnum mæli á sérhverju lokuðu og takmörkuðu hlutmengi af
$X$, þá er  markfallið $f$ samfellt
á $X$.   Setning Morera gefur okkur meira ef föllin eru fáguð:

\begin{se}
Ef $\set{f_n}$ er runa af fáguðum föllum á opnu hlutmengi $X$ af
$\C$, sem er samleitin í jöfnum mæli á sérhverju lokuðu og takmörkuðu
hlutmengi af $X$, þá er markfallið $f$ fágað og 
$$
\lim_{n\to \infty} f_n'(z)=f'(z)\qquad z\in X.
$$\end{se}

\smallskip

Við getum eins tekið fyrir óendanlegar raðir $\sum_{n=0}^\infty f_n$
af fáguðum föllum og fáum að markfallið
 $$f(z) = \sum_{n=0}^\infty f_n(z), \qquad z\in X,
 $$
er fágað, ef hlutsummurnar $s_N(z)=\sum_{n=0}^N f_n(z)$ eru samleitnar í jöfnum
mæli á sérhverju lokuðu og takmörkuðu hlutmengi af $X$ og þá má reikna
út $f'$ með því að deilda röðina lið fyrir lið,
 $$f'(z) = \sum_{n=0}^\infty f_n'(z), \qquad z\in X.
 $$


\section{Samleitnar veldaraðir}  


\subsection*{Liðun í veldaröð}

Látum $X$ vera opið mengi í $\C$, $f\in \O(X)$, $\varrho>0$ vera
þannig að $\overline S(\alpha,\varrho)\subset X$
og $f\in\O(X)$.  Þá gefur
Cauchy-formúlan okkur
 \begin{equation*}f(z)=\dfrac 1{2\pi i}\int_{\partial S(\alpha,\varrho)}
\dfrac{f(\zeta)}{\zeta-z}\, d\zeta, \qquad z\in S(\alpha,\varrho).
\label{10.4.1}
 \end{equation*}
Við athugum að 
 $$\dfrac 1{\zeta-z}=\dfrac 1{(\zeta-\alpha)-(z-\alpha)}=
\dfrac 1{\zeta-\alpha}\cdot\dfrac 1{1-(z-\alpha)/(\zeta-\alpha)}.
 $$
Nú er $|z-\alpha|/|\zeta-\alpha|<1$ ef $z\in S(\alpha,\varrho)$ og
$\zeta\in\partial S(\alpha,\varrho)$ og þar með getum við liðað
þáttinn lengst til hægri í kvótaröð
og fáum
 \begin{equation*}\dfrac 1{\zeta-z}=\dfrac 1{\zeta-\alpha}\sum_{n=0}\sp \infty
\dfrac{(z-\alpha)\sp n}{(\zeta-\alpha)\sp n}=\sum_{n=0}\sp \infty
\dfrac{(z-\alpha)\sp n}{(\zeta-\alpha)\sp {n+1}}.\label{10.4.2}
 \end{equation*}
Röðin er greinilega samleitin í jöfnum mæli fyrir öll
$\zeta\in\partial S(\alpha,\varrho)$ og öll $z\in \bar
S(\alpha,\varrho-\varepsilon)$, $\varepsilon<\varrho$, og því megum
við skipta á óendalegu summunni og heildinu í (\ref{10.4.1}).
Það gefur
 $$
f(z)=\sum_{n=0}\sp \infty \bigg( \dfrac 1{2\pi i}
\int_{\partial S(\alpha,\varrho)}
\dfrac {f(\zeta)}{(\zeta-\alpha)\sp{n+1}}\, d\zeta\bigg)
(z-\alpha)\sp n.
 $$
Samkvæmt Cauchy-formúlunni fyrir afleiður er
 $$\dfrac 1{2\pi i}
\int_{\partial S(\alpha,\varrho)}
\dfrac {f(\zeta)}{(\zeta-\alpha)\sp{n+1}}\, d\zeta=
\dfrac{f\sp{(n)}(\alpha)}{n!}.
 $$
Niðurstaða þessa útreiknings er: 


\begin{se} $X$ er opið hlutmengi af $\C$, $\alpha\in X$,
$\overline S(\alpha,\varrho)\subset X$ og $f\in \O(X)$,
þá er unnt að setja  $f$ fram með samleitinni veldaröð
á skífunni $S(\alpha,\varrho)$,
 \begin{equation*}f(z)=\sum_{n=0}\sp \infty a_n(z-\alpha)\sp n,
\qquad z\in S(\alpha,\varrho),
\label{10.4.3}
 \end{equation*}
þar sem stuðlarnir $a_n$ eru ótvírætt ákvarðaðir og eru gefnir með 
 \begin{equation*}a_n=\dfrac {f\sp{(n)}(\alpha)}{n!}.\label{10.4.4}
 \end{equation*}
Samleitnigeisli raðarinnar er stærri en eða jafn fjarlægðinni
frá $\alpha$ út á  jaðar $X$.
\end{se}


\begin{sk}
Ef $X$ er opið hlutmengi af $\C$, $\alpha\in X$ og $f\in \O(X)$, þá
kallast veldaröðin
 \begin{equation*}\sum\limits_{n=0}^\infty \dfrac{f^{(n)}(\alpha)}{n!}(z-\alpha)^n,\label{10.4.5}
 \end{equation*}
{\it Taylor-röð\index{Taylor-röð}} fágaða fallsins $f$ í punktinum $\alpha$.  Ef
$\alpha=0$, þá kallast hún {\it Maclaurin-röð\index{Maclaurin-röð}}
fágaða fallsins $f$. 
\end{sk}


\subsection*{Núllstöðvar fágaðra falla}



\begin{sk}
Látum  $X$ vera opið hlutmengi í $\C$, $\alpha\in X$ og $f\in \O(X)$.
Punkturinn $\alpha$ nefnist {\it
núllstöð\index{núllstöð}} fágaða
fallsins $f$ ef $f(\alpha)=0$ og mengið\index{núllstöð!mengi} ${\cal N}(f)=\set{\alpha\in X;
f(\alpha)=0}$ kallast {\it núllstöðvamengi\index{núllstöðvamengi}}
fágaða fallsins $f$. Ef $f$ er ekki núllfallið í
$S(\alpha,\varrho)$, þar sem $\varrho>0$, 
þá er til minnsta gildi $m>0$ á $n$ þannig að
$f^{(n)}(\alpha)\neq 0$.  
Talan $m$  nefnist {\it stig\index{stig}}
núllstöðvarinnar\index{stig!núllstöðvar}
$\alpha$.  Ef fallið $f$ er núll í heilli grennd um 
$\alpha$, þá segjum við að $f$ hafi núllstöð af {\it óendanlegu
stigi\index{óendanlegt stig}\index{stig!óendanlegt}\index{stig}} í $\alpha$.
\end{sk}


Eins og fyrir margliður þá er hægt að þátta núllstöðvar úr fáguðum föllum:


\begin{se} \label{se:10.4.3}  Fall $f\in \O(X)$ hefur núllstöð af stigi $m>0$ í punktinum
$\alpha\in X$ þá og því aðeins að til sé $g\in \O(X)$ þannig að
$g(\alpha)\neq 0$ og
 \begin{equation*}f(z)=(z-\alpha)\sp mg(z), \qquad z\in X.
\label{10.4.6}
 \end{equation*}
\end{se}


\section{Samsemdarsetningin}

\noindent
Við skulum rifja það upp að {\it svæði\index{svæði}} er opið
samanhangandi mengi, en það þýðir að sérhverja tvo punkta
$\alpha$ og $\beta$ í $X$ er unnt að tengja saman með vegi í $X$.  Ef
$A$ er hlutmengi í $\C$, þá er punktur $\alpha\in A$ sagður vera
{\it einangraður\index{einangraður punktur}} í $A$ ef til er
$\varepsilon>0$ þannig að $A\cap S(\alpha,\varepsilon)=\set{\alpha}$.
Mengi sem samanstendur af einangruðum punktum í $A$ er sagt vera {\it
dreift\index{dreift mengi}} í $A$.   Athugið að þetta þýðir að ekki er til
nein runa af {\it ólíkum } punktum í $A$ sem er samleitin og hefur
markgildi í $A$.

\begin{se}\tx{Samsemdarsetning I\index{samsemdarsetning}}  
Ef $X$ er svæði í $\C$, $f,g\in \O(X)$ og 
til er punktur ${\alpha}$ í $X$ þannig að
$f^{(n)}({\alpha})=g^{(n)}({\alpha})$ fyrir öll $n\geq 0$, 
þá er $f(z)=g(z)$ fyrir öll $z\in X$.  
\end{se}


\figura{fig0910}{{\small Mynd:  Punktar tengdir með ferli}}



\begin{se}
  Ef $X$ er svæði og $f\in \O(X)$ er ekki núllfallið, þá er
núllstöðvamengi ${\cal N}(f)=\set{z\in X; f(z)=0}$ fallsins $f$ dreift hlutmengi af $X$.
\end{se}


\begin{so} Látum $\alpha$ vera núllstöð fallsins $f$ og gerum ráð fyrir að
hún sé af stigi $m>0$.  Samkvæmt setningu \ref{se:10.4.3} er til fall
$g\in \O(X)$ þannig að $f(z)=(z-\alpha)\sp mg(z)$ fyrir öll $z\in X$
og $g(\alpha)\neq 0$.  Fyrst $g$ er samfellt, þá er til
$\varepsilon>0$, þannig að $g(z)\neq 0$ fyrir öll $z\in
S(\alpha,\varepsilon)$.  Við höfum því ${\cal N}(f)\cap S\sp
*(\alpha,\varepsilon)=\varnothing$ og þar með er ${\cal N}(f)$ dreift
mengi. 
\end{so}

\medskip
Við fáum nú enn sterkari útgáfu af samsemdarsetningunni:

\begin{se}\tx{Samsemdarsetning II\index{samsemdarsetning}}   
Ef $X$ er svæði, $f,g\in \O(X)$ og $f(a_j)=g(a_j)$ þar sem
$\set{a_j}$ er runa af ólíkum punktum, sem hefur markgildi $a\in X$,
þá er $f(z)=g(z)$ fyrir öll $z\in X$.  
\end{se}

\medskip
Samsemdarsetningin hefur mikla þýðingu.  Hún segir okkur meðal
annars, að ef $f$ er fall sem gefið er með veldaröð á bili $I$ á $\R$
og hægt er að útvíkka $f$ yfir í fágað fall á svæði $X$ í $\C$ sem
inniheldur $I$, þá er útvíkkunin ótvírætt ákvörðuð.  
Hún segir okkur einnig að 
$e\sp z=e\sp{x+iy}=e\sp x(\cos y+i\sin y)$ sé eina mögulega fágaða
útvíkkunin á veldisvísisfallinu $x\mapsto e\sp x$ og að höfuðgrein
lografallsins $\Log z$ sé eina mögulega fágaða framlengingin af
náttúrlega lografallinu $\ln x$ yfir í mengið $\C\setminus\set{x\in
\R; x\leq 0}$.  



\section{Hágildislögmálið}

\noindent
Eftirfarandi setning er merkilegt hjálpartæki til þess að sanna alls
konar ójöfnur fyrir $|f|$, þar sem $f$ er fágað fall:


\begin{se}\label{set10.9.1}\tx{Hágildislögmál I\index{hágildislögmál}} 
 Ef $X$ er svæði og $f\in \O(X)$, þá getur
$|f(z)|$ ekki haft staðbundið hágildi í $X$ nema $f$ sé fastafall.
\end{se}

\begin{se}\tx{Hágildislögmál II\index{hágildislögmál}}  Látum $X$ vera takmarkað svæði $f\in
\O(X)\cap C(\overline X)$ (samfellt á lokuninni $\overline X$).  Þá
tekur $|f(z)|$ hágildi á jaðri svæðisins $\partial X$.
\end{se}


\section{Vafningstölur vega}

\noindent
Látum $\gamma:[a,b]\to \C$ vera feril og $p$ vera punkt sem liggur
ekki á ferlinum.  Þá er hægt að skrifa 
 $$\gamma(t)=p+r(t)e^{i\theta(t)}, \qquad r=|\gamma(t)-p|, \qquad t\in [a,b],
 $$
þar sem fallið $\theta:[a,b]\to \R$ kallast  {\it horn fyrir
ferilinn\index{horn!fyrir feril} $\gamma$ mælt frá punktinum} $p$.
Fallið $\gamma$ er samfellt og af því leiðir að hægt er að velja
$\theta$ samfellt.  Ef $\gamma$ er vegur, þá er fallið $\gamma$
samfellt og samfellt deildanlegt á köflum og af því leiðir að hægt er
að velja $\theta$ með sömu eiginleika.  Sönnun á þessum staðreynum er
alls ekki flókin, en við látum hana eiga sig.  Fallið $\theta$ er ekki
ótvírætt ákvarðað, en mismunur á tveimur hornum $\theta$ og $\varphi$
fyrir ferillinn $\gamma$ mælt frá $p$ er fast heiltölumargfeldi af
$2\pi$. Þetta segir okkur að mismunurinn $\theta(b)-\theta(a)$ sé
óháður því hvernig hornið er valið. Ef ferillinn $\gamma$ er lokaður,
þá er $e^{i\theta(b)}=e^{i\theta(a)}$, sem segir okkur að
$\theta(b)-\theta(a)$ sé heiltölumargfeldi af $2\pi$.

\begin{sk}
Ef $\theta$ er samfellt horn fyrir ferilinn $\gamma$ mælt frá
punktinum $p$, þá kallast talan
$$ \theta(b)-\theta(a) $$
{\it hornauki ferilsins\index{hornauki ferils} $\gamma$ séð frá
punktinum $p$.} Ef $\gamma$ er lokaður ferill, þá nefnist heiltalan
$$ I(\gamma,p)=\dfrac 1{2\pi}(\theta(b)-\theta(a)) $$
{\it vafningstala
ferilsins\index{ferill!vafningstala}\index{vafningstala ferils}
$\gamma$ með tilliti til punktsins $p$}.
Við segjum að $\gamma$ {\it vefjist utan um\index{vefjast utan um}} $p$, ef
$I(\gamma,p)\neq 0$.  Mengi allra punkta $p$ sem liggja ekki á
ferlinum og ferillinn vefst utan um köllum við {\it innmengi
ferilsins\index{innmengi ferils}} $\gamma$ og við táknum það með
$I(\gamma)$.
\end{sk}

\figura{fig0911}{{\small Mynd: Hornauki }}

\noindent
Ef $\gamma$ er lokaður vegur, þá höfum við formúlu fyrir
vafningstölunni:


\begin{se}  Ef $\gamma$ er lokaður vegur, þá er
 $$I(\gamma,p)=\dfrac 1{2\pi i}\int_\gamma \dfrac{dz}{z-p}, 
\qquad p\in \C\setminus \mynd{\gamma}.
 $$
\end{se}

 Lítum nú á mengið $X=\C\setminus \mynd{\gamma}$ sem samanstendur af
öllum punktum $p$ sem eru ekki á ferlinum.    Það er hægt að skrifa
$X$ sem sammengi $X=\cup X_i$, $i\in I$ af sundurlægum svæðum $X_i$,
þar sem $I$ er eitthvert endanlegt eða teljanlega óendanlegt mengi.
Þessi mengi $X_i$ kallast {\it samhengisþættir\index{samhengisþáttur}}
mengisins $X$. Á sérhverjum samhengisþætti  $X_i$ er vafningstalan
fasti sem fall af $p$, því
$$
X_i\ni p\mapsto I(\gamma,p) = \dfrac 1{2\pi i}\int_\gamma \dfrac
{dz}{z-p}, 
$$

 
\noindent
er heiltölugilt fágað fall.  Eitt mengjanna $X_i$ er ótakmarkað og
við sjáum á formúlunni að $I(\gamma,p)\to 0$ ef $|p|\to +\infty$.
Þar með er vafningstalan jöfn $0$ á ótakmarkaða samhengisþættinum.


Mjög létt er að ákvarða vafningstölur fyrir alla skikkanlega
vegi.   Við tökum einn punkt í hverjum samhengisþætti í
$X\setminus\mynd \gamma$  og drögum beint línustrik frá honum yfir í
ótakmarkaða samhengisþáttinn.  Gæta verður þess að í öllum
skurðpunktum línunnar og vegarins sé snertivigurinn við veginn  ekki
samsíða línunni.    Við merkjum alla skurðpunkta, sem eru þannig að
vegurinn sker línuna frá hægri til vinstri séð frá punktinum $p$, 
með tölunni $1$, og við
merkjum hina punktana, sem eru þá þannig að vegurinn sker línuna frá
vinstri til hægri, með tölunni $-1$.  

\figura{fig0912}{{\small Mynd: Talning á skurðpunktum}}

\noindent
Við leggjum síðan saman allar tölur á sama línustriki.  Summan er
vafningstala fyrir alla punkta í samhengisþættinum, sem inniheldur 
upphafspunkt striksins. 

\figura{fig0913}{Mynd: Í samhengisþáttunum standa vafningstölurnar.}

 
\section {Einfaldlega samanhangandi svæði}

\noindent
Við höfum séð að um stjörnusvæði $X$ gildir að
vegheildi sérhvers fágaðs falls $f$ á $X$ yfir sérhvern lokaðan veg
er $0$.  Við sönnuðum þetta með því að sýna fram á að sérhvert fágað
fall $f$ á stjörnusvæði hafi stofnfall.  Hægt er að alhæfa þetta yfir
á almennari flokk mengja:

\begin{sk}  Opið mengi $X$ er sagt vera {\it einfaldlega
samanhangandi\index{einfaldrelga samhangandi mengi}} ef
$I(\gamma)\subset X$ fyrir sérhvern lokaðan veg $\gamma$ í $X$.
\end{sk}

Innmengi vegarins $\gamma$ samanstendur af öllum punktum $p$ sem
$\gamma$ vefst utanum, þar sem við segjum að $\gamma$ vefjist utanum 
$p$ ef vafningstalan $I(\gamma,p)$ er frábrugðin $0$.
Skilyrðið í skilgreiningunni segir því að í einfaldlega samanhangandi
mengi geti lokaðir vegir einungis vafist utanum punkta í $X$ og þar
með að þeir geti ekki vafist utan um punkta í fyllimenginu
$\C\setminus X$.  Þetta þýðir að mengið $X$ hafi engin göt.  Sem dæmi
má nefna að allar hringskífur eru einfaldlega samanhangandi, en
hringkragar eru það ekki.    

\figura{fig0914}{{\small Mynd: Einfaldlega  og ekki
einfaldlega samanhangandi svæði}}

\noindent
Einfaldlega samanhangandi svæði einkennast af fjölbreytilegum eiginleikum:


\begin{se}  Látum $X$ vera svæði í $\C$.  Þá er eftirfarandi jafngilt:

\smallskip\noindent
(i)  $X$ er einfaldlega samanhangandi.

\smallskip\noindent
(ii) Sérhvert fágað fall á $X$ hefur stofnfall.

\smallskip\noindent
(iii) Fyrir sérhvert $f\in \O(X)$ og sérhvern lokaðan veg $\gamma$ í
$X$ er
 \begin{equation*}\int_\gamma f(\zeta) \, d\zeta = 0.
\label{10.11.1}
 \end{equation*}
\smallskip\noindent
(iv) Fyrir sérhvert $f\in \O(X)$ og sérhvern lokaðan veg $\gamma$ í
$X$ er
 \begin{equation*}f(z)I(\gamma,z) = \dfrac 1{2\pi i}\int_\gamma \dfrac{f(\zeta)}
{\zeta-z} \, d\zeta.
\label{10.11.2}
 \end{equation*}
\smallskip\noindent
(v)  Sérhvert núllstöðvalaust fágað fall á $X$ hefur logra, þ.e.~ef
$f\in \O(X)$ og ${\cal N}(f)=\varnothing$, þá er til $g\in \O(X)$ þannig að
$f(z)=e^{g(z)}$, $z\in X$.

\smallskip\noindent
(vi) Sérhvert núllstöðvalaust fágað fall á $X$ hefur fágaða $n$-tu rót
fyrir öll $n\geq 1$, þ.e.~ef
$f\in \O(X)$ og ${\cal N}(f)=\varnothing$, þá er til $h\in \O(X)$ þannig að
$f(z)=h(z)^n$, $z\in X$.

\smallskip\noindent
(vii) Sérhvert núllstöðvalaust fágað fall á $X$ hefur fágaða aðra rót.
\end{se}
 

